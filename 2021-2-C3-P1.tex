% (find-LATEX "2021-2-C3-P1.tex")
% (defun c () (interactive) (find-LATEXsh "lualatex -record 2021-2-C3-P1.tex" :end))
% (defun C () (interactive) (find-LATEXsh "lualatex 2021-2-C3-P1.tex" "Success!!!"))
% (defun D () (interactive) (find-pdf-page      "~/LATEX/2021-2-C3-P1.pdf"))
% (defun d () (interactive) (find-pdftools-page "~/LATEX/2021-2-C3-P1.pdf"))
% (defun e () (interactive) (find-LATEX "2021-2-C3-P1.tex"))
% (defun o () (interactive) (find-LATEX "2021-2-C3-P1.tex"))
% (defun u () (interactive) (find-latex-upload-links "2021-2-C3-P1"))
% (defun v () (interactive) (find-2a '(e) '(d)))
% (defun d0 () (interactive) (find-ebuffer "2021-2-C3-P1.pdf"))
% (defun cv () (interactive) (C) (ee-kill-this-buffer) (v) (g))
%          (code-eec-LATEX "2021-2-C3-P1")
% (find-pdf-page   "~/LATEX/2021-2-C3-P1.pdf")
% (find-sh0 "cp -v  ~/LATEX/2021-2-C3-P1.pdf /tmp/")
% (find-sh0 "cp -v  ~/LATEX/2021-2-C3-P1.pdf /tmp/pen/")
%     (find-xournalpp "/tmp/2021-2-C3-P1.pdf")
%   file:///home/edrx/LATEX/2021-2-C3-P1.pdf
%               file:///tmp/2021-2-C3-P1.pdf
%           file:///tmp/pen/2021-2-C3-P1.pdf
% http://angg.twu.net/LATEX/2021-2-C3-P1.pdf
% (find-LATEX "2019.mk")
% (find-CN-aula-links "2021-2-C3-P1" "3" "c3m212p1" "c3p1")

% «.defs»		(to "defs")
% «.defs-T-and-B»	(to "defs-T-and-B")
% «.title»		(to "title")
% «.regras-e-dicas»	(to "regras-e-dicas")
% «.questao-1»		(to "questao-1")
% «.questao-2-abcde»	(to "questao-2-abcde")
% «.questao-2-fghi»	(to "questao-2-fghi")
% «.questao-2-jk»	(to "questao-2-jk")
% «.gabarito-maxima»	(to "gabarito-maxima")
%
% «.djvuize»		(to "djvuize")



% <videos>
% Video (not yet):
% (find-ssr-links     "c3m212p1" "2021-2-C3-P1")
% (code-eevvideo      "c3m212p1" "2021-2-C3-P1")
% (code-eevlinksvideo "c3m212p1" "2021-2-C3-P1")
% (find-c3m212p1video "0:00")

\documentclass[oneside,12pt]{article}
\usepackage[colorlinks,citecolor=DarkRed,urlcolor=DarkRed]{hyperref} % (find-es "tex" "hyperref")
\usepackage{amsmath}
\usepackage{amsfonts}
\usepackage{amssymb}
\usepackage{pict2e}
\usepackage[x11names,svgnames]{xcolor} % (find-es "tex" "xcolor")
\usepackage{colorweb}                  % (find-es "tex" "colorweb")
%\usepackage{tikz}
%
% (find-dn6 "preamble6.lua" "preamble0")
%\usepackage{proof}   % For derivation trees ("%:" lines)
%\input diagxy        % For 2D diagrams ("%D" lines)
%\xyoption{curve}     % For the ".curve=" feature in 2D diagrams
%
\usepackage{edrx21}               % (find-LATEX "edrx21.sty")
\input edrxaccents.tex            % (find-LATEX "edrxaccents.tex")
\input edrx21chars.tex            % (find-LATEX "edrx21chars.tex")
\input edrxheadfoot.tex           % (find-LATEX "edrxheadfoot.tex")
\input edrxgac2.tex               % (find-LATEX "edrxgac2.tex")
%\usepackage{emaxima}              % (find-LATEX "emaxima.sty")
%
%\usepackage[backend=biber,
%   style=alphabetic]{biblatex}            % (find-es "tex" "biber")
%\addbibresource{catsem-slides.bib}        % (find-LATEX "catsem-slides.bib")
%
% (find-es "tex" "geometry")
\usepackage[a6paper, landscape,
            top=1.5cm, bottom=.25cm, left=1cm, right=1cm, includefoot
           ]{geometry}
%
\begin{document}

\catcode`\^^J=10
\directlua{dofile "dednat6load.lua"}  % (find-LATEX "dednat6load.lua")
%L dofile "2022pict2e.lua"           -- (find-LATEX "2022pict2e.lua")
%L Pict2e.__index.suffix = "%"

% \pu
% \def\pictgridstyle{\color{GrayPale}\linethickness{0.3pt}}
% \def\pictaxesstyle{\linethickness{0.5pt}}
% % %L dofile "edrxtikz.lua"  -- (find-LATEX "edrxtikz.lua")
% % %L dofile "edrxpict.lua"  -- (find-LATEX "edrxpict.lua")
% % \pu

% «defs»  (to ".defs")
% (find-LATEX "edrx21defs.tex" "colors")
% (find-LATEX "edrx21.sty")

\def\u#1{\par{\footnotesize \url{#1}}}

\def\drafturl{http://angg.twu.net/LATEX/2021-2-C3.pdf}
\def\drafturl{http://angg.twu.net/2021.2-C3.html}
\def\draftfooter{\tiny \href{\drafturl}{\jobname{}} \ColorBrown{\shorttoday{} \hours}}

\def\derivs{\mathsf{derivs}}
\def\frt#1{\frac{f^{(#1)}(0)}{#1!}}
\def\ddy{\frac{d}{dy}}


% «defs-T-and-B»  (to ".defs-T-and-B")
% (c3m202p1p 6 "questao-2")
% (c3m202p1a   "questao-2")
\long\def\ColorOrange#1{{\color{orange!90!black}#1}}
\def\T(Total: #1 pts){{\bf(Total: #1)}}
\def\T(Total: #1 pts){{\bf(Total: #1 pts)}}
\def\T(Total: #1 pts){\ColorRed{\bf(Total: #1 pts)}}
\def\B       (#1 pts){\ColorOrange{\bf(#1 pts)}}



%  _____ _ _   _                               
% |_   _(_) |_| | ___   _ __   __ _  __ _  ___ 
%   | | | | __| |/ _ \ | '_ \ / _` |/ _` |/ _ \
%   | | | | |_| |  __/ | |_) | (_| | (_| |  __/
%   |_| |_|\__|_|\___| | .__/ \__,_|\__, |\___|
%                      |_|          |___/      
%
% «title»  (to ".title")
% (c3m212p1p 1 "title")
% (c3m212p1a   "title")

\thispagestyle{empty}

\begin{center}

\vspace*{1.2cm}

{\bf \Large Cálculo 3 - 2021.2}

\bsk

Primeira prova (P1)

\bsk

Eduardo Ochs - RCN/PURO/UFF

\url{http://angg.twu.net/2021.2-C3.html}

\end{center}

\newpage

% «regras-e-dicas»  (to ".regras-e-dicas")
% (c3m212p1p 3 "regras-e-dicas")
% (c3m212p1a   "regras-e-dicas")

{\bf Regras e dicas}

As regras e dicas são as mesmas dos mini-testes,

\ssk

\url{http://angg.twu.net/LATEX/2020-2-C3-MT1.pdf}

\url{http://angg.twu.net/LATEX/2020-2-C3-MT2.pdf}

\ssk

exceto que a prova vai ser disponibilizada às 22:30 do dia

21/janeiro/2022 e deve ser entregue até as 10:30 do dia

23/janeiro/2022.

\bsk

Pra fazer essa prova você vai precisar de idéias que a gente

viu durante o curso todo. Se você precisar saber onde estão

as idéias necessárias pra resolver algum item pergunte

\ColorRed{no grupo do Telegram da turma} que eu respondo com um

link pros slides, vídeos, ou livros em que aquela idéia

aparece.

\newpage

% «questao-1»  (to ".questao-1")
% (c3m212p1p 3 "questao-1")
% (c3m212p1a   "questao-1")
% (c3m212tap 15 "exercicio-5")
% (c3m212taa    "exercicio-5")

{\bf Questão 1.}

\T(Total: 5.0 pts)

Esta questão é baseada no Exercício 5

do PDF sobre Séries de Taylor.

\ssk

Digamos que $f,g:\R→\R$ são funções suaves,

$x_0∈\R$, $y=y(x)=f(x)$, $z=z(y)=g(y)$, $h=g∘f$.

\bsk
\bsk

Calcule $\derivs_{x_0}^3(h(x))$ em ``notação de físicos''.

\bsk
\bsk

Dica: você pode começar fazendo MUITAS contas pequenas

e traduções simples que você sabe que são verdade ---

por exemplo ``$z=z(y(x))=g(f(x))$'', ``$z_x = \ddx g(f(x))$'',

``$z_y = z_y(y) = \ddy g(y) = g'(y) = g'(y(x))$'', etc.



\newpage


% «questao-2-abcde»  (to ".questao-2-abcde")
% (c3m212p1p 4 "questao-2-abcde")
% (c3m212p1a   "questao-2-abcde")

{\bf Questão 2.}

\T(Total: 5.0 pts)

Sejam $x=x(t)=f(t)$, $y=y(t)=g(t)$,

$z=z(x,y)=H(x,y)=x+y-5$,

$P(t)=(x(t),y(t))=(4-t^2,3+\ColorRed{t})$,

$t_0=0$.

\bsk


Calcule:

a) \B(0.2 pts) $x_0$, $y_0$, $P(0), P(1), P(-1)$,

b) \B(0.3 pts) $P'(t), P''(t), P'(0), P'(1), P'(-1), P''(0)$.


\msk

Represente graficamente num gráfico só:

c) \B(0.5 pts) $P(0)+P'(0), P(1)+P'(1), P(-1)+P'(-1)$,  

d) \B(0.5 pts) A trajetória $P(t)$ entre $t=-1$ e $t=+1$,

e) \B(0.5 pts) $P(0)+P''(0)$.



\newpage

% «questao-2-fghi»  (to ".questao-2-fghi")
% (c3m212p1p 5 "questao-2-fghi")
% (c3m212p1a   "questao-2-fghi")

{\bf Questão 2 (cont.)}

\msk

f) \B(0.2 pts) Faça o diagrama de numerozinhos de $H(x,y)$

para $x∈\{x_0-1, x_0, x_0+1\}$, $y∈\{y_0-1, y_0, y_0+1\}$.

\msk

g) \B(0.8 pts) Represente graficamente a superfície $z = H(x,y)$

no quadrado com $x∈[x_0-1, x_0+1]$, $y∈[y_0-1, y_0+1]$.

Faça um desenho em perspectiva improvisada com postes

ligados por cabos, como aqui:

\ssk

{\footnotesize

% (c3m212dnp 9 "figuras-3D")
% (c3m212dna   "figuras-3D")
%    http://angg.twu.net/LATEX/2021-2-C3-diag-nums.pdf#page=9
\url{http://angg.twu.net/LATEX/2021-2-C3-diag-nums.pdf#page=9}

}

\bsk

Desenhe sobre a sua figura do item anterior

as seguintes trajetórias (para $t∈[-1,1]$):

h) \B(0.2 pts) $(x(t), y(t), 0)$,

i) \B(0.8 pts) $(x(t), y(t), z(x,y))$.


\newpage

% «questao-2-jk»  (to ".questao-2-jk")
% (c3m212p1p 6 "questao-2-jk")
% (c3m212p1a   "questao-2-jk")

{\bf Questão 2 (cont.)}

\ssk

Sejam:

$Q(t) = (x(t), y(t), 0)$,

$R(t) = (x(t), y(t), z(x(t),y(t)))$.

\bsk

j) \B(0.8 pts) Calcule $R(0)$, $R'(0)$, $R''(0)$. 

k) \B(0.2 pts) Calcule $Q(0)$, $Q'(0)$, $Q''(0)$. 

% Versao original, errada:
%
% i) \B(0.8 pts) Calcule $R(0)$, $R'(0)$, $R''(0)$. 
%
% j) \B(0.2 pts) Calcule $Q(0)$, $Q'(0)$, $Q''(0)$. 



\newpage

% «gabarito-maxima»  (to ".gabarito-maxima")
% (setq eepitch-preprocess-regexp "^")
% (setq eepitch-preprocess-regexp "^%T ")
%
%T  (eepitch-maxima)
%T  (eepitch-kill)
%T  (eepitch-maxima)
%T x      : 4-t^2;
%T y      : 3+t;
%T z      : x+y-5;
%T P      : [x,y];
%T Q      : [x,y,0];
%T R      : [x,y,z];
%T P_t    : diff(P,   t);
%T P_tt   : diff(P_t, t);
%T subst([t= 0], z);
%T subst([t= 1], P);
%T subst([t= 0], P);
%T subst([t=-1], P);
%T subst(t, 0, z);
%T ["a:", subst([t=0],x), subst([t=0],y),
%T        subst([t=0],P), subst([t=1],P), subst([t=-1],P)]; 
%T ["b:", P_t, P_tt,
%T        subst([t=0],P_t), subst([t=1],P_t), subst([t=-1],P_t),
%T        subst([t=0],P_tt)];
%T ["c:", [subst([t=0],P), "+", subst([t=0],P_t)],
%T        [subst([t=1],P), "+", subst([t=1],P_t)],
%T        [subst([t=-1],P), "+", subst([t=-1],P_t)]
%T        ];
%T  d:
%T    plot2d([parametric, x, y, [t, -1, 1]]);
%T ["e:", [subst([t=0],P), "+", subst([t=0],P_tt)]];



{\bf Questão 2: gabarito parcial}

% ["a:", subst([t=0],x), subst([t=0],y),
%        subst([t=0],P), subst([t=1],P), subst([t=-1],P)]; 
% (%o14)                [a:, 4, 3, [4, 3], [3, 4], [3, 2]]
% (%i15) ["b:", P_t, P_tt,
% (%i15)        subst([t=0],P_t), subst([t=1],P_t), subst([t=-1],P_t),
% (%i15)        subst([t=0],P_tt)];
% (%o15)  [b:, [- 2 t, 1], [- 2, 0], [0, 1], [- 2, 1], [2, 1], [- 2, 0]]
% (%i16) ["c:", [subst([t=0],P), "+", subst([t=0],P_t)],
%        [subst([t=1],P), "+", subst([t=1],P_t)],
%        [subst([t=-1],P), "+", subst([t=-1],P_t)]
%        ];
% (%o16) [c:, [[4, 3], +, [0, 1]], [[3, 4], +, [- 2, 1]], [[3, 2], +, [2, 1]]]
% (%i17) 

%T x(t)   = 4-t^2;
%T y(t)   = 3+t;
%T z(x,y) = x+y-5;
%T H(x,y) = x+y-5;
%T zoft(t)   = z(x(t),y(t));
%T xyzoft(t) = [x(t), y(t), zoft(t)];
%T xyzoft(0);


a) \B(0.2 pts)
   $x_0=4$,
   $y_0=3$,
   $P(0) = (4, 3)$,

   $P(1) = (3, 4)$,
   $P(-1) = (3, 2)$

b) \B(0.3 pts)
   $P'(t)  = \VEC{-2t, 1}$,
   $P''(t) = \VEC{-2, 0}$,
   $P'(0)  = \VEC{0, 1}$,

   $P'(1)  = \VEC{- 2, 1}$,
   $P'(-1) = \VEC{2, 1}$,
   $P''(0) = \VEC{- 2, 0}$

\msk

% (find-LATEX "2022pict2e.lua" "Pict2e")
% (find-LATEX "2022pict2e.lua" "Pict2e" "b0show =")

%L s2 = Surface {
%L      unitlength="15pt", sw=v(-1,-3), ne=v(13,7),
%L      p1=v(2,-0.5), p2=v(0.5,1.5), p3=v(0,0.5),
%L      maxx=5, maxy=4, maxz=3,
%L      x0=4, y0=3, nsteps=16,
%L      -- f = function (x, y) return (x-4)^2 + (y-3)^2 end,
%L      f = function (x, y) return (x-3) + (y-2) end,
%L   }
%L
%L p = s2:base() + s2:fig()
%L
%L fP = function (t) return v3(4-t^2, 3+t, 0) end
%L fQ = function (t) return v3(4-t^2, 3+t, 0) end
%L fR = function (t) return v3(4-t^2, 3+t, (4-t^2)+(3+t)-5) end
%L trajQ = Points2(map(fQ, seq(-1, 1, 1/16))):Line()
%L trajR = Points2(map(fR, seq(-1, 1, 1/16))):Line()
%L
%L -- print(s2)
%L -- p = (s2:base() + s2:fig() + s2:twoDgrid())
%L --     :setbounds():bepcb():def("FOO")
%L    
%L    p = (s2:base() + s2:fig() + trajQ + trajR)
%L        :setbounds():bep():def("FOO")
%L p:output()

\pu


% \scalebox{1.0}{\def\colwidth{4cm}\firstcol{

% c) \B(0.5 pts)

% d) \B(0.5 pts)

% e) \B(0.5 pts)

% \msk

% f) \B(0.2 pts) (numerozinhos

% g) \B(0.8 pts)

% h) \B(0.2 pts)

% i) \B(0.8 pts)

% }\anothercol{

\unitlength=15pt

$\FOO$

Faltou desenhar:

Diagrama de numerozinhos

Vetores

% }}




\msk

% j) \B(0.8 pts)

% k) \B(0.2 pts)




%T load("/usr/share/emacs/site-lisp/maxima/emaxima.lisp")$
%T display2d:'emaxima$







%\printbibliography

\GenericWarning{Success:}{Success!!!}  % Used by `M-x cv'

\end{document}

%  ____  _             _         
% |  _ \(_)_   ___   _(_)_______ 
% | | | | \ \ / / | | | |_  / _ \
% | |_| | |\ V /| |_| | |/ /  __/
% |____// | \_/  \__,_|_/___\___|
%     |__/                       
%
% «djvuize»  (to ".djvuize")
% (find-LATEXgrep "grep --color -nH --null -e djvuize 2020-1*.tex")

 (eepitch-shell)
 (eepitch-kill)
 (eepitch-shell)
# (find-fline "~/2021.2-C3/")
# (find-fline "~/LATEX/2021-2-C3/")
# (find-fline "~/bin/djvuize")

cd /tmp/
for i in *.jpg; do echo f $(basename $i .jpg); done

f () { rm -v $1.pdf;  textcleaner -f 50 -o  5 $1.jpg $1.png; djvuize $1.pdf; xpdf $1.pdf }
f () { rm -v $1.pdf;  textcleaner -f 50 -o 10 $1.jpg $1.png; djvuize $1.pdf; xpdf $1.pdf }
f () { rm -v $1.pdf;  textcleaner -f 50 -o 20 $1.jpg $1.png; djvuize $1.pdf; xpdf $1.pdf }

f () { rm -fv $1.png $1.pdf; djvuize $1.pdf }
f () { rm -fv $1.png $1.pdf; djvuize WHITEBOARDOPTS="-m 1.0 -f 15" $1.pdf; xpdf $1.pdf }
f () { rm -fv $1.png $1.pdf; djvuize WHITEBOARDOPTS="-m 1.0 -f 30" $1.pdf; xpdf $1.pdf }
f () { rm -fv $1.png $1.pdf; djvuize WHITEBOARDOPTS="-m 1.0 -f 45" $1.pdf; xpdf $1.pdf }
f () { rm -fv $1.png $1.pdf; djvuize WHITEBOARDOPTS="-m 0.5" $1.pdf; xpdf $1.pdf }
f () { rm -fv $1.png $1.pdf; djvuize WHITEBOARDOPTS="-m 0.25" $1.pdf; xpdf $1.pdf }
f () { cp -fv $1.png $1.pdf       ~/2021.2-C3/
       cp -fv        $1.pdf ~/LATEX/2021-2-C3/
       cat <<%%%
% (find-latexscan-links "C3" "$1")
%%%
}

f 20201213_area_em_funcao_de_theta
f 20201213_area_em_funcao_de_x
f 20201213_area_fatias_pizza



%  __  __       _        
% |  \/  | __ _| | _____ 
% | |\/| |/ _` | |/ / _ \
% | |  | | (_| |   <  __/
% |_|  |_|\__,_|_|\_\___|
%                        
% <make>

 (eepitch-shell)
 (eepitch-kill)
 (eepitch-shell)
# (find-LATEXfile "2019planar-has-1.mk")
make -f 2019.mk STEM=2021-2-C3-P1 veryclean
make -f 2019.mk STEM=2021-2-C3-P1 pdf

% Local Variables:
% coding: utf-8-unix
% ee-tla: "c3p1"
% ee-tla: "c3m212p1"
% End:
