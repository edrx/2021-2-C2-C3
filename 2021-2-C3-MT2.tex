% (find-LATEX "2021-2-C3-MT2.tex")
% (defun c () (interactive) (find-LATEXsh "lualatex -record 2021-2-C3-MT2.tex" :end))
% (defun C () (interactive) (find-LATEXsh "lualatex 2021-2-C3-MT2.tex" "Success!!!"))
% (defun D () (interactive) (find-pdf-page      "~/LATEX/2021-2-C3-MT2.pdf"))
% (defun d () (interactive) (find-pdftools-page "~/LATEX/2021-2-C3-MT2.pdf"))
% (defun e () (interactive) (find-LATEX "2021-2-C3-MT2.tex"))
% (defun o () (interactive) (find-LATEX "2021-2-C3-MT2.tex"))
% (defun u () (interactive) (find-latex-upload-links "2021-2-C3-MT2"))
% (defun v () (interactive) (find-2a '(e) '(d)))
% (defun d0 () (interactive) (find-ebuffer "2021-2-C3-MT2.pdf"))
% (defun cv () (interactive) (C) (ee-kill-this-buffer) (v) (g))
%          (code-eec-LATEX "2021-2-C3-MT2")
% (find-pdf-page   "~/LATEX/2021-2-C3-MT2.pdf")
% (find-sh0 "cp -v  ~/LATEX/2021-2-C3-MT2.pdf /tmp/")
% (find-sh0 "cp -v  ~/LATEX/2021-2-C3-MT2.pdf /tmp/pen/")
%     (find-xournalpp "/tmp/2021-2-C3-MT2.pdf")
%   file:///home/edrx/LATEX/2021-2-C3-MT2.pdf
%               file:///tmp/2021-2-C3-MT2.pdf
%           file:///tmp/pen/2021-2-C3-MT2.pdf
% http://angg.twu.net/LATEX/2021-2-C3-MT2.pdf
% (find-LATEX "2019.mk")
% (find-CN-aula-links "2021-2-C3-MT2" "3" "c3m212mt2" "c3mt2")

% «.videos»		(to "videos")
% «.video-a»		(to "video-a")
% «.video-b»		(to "video-b")
% «.video-c»		(to "video-c")
% «.defs»		(to "defs")
% «.title»		(to "title")
% «.itens»		(to "itens")
% «.gabarito»		(to "gabarito")
% «.gabarito-maxima»	(to "gabarito-maxima")
% «.gabarito-5x5»	(to "gabarito-5x5")
% «.figura-homogeneas»	(to "figura-homogeneas")
%
% «.djvuize»		(to "djvuize")



% «videos»  (to ".videos")
% «video-a»  (to ".video-a")
% (c3m212mt2a     "video-a")
% (find-ssr-links     "c3m212mt2a" "2021-2-C3-MT2" "Rz01pLaL9Z0")
% (code-eevvideo      "c3m212mt2a" "2021-2-C3-MT2" "Rz01pLaL9Z0")
% (code-eevlinksvideo "c3m212mt2a" "2021-2-C3-MT2" "Rz01pLaL9Z0")
% (find-c3m212mt2avideo "0:00")
% (find-c3m212mt2avideo "0:40" "Diagramas de sinais")
% (find-c3m212mt2avideo "2:25" "zeros na diagonal")

% «video-b»  (to ".video-b")
% (c3m212mt2a     "video-b")
% (find-ssr-links     "c3m212mt2b" "2021-2-C3-MT2-2" "lDvAMU5aNvc")
% (code-eevvideo      "c3m212mt2b" "2021-2-C3-MT2-2" "lDvAMU5aNvc")
% (code-eevlinksvideo "c3m212mt2b" "2021-2-C3-MT2-2" "lDvAMU5aNvc")
% (find-c3m212mt2bvideo "0:00")
% (find-c3m212mt2bvideo "2:25" "rodar a figura")
% (find-c3m212mt2bvideo "3:10" "acrescentei um plano horizontal")
% (find-c3m212mt2bvideo "4:40" "olhar pra segunda derivada")
% (find-c3m212mt2bvideo "5:50" "f(x,y) = xy")
% (find-c3m212mt2bvideo "7:50" "raízes reais")

% «video-c»  (to ".video-c")
% (c3m212mt2a     "video-c")
% (find-ssr-links     "c3m212mt2c" "2021-2-C3-MT2-3" "cL3G-t3mILs")
% (code-eevvideo      "c3m212mt2c" "2021-2-C3-MT2-3" "cL3G-t3mILs")
% (code-eevlinksvideo "c3m212mt2c" "2021-2-C3-MT2-3" "cL3G-t3mILs")
% (find-yttranscript-links "c3m212mt2c" "cL3G-t3mILs")
% (find-c3m212mt2cvideo "0:00")
% (find-c3m212mt2cvideo "1:45" "x0 e y0")
% (find-c3m212mt2cvideo "2:38" "uma outra linha de zeros aqui")
% (find-c3m212mt2cvideo "4:06" "f(2,1) = f(-2,-1)")
% (find-c3m212mt2cvideo "7:25" "e se α=2?")
% (find-c3m212mt2cvideo "10:35" "(x,y) = (0.5,1)")
% (find-c3m212mt2cvideo "10:42" "e eu multiplico esse ponto por 4? z=20...")
% (find-c3m212mt2cvideo "13:02" "esse z e' 4 vezes esse z")


\documentclass[oneside,12pt]{article}
\usepackage[colorlinks,citecolor=DarkRed,urlcolor=DarkRed]{hyperref} % (find-es "tex" "hyperref")
\usepackage{amsmath}
\usepackage{amsfonts}
\usepackage{amssymb}
\usepackage{pict2e}
\usepackage[x11names,svgnames]{xcolor} % (find-es "tex" "xcolor")
\usepackage{colorweb}                  % (find-es "tex" "colorweb")
%\usepackage{tikz}
%
% (find-dn6 "preamble6.lua" "preamble0")
%\usepackage{proof}   % For derivation trees ("%:" lines)
%\input diagxy        % For 2D diagrams ("%D" lines)
%\xyoption{curve}     % For the ".curve=" feature in 2D diagrams
%
\usepackage{edrx21}               % (find-LATEX "edrx21.sty")
\input edrxaccents.tex            % (find-LATEX "edrxaccents.tex")
\input edrx21chars.tex            % (find-LATEX "edrx21chars.tex")
\input edrxheadfoot.tex           % (find-LATEX "edrxheadfoot.tex")
\input edrxgac2.tex               % (find-LATEX "edrxgac2.tex")
%\usepackage{emaxima}              % (find-LATEX "emaxima.sty")
%
%\usepackage[backend=biber,
%   style=alphabetic]{biblatex}            % (find-es "tex" "biber")
%\addbibresource{catsem-slides.bib}        % (find-LATEX "catsem-slides.bib")
%
% (find-es "tex" "geometry")
\usepackage[a6paper, landscape,
            top=1.5cm, bottom=.25cm, left=1cm, right=1cm, includefoot
           ]{geometry}
%
\begin{document}

\catcode`\^^J=10
\directlua{dofile "dednat6load.lua"}  % (find-LATEX "dednat6load.lua")
%
%L dofile "2021pict2e.lua"           -- (find-LATEX "2021pict2e.lua")
%L Pict2e.__index.suffix = "%"
\pu
\def\pictgridstyle{\color{GrayPale}\linethickness{0.3pt}}
\def\pictaxesstyle{\linethickness{0.5pt}}

% %L dofile "edrxtikz.lua"  -- (find-LATEX "edrxtikz.lua")
% %L dofile "edrxpict.lua"  -- (find-LATEX "edrxpict.lua")
% \pu

% «defs»  (to ".defs")
% (find-LATEX "edrx21defs.tex" "colors")
% (find-LATEX "edrx21.sty")

\def\u#1{\par{\footnotesize \url{#1}}}

\def\drafturl{http://angg.twu.net/LATEX/2021-2-C3.pdf}
\def\drafturl{http://angg.twu.net/2021.2-C3.html}
\def\draftfooter{\tiny \href{\drafturl}{\jobname{}} \ColorBrown{\shorttoday{} \hours}}



%  _____ _ _   _                               
% |_   _(_) |_| | ___   _ __   __ _  __ _  ___ 
%   | | | | __| |/ _ \ | '_ \ / _` |/ _` |/ _ \
%   | | | | |_| |  __/ | |_) | (_| | (_| |  __/
%   |_| |_|\__|_|\___| | .__/ \__,_|\__, |\___|
%                      |_|          |___/      
%
% «title»  (to ".title")
% (c3m212mt2p 1 "title")
% (c3m212mt2a   "title")

\thispagestyle{empty}

\begin{center}

\vspace*{1.2cm}

{\bf \Large Cálculo 3 - 2021.2}

\bsk

Mini-teste 2

\bsk

Eduardo Ochs - RCN/PURO/UFF

\url{http://angg.twu.net/2021.2-C3.html}

\end{center}

\newpage



{\bf Avisos}

As regras vão ser as mesmas do mini-teste 1.

\msk

As questões deste mini-teste vão ser baseadas

nos exercícios 4 e 6 deste PDF:

\ssk

{\footnotesize

\url{http://angg.twu.net/LATEX/2021-2-C3-diag-nums.pdf}

}

% (c3m212dnp 11 "exercicio-4")
% (c3m212dna    "exercicio-4")
% (c3m212dnp 13 "exercicio-6")
% (c3m212dna    "exercicio-6")


\msk

As questões vão ser disponibilizadas às 20:40 da sexta

7/janeiro/2021 e vocês vão ter 24 horas pra entregar

as respostas.


\newpage

% «itens»  (to ".itens")
% (c3m212mt2p 3 "itens")
% (c3m212mt2a   "itens")

\scalebox{0.9}{\def\colwidth{12cm}\firstcol{

Para cada uma das superfícies abaixo faça o diagrama de

numerozinhos e o diagrama de sinais dela. As definições

vão ser iguais às do PDF sobre diagramas de numerozinhos,

mas aqui vamos usar $x_0=\ColorRed{4}$ e $y_0=\ColorRed{3}$. 

\msk

a) $Δx$

b) $Δy$

c) $Δy-Δx$

d) $(Δy-Δx)^2$

e) $Δy(Δy-Δx)$

\bsk
\bsk

Você pode fazer os diagrama de numerozinhos usando só

os 9 pontos com $Δx,Δy∈\{-1,0,1\}$, mas se você não

conseguir descobrir qual é o diagrama de sinais do

último item usando só esses 9 pontos você pode fazer

um diagrama de numerozinhos maior, com os 25 pontos

com $Δx,Δy∈\{-2,-1,0,1,-2\}$.


%}\anothercol{
}}


\newpage

% «gabarito»  (to ".gabarito")
% (c3m212mt2p 4 "gabarito")
% (c3m212mt2a   "gabarito")

\vspace*{-0.7cm}

{\bf Gabarito}

%L Pict2e.new()
%L   :setbounds(v(0,0), v(5,4))
%L     -- :grid()
%L     :add("#1")
%L     :axesandticks()
%L   :add("#2")
%L   :bepc()
%L   :def("funcaof#1#2")
%L   :output()
\pu

%L nineputs = function (str)
%L     local xl,xc,xr = 3,4,5
%L     local ya,yc,yb = 4,3,2
%L     local ns = split(str)
%L     local out = ""
%L     local put = function (x, y, idx)
%L         -- out = out..format("\\put(%s,%s){\\cell{%s}}\n", x, y, ns[idx] or "?")
%L         out = out..format("\\put(%s,%s){\\cell{\\text{%s}}}\n", x, y, ns[idx] or "?")
%L       end
%L     put(xl,ya,1); put(xc,ya,2); put(xr,ya,3);
%L     put(xl,yc,4); put(xc,yc,5); put(xr,yc,6);
%L     put(xl,yb,7); put(xc,yb,8); put(xr,yb,9); 
%L     return out
%L   end
\pu

\def\nineputs#1{\funcaof{\expr{nineputs("#1")}}{}}

% (find-LATEX "edrx21.sty" "picture-cells")
\unitlength=8pt
\celllower=2.5pt
\def\cellfont{\footnotesize}
\def\cellfont{\scriptsize}

\msk

$\begin{array}{ccl}
 Δx        &=& \nineputs{-1 0 1  -1 0 1  -1 0 1} \\[15pt]
 Δy        &=& \nineputs{1 1 1  0 0 0  -1 -1 -1} \\[15pt]
 Δy-Δx     &=& \nineputs{2 1 0  1 0 -1  0 -1 -2} \\[15pt]
 (Δy-Δx)^2 &=& \nineputs{4 1 0  1 0 1  0 1 4}    \\[15pt]
 Δy(Δy-Δx) &=& \nineputs{2 1 0  0 0 0  0 1 2}    \\
 \end{array}
$


% «gabarito-maxima»  (to ".gabarito-maxima")
% (setq eepitch-preprocess-regexp "^")
% (setq eepitch-preprocess-regexp "^%T ")
%
%T  (eepitch-maxima)
%T  (eepitch-kill)
%T  (eepitch-maxima)
%T x0 : 4;
%T y0 : 3;
%T Dx(x) := x - x0;
%T Dy(y) := y - y0;
%T z(x,y) := Dy(y) * (Dy(y) - Dx(x));
%T z(0,0);
%T plot3d (z(x,y), [x, x0-1, x0+1], [y, y0-1, y0+1]);
%T plot3d (z(x,y), [x, x0-2, x0+2], [y, y0-2, y0+2]);
%T plot3d (z(x,y), [x, x0-4, x0+4], [y, y0-4, y0+4]);
%T cap(z) := max(min(z, 0.1), -0.1);
%T plot3d ( cap(z(x,y)),    [x, x0-1, x0+1], [y, y0-1, y0+1] );
%T plot3d ([cap(z(x,y)), 0, [x, x0-1, x0+1], [y, y0-1, y0+1]]);
%T plot3d ([    z(x,y) , 0, [x, x0-1, x0+1], [y, y0-1, y0+1]]);
%T 
%T plot2d ([contour, z(x,y)], [x, x0-4, x0+4], [y, y0-4, y0+4]);


\newpage

% «gabarito-5x5»  (to ".gabarito-5x5")
% (c3m212mt2p 5 "gabarito-5x5")
% (c3m212mt2a   "gabarito-5x5")

%L Pict2e.new()
%L   :setbounds(v(0,0), v(6,5))
%L     -- :grid()
%L     -- :add("#1")
%L     :axesandticks()
%L     :run(function (p)
%L         for y=3+2,3-2,-1 do
%L           for x=4-2,4+2 do
%L             local Dx,Dy = x-4,y-3
%L             p:puttext(v(x,y), Dy * (Dy - Dx))
%L           end
%L         end
%L       end)
%L   :bepc()
%L   :def("GabBig")
%L   :output()
\pu

\vspace*{5.5cm}

$Δy(Δy-Δx) \;\;=\;\; \GabBig$


\newpage

Nós vamos usar algumas idéias deste mini-teste

pra entender máximos e mínimos de superfícies.

Assista os vídeos:

\bsk

12/jan/2022, primeira metade:

{\footnotesize

\url{http://angg.twu.net/eev-videos/2021-2-C3-MT2.mp4}

\url{https://www.youtube.com/watch?v=Rz01pLaL9Z0}

}

\msk

12/jan/2022, segunda metade:

{\footnotesize

\url{http://angg.twu.net/eev-videos/2021-2-C3-MT2-2.mp4}

\url{https://www.youtube.com/watch?v=lDvAMU5aNvc}

}

\bsk

14/jan/2022 (sobre funções homogêneas):

{\footnotesize

\url{http://angg.twu.net/eev-videos/2021-2-C3-MT2-3.mp4}

\url{https://www.youtube.com/watch?v=cL3G-t3mILs}

}






\newpage

% «figura-homogeneas»  (to ".figura-homogeneas")
% (c3m212mt2p 7 "figura-homogeneas")
% (c3m212mt2a   "figura-homogeneas")

\unitlength=15pt

%L Pict2e.new()
%L   :setbounds(v(-5,-5), v(5,5))
%L     :axesandticks()
%L     :run(function (p)
%L         for y=-4,4 do
%L           for x=-4,4 do
%L             p:puttext(v(x,y), (x-y) * (x+2*y))
%L           end
%L         end
%L       end)
%L   :bepc()
%L   :def("Foo")
%L   :output()
\pu

$$(x-y) (x+2y)
  \;\;=\;\;
  \Foo
$$




%\printbibliography

\GenericWarning{Success:}{Success!!!}  % Used by `M-x cv'

\end{document}

%  ____  _             _         
% |  _ \(_)_   ___   _(_)_______ 
% | | | | \ \ / / | | | |_  / _ \
% | |_| | |\ V /| |_| | |/ /  __/
% |____// | \_/  \__,_|_/___\___|
%     |__/                       
%
% «djvuize»  (to ".djvuize")
% (find-LATEXgrep "grep --color -nH --null -e djvuize 2020-1*.tex")

 (eepitch-shell)
 (eepitch-kill)
 (eepitch-shell)
# (find-fline "~/2021.2-C3/")
# (find-fline "~/LATEX/2021-2-C3/")
# (find-fline "~/bin/djvuize")

cd /tmp/
for i in *.jpg; do echo f $(basename $i .jpg); done

f () { rm -v $1.pdf;  textcleaner -f 50 -o  5 $1.jpg $1.png; djvuize $1.pdf; xpdf $1.pdf }
f () { rm -v $1.pdf;  textcleaner -f 50 -o 10 $1.jpg $1.png; djvuize $1.pdf; xpdf $1.pdf }
f () { rm -v $1.pdf;  textcleaner -f 50 -o 20 $1.jpg $1.png; djvuize $1.pdf; xpdf $1.pdf }

f () { rm -fv $1.png $1.pdf; djvuize $1.pdf }
f () { rm -fv $1.png $1.pdf; djvuize WHITEBOARDOPTS="-m 1.0 -f 15" $1.pdf; xpdf $1.pdf }
f () { rm -fv $1.png $1.pdf; djvuize WHITEBOARDOPTS="-m 1.0 -f 30" $1.pdf; xpdf $1.pdf }
f () { rm -fv $1.png $1.pdf; djvuize WHITEBOARDOPTS="-m 1.0 -f 45" $1.pdf; xpdf $1.pdf }
f () { rm -fv $1.png $1.pdf; djvuize WHITEBOARDOPTS="-m 0.5" $1.pdf; xpdf $1.pdf }
f () { rm -fv $1.png $1.pdf; djvuize WHITEBOARDOPTS="-m 0.25" $1.pdf; xpdf $1.pdf }
f () { cp -fv $1.png $1.pdf       ~/2021.2-C3/
       cp -fv        $1.pdf ~/LATEX/2021-2-C3/
       cat <<%%%
% (find-latexscan-links "C3" "$1")
%%%
}

f 20201213_area_em_funcao_de_theta
f 20201213_area_em_funcao_de_x
f 20201213_area_fatias_pizza



%  __  __       _        
% |  \/  | __ _| | _____ 
% | |\/| |/ _` | |/ / _ \
% | |  | | (_| |   <  __/
% |_|  |_|\__,_|_|\_\___|
%                        
% <make>

 (eepitch-shell)
 (eepitch-kill)
 (eepitch-shell)
# (find-LATEXfile "2019planar-has-1.mk")
make -f 2019.mk STEM=2021-2-C3-MT2 veryclean
make -f 2019.mk STEM=2021-2-C3-MT2 pdf

% Local Variables:
% coding: utf-8-unix
% ee-tla: "c3mt2"
% ee-tla: "c3m212mt2"
% End:
