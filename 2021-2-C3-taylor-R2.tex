% (find-LATEX "2021-2-C3-taylor-R2.tex")
% (defun c () (interactive) (find-LATEXsh "lualatex -record 2021-2-C3-taylor-R2.tex" :end))
% (defun C () (interactive) (find-LATEXsh "lualatex 2021-2-C3-taylor-R2.tex" "Success!!!"))
% (defun D () (interactive) (find-pdf-page      "~/LATEX/2021-2-C3-taylor-R2.pdf"))
% (defun d () (interactive) (find-pdftools-page "~/LATEX/2021-2-C3-taylor-R2.pdf"))
% (defun e () (interactive) (find-LATEX "2021-2-C3-taylor-R2.tex"))
% (defun o () (interactive) (find-LATEX "2021-1-C3-matriz-jacobiana.tex"))
% (defun u () (interactive) (find-latex-upload-links "2021-2-C3-taylor-R2"))
% (defun v () (interactive) (find-2a '(e) '(d)))
% (defun d0 () (interactive) (find-ebuffer "2021-2-C3-taylor-R2.pdf"))
% (defun cv () (interactive) (C) (ee-kill-this-buffer) (v) (g))
%          (code-eec-LATEX "2021-2-C3-taylor-R2")
% (find-pdf-page   "~/LATEX/2021-2-C3-taylor-R2.pdf")
% (find-sh0 "cp -v  ~/LATEX/2021-2-C3-taylor-R2.pdf /tmp/")
% (find-sh0 "cp -v  ~/LATEX/2021-2-C3-taylor-R2.pdf /tmp/pen/")
%     (find-xournalpp "/tmp/2021-2-C3-taylor-R2.pdf")
%   file:///home/edrx/LATEX/2021-2-C3-taylor-R2.pdf
%               file:///tmp/2021-2-C3-taylor-R2.pdf
%           file:///tmp/pen/2021-2-C3-taylor-R2.pdf
% http://angg.twu.net/LATEX/2021-2-C3-taylor-R2.pdf
% (find-LATEX "2019.mk")
% (find-CN-aula-links "2021-2-C3-taylor-R2" "3" "c3m212t2" "c3t2")

% «.videos-antigos»	(to "videos-antigos")
%
% «.defs»		(to "defs")
% «.title»		(to "title")
% «.introducao»		(to "introducao")
%
% «.djvuize»		(to "djvuize")



% «videos-antigos»  (to ".videos-antigos")
% (c3m211cna "video-1")
% (c3m211cna "video-2")


% <videos>
% Video (not yet):
% (find-ssr-links     "c3m212t2" "2021-2-C3-taylor-R2")
% (code-eevvideo      "c3m212t2" "2021-2-C3-taylor-R2")
% (code-eevlinksvideo "c3m212t2" "2021-2-C3-taylor-R2")
% (find-c3m212t2video "0:00")

\documentclass[oneside,12pt]{article}
\usepackage[colorlinks,citecolor=DarkRed,urlcolor=DarkRed]{hyperref} % (find-es "tex" "hyperref")
\usepackage{amsmath}
\usepackage{amsfonts}
\usepackage{amssymb}
\usepackage{pict2e}
\usepackage[x11names,svgnames]{xcolor} % (find-es "tex" "xcolor")
\usepackage{colorweb}                  % (find-es "tex" "colorweb")
%\usepackage{tikz}
%
% (find-dn6 "preamble6.lua" "preamble0")
%\usepackage{proof}   % For derivation trees ("%:" lines)
%\input diagxy        % For 2D diagrams ("%D" lines)
%\xyoption{curve}     % For the ".curve=" feature in 2D diagrams
%
\usepackage{edrx21}               % (find-LATEX "edrx21.sty")
\input edrxaccents.tex            % (find-LATEX "edrxaccents.tex")
\input edrx21chars.tex            % (find-LATEX "edrx21chars.tex")
\input edrxheadfoot.tex           % (find-LATEX "edrxheadfoot.tex")
\input edrxgac2.tex               % (find-LATEX "edrxgac2.tex")
%\usepackage{emaxima}              % (find-LATEX "emaxima.sty")
%
%\usepackage[backend=biber,
%   style=alphabetic]{biblatex}            % (find-es "tex" "biber")
%\addbibresource{catsem-slides.bib}        % (find-LATEX "catsem-slides.bib")
%
% (find-es "tex" "geometry")
\usepackage[a6paper, landscape,
            top=1.5cm, bottom=.25cm, left=1cm, right=1cm, includefoot
           ]{geometry}
%
\begin{document}

%\catcode`\^^J=10
%\directlua{dofile "dednat6load.lua"}  % (find-LATEX "dednat6load.lua")
%
% %L dofile "2021pict2e.lua"           -- (find-LATEX "2021pict2e.lua")
% %L Pict2e.__index.suffix = "%"
% \pu
% \def\pictgridstyle{\color{GrayPale}\linethickness{0.3pt}}
% \def\pictaxesstyle{\linethickness{0.5pt}}

% %L dofile "edrxtikz.lua"  -- (find-LATEX "edrxtikz.lua")
% %L dofile "edrxpict.lua"  -- (find-LATEX "edrxpict.lua")
% \pu

% «defs»  (to ".defs")
% (find-LATEX "edrx21defs.tex" "colors")
% (find-LATEX "edrx21.sty")

\def\u#1{\par{\footnotesize \url{#1}}}

\def\drafturl{http://angg.twu.net/LATEX/2021-2-C3.pdf}
\def\drafturl{http://angg.twu.net/2021.2-C3.html}
\def\draftfooter{\tiny \href{\drafturl}{\jobname{}} \ColorBrown{\shorttoday{} \hours}}



%  _____ _ _   _                               
% |_   _(_) |_| | ___   _ __   __ _  __ _  ___ 
%   | | | | __| |/ _ \ | '_ \ / _` |/ _` |/ _ \
%   | | | | |_| |  __/ | |_) | (_| | (_| |  __/
%   |_| |_|\__|_|\___| | .__/ \__,_|\__, |\___|
%                      |_|          |___/      
%
% «title»  (to ".title")
% (c3m212t2p 1 "title")
% (c3m212t2a   "title")

\thispagestyle{empty}

\begin{center}

\vspace*{1.2cm}

{\bf \Large Cálculo 3 - 2021.2}

\bsk

Aula 26: Taylor em $\R^2$

\bsk

Eduardo Ochs - RCN/PURO/UFF

\url{http://angg.twu.net/2021.2-C3.html}

\end{center}

\newpage

% «introducao»  (to ".introducao")
% (c3m212t2p 2 "introducao")
% (c3m212t2a   "introducao")

{\bf Introdução}

Na P1 vocês viram como calcular a série de Taylor

de uma função $f:\R→\R$ truncada até qualquer grau...

\msk

Agora nós vamos ver um pouco do que acontece quando

a gente calcula a primeira e a segunda derivada de

funções de $\R^m$ em $\R^n$, onde $m,n∈\{1,2\}$.

\bsk

Comece dando uma olhada neste PDF e nestes vídeos

do semestre passado:




{\footnotesize

% (c3m211ja "title")
% (c3m211ja "title" "Aula 21: a matriz jacobiana")
%    http://angg.twu.net/LATEX/2021-1-C3-matriz-jacobiana.pdf
\url{http://angg.twu.net/LATEX/2021-1-C3-matriz-jacobiana.pdf}

\ssk

% (c3m211ja "video-1")
\url{http://angg.twu.net/eev-videos/2021-1-C3-matriz-jacobiana.mp4}

\url{https://www.youtube.com/watch?v=kMGtZk5er9w}

\ssk

% (c3m211ja "video-2")
\url{http://angg.twu.net/eev-videos/2021-1-C3-matriz-jacobiana-2.mp4}

\url{https://www.youtube.com/watch?v=D_YKka3RG9E}

\ssk


}



\newpage

{\bf Algumas superfícies de primeiro grau}


\scalebox{0.85}{\def\colwidth{12cm}\firstcol{

Uma superfície $z=F(x,y)$ é uma função homogênea

de primeiro grau quando ela é desta forma daqui:

$F(x,y)=ax+by$.

\msk

Vamos começar verificando que você sabe desenhar

o diagramas de numerozinhos dessas funções desse

tipo bem rápido --- que você já sabe que padrões

eles obedecem e que você consegue desenhar cada

um em poucos segundos.

\msk

{\bf Exercício 1.}

Desenhe o diagrama de numerozinhos $5×5$ de cada

uma das funções abaixo. Use $x,y∈\{-2,-1,0,1,2\}$.

\msk

a) $-3x$

b) $2y$

c) $4x-y$

d) $-4x-3y$

%}\anothercol{
}}


\newpage

{\bf Algumas curvas de nível}


\scalebox{0.8}{\def\colwidth{12cm}\firstcol{

Leia a definição do Bortolossi de curvas de nível nas

páginas 97 até 100 do capítulo 3 dele.

\bsk

{\bf Exercício 2}

Desenhe pelo menos 5 curvas de nível sobre cada um dos

diagramas de numerozinhos que você fez no exercício 1,

e do lado de cada uma dela escreva o valor de $z$ nela ---

por exemplo, `$z=0$', `$z=4$', $z=-42$, etc.

\bsk

{\bf Exercício 3}

Em cada um dos seus 4 diagramas de numerozinhos

escolha uma curva de nível dele {\sl na qual $z≠0$} --- ela

vai ser uma reta --- e dê uma parametrização para ela.

Lembre que retas parametrizadas em $\R^2$ têm essa forma aqui:
%
$$r \;\;=\;\; \setofst{(α,β)+t\VEC{γ,δ}}{t∈\R}$$

%}\anothercol{
}}

\newpage

{\bf Exercício 4}

Agora entenda a definição de ``vetor gradiente'' do Bortolossi.

Ela está na página 298 do livro, no capítulo 8.

Para cada umas suas quatro superfícies:

\ssk

a) calcule $∇z(x,y)$,

b) verifique que $∇z(x,y)$ não depende do ponto $(x,y)$,

c) calcule $∇z(2,3)$,

d) calcule $∇z(x,y)·\VEC{γ,δ}$, onde `$·$' é o produto

escalar e $\VEC{γ,δ}$ é o vetor diretor da reta parametrizada

dessa superfície que você encontrou no exercício 3.

e) Verifique que $∇z(x,y) ⊥ \VEC{γ,δ}$, isto é, que o gradiente

e o vetor diretor são ortogonais.




% (c3m211mt2a "title")
% (c3m211mt2a "title" "Mini-teste 2")
% (c3mgrada "title")
% (c3mgrada "title" "Aula 23: o vetor gradiente")






%\printbibliography

\GenericWarning{Success:}{Success!!!}  % Used by `M-x cv'

\end{document}

%  ____  _             _         
% |  _ \(_)_   ___   _(_)_______ 
% | | | | \ \ / / | | | |_  / _ \
% | |_| | |\ V /| |_| | |/ /  __/
% |____// | \_/  \__,_|_/___\___|
%     |__/                       
%
% «djvuize»  (to ".djvuize")
% (find-LATEXgrep "grep --color -nH --null -e djvuize 2020-1*.tex")

 (eepitch-shell)
 (eepitch-kill)
 (eepitch-shell)
# (find-fline "~/2021.2-C3/")
# (find-fline "~/LATEX/2021-2-C3/")
# (find-fline "~/bin/djvuize")

cd /tmp/
for i in *.jpg; do echo f $(basename $i .jpg); done

f () { rm -v $1.pdf;  textcleaner -f 50 -o  5 $1.jpg $1.png; djvuize $1.pdf; xpdf $1.pdf }
f () { rm -v $1.pdf;  textcleaner -f 50 -o 10 $1.jpg $1.png; djvuize $1.pdf; xpdf $1.pdf }
f () { rm -v $1.pdf;  textcleaner -f 50 -o 20 $1.jpg $1.png; djvuize $1.pdf; xpdf $1.pdf }

f () { rm -fv $1.png $1.pdf; djvuize $1.pdf }
f () { rm -fv $1.png $1.pdf; djvuize WHITEBOARDOPTS="-m 1.0 -f 15" $1.pdf; xpdf $1.pdf }
f () { rm -fv $1.png $1.pdf; djvuize WHITEBOARDOPTS="-m 1.0 -f 30" $1.pdf; xpdf $1.pdf }
f () { rm -fv $1.png $1.pdf; djvuize WHITEBOARDOPTS="-m 1.0 -f 45" $1.pdf; xpdf $1.pdf }
f () { rm -fv $1.png $1.pdf; djvuize WHITEBOARDOPTS="-m 0.5" $1.pdf; xpdf $1.pdf }
f () { rm -fv $1.png $1.pdf; djvuize WHITEBOARDOPTS="-m 0.25" $1.pdf; xpdf $1.pdf }
f () { cp -fv $1.png $1.pdf       ~/2021.2-C3/
       cp -fv        $1.pdf ~/LATEX/2021-2-C3/
       cat <<%%%
% (find-latexscan-links "C3" "$1")
%%%
}

f 20201213_area_em_funcao_de_theta
f 20201213_area_em_funcao_de_x
f 20201213_area_fatias_pizza



%  __  __       _        
% |  \/  | __ _| | _____ 
% | |\/| |/ _` | |/ / _ \
% | |  | | (_| |   <  __/
% |_|  |_|\__,_|_|\_\___|
%                        
% <make>

 (eepitch-shell)
 (eepitch-kill)
 (eepitch-shell)
# (find-LATEXfile "2019planar-has-1.mk")
make -f 2019.mk STEM=2021-2-C3-taylor-R2 veryclean
make -f 2019.mk STEM=2021-2-C3-taylor-R2 pdf

% Local Variables:
% coding: utf-8-unix
% ee-tla: "c3t2"
% ee-tla: "c3m212t2"
% End:

