% (find-LATEX "2021-2-C2-mud-var-gamb.tex")
% (defun c () (interactive) (find-LATEXsh "lualatex -record 2021-2-C2-mud-var-gamb.tex" :end))
% (defun C () (interactive) (find-LATEXsh "lualatex 2021-2-C2-mud-var-gamb.tex" "Success!!!"))
% (defun D () (interactive) (find-pdf-page      "~/LATEX/2021-2-C2-mud-var-gamb.pdf"))
% (defun d () (interactive) (find-pdftools-page "~/LATEX/2021-2-C2-mud-var-gamb.pdf"))
% (defun e () (interactive) (find-LATEX "2021-2-C2-mud-var-gamb.tex"))
% (defun o () (interactive) (find-LATEX "2021-2-C2-mud-var-gamb.tex"))
% (defun u () (interactive) (find-latex-upload-links "2021-2-C2-mud-var-gamb"))
% (defun v () (interactive) (find-2a '(e) '(d)))
% (defun d0 () (interactive) (find-ebuffer "2021-2-C2-mud-var-gamb.pdf"))
% (defun cv () (interactive) (C) (ee-kill-this-buffer) (v) (g))
%          (code-eec-LATEX "2021-2-C2-mud-var-gamb")
% (find-pdf-page   "~/LATEX/2021-2-C2-mud-var-gamb.pdf")
% (find-sh0 "cp -v  ~/LATEX/2021-2-C2-mud-var-gamb.pdf /tmp/")
% (find-sh0 "cp -v  ~/LATEX/2021-2-C2-mud-var-gamb.pdf /tmp/pen/")
%     (find-xournalpp "/tmp/2021-2-C2-mud-var-gamb.pdf")
%   file:///home/edrx/LATEX/2021-2-C2-mud-var-gamb.pdf
%               file:///tmp/2021-2-C2-mud-var-gamb.pdf
%           file:///tmp/pen/2021-2-C2-mud-var-gamb.pdf
% http://angg.twu.net/LATEX/2021-2-C2-mud-var-gamb.pdf
% (find-LATEX "2019.mk")
% (find-CN-aula-links "2021-2-C2-mud-var-gamb" "2" "c2m212mv" "c2mv")

% «.defs»		(to "defs")
% «.title»		(to "title")
% «.introducao»		(to "introducao")
% «.introducao-2»	(to "introducao-2")
% «.exemplo»		(to "exemplo")
% «.limites-int»	(to "limites-int")
% «.importante»		(to "importante")
% «.caixinhas»		(to "caixinhas")
%
% «.djvuize»		(to "djvuize")



% <videos>
% Video (not yet):
% (find-ssr-links     "c2m212mv" "2021-2-C2-mud-var-gamb" "{naoexiste}")
% (code-eevvideo      "c2m212mv" "2021-2-C2-mud-var-gamb")
% (code-eevlinksvideo "c2m212mv" "2021-2-C2-mud-var-gamb")
% (find-c2m212mvvideo "0:00")

\documentclass[oneside,12pt]{article}
\usepackage[colorlinks,citecolor=DarkRed,urlcolor=DarkRed]{hyperref} % (find-es "tex" "hyperref")
\usepackage{amsmath}
\usepackage{amsfonts}
\usepackage{amssymb}
\usepackage{pict2e}
\usepackage[x11names,svgnames]{xcolor} % (find-es "tex" "xcolor")
\usepackage{colorweb}                  % (find-es "tex" "colorweb")
%\usepackage{tikz}
%
% (find-dn6 "preamble6.lua" "preamble0")
%\usepackage{proof}   % For derivation trees ("%:" lines)
%\input diagxy        % For 2D diagrams ("%D" lines)
%\xyoption{curve}     % For the ".curve=" feature in 2D diagrams
%
\usepackage{edrx21}               % (find-LATEX "edrx21.sty")
\input edrxaccents.tex            % (find-LATEX "edrxaccents.tex")
\input edrx21chars.tex            % (find-LATEX "edrx21chars.tex")
\input edrxheadfoot.tex           % (find-LATEX "edrxheadfoot.tex")
\input edrxgac2.tex               % (find-LATEX "edrxgac2.tex")
%\usepackage{emaxima}              % (find-LATEX "emaxima.sty")
%
%\usepackage[backend=biber,
%   style=alphabetic]{biblatex}            % (find-es "tex" "biber")
%\addbibresource{catsem-slides.bib}        % (find-LATEX "catsem-slides.bib")
%
% (find-es "tex" "geometry")
\usepackage[a6paper, landscape,
            top=1.5cm, bottom=.25cm, left=1cm, right=1cm, includefoot
           ]{geometry}
%
\begin{document}

%\catcode`\^^J=10
%\directlua{dofile "dednat6load.lua"}  % (find-LATEX "dednat6load.lua")
%
% %L dofile "2021pict2e.lua"           -- (find-LATEX "2021pict2e.lua")
% %L Pict2e.__index.suffix = "%"
% \pu
% \def\pictgridstyle{\color{GrayPale}\linethickness{0.3pt}}
% \def\pictaxesstyle{\linethickness{0.5pt}}

% %L dofile "edrxtikz.lua"  -- (find-LATEX "edrxtikz.lua")
% %L dofile "edrxpict.lua"  -- (find-LATEX "edrxpict.lua")
% \pu

% «defs»  (to ".defs")
% (find-LATEX "edrx21defs.tex" "colors")
% (find-LATEX "edrx21.sty")

\def\u#1{\par{\footnotesize \url{#1}}}

\def\drafturl{http://angg.twu.net/LATEX/2021-2-C2.pdf}
\def\drafturl{http://angg.twu.net/2021.2-C2.html}
\def\draftfooter{\tiny \href{\drafturl}{\jobname{}} \ColorBrown{\shorttoday{} \hours}}



%  _____ _ _   _                               
% |_   _(_) |_| | ___   _ __   __ _  __ _  ___ 
%   | | | | __| |/ _ \ | '_ \ / _` |/ _` |/ _ \
%   | | | | |_| |  __/ | |_) | (_| | (_| |  __/
%   |_| |_|\__|_|\___| | .__/ \__,_|\__, |\___|
%                      |_|          |___/      
%
% «title»  (to ".title")
% (c2m212mvp 1 "title")
% (c2m212mva   "title")

\thispagestyle{empty}

\begin{center}

\vspace*{1.2cm}

{\bf \Large Cálculo 2 - 2021.2}

\bsk

Aula 26: mudança de variável

por gambiarras

\bsk

Eduardo Ochs - RCN/PURO/UFF

\url{http://angg.twu.net/2021.2-C2.html}

\end{center}

\newpage

% «introducao»  (to ".introducao")
% (c2m212mvp 2 "introducao")
% (c2m212mva   "introducao")

{\bf Introdução}

No último PDF e na P1 a gente viu como fazer

``integração por substituição'' de um jeito mais ou menos

fácil de formalizar... agora a gente vai ver o método

que os livros usam, que nos permite fazer as contas bem

rápido, mas que usa várias gambiarras, algumas delas

bem difíceis de formalizar.

\msk

Os nomes ``integração por substituição'' e ``integração

por mudança de variável'' costumam ser equivalentes.

Vou me referir ao método que a gente vai ver agora como

``mudança de variável'', ``mudança de variável por

gambiarras'', ``MV'', ou ``MVG'', pra gente poder usar

o termo ``substituição'' pro `[:=]'.

\newpage

% «introducao-2»  (to ".introducao-2")
% (c2m212mvp 3 "introducao-2")
% (c2m212mva   "introducao-2")

{\bf Introdução (2)}

\scalebox{0.95}{\def\colwidth{12cm}\firstcol{

Cada livro usa convenções um pouco diferentes pra como

escrever as contas por MVG. Eu vou usar a convenção do

exemplo do próximo slide, em que a resolução da integral

fica à esquerda e as caixinhas indicando os truques que

usamos em \ColorRed{cada} MV ficam à direita, separadas da contas

da integral.

\msk

A primeira caixinha tem os truques pra mudar

da variável $x$ pra variável $u$ e pra voltar de $u$ pra $x$.

\ssk

A segunda caixinha tem os truques pra mudar

da variável $u$ pra variável $v$ e pra voltar de $v$ pra $u$.

\ssk

A terceira caixinha tem os truques pra mudar

da variável $v$ pra variável $w$ e pra voltar de $w$ pra $v$.

\ssk

A quarta caixinha tem os truques pra mudar

da variável $w$ pra variável $y$ e pra voltar de $y$ pra $w$.

%}\anothercol{
}}


\newpage

% «exemplo»  (to ".exemplo")
% (c2m212mvp 4 "exemplo")
% (c2m212mva   "exemplo")
% (c2m211isp 7 "exemplo-contas-2")
% (c2m211isa   "exemplo-contas-2")
% (c2m211p1p 15 "gabarito-2-2020.2")
% (c2m211p1a    "gabarito-2-2020.2")

$$\scalebox{1.25}{$
 \begin{array}{l}
   \begin{array}{l}
   \intx {\frac{3 \cos{\left (2 + \sqrt{3 x + 4} \right )}}
         {2 \sqrt{3 x + 4}}
         } \\
   = \intu {\frac{\cos{\left (2 + \sqrt{u + 4} \right )}}
           {2 \sqrt{u + 4}}
           } \\
   = \intv {\frac{\cos{\left (2 + \sqrt{v} \right )}}
           {2 \sqrt{v}}
           } \\
   = \intw {\cos{\left (2 + w \right )}
           } \\
   = \inty {\cos y}
           \\
   = \sen y \\
   = \sen \left( 2+w \right) \\
   = \sen \left( 2+\sqrt{v} \right) \\
   = \sen \left( 2+\sqrt{u+4} \right) \\
   = \sen \left( 2+\sqrt{3x+4} \right) \\
   \end{array}
   %
   \begin{array}{c}
     \bsm{u = 3x \\ \frac{du}{dx} = 3 \\ du = 3\,dx \\ dx = \frac13 du}
     \\[15pt]
     \bsm{v = u+4 \\ du=dv }
     \\[5pt]
     \bsm{w = \sqrt{v} \\ \frac{dw}{dv} = \frac12 v^{-1/2} = \frac{1}{2\sqrt{v}} \\}
     \\[5pt]
     \bsm{y = 2+w \\ dy=dw }
     \\[60pt]
   \end{array}
   %
  \end{array}
  $}
$$

\newpage

% «limites-int»  (to ".limites-int")
% (c2m212mvp 5 "limites-int")
% (c2m212mva   "limites-int")

{\bf Limites de integração}

A coluna da esquerda tem uma série de integrais sem

limites de integração --- a gente está trabalhando numa

notação abreviada em que os limites de integração foram

apagados. Eles podem ser recolocados de novo no final,

quando a gente for transformar essas contas abreviadas

numa versão ``desabreviada'' delas.

\msk

Os limites de integração em $x$ são diferentes

dos limites de integração em $u$, que são diferentes

dos limites de integração em $v$, que são diferentes

dos limites de integração em $w$, que são diferentes

dos limites de integração em $y$.

\msk

Detalhes em breve!


\newpage


\def\expr#1{〈\mathsf{expr_{#1}}〉}


A coluna da esquerda tem uma série de igualdades.

Ela é da forma $\expr1 = \expr2 = \ldots = \expr{n}$,

mas a gente escreve essa série de igualdades na

vertical.

\msk

Repare que na coluna da esquerda

``as variáveis não se misturam'':

$\expr{1}$ e $\expr{10}$ são ``expressões em $x$'',

$\expr{2}$ e $\expr{9}$ são ``expressões em $u$'',

$\expr{3}$ e $\expr{8}$ são ``expressões em $v$'',

$\expr{4}$ e $\expr{7}$ são ``expressões em $w$'',

$\expr{5}$ e $\expr{6}$ são ``expressões em $y$''.


\newpage

% «importante»  (to ".importante")
% (c2m212mvp 7 "importante")
% (c2m212mva   "importante")

{\bf A regra mais importante de todas}


\scalebox{0.85}{\def\colwidth{9cm}\firstcol{

Na coluna da esquerda cada expressão é

uma expressão ``em uma variável só''.

Se você escrever algo como
%
$$\inty{\cos(2+w)}$$

Isso é um \ColorRed{ERRO CONCEITUAL GRAVÍSSIMO}

e a sua questão é \ColorRed{ZERADA}.


\bsk



A gente não vai ter tempo de ver o porquê disso...

O motivo é que com essa proibição o método pra

``desabreviar'' as contas fica simples ---

sem essa proibição ele fica BEM mais complicado,

e a gente precisaria de uns truques de ``notação

de físicos'', que é um assunto bem difícil de

Cálculo 3, pra definir o método de desabreviação.

%}\anothercol{
}}

\newpage

% «caixinhas»  (to ".caixinhas")
% (c2m212mvp 8 "caixinhas")
% (c2m212mva   "caixinhas")

{\bf As caixinhas de truques}

As caixinhas de truques da MVG têm uma sintaxe

\ColorRed{BEM} diferente das caixinhas do `[:=]'.

Pra enfatizar isso a gente usa `$=$'s dentro delas,

não `$:=$'s, e a gente escreve elas separadas

do resto, à direita.


\msk

Dê uma olhada nas 9 primeiras páginas daqui:

\ssk

{\footnotesize

% (c3m212nfp 1 "title")
% (c3m212nfa   "title")
%    http://angg.twu.net/LATEX/2021-2-C3-notacao-de-fisicos.pdf
\url{http://angg.twu.net/LATEX/2021-2-C3-notacao-de-fisicos.pdf}

}

\msk

Dentro cada caixinha de truques da MVG a gente

vai usar algumas expressões que só podem ser

formalizadas \ColorRed{direito} usando a ``notação de físicos'',

que a gente vai ver com detalhes em C3...

Vou mostrar como ``ler em voz alta'' uma caixinha

e a gente vai tentar usar elas meio de improviso.


\newpage

{\bf Lendo uma caixinha de truques em voz alta}

$$\bmat{u = 3x \\ \frac{du}{dx} = 3 \\ du = 3\,dx \\ dx = \frac13 du}$$

Digamos que $u$ e $x$ são variáveis dependentes,

que obedecem a equação $u=3x$.

Então podemos tratar $u$ como uma função de $x$,

e temos $\frac{du}{dx} = \frac{d}{dx}(u(x)) = \frac{d}{dx}(3x) = \frac{d}{dx}(u(x)) = 3$.

Multiplicando os dois lados de $\frac{du}{dx} = 3$ por $dx$

obtemos $du = 3\,dx$; e multiplicando os dois lados de

$du = 3\,dx$ por $\frac13$ obtemos $dx = \frac{1}{3}$.

\newpage

Na caixinha
%
$$\bmat{u = 3x \\ \frac{du}{dx} = 3 \\ du = 3\,dx \\ dx = \frac13 du}$$

\msk

as duas últimas linhas são igualdades entre expressões

incompletas. Você viu na P1 como substituir expressões

incompletas, como parênteses, bananas e lentes...

\msk

Em expressões das formas `$\intx{\ldots}$' e `$\intu{\ldots}$' o `$dx$'

e o `$du$' fazem papel de ``fecha parênteses'', e as igualdades

$du = 3\,dx$ e $dx = \frac13 du$ indicam substituições que você vai

poder fazer nas integrais do lado esquerda que vão ser

\ColorRed{parecidas} com as da questão 2 da P1.



\newpage

{\bf Exercício 1.}

Reescreva os exemplos 1 a 4 da seção 6.2 do livro do

Daniel Miranda na notação que eu disse que nós vamos

usar, em que todas caixinhas de truques são escritas

explicitamente.

\msk

Link:

\ssk

{\scriptsize

\url{http://hostel.ufabc.edu.br/~daniel.miranda/calculo/calculo.pdf\#page=189}

}


% sobre
% 
% ;; (find-books "__analysis/__analysis.el" "miranda")
% ;; (find-es "ead" "daniel-miranda")
% ;; http://hostel.ufabc.edu.br/~daniel.miranda/calculo/calculo.pdf
% 
% 
% ;; (find-dmirandacalcpage 189 "6.2 Integração por Substituição")






%\printbibliography

\GenericWarning{Success:}{Success!!!}  % Used by `M-x cv'

\end{document}

%  ____  _             _         
% |  _ \(_)_   ___   _(_)_______ 
% | | | | \ \ / / | | | |_  / _ \
% | |_| | |\ V /| |_| | |/ /  __/
% |____// | \_/  \__,_|_/___\___|
%     |__/                       
%
% «djvuize»  (to ".djvuize")
% (find-LATEXgrep "grep --color -nH --null -e djvuize 2020-1*.tex")

 (eepitch-shell)
 (eepitch-kill)
 (eepitch-shell)
# (find-fline "~/2021.2-C2/")
# (find-fline "~/LATEX/2021-2-C2/")
# (find-fline "~/bin/djvuize")

cd /tmp/
for i in *.jpg; do echo f $(basename $i .jpg); done

f () { rm -v $1.pdf;  textcleaner -f 50 -o  5 $1.jpg $1.png; djvuize $1.pdf; xpdf $1.pdf }
f () { rm -v $1.pdf;  textcleaner -f 50 -o 10 $1.jpg $1.png; djvuize $1.pdf; xpdf $1.pdf }
f () { rm -v $1.pdf;  textcleaner -f 50 -o 20 $1.jpg $1.png; djvuize $1.pdf; xpdf $1.pdf }

f () { rm -fv $1.png $1.pdf; djvuize $1.pdf }
f () { rm -fv $1.png $1.pdf; djvuize WHITEBOARDOPTS="-m 1.0 -f 15" $1.pdf; xpdf $1.pdf }
f () { rm -fv $1.png $1.pdf; djvuize WHITEBOARDOPTS="-m 1.0 -f 30" $1.pdf; xpdf $1.pdf }
f () { rm -fv $1.png $1.pdf; djvuize WHITEBOARDOPTS="-m 1.0 -f 45" $1.pdf; xpdf $1.pdf }
f () { rm -fv $1.png $1.pdf; djvuize WHITEBOARDOPTS="-m 0.5" $1.pdf; xpdf $1.pdf }
f () { rm -fv $1.png $1.pdf; djvuize WHITEBOARDOPTS="-m 0.25" $1.pdf; xpdf $1.pdf }
f () { cp -fv $1.png $1.pdf       ~/2021.2-C2/
       cp -fv        $1.pdf ~/LATEX/2021-2-C2/
       cat <<%%%
% (find-latexscan-links "C2" "$1")
%%%
}

f 20201213_area_em_funcao_de_theta
f 20201213_area_em_funcao_de_x
f 20201213_area_fatias_pizza



%  __  __       _        
% |  \/  | __ _| | _____ 
% | |\/| |/ _` | |/ / _ \
% | |  | | (_| |   <  __/
% |_|  |_|\__,_|_|\_\___|
%                        
% <make>

 (eepitch-shell)
 (eepitch-kill)
 (eepitch-shell)
# (find-LATEXfile "2019planar-has-1.mk")
make -f 2019.mk STEM=2021-2-C2-mud-var-gamb veryclean
make -f 2019.mk STEM=2021-2-C2-mud-var-gamb pdf

% Local Variables:
% coding: utf-8-unix
% ee-tla: "c2mv"
% ee-tla: "c2m212mv"
% End:
