% (find-LATEX "2021-2-C2-P2.tex")
% (defun c () (interactive) (find-LATEXsh "lualatex -record 2021-2-C2-P2.tex" :end))
% (defun C () (interactive) (find-LATEXsh "lualatex 2021-2-C2-P2.tex" "Success!!!"))
% (defun D () (interactive) (find-pdf-page      "~/LATEX/2021-2-C2-P2.pdf"))
% (defun d () (interactive) (find-pdftools-page "~/LATEX/2021-2-C2-P2.pdf"))
% (defun e () (interactive) (find-LATEX "2021-2-C2-P2.tex"))
% (defun o () (interactive) (find-LATEX "2021-2-C3-P2.tex"))
% (defun o () (interactive) (find-LATEX "2021-2-C2-P2.tex"))
% (defun o () (interactive) (find-LATEX "2021-2-C2-P1.tex"))
% (defun u () (interactive) (find-latex-upload-links "2021-2-C2-P2"))
% (defun v () (interactive) (find-2a '(e) '(d)))
% (defun d0 () (interactive) (find-ebuffer "2021-2-C2-P2.pdf"))
% (defun cv () (interactive) (C) (ee-kill-this-buffer) (v) (g))
%          (code-eec-LATEX "2021-2-C2-P2")
% (find-pdf-page   "~/LATEX/2021-2-C2-P2.pdf")
% (find-sh0 "cp -v  ~/LATEX/2021-2-C2-P2.pdf /tmp/")
% (find-sh0 "cp -v  ~/LATEX/2021-2-C2-P2.pdf /tmp/pen/")
%     (find-xournalpp "/tmp/2021-2-C2-P2.pdf")
%   file:///home/edrx/LATEX/2021-2-C2-P2.pdf
%               file:///tmp/2021-2-C2-P2.pdf
%           file:///tmp/pen/2021-2-C2-P2.pdf
% http://angg.twu.net/LATEX/2021-2-C2-P2.pdf
% (find-LATEX "2019.mk")
% (find-CN-aula-links "2021-2-C2-P2" "2" "c2m212p2" "c2p2")

% «.defs»		(to "defs")
% «.defs-T-and-B»	(to "defs-T-and-B")
% «.title»		(to "title")
% «.regras-e-dicas»	(to "regras-e-dicas")
%
% «.questao-1»		(to "questao-1")
% «.questao-2-intro»	(to "questao-2-intro")
% «.questao-2»		(to "questao-2")
% «.questao-3-intro»	(to "questao-3-intro")
% «.questao-3-intro-2»	(to "questao-3-intro-2")
% «.questao-3»		(to "questao-3")
% «.questao-3-bcd»	(to "questao-3-bcd")
% «.questao-3-maxima»	(to "questao-3-maxima")
%
% «.questao-1-gab»	(to "questao-1-gab")
% «.questao-2-gab»	(to "questao-2-gab")
% «.ideias»		(to "ideias")
%
% «.djvuize»		(to "djvuize")



% <videos>
% Video (not yet):
% (find-ssr-links     "c2m212p2" "2021-2-C2-P2" "{naoexiste}")
% (code-eevvideo      "c2m212p2" "2021-2-C2-P2")
% (code-eevlinksvideo "c2m212p2" "2021-2-C2-P2")
% (find-c2m212p2video "0:00")

\documentclass[oneside,12pt]{article}
\usepackage[colorlinks,citecolor=DarkRed,urlcolor=DarkRed]{hyperref} % (find-es "tex" "hyperref")
\usepackage{amsmath}
\usepackage{amsfonts}
\usepackage{amssymb}
\usepackage{pict2e}
\usepackage[x11names,svgnames]{xcolor} % (find-es "tex" "xcolor")
\usepackage{colorweb}                  % (find-es "tex" "colorweb")
%\usepackage{tikz}
%
% (find-dn6 "preamble6.lua" "preamble0")
%\usepackage{proof}   % For derivation trees ("%:" lines)
%\input diagxy        % For 2D diagrams ("%D" lines)
%\xyoption{curve}     % For the ".curve=" feature in 2D diagrams
%
\usepackage{edrx21}               % (find-LATEX "edrx21.sty")
\input edrxaccents.tex            % (find-LATEX "edrxaccents.tex")
\input edrx21chars.tex            % (find-LATEX "edrx21chars.tex")
\input edrxheadfoot.tex           % (find-LATEX "edrxheadfoot.tex")
\input edrxgac2.tex               % (find-LATEX "edrxgac2.tex")
%\usepackage{emaxima}              % (find-LATEX "emaxima.sty")
%
%\usepackage[backend=biber,
%   style=alphabetic]{biblatex}            % (find-es "tex" "biber")
%\addbibresource{catsem-slides.bib}        % (find-LATEX "catsem-slides.bib")
%
% (find-es "tex" "geometry")
\usepackage[a6paper, landscape,
            top=1.5cm, bottom=.25cm, left=1cm, right=1cm, includefoot
           ]{geometry}
%
\begin{document}

%\catcode`\^^J=10
%\directlua{dofile "dednat6load.lua"}  % (find-LATEX "dednat6load.lua")
%
% %L dofile "2021pict2e.lua"           -- (find-LATEX "2021pict2e.lua")
% %L Pict2e.__index.suffix = "%"
% \pu
% \def\pictgridstyle{\color{GrayPale}\linethickness{0.3pt}}
% \def\pictaxesstyle{\linethickness{0.5pt}}

% %L dofile "edrxtikz.lua"  -- (find-LATEX "edrxtikz.lua")
% %L dofile "edrxpict.lua"  -- (find-LATEX "edrxpict.lua")
% \pu

% «defs»  (to ".defs")
% (find-LATEX "edrx21defs.tex" "colors")
% (find-LATEX "edrx21.sty")

\def\u#1{\par{\footnotesize \url{#1}}}

\def\drafturl{http://angg.twu.net/LATEX/2021-2-C2.pdf}
\def\drafturl{http://angg.twu.net/2021.2-C2.html}
\def\draftfooter{\tiny \href{\drafturl}{\jobname{}} \ColorBrown{\shorttoday{} \hours}}

% «defs-T-and-B»  (to ".defs-T-and-B")
% (c3m202p1p 6 "questao-2")
% (c3m202p1a   "questao-2")
\long\def\ColorOrange#1{{\color{orange!90!black}#1}}
\def\T(Total: #1 pts){{\bf(Total: #1)}}
\def\T(Total: #1 pts){{\bf(Total: #1 pts)}}
\def\T(Total: #1 pts){\ColorRed{\bf(Total: #1 pts)}}
\def\B       (#1 pts){\ColorOrange{\bf(#1 pts)}}



%  _____ _ _   _                               
% |_   _(_) |_| | ___   _ __   __ _  __ _  ___ 
%   | | | | __| |/ _ \ | '_ \ / _` |/ _` |/ _ \
%   | | | | |_| |  __/ | |_) | (_| | (_| |  __/
%   |_| |_|\__|_|\___| | .__/ \__,_|\__, |\___|
%                      |_|          |___/      
%
% «title»  (to ".title")
% (c2m212p2p 1 "title")
% (c2m212p2a   "title")

\thispagestyle{empty}

\begin{center}

\vspace*{1.2cm}

{\bf \Large Cálculo 2 - 2021.2}

\bsk

Segunda prova (P2)

\bsk

Eduardo Ochs - RCN/PURO/UFF

\url{http://angg.twu.net/2021.2-C2.html}

\end{center}

\newpage

% «regras-e-dicas»  (to ".regras-e-dicas")
% (c2m212p2p 2 "regras-e-dicas")
% (c2m212p2a   "regras-e-dicas")

{\bf Regras e dicas}

As regras e dicas são as mesmas dos mini-testes,

exceto que a prova será disponibilizada às 22:00

da quinta, 3/fev/2022, e você deverá entregá-la

até as 22:00 do sábado, 5/fev/2022.



\bsk

\newpage

{\bf Importante}

\ssk

Os símbolos ``[MVG]'', ``[RP2]'' e ``[EVS]''

nas questões são referências a estes PDFs:

\ssk

{\footnotesize

% (c2m212mvp 2)
%    http://angg.twu.net/LATEX/2021-2-C2-mud-var-gamb.pdf
\url{http://angg.twu.net/LATEX/2021-2-C2-mud-var-gamb.pdf}

% (c2m212rp2p 5)
%    http://angg.twu.net/LATEX/2021-2-C2-revisao-pra-P2.pdf
\url{http://angg.twu.net/LATEX/2021-2-C2-revisao-pra-P2.pdf}

% (c2m211edovsp 2 "introducao")
% (c2m211edovs    "introducao")
%    http://angg.twu.net/LATEX/2021-2-C2-edovs.pdf
\url{http://angg.twu.net/LATEX/2021-2-C2-edovs.pdf}

}

\bsk

Nas questões 1 e 2 desta prova você {\bf VAI TER QUE}

usar a notação das caixinhas de truques do [MVG].

% (c2m212rp2p 5 "exercicio-3")
% (c2m212rp2a   "exercicio-3")
% (find-books "__analysis/__analysis.el" "miranda")



% \msk
% 
% Várias das questões desta prova são baseadas nos
% 
% exercícios sobre funções homogêneas que nós discutimos
% 
% na última aula antes da prova. Leia o log aqui:


\newpage

% «questao-1»  (to ".questao-1")
% (c2m212p2p 3 "questao-1")
% (c2m212p2a   "questao-1")
% (c2m212rp2p 5 "exercicio-3")
% (c2m212rp2a   "exercicio-3")

{\bf Questão 1.}

\T(Total: 3.0 pts)

Obs: esta questão é muito parecida

com o exercício 3 do [RP2].

% você integrou isto aqui:
%
%$$\intth {(\cos θ)^2(\sen θ)^2·\sen θ}$$

\bsk
\bsk

a) \B(2.0 pts) Use a substituição $c = \cos θ$ pra integrar
%
$$\intth {(\sen θ)^3(\cos θ)^3}.$$

b) \B(1.0 pts) Confira a sua resposta derivando-a.


\newpage

% «questao-2-intro»  (to ".questao-2-intro")
% (c2m211cc2p 8 "d-arcsen-3")
% (c2m211cc2a   "d-arcsen-3")
% (find-LATEXgrep "grep --color=auto -niH --null -e trigo 2*.tex")
% (find-dmirandacalcpage 263 "8.4 Substituição Trigonométrica")

{\bf Introdução à questão 2}


Os livros costumam ensinar Substituição Trigonométrica

exatamente do jeito que o Daniel Miranda faz na seção 8.4

do livro dele --- que é um jeito que só funciona pra pessoas

com muito mais memória que eu. Pra mim o jeito que usa

a caixinha de truques que está explicada aqui:

\msk

{\footnotesize

% (c2m211cc2p 8 "d-arcsen-3")
% (c2m211cc2a   "d-arcsen-3")
%    http://angg.twu.net/LATEX/2021-1-C2-contas-em-C2.pdf#page=8
\url{http://angg.twu.net/LATEX/2021-1-C2-contas-em-C2.pdf\#page=8}

}

\msk

é bem mais fácil...

Você {\bf VAI TER QUE} usá-lo na questão 2.

\newpage

{\bf Introdução à questão 2 (cont.)}

\ssk

Dá pra resolver a questão 2 usando mudanças

de variáveis bem simples. O objetivo da questão 2

é treinar você pra resolver integrais que só podem

ser resolvidas por substituição trigonométrica,

então se você resolver a 2 por outros métodos você

vai estar treinando outras coisas e contornando o

objetivo da questão. Releia o ``VAI TER QUE''

que eu pus em negrito no slide anterior!

\bsk

Em alguns outros lugares da prova eu também pedi

que a resolução seja feita por um determinando

método ou posta num determinado formato.

Recomendo que você leve esses pedidos muito a sério.




\newpage

% «questao-2»  (to ".questao-2")
% (c2m212p2p 6 "questao-2")
% (c2m212p2a   "questao-2")

{\bf Questão 2.}

\T(Total: 3.0 pts)

\ssk

a) \B(2.0 pts) Use o método do slide anterior pra integrar
%
$$\ints {s \sqrt{1 - s^2}}.$$

b) \B(1.0 pts) Confira a sua resposta derivando-a.


% (setq eepitch-preprocess-regexp "^")
% (setq eepitch-preprocess-regexp "^%T ")
%
%T  (eepitch-maxima)
%T  (eepitch-kill)
%T  (eepitch-maxima)
%T rs : sqrt(1 - s^2);
%T integrate(s * rs, s);
%T integrate(s^2 * rs, s);
%T integrate(s^2 * rs, s) - asin(s) / 8;



\newpage

% «questao-3-intro»  (to ".questao-3-intro")
% (c2m212p2p 7 "questao-3-intro")
% (c2m212p2a   "questao-3-intro")

{\bf Introdução à questão 3}

\ssk

Na parte do curso sobre EDOs com variáveis separáveis

nos vimos um \ColorRed{método} para resolvê-las...

\msk

Nós vimos porque o método funciona --- a ``demonstração''

dele tem umas gambiarras bem difíceis de formalizar --- e

vimos que podemos \ColorRed{aplicar} o método mesmo sem entender

ele direito, só usando o `$[:=]$'...

\bsk

Pra ``aplicar o método'' nós primeiro usamos o `$[:=]$'

pra obter várias ``soluções gerais'', e depois pra procurar

a solução que passa por um certo ponto $(x,y)$ dado nós

precisamos escolher a ``solução geral'' certa, ajustar

umas constantes nela, e depois precisamos testar essa

solução.




\newpage

% «questao-3-intro-2»  (to ".questao-3-intro-2")
% (c2m212p2p 8 "questao-3-intro-2")
% (c2m212p2a   "questao-3-intro-2")


\msk

{\bf Introdução à questão 3 (cont.)}

\ssk

\def\ssqrt#1{\ColorRed{#1}\sqrt{\mathstrut\ldots}}

Nos slides 19 e 20 do [EVS] eu falei sobre  de três tipos

de soluções gerais: um com `$\ColorRed{\pm}\sqrt{\ldots}$' , um com
`$\ColorRed{+}\sqrt{\ldots}$',

e um com `$\ColorRed{-}\sqrt{\ldots}$'.

\msk

Às vezes os enunciados vão dizer qual tipo você vai ter

que usar e às vezes você vai ter que descobrir o tipo

certo.


\newpage

% (c2m211edovsp 21 "duas-formulas")
% (c2m211edovsa    "duas-formulas")

{\bf Uma dica pra questão 3}

\ssk

No slide 21 do [EVS] eu defini as fórmulas

[EDOVSG1] e [EDOVSG2]. A [EDOVSG1]

é bem grande, e eu recomendo \underline{{\bf MUITO}} que

antes de fazer a questão 3 você copie ela numa

folha de papel e recorte-a --- como você fez com

a [S2I] no início do curso, aqui:

\ssk

{\footnotesize

% (c2m212introp 10 "exercicio-1")
% (c2m212introa    "exercicio-1")
%    http://angg.twu.net/LATEX/2021-2-C2-intro.pdf#page=10
\url{http://angg.twu.net/LATEX/2021-2-C2-intro.pdf#page=10}

}





\newpage

% «questao-3»  (to ".questao-3")
% (c2m212p2p 8 "questao-3")
% (c2m212p2a   "questao-3")
% (c2m211edovsp 21 "duas-formulas")
% (c2m211edovsa    "duas-formulas")

\def\rq{\ColorRed{?}}


\sa{EDOVSG1}{
  \left(
  \begin{array}{rcl}
  \D \frac{dy}{dx} &=& \D \frac{f(x)}{g(y)} \\ [10pt]
        g(y) \, dy &=&   f(x) \, dx         \\ [5pt]
      ∫ g(y) \, dy &=& ∫ f(x) \, dx         \\
   \rotl{=}\ph{mm} & & \ph{mm}\rotl{=}      \\[-5pt]
        G(y) + C_1 & & F(x) + C_2           \\
                                               [5pt]
        G(y) + C_1 &=& F(x) + C_2           \\ [5pt]
        G(y)       &=& F(x) + C_2 - C_1     \\
                   &=& F(x) + C_3           \\
                                               [5pt]
      G^{-1}(G(y)) &=& G^{-1}(F(x) + C_3)   \\
   \rotl{=}\ph{mm} & &                      \\[-5pt]
          y\ph{mm} & &                      \\
  \end{array}
  \right)
  }
\sa{EDOVSG2}{
  \left(
  \begin{array}{rcl}
  \D \frac{dy}{dx} &=& \D \frac{f(x)}{g(y)} \\ [10pt]
                 y &=& G^{-1}(F(x) + C_3)   \\
  \end{array}
  \right)
  }

\sa{Edovsg1}{\text{[EDOVSG1]}}
\sa{Edovsg2}{\text{[EDOVSG2]}}
\sa{Daigoro}{\text{[Daigoro]}}
\sa{Itto}   {\text{[Itto]}}
\sa{Ogami}  {\text{[Ogami]}}
\sa{SDaigoro}{\bsm{
    f(x) = \rq \\
    g(y) = \rq \\
    F(x) = \rq \\
    G(y) = \rq \\
    G^{-1}(y) = \rq \\
  }}
\sa{SItto}{\bsm{
    f(x) = \rq \\
    g(y) = \rq \\
    F(x) = \rq \\
    G(y) = \rq \\
    G^{-1}(y) = \rq \\
    C_3 = \rq \\
  }}

{\bf Questão 3.}

\T(Total: 5.0 pts)

Seja $(*)$ esta EDO:
%
$$\frac{dy}{dx} = -\frac{4x}{y} \qquad (*)$$

a) \B(1.0 pts) Encontre uma substituição para a [EDOVSG2]

que dá a solução geral da EDO $(*)$ com `$\ssqrt{\pm}$' e chame o

resultado desta substituição de [Daigoro]. A sua resposta

deve ser algo desta forma:
%
$$\ga{Daigoro} \;\;=\;\; \ga{Edovsg2} \ga{SDaigoro} \;\;=\;\; (\rq)$$


\newpage

% «questao-3-bcd»  (to ".questao-3-bcd")
% (c2m212p2p 9 "questao-3-bcd")
% (c2m212p2a   "questao-3-bcd")

{\bf Questão 3 (cont.)}

\msk


b) \B(1.0 pts) Encontre uma substituição para a

[EDOVSG2] que dá a solução da EDO $(*)$ que

passa pelo ponto $(x,y)=(2,3)$. Aqui você vai

ter que usar `$\ssqrt{+}$' ou `$\ssqrt{-}$', e a sua

resposta deve ser algo desta forma:
%
$$\ga{Itto} \;\;=\;\; \ga{Edovsg2} \ga{SItto} \;\;=\;\; (\rq)$$


c) \B(1.0 pts) Verifique a sua solução obedece $\frac{dy}{dx}=-\frac{4x}{y}$.

d) \B(1.0 pts) Verifique a sua solução passa pelo ponto $(2,3)$.



\newpage

{\bf Questão 3 (cont.)}

\msk

e) \B(1.0 pts) Mostre qual é o resultado de aplicar

a substituição que você obteve no item (b) na

[EDOVSG1]. O resultado deve ser algo desta forma:
%
$$\ga{Ogami} \;\;=\;\; \ga{Edovsg1} \ga{SItto} \;\;=\;\; (\rq)$$


\newpage


% «questao-3-maxima»  (to ".questao-3-maxima")
% (setq eepitch-preprocess-regexp "^")
% (setq eepitch-preprocess-regexp "^%T ?")
%
%T  (eepitch-maxima)
%T  (eepitch-kill)
%T  (eepitch-maxima)
%T load("/usr/share/emacs/site-lisp/maxima/emaxima.lisp")$
%T display2d:'emaxima$
%T
%T  (eepitch-maxima)
%T  (eepitch-kill)
%T  (eepitch-maxima)
%T dydx    :  'diff(y,x);
%T myeq    : ('diff(y,x) = - 4*x/y);
%T mysolg  : ode2 (myeq, y, x);
%T mysol1  : ic1(mysolg, x=4, y=6);
%T mysol1  : ic1(mysolg, x=2, y=3);
%T mysol2  : solve(mysol1, y);
%T x0      : 2;
%T y0      : 3;
%T mysol3  : subst([x=x0],       mysol2);
%T mysol3  : subst([x=x0, y=y0], mysol2);
%T my_y    : mysol2[2];
%T my_y    : rhs(mysol2[2]);
%T my_y_x  : diff(my_y, x);
%T subst([],                    myeq);
%T subst([dydx=my_y_x],         myeq);
%T subst([dydx=my_y_x, y=my_y], myeq);
%T 
%T solve(ode2(('diff(y,x)=-4*x/y), y, x), y);

\newpage

% «questao-1-gab»  (to ".questao-1-gab")
% (c2m212p2p 14 "questao-1-gab")
% (c2m212p2a    "questao-1-gab")
% (c2m212rp2p 4 "exercicio-2")
% (c2m212rp2a   "exercicio-2")

{\bf Questão 1: gabarito}

\sa{caixa s=sen theta}{
    s = \sen θ \\
    \frac{ds}{dθ} = \frac{d}{dθ} \sen θ = \cos θ \\
    ds = \cos θ \, dθ \\
    (\cosθ)^2 + (\senθ)^2 = 1 \\
    (\cosθ)^2 = 1 - (\senθ)^2 \\
    (\cosθ)^2 = 1 - s^2 \\
  }
\sa{caixa c=cos theta}{
    c = \sen θ \\
    \frac{dc}{dθ} = \frac{d}{dθ} \cos θ = -\sen θ \\
    dc = -\sen θ \, dθ \\
    (-1)dc = \sen θ \, dθ \\
    (\cosθ)^2 + (\senθ)^2 = 1 \\
    (\senθ)^2 = 1 - (\cosθ)^2 \\
    (\senθ)^2 = 1 - c^2 \\
  }
\sa{caixa s=sen theta trig}{
    s = \sen θ \\
    \frac{ds}{dθ} = \frac{d}{dθ} \sen θ = \cos θ \\
    ds = \cos θ \, dθ\\
    1 - s^2 = \cos^2 θ \\
    \sqrt{1 - s^2} = \cos θ \\
    θ = \arcsen s \\
  }


\scalebox{0.85}{\def\colwidth{12cm}\firstcol{

\msk

a) \;\;
%
$\begin{array}{l}
  \intth {(\sen θ)^3(\cos θ)^3} \\
  = \intth {(\sen θ)^2(\cos θ)^3(\sen θ)} \\
  = \intc {(1-c^2)(c)^3(-1)} \\
  = \intc {(c^2-1)(c)^3} \\
  = \intc {c^5-c^3} \\
  = \frac{c^6}{6} - \frac{c^4}{4} \\
  = \frac{(\cos θ)^6}{6} - \frac{(\cos θ)^4}{4} \\
  \end{array}
  \qquad
  \bsm{\ga{caixa c=cos theta}}
$

\bsk
\bsk

b) \;\; $\begin{array}{l}
  \frac{d}{dθ} \left( \frac{(\cos θ)^6}{6} - \frac{(\cos θ)^4}{4} \right) \\
  = 6\frac{(\cos θ)^5}{6}(\cos'θ) - 4\frac{(\cos θ)^3}{4} (\cos'θ) \\
  = ((\cos θ)^5 - (\cos θ)^3) (-\sen θ) \\
  = (\cos θ)^3((\cos θ)^2 - 1) (-\sen θ) \\
  = (\cos θ)^3(-1)(\sen θ)^2 (-\sen θ) \\
  = (\sen θ)^3(\cos θ)^3 \\
  \end{array}
$

%}\anothercol{
}}


\newpage

% «questao-2-gab»  (to ".questao-2-gab")
% (c2m212p2p 15 "questao-2-gab")
% (c2m212p2a    "questao-2-gab")

{\bf Questão 2: gabarito}

\scalebox{0.8}{\def\colwidth{12cm}\firstcol{

a) \;\;
%
$\begin{array}{l}
 \ints {s \sqrt{1 - s^2}} \\
 = \intth {(\sen θ) (\cos θ) (\cos θ)} \\
 = \intth {(\cos θ)^2 (\sen θ)} \\
 = \intc {c^2·(-1)} \\
 = -\frac{c^3}{3} \\
 = -\frac{(\cos θ)^3}{3} \\
 = -\frac{\sqrt{1-s^2}^3}{3} \\
 \end{array}
 %
 \qquad
 \begin{array}{c}
   \bsm{\ga{caixa s=sen theta trig}} \\[25pt]
   \bsm{\ga{caixa c=cos theta}}
 \end{array}
$

\bsk

b) \;\;
%
$\begin{array}{l}
 \frac{d}{ds} \left( -\frac{\sqrt{1-s^2}^3}{3} \right) \\
 = \frac{d}{ds} \left( -\frac{1}{3}(1-s^2)^{3/2} \right) \\
 = -\frac{1}{3} \frac{3}{2} (1-s^2)^{1/2} · \frac{d}{ds} (1-s^2)\\
 = -\frac{1}{3} \frac{3}{2} (1-s^2)^{1/2} · (-2s) \\
 = s \sqrt{1-s^2} \\
 \end{array}
$

%}\anothercol{
}}


%\bmat{\ga{caixa c=cos theta}}
%\bmat{\ga{caixa s=sen theta trig}}


\newpage

{\bf Questão 3: mini-gabarito}

\sa{Edovsg1}{\text{[EDOVSG1]}}
\sa{Edovsg2}{\text{[EDOVSG2]}}
\sa{Daigoro}{\text{[Daigoro]}}
\sa{Itto}   {\text{[Itto]}}
\sa{Ogami}  {\text{[Ogami]}}
\sa{SDaigoro}{\bsm{
    f(x) = \rq \\
    g(y) = \rq \\
    F(x) = \rq \\
    G(y) = \rq \\
    G^{-1}(y) = \rq \\
  }}
\sa{SDaigoro2}{\bsm{
    f(x) = -4x \\
    g(y) = y \\
    F(x) = -2x^2 \\
    G(y) = y^2/2 \\
    G^{-1}(y) = \sqrt{2y} \\
  }}
\sa{SItto}{\bsm{
    f(x) = -4x \\
    g(y) = y \\
    F(x) = -2x^2 \\
    G(y) = y^2/2 \\
    G^{-1}(y) = \sqrt{2y} \\
    C_3 = 25/2 \\
  }}
\sa{EDOVSG2}{
  \left(
  \begin{array}{rcl}
  \D \frac{dy}{dx} &=& \D \frac{f(x)}{g(y)} \\ [10pt]
                 y &=& G^{-1}(F(x) + C_3)   \\
  \end{array}
  \right)
  }
\sa{DAIGORO}{
  \left(
  \begin{array}{rcl}
  \D \frac{dy}{dx} &=& \D \frac{-4x}{y} \\ [10pt]
                 y &=& +\sqrt{2(-2x^2 + C_3)} \\
  \end{array}
  \right)
  }
\sa{ITTO}{
  \left(
  \begin{array}{rcl}
  \D \frac{dy}{dx} &=& \D \frac{-4x}{y} \\ [10pt]
                 y &=& +\sqrt{2(-2x^2 + 25/2)} \\
  \end{array}
  \right)
  }


\scalebox{0.6}{\def\colwidth{12cm}\firstcol{

$$\begin{array}{rcl}
  \ga{Edovsg2} &=& \ga{EDOVSG2} \\[20pt]
  \ga{Daigoro} \;\;=\;\; \ga{Edovsg2} \ga{SDaigoro2} 
               &=& \ga{DAIGORO} \\[20pt]
  \ga{Itto}    \;\;=\;\; \ga{Edovsg2} \ga{SItto}
               &=& \ga{ITTO} \\
  \end{array}
$$

$$\begin{array}{l}
  y = \sqrt{2(-2x^2+C_3)} \\
  3 = \sqrt{2(-2(2)^2+C_3)} = \sqrt{- 16 + 2C_3}\\
  9 = - 16 + 2C_3 \\
  2C_3 = 25 \\
  C_3 = 25/2 \\
  y = \sqrt{2(-2x^2+25/2)}  = \sqrt{25-4x^2} \\
  \frac{dy}{dx} = \frac{d}{dx} (25-4x^2)^{1/2}
                = \frac{1}{2} (25-4x^2)^{-1/2} · (-8x)
                = \frac{-4}{\sqrt{25-4x^2}} \\
  \frac{-4x}{y} = \frac{-4}{\sqrt{25-4x^2}} \\
  \end{array}
$$

}\anothercol{
}}





% «ideias»  (to ".ideias")
% (c2m212p2p 2 "ideias")
% (c2m212p2a   "ideias")

%Uma questão de variáveis separáveis
% (c2m211edovsp 4 "campos-dirs")
% (c2m211edovs    "campos-dirs")

%Uma questão de subst trigonométrica
% (c2m212intsp 10 "exercicio-4")
% (c2m212intsa    "exercicio-4")

%Dica: não misture as variáveis
% (c2m212mvp 7 "importante")
% (c2m212mva   "importante")




%\printbibliography

\GenericWarning{Success:}{Success!!!}  % Used by `M-x cv'

\end{document}

%  ____  _             _         
% |  _ \(_)_   ___   _(_)_______ 
% | | | | \ \ / / | | | |_  / _ \
% | |_| | |\ V /| |_| | |/ /  __/
% |____// | \_/  \__,_|_/___\___|
%     |__/                       
%
% «djvuize»  (to ".djvuize")
% (find-LATEXgrep "grep --color -nH --null -e djvuize 2020-1*.tex")

 (eepitch-shell)
 (eepitch-kill)
 (eepitch-shell)
# (find-fline "~/2021.2-C2/")
# (find-fline "~/LATEX/2021-2-C2/")
# (find-fline "~/bin/djvuize")

cd /tmp/
for i in *.jpg; do echo f $(basename $i .jpg); done

f () { rm -v $1.pdf;  textcleaner -f 50 -o  5 $1.jpg $1.png; djvuize $1.pdf; xpdf $1.pdf }
f () { rm -v $1.pdf;  textcleaner -f 50 -o 10 $1.jpg $1.png; djvuize $1.pdf; xpdf $1.pdf }
f () { rm -v $1.pdf;  textcleaner -f 50 -o 20 $1.jpg $1.png; djvuize $1.pdf; xpdf $1.pdf }

f () { rm -fv $1.png $1.pdf; djvuize $1.pdf }
f () { rm -fv $1.png $1.pdf; djvuize WHITEBOARDOPTS="-m 1.0 -f 15" $1.pdf; xpdf $1.pdf }
f () { rm -fv $1.png $1.pdf; djvuize WHITEBOARDOPTS="-m 1.0 -f 30" $1.pdf; xpdf $1.pdf }
f () { rm -fv $1.png $1.pdf; djvuize WHITEBOARDOPTS="-m 1.0 -f 45" $1.pdf; xpdf $1.pdf }
f () { rm -fv $1.png $1.pdf; djvuize WHITEBOARDOPTS="-m 0.5" $1.pdf; xpdf $1.pdf }
f () { rm -fv $1.png $1.pdf; djvuize WHITEBOARDOPTS="-m 0.25" $1.pdf; xpdf $1.pdf }
f () { cp -fv $1.png $1.pdf       ~/2021.2-C2/
       cp -fv        $1.pdf ~/LATEX/2021-2-C2/
       cat <<%%%
% (find-latexscan-links "C2" "$1")
%%%
}

f 20201213_area_em_funcao_de_theta
f 20201213_area_em_funcao_de_x
f 20201213_area_fatias_pizza



%  __  __       _        
% |  \/  | __ _| | _____ 
% | |\/| |/ _` | |/ / _ \
% | |  | | (_| |   <  __/
% |_|  |_|\__,_|_|\_\___|
%                        
% <make>

 (eepitch-shell)
 (eepitch-kill)
 (eepitch-shell)
# (find-LATEXfile "2019planar-has-1.mk")
make -f 2019.mk STEM=2021-2-C2-P2 veryclean
make -f 2019.mk STEM=2021-2-C2-P2 pdf

% Local Variables:
% coding: utf-8-unix
% ee-tla: "c2p2"
% ee-tla: "c2m212p2"
% End:

