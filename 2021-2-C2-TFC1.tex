% (find-LATEX "2021-2-C2-TFC1.tex")
% (defun c () (interactive) (find-LATEXsh "lualatex -record 2021-2-C2-TFC1.tex" :end))
% (defun C () (interactive) (find-LATEXsh "lualatex 2021-2-C2-TFC1.tex" "Success!!!"))
% (defun D () (interactive) (find-pdf-page      "~/LATEX/2021-2-C2-TFC1.pdf"))
% (defun d () (interactive) (find-pdftools-page "~/LATEX/2021-2-C2-TFC1.pdf"))
% (defun e () (interactive) (find-LATEX "2021-2-C2-TFC1.tex"))
% (defun o () (interactive) (find-LATEX "2021-2-C2-TFC1.tex"))
% (defun u () (interactive) (find-latex-upload-links "2021-2-C2-TFC1"))
% (defun v () (interactive) (find-2a '(e) '(d)))
% (defun d0 () (interactive) (find-ebuffer "2021-2-C2-TFC1.pdf"))
% (defun cv () (interactive) (C) (ee-kill-this-buffer) (v) (g))
%          (code-eec-LATEX "2021-2-C2-TFC1")
% (find-pdf-page   "~/LATEX/2021-2-C2-TFC1.pdf")
% (find-sh0 "cp -v  ~/LATEX/2021-2-C2-TFC1.pdf /tmp/")
% (find-sh0 "cp -v  ~/LATEX/2021-2-C2-TFC1.pdf /tmp/pen/")
%     (find-xournalpp "/tmp/2021-2-C2-TFC1.pdf")
%   file:///home/edrx/LATEX/2021-2-C2-TFC1.pdf
%               file:///tmp/2021-2-C2-TFC1.pdf
%           file:///tmp/pen/2021-2-C2-TFC1.pdf
% http://angg.twu.net/LATEX/2021-2-C2-TFC1.pdf
% (find-LATEX "2019.mk")
% (find-CN-aula-links "2021-2-C2-TFC1" "2" "c2m212tfc1" "c2t1")

% «.video-1»		(to "video-1")
%
% «.defs»		(to "defs")
% «.defs-parabola»	(to "defs-parabola")
% «.title»		(to "title")
% «.intro-1»		(to "intro-1")
% «.intro-2»		(to "intro-2")
% «.intro-3»		(to "intro-3")
% «.exemplo-1»		(to "exemplo-1")
% «.exemplo-1-left»	(to "exemplo-1-left")
% «.exercicio-1»	(to "exercicio-1")
% «.exercicio-2»	(to "exercicio-2")
% «.descontinuidades»	(to "descontinuidades")
% «.tfc1-complicado-1»	(to "tfc1-complicado-1")
% «.exercicio-3»	(to "exercicio-3")
% «.tfc1-complicado-3»	(to "tfc1-complicado-3")
% «.tfc2-exemplo»	(to "tfc2-exemplo")
% «.tfc2-exemplo-2»	(to "tfc2-exemplo-2")
%
% «.djvuize»		(to "djvuize")



% «video-1»  (to ".video-1")
% (c2m212tfc1a    "video-1")
% (find-ssr-links     "c2m212tfc1" "2021-2-C2-TFC1" "XvzrNtle-c0")
% (code-eevvideo      "c2m212tfc1" "2021-2-C2-TFC1" "XvzrNtle-c0")
% (code-eevlinksvideo "c2m212tfc1" "2021-2-C2-TFC1" "XvzrNtle-c0")
% (find-c2m212tfc1video "0:00")
% (find-c2m212tfc1video "0:00" "Preparação pro Mini-Teste 3")

\documentclass[oneside,12pt]{article}
\usepackage[colorlinks,citecolor=DarkRed,urlcolor=DarkRed]{hyperref} % (find-es "tex" "hyperref")
\usepackage{amsmath}
\usepackage{amsfonts}
\usepackage{amssymb}
\usepackage{pict2e}
\usepackage[x11names,svgnames]{xcolor} % (find-es "tex" "xcolor")
\usepackage{colorweb}                  % (find-es "tex" "colorweb")
%\usepackage{tikz}
%
% (find-dn6 "preamble6.lua" "preamble0")
%\usepackage{proof}   % For derivation trees ("%:" lines)
%\input diagxy        % For 2D diagrams ("%D" lines)
%\xyoption{curve}     % For the ".curve=" feature in 2D diagrams
%
\usepackage{edrx21}               % (find-LATEX "edrx21.sty")
\input edrxaccents.tex            % (find-LATEX "edrxaccents.tex")
\input edrx21chars.tex            % (find-LATEX "edrx21chars.tex")
\input edrxheadfoot.tex           % (find-LATEX "edrxheadfoot.tex")
\input edrxgac2.tex               % (find-LATEX "edrxgac2.tex")
%\usepackage{emaxima}              % (find-LATEX "emaxima.sty")
%
%\usepackage[backend=biber,
%   style=alphabetic]{biblatex}            % (find-es "tex" "biber")
%\addbibresource{catsem-slides.bib}        % (find-LATEX "catsem-slides.bib")
%
% (find-es "tex" "geometry")
\usepackage[a6paper, landscape,
            top=1.5cm, bottom=.25cm, left=1cm, right=1cm, includefoot
           ]{geometry}
%
\begin{document}

\catcode`\^^J=10
\directlua{dofile "dednat6load.lua"}  % (find-LATEX "dednat6load.lua")
%L dofile "2021pict2e.lua"           -- (find-LATEX "2021pict2e.lua")
%L Pict2e.__index.suffix = "%"
\pu
\def\pictgridstyle{\color{GrayPale}\linethickness{0.3pt}}
\def\pictaxesstyle{\linethickness{0.5pt}}

% «defs»  (to ".defs")
% (find-LATEX "edrx21defs.tex" "colors")
% (find-LATEX "edrx21.sty")

\def\u#1{\par{\footnotesize \url{#1}}}

\def\drafturl{http://angg.twu.net/LATEX/2021-2-C2.pdf}
\def\drafturl{http://angg.twu.net/2021.2-C2.html}
\def\draftfooter{\tiny \href{\drafturl}{\jobname{}} \ColorBrown{\shorttoday{} \hours}}

\def\mname#1{[\mathsf{#1}]}


% «defs-parabola»  (to ".defs-parabola")
% (find-LATEXgrep "grep --color=auto -nH --null -e Pict2e *.tex")
% (find-LATEX "2021-2-C2-def-integral.lua")
%
%L fp = function (x)
%L     return 4 - (x-2)^2
%L   end
%L fp_x0 = 1
%L fp_xr = 5
%L
%L f_direita = function (x)
%L     local xl = (x - fp_xr)*eps + fp_x0
%L     return fp(xl)
%L   end
%L
%L define_pwi = function (defname)
%L     pwi = Piecewisify.new(fp, seq(0, 4, 0.25))
%L     makepwir = function (thiseps)
%L         eps = thiseps
%L         pwir = Piecewisify.new(f_direita, seq(fp_xr, fp_xr+1, 1/8))
%L         return pwir
%L       end
%L     Pwil = function (thiseps)
%L         makepwir(thiseps)
%L         return Pict2e.new():add(pwi:pol(min(fp_x0, fp_x0+eps),
%L                                         max(fp_x0, fp_x0+eps), "*"):color("orange"))
%L                         -- :add(pwir:pol(fp_xr, fp_xr+1, "*"):color("orange"))
%L                         -- :add(pwir:pw(fp_xr, fp_xr+1))
%L       end
%L     Pwir = function (thiseps)
%L         makepwir(thiseps)
%L         return Pict2e.new():add(pwi:pol(min(fp_x0, fp_x0+eps),
%L                                         max(fp_x0, fp_x0+eps), "*"):color("orange"))
%L                            :add(pwir:pol(fp_xr, fp_xr+1, "*"):color("orange"))
%L                            :add(pwir:pw(fp_xr, fp_xr+1))
%L       end
%L     Pict2e.new()
%L       :setbounds(v(0,0), v(8,5))
%L         :grid()
%L         :add("#1")
%L         :axesandticks()
%L         :add(pwi:pw(0, 4))  -- parabola
%L       :bepc()
%L       :def(defname.."#1")
%L       :output()
%L   end
%L
%L define_pwi("ParR")
%L
%L use_f21343 = function ()
%L     f21343 = function (x)
%L         if x <= 1 then return 2-x end    -- (0,2)--(1,1)
%L         if x <= 2 then return 2*x-1 end  -- (1,1)--(2,3)
%L         if x <= 3 then return x+1 end    -- (2,3)--(3,4)
%L         return 7-x                       -- (3,4)--(4,3)
%L       end
%L     fp    = f21343
%L     fp_x0 = 2
%L     define_pwi("ParR")
%L   end
%L
%L use_f201343 = function ()
%L     f201343 = function (x)
%L         if x <= 1 then return 2-2*x end  -- (0,2)--(1,0)
%L         if x <  2 then return x-1 end    -- (1,0)--(2,1)
%L         if x == 2 then return 2 end      -- (2,2)
%L         if x <= 3 then return x+1 end    -- (2,3)--(3,4)
%L         return 7-x                       -- (3,4)--(4,3)
%L       end
%L     fp    = f201343
%L     fp_x0 = 2
%L     define_pwi("ParR")
%L   end
%L
%L -- use_f21343()
%L -- use_f201343()
%L
\pu





%  _____ _ _   _                               
% |_   _(_) |_| | ___   _ __   __ _  __ _  ___ 
%   | | | | __| |/ _ \ | '_ \ / _` |/ _` |/ _ \
%   | | | | |_| |  __/ | |_) | (_| | (_| |  __/
%   |_| |_|\__|_|\___| | .__/ \__,_|\__, |\___|
%                      |_|          |___/      
%
% «title»  (to ".title")
% (c2m212tfc1p 1 "title")
% (c2m212tfc1a   "title")

\thispagestyle{empty}

\begin{center}

\vspace*{1.2cm}

{\bf \Large Cálculo 2 - 2021.2}

\bsk

Aula 20: o TFC1

\bsk

Eduardo Ochs - RCN/PURO/UFF

\url{http://angg.twu.net/2021.2-C2.html}

\end{center}

\newpage

%  ___       _                 _                       
% |_ _|_ __ | |_ _ __ ___   __| |_   _  ___ __ _  ___  
%  | || '_ \| __| '__/ _ \ / _` | | | |/ __/ _` |/ _ \ 
%  | || | | | |_| | | (_) | (_| | |_| | (_| (_| | (_) |
% |___|_| |_|\__|_|  \___/ \__,_|\__,_|\___\__,_|\___/ 
%                                                      
% «intro-1»  (to ".intro-1")
% (c2m212tfc1p 2 "intro-1")
% (c2m212tfc1a   "intro-1")

{\bf Introdução}

\scalebox{0.75}{\def\colwidth{12cm}\firstcol{

Digamos que $f:[a,b] \to \R$ é uma função integrável.

Digamos que $c∈[a,b]$.

Digamos que a função $F:[a,b] \to \R$ é \ColorRed{definida} por:
%
$$F(t) \;\; = \Intx{c}{t}{f(x)}.$$

O TFC1 tem duas versões.

A versão mais simples diz o seguinte:

se a função $f$ é contínua então para todo $t∈(a,b)$ vale:
%
$$F'(t) \;\; = f(t). \qquad \qquad (*)$$

A versão mais complicada do TFC1, que vamos ver

depois, não supõe que a função $f$ é contínua.

\msk

Nós vamos ver um argumento visual que mostra que

a igualdade $(*)$ é verdade. Esse argumento visual é

\ColorRed{quase} uma demonstração formal, num sentido que eu

vou explicar depois.

}}



\newpage

% «intro-2»  (to ".intro-2")
% (c2m212tfc1p 3 "intro-2")
% (c2m212tfc1a   "intro-2")

{\bf Introdução (2)}

\scalebox{0.75}{\def\colwidth{12cm}\firstcol{

Digamos que $f:[a,b] \to \R$ é uma função \ColorRed{contínua}.

Digamos que $c∈[a,b]$.

Digamos que a função $F:[a,b] \to \R$ é \ColorRed{definida} por:
%
$$F(t) \;\; = \Intx{c}{t}{f(x)}.$$

\def\eqq{\overset{\ColorRed{???}}{=}}

Então:
%
$$\begin{array}{rcl}
  F'(t) &=& \D \lim_{ε→0} \frac{F(t+ε)-F(t)}{ε} \\
        &=& \D \lim_{ε→0} \frac{ \Intx{c}{t+ε}{f(x)} - \Intx{c}{t}{f(x)} }{ε} \\
        &=& \D \lim_{ε→0} \frac{ \Intx{t}{t+ε}{f(x)} }{ε} \\[12pt]
        &=& \D \lim_{ε→0} \frac{1}{ε} \Intx{t}{t+ε}{f(x)}  \\[12pt]
        &\eqq& f(t) \\
  \end{array}
$$


}}


\newpage

% «intro-3»  (to ".intro-3")
% (c2m212tfc1p 4 "intro-3")
% (c2m212tfc1a   "intro-3")

{\bf Introdução (3)}

Digamos que $f:[a,b] \to \R$ é uma função \ColorRed{contínua}.

Digamos que $c∈[a,b]$.

Digamos que a função $F:[a,b] \to \R$ é \ColorRed{definida} por:
%
$$F(t) \;\; = \Intx{c}{t}{f(x)}.$$

O nosso argumento visual vai mostrar que:
%
$$\begin{array}{rcl}
  \D \lim_{ε→0} \frac{1}{ε} \Intx{t}{t+ε}{f(x)}
  &=& f(t). \\
  \end{array}
$$



\newpage

%  _____                          _         _ 
% | ____|_  _____ _ __ ___  _ __ | | ___   / |
% |  _| \ \/ / _ \ '_ ` _ \| '_ \| |/ _ \  | |
% | |___ >  <  __/ | | | | | |_) | | (_) | | |
% |_____/_/\_\___|_| |_| |_| .__/|_|\___/  |_|
%                          |_|                
%
% «exemplo-1»  (to ".exemplo-1")
% (c2m212tfc1p 5 "exemplo-1")
% (c2m212tfc1a   "exemplo-1")

\scalebox{1.0}{\def\colwidth{5cm}\firstcol{

{\bf Primeiro exemplo:}

$f(x)$ é a nossa parábola

preferida, e $t=1$.

\msk

Primeira figura: $ε=2$.

Segunda figura: $ε=1$.

Terceira figura: $ε=1/2$.

\msk

À esquerda: $\Intx{t}{t+ε}{f(x)}$.

À direita: $\frac{1}{ε}\Intx{t}{t+ε}{f(x)}$.

\msk

Repare que a área em

laranja à esquerda sempre

tem base $ε$ e a área em

laranja à direita sempre

tem base $ε·\frac{1}{ε}=1$.


}\anothercol{

\unitlength=10pt

%$$\ParR{}$$
$$\ParR{\expr{Pwir(2)}}$$
$$\ParR{\expr{Pwir(1)}}$$
$$\ParR{\expr{Pwir(1/2)}}$$

}}




\newpage

\unitlength=25pt

\def\myint{\Intx{1}{1+ε}{f(x)}}
\def\myinte#1{
  $\begin{array}{rl}
   \D             \myint & \text{e} \\[15pt]
   \D \frac{1}{ε} \myint & \text{quando $ε=#1$:} \\
   \end{array}
  $}

\msk

\myinte{2}
$$\ParR{\expr{Pwir(2)}}$$
\newpage
\myinte{1}
$$\ParR{\expr{Pwir(1)}}$$
\newpage
\myinte{1/2}
$$\ParR{\expr{Pwir(1/2)}}$$
\newpage
\myinte{1/4}
$$\ParR{\expr{Pwir(1/4)}}$$
\newpage
\myinte{1/8}
$$\ParR{\expr{Pwir(1/8)}}$$
\newpage
\myinte{1/16}
$$\ParR{\expr{Pwir(1/16)}}$$
\newpage
\myinte{1/32}
$$\ParR{\expr{Pwir(1/32)}}$$
\newpage
\myinte{1/64}
$$\ParR{\expr{Pwir(1/64)}}$$




\newpage

% «exemplo-1-left»  (to ".exemplo-1-left")
% (c2m212tfc1p 14 "exemplo-1-left")
% (c2m212tfc1a    "exemplo-1-left")

\scalebox{1.0}{\def\colwidth{5cm}\firstcol{

{\bf Agora com $ε$ negativo!...}

\msk

$f(x)$ é a nossa parábola

preferida, e $t=1$.

\msk

Primeira figura: $ε=-1$.

Segunda figura: $ε=-1/2$.

Terceira figura: $ε=-1/4$.

\msk

À esquerda: $\Intx{t}{t+ε}{f(x)}$.

À direita: $\frac{1}{ε}\Intx{t}{t+ε}{f(x)}$.

% \msk
% 
% Repare que a área em
% 
% laranja à esquerda sempre
% 
% tem base $ε$ e a área em
% 
% laranja à direita sempre
% 
% tem base $ε·\frac{1}{ε}=1$.


}\anothercol{

\unitlength=10pt

%$$\ParR{}$$
$$\ParR{\expr{Pwir(-1)}}$$
$$\ParR{\expr{Pwir(-1/2)}}$$
$$\ParR{\expr{Pwir(-1/4)}}$$

}}

\newpage

\myinte{-1}
$$\ParR{\expr{Pwir(-1)}}$$
\newpage
\myinte{-1/2}
$$\ParR{\expr{Pwir(-1/2)}}$$
\newpage
\myinte{-1/4}
$$\ParR{\expr{Pwir(-1/4)}}$$
\newpage
\myinte{-1/8}
$$\ParR{\expr{Pwir(-1/8)}}$$
\newpage
\myinte{-1/16}
$$\ParR{\expr{Pwir(-1/16)}}$$
\newpage
\myinte{-1/32}
$$\ParR{\expr{Pwir(-1/32)}}$$
\newpage
\myinte{-1/64}
$$\ParR{\expr{Pwir(-1/64)}}$$


\newpage

%  _____                   _      _         _ 
% | ____|_  _____ _ __ ___(_) ___(_) ___   / |
% |  _| \ \/ / _ \ '__/ __| |/ __| |/ _ \  | |
% | |___ >  <  __/ | | (__| | (__| | (_) | | |
% |_____/_/\_\___|_|  \___|_|\___|_|\___/  |_|
%                                             
% «exercicio-1»  (to ".exercicio-1")
% (c2m212tfc1p 22 "exercicio-1")
% (c2m212tfc1a    "exercicio-1")

% (find-LATEX "edrx21defs.tex" "firstcol-anothercol")

%L use_f21343()
\pu

\scalebox{1.0}{\def\colwidth{6cm}\firstcol{

{\bf Exercício 1.}

Seja $f(x)$ a função à direita.

Seja $t=2$.

\msk

a) Desenhe $\frac{1}{ε}\Intx{t}{t+ε}{f(x)}$

para $ε=2$, $ε=1$, $ε=1/2$. 

\msk

b) Desenhe $\frac{1}{ε}\Intx{t}{t+ε}{f(x)}$

para $ε=-2$, $ε=-1$, $ε=-1/2$. 

\msk

Dica: comece entendendo as

áreas em laranja à direita!

\msk

c) Quanto você acha que dá

$\lim_{ε→0^+} \frac{1}{ε} \Intx{t}{t+ε}{f(x)}$?

\msk

d) Quanto você acha que dá

$\lim_{ε→0^-} \frac{1}{ε} \Intx{t}{t+ε}{f(x)}$?

}\hspace*{-1cm}\anothercol{

\unitlength=7.5pt

\def\PPP#1{\ParR{\expr{Pwil(#1)}}}

$$\PPP{2} \quad \PPP{-2}$$
$$\PPP{1} \quad \PPP{-1}$$
$$\PPP{1/2} \quad \PPP{-1/2}$$


}}

\newpage

%  _____                   _      _         ____  
% | ____|_  _____ _ __ ___(_) ___(_) ___   |___ \ 
% |  _| \ \/ / _ \ '__/ __| |/ __| |/ _ \    __) |
% | |___ >  <  __/ | | (__| | (__| | (_) |  / __/ 
% |_____/_/\_\___|_|  \___|_|\___|_|\___/  |_____|
%                                                 
% «exercicio-2»  (to ".exercicio-2")
% (c2m212tfc1p 23 "exercicio-2")
% (c2m212tfc1a    "exercicio-2")
% (find-LATEX "edrx21defs.tex" "firstcol-anothercol")

%L use_f201343()
\pu

\scalebox{1.0}{\def\colwidth{6cm}\firstcol{

{\bf Exercício 2.}

Seja $f(x)$ a função à direita.

Seja $t=2$.

\msk

a) Desenhe $\frac{1}{ε}\Intx{t}{t+ε}{f(x)}$

para $ε=2$, $ε=1$, $ε=1/2$. 

\msk

b) Desenhe $\frac{1}{ε}\Intx{t}{t+ε}{f(x)}$

para $ε=-2$, $ε=-1$, $ε=-1/2$. 

\msk

Dica: comece entendendo as

áreas em laranja à direita!

\msk

c) Quanto você acha que dá

$\lim_{ε→0^+} \frac{1}{ε} \Intx{t}{t+ε}{f(x)}$?

\msk

d) Quanto você acha que dá

$\lim_{ε→0^-} \frac{1}{ε} \Intx{t}{t+ε}{f(x)}$?

}\hspace*{-1cm}\anothercol{

\unitlength=7.5pt

\def\PPP#1{\ParR{\expr{Pwil(#1)}}}

$$\PPP{2} \quad \PPP{-2}$$
$$\PPP{1} \quad \PPP{-1}$$
$$\PPP{1/2} \quad \PPP{-1/2}$$


}}

\newpage

% «descontinuidades»  (to ".descontinuidades")
% (c2m212tfc1p 24 "descontinuidades")
% (c2m212tfc1a    "descontinuidades")

{\bf Descontinuidades}

%L f_parabola_preferida = function (x)
%L     return 4 - (x-2)^2
%L   end
%L f_parabola_complicada = function (x)
%L     if x <= 4 then return f_parabola_preferida(x) end
%L     if x <  5 then return 5 - x end
%L     if x <  6 then return 7 - x end
%L     if x <  7 then return 3 end
%L     if x == 7 then return 4 end
%L     return 0.5
%L   end
%L f_funcao_complicada = f_parabola_complicada
%L 
%L pwi = Piecewisify.new(f_funcao_complicada, seq(0, 4, 0.25), 5, 6, 7)
%L 
%L Pict2e.new()
%L   :setbounds(v(0,0), v(8,5))
%L     :grid()
%L     :add("#1")
%L     :axesandticks()
%L     :add(pwi:pw(0, 8))  -- f
%L   :bepc()
%L   :def("ParCoWith#1")
%L   :output()
\pu


\scalebox{0.65}{\def\colwidth{13cm}\firstcol{

Digamos que $f:[a,b]→\R$ é uma função qualquer.

Vamos definir o conjunto dos pontos de descontinuidade da $f$,

ou, pra abreviar, o ``conjunto das descontinuidades da $f$'', assim:
%
$$\mathsf{desc}(f) \;\; = \;\;
  \setofst{x∈[a,b]}{f \text{ é descontinua em $x$}}
$$

A expressão ``$f$ tem um número finito de pontos de descontinuidade'',

que eu vou abreviar pra ``$f$ tem finitas descontinuidades'' apesar

disso soar bem estranho em português, vai querer dizer:
%
$$\mathsf{desc}(f) \text{\;\;é um conjunto finito}$$

O conjunto vazio é finito, então toda $f$ contínua ``tem finitas

descontinuidades''. Essa função aqui tem finitas descontinuidades:
%
$$\unitlength=7.5pt
  \ParCoWith{}
$$


A função de Dirichlet, que nós vimos aqui,

% (c2m211somas2p 46 "dirichlet")
% (c2m211somas2a    "dirichlet")

\ssk

{\footnotesize

\url{http://angg.twu.net/LATEX/2021-1-C2-somas-2.pdf\#page=46}

}

tem infinitas descontinuidades.


%}\anothercol{
}}


\newpage

% «tfc1-complicado-1»  (to ".tfc1-complicado-1")
% (c2m212tfc1p 25 "tfc1-complicado-1")
% (c2m212tfc1a    "tfc1-complicado-1")

{\bf A versão complicada do TFC1}

Vou dizer que uma função $f:[a,b]→\R$ é ``boa''

quando ela é integrável e tem finitas descontinuidades.

\msk

(O termo ``função boa'' é péssimo de propósito ---

é pra deixar óbvio que essa é uma definição temporária,

que vai valer só durante poucos slides...)

\msk

Vou dizer que uma função $G:[a,b]→\R$ obedece
%
$$G'(x) = f(x)$$

quando $G$ for contínua em $[a,b]$ e $G$ obedecer isto aqui:
%
$$∀x∈((a,b) \; ∖ \; \mathsf{desc}(f)). \; G'(x)=f(x)$$

ou seja, neste caso ``$G'(x) = f(x)$'' é uma abreviação

pra algo complicado.


\newpage

% «exercicio-3»  (to ".exercicio-3")
% (c2m212tfc1p 26 "exercicio-3")
% (c2m212tfc1a    "exercicio-3")

{\bf A versão complicada do TFC1 (2)}

%L Pict2e.new()
%L   :setbounds(v(0,-2), v(4,3))
%L     :grid()
%L     :axesandticks()
%L     :add(pictpiecewise("(0,1)--(1,1)o (1,2)c--(2,2)o (2,-1)c--(4,-1)")
%L          :as("\\linethickness{1pt}"))
%L   :bepc()
%L   :def("TFCcomplicEx")
%L   :output()
\pu


\scalebox{0.95}{\def\colwidth{12cm}\firstcol{

Antes de prosseguir vamos fazer um exercício.

\bsk

{\bf Exercício 3.}

Seja:
$$f(x) \;\;=\;\;
  \unitlength=7.5pt
  \TFCcomplicEx
$$

a) Qual é o domínio da $f$? (Ele está ``implícito no gráfico''...)

b) Encontre uma função $G$ que obedece $G'(x)=f(x)$ e $G(0)=0$.

c) Encontre uma função $H$ que obedece $H'(x)=f(x)$ e $H(0)=1$.

d) Faça o gráfico da função $M(x) = H(x) - G(x)$.

e) Encontre uma função $K$ que obedece $K'(x)=f(x)$ e $K(4)=-1$.

%}\anothercol{
}}



\newpage

% «tfc1-complicado-3»  (to ".tfc1-complicado-3")
% (c2m212tfc1p 27 "tfc1-complicado-3")
% (c2m212tfc1a    "tfc1-complicado-3")

{\bf A versão complicada do TFC1 (3)}

\scalebox{0.6}{\def\colwidth{9cm}\firstcol{

Digamos que $f:[a,b]→\R$ é ``boa''.

Digamos que $c∈[a,b]$ e que $G'(x) = f(x)$.

Digamos que
%
$$F(x) \;\; = \;\; \Intt{c}{x}{f(t)}.$$

Então $F$ e $G$ ``diferem por uma constante'',

como as funções $G$, $H$ e $K$ do exercício 3.

Isso é o ``TFC1 na versão complicada''.

Eu não vou demonstrá-lo. \quad $\frown$

\msk

Seja $k$ essa constante. Temos:
%
$$∀x∈[a,b]. \; G(x) = F(x) + k.$$

\msk

Isso tem um monte de consequências bacanas.

Por exemplo: $F(c) = 0$, $G(c) = k$, e,

se $α,β∈[a,b]$,
%
$$\begin{array}{rcl}
  \Intt{α}{β}{f(t)} &=& \Intt{c}{β}{f(t)} - \Intt{c}{α}{f(t)} \\
                    &=& F(β) - F(α) \\
                    &=& (G(β) - k) - (G(α) - k) \\
                    &=& G(β) - G(α). \\
  \end{array}
$$

}\anothercol{

Isso nos dá um \ColorRed{método} pra calcular integrais

da função $f$. Se $α,β∈[a,b]$,

\msk

1) encontramos \ColorRed{uma} solução $G(x)$

da EDO $G'(x) = f(x)$,

\msk

2) usamos a fórmula
%
$$\Intt{α}{β}{f(t)} \;\; = \;\; G(β) - G(α).$$

\msk

Você viu no exercício anterior que a EDO

$G'(x) = f(x)$ tem infinitas soluções...

Qualquer solução serve, e não precisamos

calcular a constante $k$.

\bsk
\bsk

\ColorRed{Esse método é o TFC2.}

}}

\newpage

{\bf O TFC2}

\msk

Digamos que $f:[a,b]→\R$ é ``boa''.

Digamos que $α,β∈[a,b]$ e que $G'(x) = f(x)$.

\msk

Então:

$$\Intt{α}{β}{f(t)} \;\; = \;\; G(β) - G(α).$$

% \qquad \mname{TFC2}


\newpage

% «tfc2-exemplo»  (to ".tfc2-exemplo")
% (c2m212tfc1p 29 "tfc2-exemplo")
% (c2m212tfc1a    "tfc2-exemplo")

{\bf TFC2: um exemplo}

\ssk

A nossa parábola preferida é $f(x) = 4 - (x-2)^2$,

ou seja, $f(x) = 4x - x^2$.

Digamos que $G(x) = 2x^2 - \frac{x^3}{3}$.

Então $G'(x) = f(x)$, e o resultado desta

substituição aqui vai dar uma igualdade verdadeira...


$$\left(
  \Intt{α}{β}{f(t)} \;\; = \;\; G(β) - G(α)
  \right)
  \;
  \bmat{
    f(x) := 4x - x^2 \\
    G(x) := 2x^2 - \frac{x^3}{3} \\
    β := 4 \\
    α := 0 \\
  }
$$

\newpage

% «tfc2-exemplo-2»  (to ".tfc2-exemplo-2")
% (c2m212tfc1p 30 "tfc2-exemplo-2")
% (c2m212tfc1a    "tfc2-exemplo-2")

{\bf TFC2: um exemplo (2)}

\scalebox{0.8}{\def\colwidth{12cm}\firstcol{

Temos:
%
$$\begin{array}{c}
    \left(
    \D \Intt{α}{β}{f(t)} \;\; = \;\; G(β) - G(α)
    \right)
    \;
    \bmat{
      f(x) := 4 - (x-2)^2 \\
      G(x) := 2x^2 - \frac{x^3}{3} \\
      β := 4 \\
      α := 0 \\
    }
  \\[25pt]
    = \;\;
    \left( \D
    \Intt{0}{4}{4-(t-2)^2} \;\; = \;\;
      \left(2·4^2-\frac{4^3}{3}\right) -
      \left(2·0^2-\frac{0^3}{3}\right)
    \right)
    \;
  \\
  \end{array}
$$

e:
%
$$\begin{array}{rcl}
    \D \Intt{0}{4}{4-(t-2)^2}
    &=&
      \left(2·4^2-\frac{4^3}{3}\right) -
      \left(2·0^2-\frac{0^3}{3}\right) \\
    &=&
      \left(32 - \frac{64}{3}\right) - 0 \\[5pt]
    &=& \frac{96}{3} - \frac{64}{3} \\[5pt]
    &=& \frac{32}{3}. \\
  \\
  \end{array}
$$



%}\anothercol{
}}








% (setq eepitch-preprocess-regexp "^")
% (setq eepitch-preprocess-regexp "^% ")
%
%  (eepitch-maxima)
%  (eepitch-kill)
%  (eepitch-maxima)
% f : 4 - (x-2)^2;
% expand(f);
% integrate(expand(f), x);




% (c2m212somas2p 27 "metodos-nomes")
% (c2m212somas2a    "metodos-nomes")


%
%$$F(t) \;\; = \Intx{c}{t}{f(x)}.$$








% (find-books "__analysis/__analysis.el" "apex-calculus-C2")
% (find-apexcalculuspage (+ 10 236) "5.4 The Fundamental Theorem of Calculus")
% (find-apexcalculuspage (+ 10 239)     "the ball has travelled much farther")
% (find-apexcalculuspage (+ 10 240)     "The FTC and the chain rule")
% (find-apexcalculuspage (+ 10 243)     "The mean value theorem of integration")
% (find-apexcalculuspage (+ 10 244)     "average value")
% (find-apexcalculuspage (+ 10 246)     "Exercises 5.4")

% (find-books "__analysis/__analysis.el" "thomas")
% (find-thomas11-1page (+  60  356) "5.4 The fundamental theorem of calculus")
% (find-thomas11-1page (+  60  358)     "Fundamental theorem, part 1")
% (find-thomas11-1page (+  60  361)     "Fundamental theorem, part 2 (the evaluation theorem)")



%\printbibliography

\GenericWarning{Success:}{Success!!!}  % Used by `M-x cv'

\end{document}

%  ____  _             _         
% |  _ \(_)_   ___   _(_)_______ 
% | | | | \ \ / / | | | |_  / _ \
% | |_| | |\ V /| |_| | |/ /  __/
% |____// | \_/  \__,_|_/___\___|
%     |__/                       
%
% «djvuize»  (to ".djvuize")
% (find-LATEXgrep "grep --color -nH --null -e djvuize 2020-1*.tex")

 (eepitch-shell)
 (eepitch-kill)
 (eepitch-shell)
# (find-fline "~/2021.2-C2/")
# (find-fline "~/LATEX/2021-2-C2/")
# (find-fline "~/bin/djvuize")

cd /tmp/
for i in *.jpg; do echo f $(basename $i .jpg); done

f () { rm -v $1.pdf;  textcleaner -f 50 -o  5 $1.jpg $1.png; djvuize $1.pdf; xpdf $1.pdf }
f () { rm -v $1.pdf;  textcleaner -f 50 -o 10 $1.jpg $1.png; djvuize $1.pdf; xpdf $1.pdf }
f () { rm -v $1.pdf;  textcleaner -f 50 -o 20 $1.jpg $1.png; djvuize $1.pdf; xpdf $1.pdf }

f () { rm -fv $1.png $1.pdf; djvuize $1.pdf }
f () { rm -fv $1.png $1.pdf; djvuize WHITEBOARDOPTS="-m 1.0 -f 15" $1.pdf; xpdf $1.pdf }
f () { rm -fv $1.png $1.pdf; djvuize WHITEBOARDOPTS="-m 1.0 -f 30" $1.pdf; xpdf $1.pdf }
f () { rm -fv $1.png $1.pdf; djvuize WHITEBOARDOPTS="-m 1.0 -f 45" $1.pdf; xpdf $1.pdf }
f () { rm -fv $1.png $1.pdf; djvuize WHITEBOARDOPTS="-m 0.5" $1.pdf; xpdf $1.pdf }
f () { rm -fv $1.png $1.pdf; djvuize WHITEBOARDOPTS="-m 0.25" $1.pdf; xpdf $1.pdf }
f () { cp -fv $1.png $1.pdf       ~/2021.2-C2/
       cp -fv        $1.pdf ~/LATEX/2021-2-C2/
       cat <<%%%
% (find-latexscan-links "C2" "$1")
%%%
}

f 20201213_area_em_funcao_de_theta
f 20201213_area_em_funcao_de_x
f 20201213_area_fatias_pizza



%  __  __       _        
% |  \/  | __ _| | _____ 
% | |\/| |/ _` | |/ / _ \
% | |  | | (_| |   <  __/
% |_|  |_|\__,_|_|\_\___|
%                        
% <make>

 (eepitch-shell)
 (eepitch-kill)
 (eepitch-shell)
# (find-LATEXfile "2019planar-has-1.mk")
make -f 2019.mk STEM=2021-2-C2-TFC1 veryclean
make -f 2019.mk STEM=2021-2-C2-TFC1 pdf

% Local Variables:
% coding: utf-8-unix
% ee-tla: "c2t1"
% ee-tla: "c2m212tfc1"
% End:
