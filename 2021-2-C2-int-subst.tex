% (find-LATEX "2021-2-C2-int-subst.tex")
% (defun c () (interactive) (find-LATEXsh "lualatex -record 2021-2-C2-int-subst.tex" :end))
% (defun C () (interactive) (find-LATEXsh "lualatex 2021-2-C2-int-subst.tex" "Success!!!"))
% (defun D () (interactive) (find-pdf-page      "~/LATEX/2021-2-C2-int-subst.pdf"))
% (defun d () (interactive) (find-pdftools-page "~/LATEX/2021-2-C2-int-subst.pdf"))
% (defun e () (interactive) (find-LATEX "2021-2-C2-int-subst.tex"))
% (defun o () (interactive) (find-LATEX "2021-1-C2-int-subst.tex"))
% (defun u () (interactive) (find-latex-upload-links "2021-2-C2-int-subst"))
% (defun v () (interactive) (find-2a '(e) '(d)))
% (defun d0 () (interactive) (find-ebuffer "2021-2-C2-int-subst.pdf"))
% (defun cv () (interactive) (C) (ee-kill-this-buffer) (v) (g))
%          (code-eec-LATEX "2021-2-C2-int-subst")
% (find-pdf-page   "~/LATEX/2021-2-C2-int-subst.pdf")
% (find-sh0 "cp -v  ~/LATEX/2021-2-C2-int-subst.pdf /tmp/")
% (find-sh0 "cp -v  ~/LATEX/2021-2-C2-int-subst.pdf /tmp/pen/")
%     (find-xournalpp "/tmp/2021-2-C2-int-subst.pdf")
%   file:///home/edrx/LATEX/2021-2-C2-int-subst.pdf
%               file:///tmp/2021-2-C2-int-subst.pdf
%           file:///tmp/pen/2021-2-C2-int-subst.pdf
% http://angg.twu.net/LATEX/2021-2-C2-int-subst.pdf
% (find-LATEX "2019.mk")
% (find-CN-aula-links "2021-2-C2-int-subst" "2" "c2m212ints" "c2ints")

% «.video-1»		(to "video-1")
% «.video-2»		(to "video-2")
% «.video-3»		(to "video-3")
% «.defs»		(to "defs")
% «.title»		(to "title")
% «.exercicio-1»	(to "exercicio-1")
% «.exercicio-2»	(to "exercicio-2")
% «.x^-2»		(to "x^-2")
% «.1-then-2»		(to "1-then-2")
% «.S2-proof-1»		(to "S2-proof-1")
% «.dfi»		(to "dfi")
% «.exercicio-3»	(to "exercicio-3")
% «.mais-algumas»	(to "mais-algumas")
% «.exercicio-4»	(to "exercicio-4")
% «.um-exemplo»		(to "um-exemplo")
%
% «.djvuize»		(to "djvuize")



% «video-1»  (to ".video-1")
% (c2m212intsa    "video-1")
% (find-ssr-links     "c2m212ints" "2021-2-C2-int-subst" "YbVfNi-xGNw")
% (code-eevvideo      "c2m212ints" "2021-2-C2-int-subst" "YbVfNi-xGNw")
% (code-eevlinksvideo "c2m212ints" "2021-2-C2-int-subst" "YbVfNi-xGNw")
% (find-yttranscript-links "c2m212ints" "YbVfNi-xGNw")
% (find-c2m212intsvideo  "0:00" "12/jan/2022")
% (find-c2m212intsvideo  "0:40" "a f é a derivada de G, mas tem uns detalhes")
% (find-c2m212intsvideo  "1:38" "calcular a integral de 2 até 4")
% (find-c2m212intsvideo  "1:59" "e a gente também consegue calcular essa área por G(4)-G(2)")
% (find-c2m212intsvideo  "3:10" "exercício 1 sobre diferença")
% (find-c2m212intsvideo  "3:25" "aqui tem um caso fácil do TFC2")
% (find-c2m212intsvideo  "3:33" "definição do TFC2")
% (find-c2m212intsvideo  "3:43" "aplicar o TFC2")
% (find-c2m212intsvideo  "3:50" "resolver algumas integrais por chutar e testar")
% (find-c2m212intsvideo  "4:10" "só que a gente vai ver isso numa outra ordem")
% (find-c2m212intsvideo  "4:28" "o primeiro slide que não é o título")
% (find-c2m212intsvideo  "4:38" "isso aqui vocês viram que era verdade no MT3")
% (find-c2m212intsvideo  "4:48" "mas isso é só a fórmula")
% (find-c2m212intsvideo  "4:50" "as hipóteses são escritas por fora")
% (find-c2m212intsvideo  "5:17" "no mini-teste 3 vocês usaram na prática")
% (find-c2m212intsvideo  "5:34" "testar casos particulares e ver quando funcionam e quando não")
% (find-c2m212intsvideo  "5:44" "dois exemplos de quando o TFC2 não funciona")
% (find-c2m212intsvideo  "5:50" "a gente não sabe o teorema, só a fórmula dele")
% (find-c2m212intsvideo  "5:59" "da mesma forma que a gente pode escrever uma igualdade errada")
% (find-c2m212intsvideo  "6:25" "essa fórmula em que a gente escolhe uma função F")
% (find-c2m212intsvideo  "6:40" "F(x) = -x^-1")
% (find-c2m212intsvideo  "7:08" "vamos tentar calcular essa integral e essa diferença")
% (find-c2m212intsvideo  "7:56" "isso aqui vai dar -2")
% (find-c2m212intsvideo  "8:05" "essa conta da esquerda ... dá algo positivo")
% (find-c2m212intsvideo  "8:40" "e como o TFC tem `teorema' no nome")
% (find-c2m212intsvideo  "8:55" "um exemplo que não envolve infinitos")
% (find-c2m212intsvideo  "9:45" "vai ser essa área, que é igual a 3")
% (find-c2m212intsvideo "10:12" "que dá 4")
% (find-c2m212intsvideo "10:32" "dessa vez a gente vai começar a usar o TFC2")
% (find-c2m212intsvideo "10:49" "a gente vai ter que usar zilhões de vezes o [:=]")
% (find-c2m212intsvideo "10:54" "isso é uma demonstração (do S2)")
% (find-c2m212intsvideo "13:30" "isso aqui é a fórmula do TFC2")
% (find-c2m212intsvideo "15:30" "aí a gente tem essa sequência de igualdades aqui")
% (find-c2m212intsvideo "15:54" "deixa eu ver se eu encontro uma explicação visual")

% «video-2»  (to ".video-2")
% (c2m212intsa    "video-2")
% (find-ssr-links     "c2m212ints2" "2021-2-C2-int-subst-2" "SKff-4NqD6I")
% (code-eevvideo      "c2m212ints2" "2021-2-C2-int-subst-2" "SKff-4NqD6I")
% (code-eevlinksvideo "c2m212ints2" "2021-2-C2-int-subst-2" "SKff-4NqD6I")
% (find-yttranscript-links "c2m212ints2" "SKff-4NqD6I")
% (find-c2m212ints2video "0:00" "19/jan/2022")
% (find-c2m212ints2video "0:00" "Integração por substituição")

% «video-3»  (to ".video-3")
% (c2m212intsa    "video-3")
% (find-ssr-links     "c2m212ints3" "2021-2-C2-int-subst-3" "mZxNYcbq9aU")
% (code-eevvideo      "c2m212ints3" "2021-2-C2-int-subst-3" "mZxNYcbq9aU")
% (code-eevlinksvideo "c2m212ints3" "2021-2-C2-int-subst-3" "mZxNYcbq9aU")
% (find-yttranscript-links "c2m212ints3" "mZxNYcbq9aU")
% (find-c2m212ints3video "0:00" "21/jan/2022")
% (find-c2m212ints3video "0:00")

% Videos antigos:
% (c2m202tfca     "video-1")
% (c2m211isa      "video-1")


\documentclass[oneside,12pt]{article}
\usepackage[colorlinks,citecolor=DarkRed,urlcolor=DarkRed]{hyperref} % (find-es "tex" "hyperref")
\usepackage{amsmath}
\usepackage{amsfonts}
\usepackage{amssymb}
\usepackage{pict2e}
\usepackage[x11names,svgnames]{xcolor} % (find-es "tex" "xcolor")
\usepackage{colorweb}                  % (find-es "tex" "colorweb")
%\usepackage{tikz}
%
% (find-dn6 "preamble6.lua" "preamble0")
%\usepackage{proof}   % For derivation trees ("%:" lines)
%\input diagxy        % For 2D diagrams ("%D" lines)
%\xyoption{curve}     % For the ".curve=" feature in 2D diagrams
%
\usepackage{edrx21}               % (find-LATEX "edrx21.sty")
\input edrxaccents.tex            % (find-LATEX "edrxaccents.tex")
\input edrx21chars.tex            % (find-LATEX "edrx21chars.tex")
\input edrxheadfoot.tex           % (find-LATEX "edrxheadfoot.tex")
\input edrxgac2.tex               % (find-LATEX "edrxgac2.tex")
%\usepackage{emaxima}              % (find-LATEX "emaxima.sty")
%
%\usepackage[backend=biber,
%   style=alphabetic]{biblatex}            % (find-es "tex" "biber")
%\addbibresource{catsem-slides.bib}        % (find-LATEX "catsem-slides.bib")
%
% (find-es "tex" "geometry")
\usepackage[a6paper, landscape,
            top=1.5cm, bottom=.25cm, left=1cm, right=1cm, includefoot
           ]{geometry}
%
\begin{document}

\catcode`\^^J=10
\directlua{dofile "dednat6load.lua"}  % (find-LATEX "dednat6load.lua")
%L dofile "2021pict2e.lua"           -- (find-LATEX "2021pict2e.lua")
%L Pict2e.__index.suffix = "%"
\pu
\def\pictgridstyle{\color{GrayPale}\linethickness{0.3pt}}
\def\pictaxesstyle{\linethickness{0.5pt}}

% «defs»  (to ".defs")
% (find-LATEX "edrx21defs.tex" "colors")
% (find-LATEX "edrx21.sty")

\def\u#1{\par{\footnotesize \url{#1}}}
\def\rq{\ColorRed{?}}

\def\drafturl{http://angg.twu.net/LATEX/2021-2-C2.pdf}
\def\drafturl{http://angg.twu.net/2021.2-C2.html}
\def\draftfooter{\tiny \href{\drafturl}{\jobname{}} \ColorBrown{\shorttoday{} \hours}}




\def\pfo#1{\ensuremath{\mathsf{[#1]}}}
\def\veq{\rotatebox{90}{$=$}}
\def\Rd{\ColorRed}
\def\D{\displaystyle}

% Difference with mathstrut
\def\Difms #1#2#3{\left. \mathstrut #3 \right|_{s=#1}^{s=#2}}
\def\Difmu #1#2#3{\left. \mathstrut #3 \right|_{u=#1}^{u=#2}}
\def\Difmx #1#2#3{\left. \mathstrut #3 \right|_{x=#1}^{x=#2}}
\def\Difmth#1#2#3{\left. \mathstrut #3 \right|_{θ=#1}^{θ=#2}}

\def\iequationbox#1#2{
    \left(
    \begin{array}{rcl}
    \D{ #1 } &=& \D{ #2 } \\
    \end{array}
    \right)
  }
\def\isubstbox#1#2#3#4#5{{
    \def\veq{\rotatebox{90}{$=$}}
    \def\ph{\phantom}
    \left(
    \begin{array}{rcl}
    \D{ #1 } &=& \D{ #2 } \\
    {\veq#3} \\
    \D{ #4 } &=& \D{ #5 } \\
    \end{array}
    \right)
  }}
\def\isubstboxT#1#2#3#4#5#6{{
    \def\veq{\rotatebox{90}{$=$}}
    \def\ph{\phantom}
    \left(
    \begin{array}{rcl}
    \multicolumn{3}{l}{\text{#6}} \\%[5pt]
    \D{ #1 } &=& \D{ #2 } \\
    {\veq#3} \\
    \D{ #4 } &=& \D{ #5 } \\
    \end{array}
    \right)
  }}
\def\isubstboxTT#1#2#3#4#5#6#7{{
    \def\veq{\rotatebox{90}{$=$}}
    \def\ph{\phantom}
    \left(
    \begin{array}{rcl}
    \multicolumn{3}{l}{\text{#6}} \\%[5pt]
    \D{ #1 } &=& \D{ #2 } \\
    {\veq#3} \\
    \D{ #4 } &=& \D{ #5 } \\
    \multicolumn{3}{l}{\text{#7}} \\%[5pt]
    \end{array}
    \right)
  }}

% Definição das fórmulas para integração por substituição.
% Algumas são pmatrizes 3x3 usando isubstbox.

\def\TFCtwo{
  \iequationbox {\Intx{a}{b}{F'(x)}}
                {\Difmx{a}{b}{F(x)}}
}
\def\TFCtwoI{
  \iequationbox {\intx{F'(x)}}
                {F(x)}
}

\def\Sone{
  \isubstbox
    {\Difmx{a}{b}{f(g(x))}}  {\Intx{a}{b}{f'(g(x))g'(x)}}
    {\ph{mmm}}
    {\Difmu{g(a)}{g(b)}{f(u)}} {\Intu{g(a)}{g(b)}{f'(u)}}
}
\def\SoneI{
  \isubstbox
    {f(g(x))} {\intx{f'(g(x))g'(x)}}
    {\ph{m}}
    {f(u)}    {\intu{f'(u)}}
}

\def\Stwo{
  \isubstboxT
    {\Difmx{a}{b}{F(g(x))}}   {\Intx{a}{b}{f(g(x))g'(x)}}
    {\ph{mmm}}
    {\Difmu{g(a)}{g(b)}{F(u)}}  {\Intu{g(a)}{g(b)}{f(u)}}
    {Se $F'(u)=f(u)$ então:}
}
\def\StwoI{
  \isubstboxT
    {F(g(x))}  {\intx{f(g(x))g'(x)}}
    {\ph{m}}
    {F(u)}     {\intu{f(u)}}
    {Se $F'(u)=f(u)$ então:}
}
\def\StwoI{
  \isubstboxTT
    {F(g(x))}  {\intx{f(g(x))g'(x)}}
    {\ph{m}}
    {F(u)}     {\intu{f(u)}}
    {Se $F'(u)=f(u)$ então:}
    {Obs: $u=g(x)$.}
}

\def\Sthree{
  \iequationbox {\Intx{a}{b}{f(g(x))g'(x)}}
                {\Intu{g(a)}{g(b)}{f(u)}}
}
\def\SthreeI{
  \iequationbox {\intx{f(g(x))g'(x)}}
                {\intu{f(u)}
                 \qquad [u=g(x)]
                }
  % [u=g(x)]
}

\def\Sthree{
  \pmat{
    \D \Intx{a}{b}{f(g(x))g'(x)} \\
    \veq \\
    \D \Intu{g(a)}{g(b)}{f(u)}
  }}

\def\SthreeI{
  \pmat{
    \D \intx{f(g(x))g'(x)} \\
       \veq \\
    \D \intu{f(u)} \\
    \text{Obs: $u=g(x)$.} \\
  }}



\def\Subst#1{\bmat{#1}}

\def\Ps  #1{\left( #1 \right) }
\def\ps  #1{     ( #1       ) }
\def\nops#1{       #1         }
\def\righte{\quad\text{e}}

\def\Rd#1{{\ColorRed{#1}}}
\def\Rdq {{\ColorRed{?}}}




%  _____ _ _   _                               
% |_   _(_) |_| | ___   _ __   __ _  __ _  ___ 
%   | | | | __| |/ _ \ | '_ \ / _` |/ _` |/ _ \
%   | | | | |_| |  __/ | |_) | (_| | (_| |  __/
%   |_| |_|\__|_|\___| | .__/ \__,_|\__, |\___|
%                      |_|          |___/      
%
% «title»  (to ".title")
% (c2m212intsp 1 "title")
% (c2m212intsa   "title")

\thispagestyle{empty}

\begin{center}

\vspace*{1.2cm}

{\bf \Large Cálculo 2 - 2021.2}

\bsk

Aula 22: integração por substituição

\bsk

Eduardo Ochs - RCN/PURO/UFF

\url{http://angg.twu.net/2021.2-C2.html}

\end{center}

\newpage

No mini-teste 3 - link:

\ssk

{\footnotesize

% (c2m212mt3p 3 "questao")
% (c2m212mt3a   "questao")
% (c2m212mt3p 4 "gabarito")
% (c2m212mt3a   "gabarito")
%    http://angg.twu.net/LATEX/2021-2-C2-MT3.pdf#page=4
\url{http://angg.twu.net/LATEX/2021-2-C2-MT3.pdf#page=4}

}

\ssk

vocês viram que quando a função $G$ ``é uma integral da $f$''

nós podemos fazer contas como esta aqui:
%
$$\Intx{2}{5}{f(x)} \;\;=\;\; G(5) - G(2)$$

Isto é um caso particular do TFC2,

que tem várias versões diferentes...

a \ColorRed{fórmula} dele é essa aqui:

$$\Intx{a}{b}{F'(x)} \;\;=\;\; \difx{a}{b}{F(x)}$$

\newpage

% «exercicio-1»  (to ".exercicio-1")
% (c2m212intsp 3 "exercicio-1")
% (c2m212intsa   "exercicio-1")



\scalebox{0.75}{\def\colwidth{12cm}\firstcol{

Neste semestre eu vou tentar 

explicar o TFC2 e as consequências dele ---

tipo: TODAS as técnicas de integração são

consequência do TFC2 --- com uma abordagem

diferente da do semestre passado.

\msk

Dê uma olhada nestes slides do semestre passado:

\ssk

{\footnotesize

% (c2m211tfcsp 2 "exercicio-1")
% (c2m211tfcsa   "exercicio-1")
% (c2m211tfcsp 10 "exercicio-2")
% (c2m211tfcsa    "exercicio-2")
% (c2m211tfcsp 12 "exercicio-3")
% (c2m211tfcsa    "exercicio-3")
%    http://angg.twu.net/LATEX/2021-1-C2-os-dois-TFCs.pdf
\url{http://angg.twu.net/LATEX/2021-1-C2-os-dois-TFCs.pdf}

}

Leia as páginas 2 até 4 dele,

a definição no fim da página 7,

e as páginas 10 até 12.

\msk

{\bf Exercício 1.}

Faça os exercícios 1, 2 e 3 do PDF acima ---

mas ao invés de fazer o 2 como eu pedi no semestre

passado faça esta versão modificada dele:
%
$$[\text{TFC2}] \pmat{F(x) := 2x^2 - \frac{x^3}{3} \\
                      F'(x) := 4x - x^2 \\
                      b:=4 \\
                      a:=0 }
  \;\;=\;\; \ColorRed{?}
$$

%}\anothercol{
}}





\newpage


% «exercicio-2»  (to ".exercicio-2")
% (c2m212intsp 4 "exercicio-2")
% (c2m212intsa   "exercicio-2")


{\bf Exercício 2.}

Assista este vídeo,

\ssk

{\footnotesize

\url{http://angg.twu.net/eev-videos/2021-2-C2-int-subst.mp4}

\url{https://www.youtube.com/watch?v=YbVfNi-xGNw}

}

e depois tente entender cada uma

das igualdades do slide 7.

\bsk

Dica: os `$=$'s do slide 7 têm montes

de significados diferentes dependendo

do contexto. Tente fazer uma lista de

significados e pronúncias.


\bsk

Obs: os próximos 3 slides não são

autocontidos -- você vai precisar

assistir o vídeo pra entendê-los.








\newpage


%       /\      ____  
% __  _|/\|    |___ \ 
% \ \/ /   _____ __) |
%  >  <   |_____/ __/ 
% /_/\_\       |_____|
%                     
% «x^-2»  (to ".x^-2")
% (c2m212intsp 5 "x^-2")
% (c2m212intsa   "x^-2")

{\bf Um caso em que o TFC2 dá um resultado errado}

\ssk

% (setq eepitch-preprocess-regexp "^")
% (setq eepitch-preprocess-regexp "^%T ?")
%
%T  (eepitch-maxima)
%T  (eepitch-kill)
%T  (eepitch-maxima)
%T C : 4;
%T cap(f) ::= min(max(-C, f), C);
%T f    :  x^-2;
%T f(x) := x^-2;
%T F    :  integrate(f, x);
%T F(x) := -1/x;
%T F(1) - F(-1);
%T 
%T integrate(f, x, -1, -0.01) +
%T integrate(f, x, 0.01, 1);
%T 
%T ? solve
%T solve(f-4, x);
%T solve(f+4, x);
%T solve(F-4, x);
%T solve(F+4, x);

% (setq eepitch-preprocess-regexp "^")
% (setq eepitch-preprocess-regexp "^%%?L ?")
%
%%L  (eepitch-lua51)
%%L  (eepitch-kill)
%%L  (eepitch-lua51)
%%L dofile "2021pict2e.lua"           -- (find-LATEX "2021pict2e.lua")
%L
%L f = function (x) return  x^-2 end
%L F = function (x) return -x^-1 end
%L 
%L cut2 = function (f, xl, xr)
%L     local yl = f(xl)
%L     local yr = f(xr)
%L     local g = function (x)
%L         if x <= xl then return f(x) end
%L         if x <= 0  then return yl   end
%L         if x == 0  then return (yl+yr)/2 end
%L         if x <= xr then return yr end
%L         return f(x)
%L       end
%L     return g
%L   end
%L fcut = cut2(f, -1/2, 1/2)
%L Fcut = cut2(F, -1/4, 1/4)
%L 
%%L = f(-1/2), f( 1/2)
%%L = F(-1/4), F( 1/4)
%%L = fcut(-0.6), fcut(-0.5), fcut(-0.4)
%%L = fcut( 0.4), fcut( 0.5), fcut( 0.6)
%%L = Fcut(-0.30), Fcut(-0.25), Fcut(-0.20)
%%L = Fcut( 0.20), Fcut( 0.25), Fcut( 0.30)
%L
%L pwif    = Piecewisify.new(fcut, seq(-4, 4, 1/8))
%L pwifpol = Pict2e.new():add(pwif:pol(-1, 1, "*")):color("orange")
%L pwiF    = Piecewisify.new(Fcut, seq(-4, 4, 1/8))
%L pwiFpol = Pict2e.new():add(pwiF:pol(-4, 4, "*")):color("orange")
%L
%L pwifl   = Piecewisify.new(fcut, seq(-1, -1/2, 1/8))
%L pwifr   = Piecewisify.new(fcut, seq(1/2, 1,   1/8))
%L
%%L = pwifpol
%%L = pwiFpol
%%L = pwifl
%%L = pwifl:topict()
%%L = pwif:pw(-1, -2)
%%L = pwif:piecewise(-1, -1/8)
%%L = pwif:piecewise(-1, -1/8):Line()
%%L = pwif:piecewise(-1, -1/8, nil, "")
%%L = "\\Line"..pwif:piecewise(-1, -1/8, nil, "")
%L
%L Piecewisify.__index.lineify = function (pwi, a, b)
%L     return "\\Line"..pwi:piecewise(a, b, nil, "")
%L   end
%%L = pwif:lineify(-1, -1/8)
%L
%L Pict2e.new()
%L   :setbounds(v(-4,-4), v(4, 4))
%L     :grid()
%L     :axesandticks()
%L     :Thick("1pt")
%L     :add(pwifpol)
%L     :add(pwif:lineify(-4, -1/2))
%L     :add(pwif:lineify(1/2, 4))
%L   :bepc()
%L   :def("Graphf")
%L   :output()
%L
%L Pict2e.new()
%L   :setbounds(v(-4,-4), v(4, 4))
%L     :grid()
%L     :axesandticks()
%L     :Thick("1pt")
%L  -- :add(pwiFpol)
%L     :add(pwiF:lineify(-4, -1/4))
%L     :add(pwiF:lineify(1/4, 4))
%L     :add(pictpiecewise("(-1,1)c (1,-1)c"))
%L   :bepc()
%L   :def("GraphF")
%L   :output()
\pu

\unitlength=10pt

Se $F(x) = -x^{-1}$

então $F'(x) = x^{-2}$, e:

$$\begin{array}{rcl}
  \Intx{-1}{1}{F'(x)} &=& \difx{-1}{1}{F(x)} \\
  \Intx{-1}{1}{x^{-2}}
        & = & \difx{-1}{1}{(- x^{-1})} \\
        & = & (- 1^{-1}) - (- (-1)^{-1}) \\
        & = & -2 \\
  \end{array}
$$

$$\Graphf
  \;\; = \;\;
  \GraphF
  \qquad
  \frown
$$


\newpage

% «1-then-2»  (to ".1-then-2")
% (c2m212intsp 6 "1-then-2")
% (c2m212intsa   "1-then-2")

{\bf Outro caso em que o TFC2 dá um resultado errado}

%L f_P1 = function (x)
%L     if x <= 1 then return 1 end
%L     return 2
%L   end
%L pwi1 = Piecewisify.new(f_P1, 1)
%L
%L Pict2e.new()
%L   :setbounds(v(-1,0), v(3,4))
%L     :grid()
%L     :add("#1")
%L     :axesandticks()
%L   :bepc()
%L   :def("PUm#1")
%L   :output()
%L
%L f_P2 = function (x)
%L     if x <= 1 then return x end
%L     return 2*x
%L   end
%L pwi2 = Piecewisify.new(f_P2, 1)
%L
%L Pict2e.new()
%L   :setbounds(v(-1,0), v(3,4))
%L     :grid()
%L     :add(pictpiecewise("(0,0)c--(1,1) (1,2)--(2,4)c"))
%L     :axesandticks()
%L   :bepc()
%L   :def("PDois#1")
%L   :output()

\pu
\unitlength=10pt
\linethickness{1pt}


$$\Intx{0}{2}{f(x)}
  \;\;=\;\;
  \difx{0}{2}{F(x)}
$$
$$\PUm{\expr{
    pwi1:pol(0, 2, "*"):color("orange")
    + pwi1:pw(-1, 3)
  }}
  \;\; = \;\;
  \PDois{}
$$
$$3 \;\; = \;\; 4-0$$




\newpage

%  ____ ____                           __   _ 
% / ___|___ \   _ __  _ __ ___   ___  / _| / |
% \___ \ __) | | '_ \| '__/ _ \ / _ \| |_  | |
%  ___) / __/  | |_) | | | (_) | (_) |  _| | |
% |____/_____| | .__/|_|  \___/ \___/|_|   |_|
%              |_|                            
%
% «S2-proof-1»  (to ".S2-proof-1")
% (c2m212intsp 7 "S2-proof-1")
% (c2m212intsa   "S2-proof-1")

\def\TfcDois{[\text{TFC2}]}
\def\DefDif {[\text{DefDif}]}
\def\DEFDIFA #1{ \difx{a}{b}{F(x)}  #1 = #1 F(b) - F(a)       }
\def\TFCDOISA#1{ \Intx{a}{b}{F'(x)} #1 = #1 \difx{a}{b}{F(x)} }
\def\TFCP    #1{ \D \left( #1 \right) }

\sa{TFC2-ap1-S}{ F(x)  := f(g(x))       \\
                 F'(x) := f'(g(x))g'(x) \\
               }
\sa{TFC2-ap1-L}{ \Intx{a}{b}{f'(g(x))g'(x)} }
\sa{TFC2-ap1-R}{ \difx{a}{b}{f(g(x))}       }

\sa{TFC2-ap2-S}{ x := u         \\
                 b := g(b)      \\
                 a := g(a)      \\
                 F(u)  := f(u)  \\
                 F'(u) := f'(u) \\
               }
\sa{TFC2-ap2-L}{ \Intu{g(a)}{g(b)}{f'(u)} }
\sa{TFC2-ap2-R}{ \difu{g(a)}{g(b)}{f(u)}       }

\sa{DEFDIF-ap1-S}{ F(x)  := f(g(x))       \\
                 }
\sa{DEFDIF-ap2-S}{ x     := u      \\
                   F(u)  := f(u)   \\
                   a     := g(a)   \\
                   b     := g(b)   \\
                 }

\sa{TFC2-ap1}  { \TFCP{ \ga{TFC2-ap1-L}         \;\;=\;\; \ga{TFC2-ap1-R}   } }
\sa{TFC2-ap2}  { \TFCP{ \ga{TFC2-ap2-L}         \;\;=\;\; \ga{TFC2-ap2-R}   } }
\sa{DEFDIF-ap1}{ \TFCP{ \difx{a}{b}{f(g(x))}    \;\;=\;\; f(g(b)) - f(g(a)) } }
\sa{DEFDIF-ap2}{ \TFCP{ \difu{g(a)}{g(b)}{f(u)} \;\;=\;\; f(g(b)) - f(g(a)) } }

\sa{S2-primeira-versao}{
 \begin{array}{rcl}
   \D \ga{TFC2-ap1-L} &=& \D \ga{TFC2-ap1-R} \\
                      &=& f(g(b)) - f(g(a))  \\[7.5pt]
                      &=& \D \ga{TFC2-ap2-R} \\[7.5pt]
                      &=& \D \ga{TFC2-ap2-L}
 \end{array}}

\vspace*{-0.75cm}

\scalebox{0.65}{\def\colwidth{14cm}\firstcol{

$$\begin{array}{rcl}
 \DefDif  &=& \TFCP{ \DEFDIFA{\;\;}  } \\
 \TfcDois &=& \TFCP{ \TFCDOISA{\;\;} } \\
 [20pt]
 \DefDif \bsm{\ga{DEFDIF-ap1-S}} &=& \ga{DEFDIF-ap1} \\
 \DefDif \bsm{\ga{DEFDIF-ap2-S}} &=& \ga{DEFDIF-ap2} \\
 [20pt]
 \TfcDois \bsm{\ga{TFC2-ap1-S}} &=& \ga{TFC2-ap1} \\
 \TfcDois \bsm{\ga{TFC2-ap2-S}} &=& \ga{TFC2-ap2} \\
 \end{array}
$$

$$\ga{S2-primeira-versao}$$


%}\anothercol{
}}



\newpage

% «dfi»     (to ".dfi")
% (c2m212intsp 8 "dfi")
% (c2m212intsa   "dfi")
% «exercicio-3»  (to ".exercicio-3")
% (c2m212intsp 8      "exercicio-3")
% (c2m212intsa        "exercicio-3")

{\bf A fórmula da derivada da função inversa}

\sa{DFI1}{\left(
    \begin{array}{rcl}
           f(g(x)) &=& x \\
      \ddx f(g(x)) &=& \ddx x \;\;=\;\; 1 \\
      \ddx f(g(x)) &=& f'(g(x))g'(x) \\
      f'(g(x))g'(x) &=& 1 \\
      g'(x) &=& \frac{1}{f'(g(x))} \\
    \end{array}
  \right)}
\sa{DFI2}{\left(
    \begin{array}{rcl}
           f(g(x)) &=& x \\
      g'(x) &=& \frac{1}{f'(g(x))} \\
    \end{array}
  \right)}
\sa{Dfi1}{[\text{DFI1}]}
\sa{Dfi2}{[\text{DFI2}]}


\scalebox{0.65}{\def\colwidth{11cm}\firstcol{

$$\begin{array}{rcc}
  \ga{Dfi1} &=& \ga{DFI1} \\[40pt]
  \ga{Dfi2} &=& \ga{DFI2} \\
  \end{array}
$$

\bsk
\bsk

{\bf Exercício 3.}

\msk

a) $\ga{Dfi1} \bmat{f(y) := e^y \\
                    f'(y) := e^y \\
                    g(x) := \ln x \\
                    g'(x) := \ln' x \\
                   } = \rq$

}\anothercol{

{\bf Exercício 3 (cont.)}

\msk

b) $\def\Sqrt{\text{sqrt}}
    \ga{Dfi2} \bmat{f (y) := y^2 \\
                    f'(y) := 2y \\
                    g (x) := \Sqrt(x) \\
                    g'(x) := \Sqrt'(x) \\
                   } = \rq
   $

\msk

c) $\ga{Dfi2} \bmat{f (y) := \sen y \\
                    f'(y) := \cos y \\
                    g (x) := \arcsen(x) \\
                    g'(x) := \arcsen'(x) \\
                   } = \rq
   $

\msk

d) $\ga{Dfi2} \bmat{x     := s \\
                    f (θ) := \sen θ \\
                    f'(θ) := \cos θ \\
                    g (s) := \arcsen(s) \\
                    g'(s) := \arcsen'(s) \\
                   } = \rq
   $

\msk

e) $\ga{Dfi2} \bmat{x := c\\
                    f (θ) := \cos θ \\
                    f'(θ) := -\sen θ \\
                    g (c) := \cos^{-1}(c) \\
                    g'(c) := (\cos^{-1})'(c) \\
                   } = \rq
   $


}}


\newpage

% «mais-algumas»  (to ".mais-algumas")
% (c2m212intsp 9 "mais-algumas")
% (c2m212intsa   "mais-algumas")

{\bf Mais algumas fórmulas que não valem sempre}

$$\begin{array}{rcl}
  (\cos x)^2 + (\sen x)^2 &=& 1 \\
  %
  [10pt]
  %
  (\sen x)^2 &=& 1 - (\cos x)^2 \\
  \sqrt{(\sen x)^2} &=& \sqrt{1 - (\cos x)^2} \\
         \sen x     &=& \sqrt{1 - (\cos x)^2} \\
  (\cos x)^2 &=& 1 - (\sen x)^2 \\
  %
  [10pt]
  %
  \sqrt{(\cos x)^2} &=& \sqrt{1 - (\sen x)^2} \\
         \cos x     &=& \sqrt{1 - (\sen x)^2} \\
  \end{array}
$$

\newpage

% «exercicio-4»  (to ".exercicio-4")
% (c2m212intsp 10 "exercicio-4")
% (c2m212intsa    "exercicio-4")

{\bf Exercício 4.}

\scalebox{0.8}{\def\colwidth{10cm}\firstcol{

a) Escolha um número entre 42 e 99.

\ColorRed{(Se você não conseguir converse com seus colegas!!!)}

\msk

b) Escolha um $α∈\R$ tal que $\sen α<0$

e verifique se $\sen α = \sqrt{1 - (\cos α)^2}$.

Dica: escolha um $α$ para o qual você sabe $\sen α$ e $\cos α$.

\msk

c) Escolha um $β∈\R$ tal que $\cos β<0$

e verifique se $\cos β = \sqrt{1 - (\cos β)^2}$.

\msk

d) Faça uma cópia do gráfico abaixo num papel
%
%L pi,sin,cos = math.pi, math.sin, math.cos
%L
%L PW = function (s) return
%L     Piecewisify.new(L("t -> "..s), seq(0, 2*pi, pi/32)):pw(0, 2*pi)
%L   end
%L
%L Pict2e.new()
%L   :setbounds(v(0,-2), v(7,2))
%L     :grid()
%L     :add(PW("sin(t)"))
%L     :add(PW("cos(t)"))
%L     :axesandticks()
%L   :bepc()
%L   :def("SinAndCos")
%L   :output()
\pu
%
\unitlength=10pt
%
$$\SinAndCos
$$
%
e desenhe sobre ela os conjuntos:

\ssk

$A \;\;=\;\; \setofst{θ∈[0,2π]}{\sen θ = \sqrt{1 - (\cos θ)^2}}$,

$B \;\;=\;\; \setofst{θ∈[0,2π]}{\cos θ = \sqrt{1 - (\sen θ)^2}}$.

}\anothercol{
}}


\newpage

{\bf Juntando fórmulas estranhas}

$$\begin{array}{rcl}
  f(g(x)) &=& x \\
    g'(x) &=& \frac{1}{f'(g(x))} \\
  e^{\ln x} &=& x \\
  \ln' x &=& \frac{1}{e^{\ln x}} \\[2.5pt]
         &=& \frac{1}{x} \\
  \Intx{a}{b}{\ln' x} &=& \difx{a}{b}{\ln x} \\
  \Intx{a}{b}{\frac{1}{x}} &=& \difx{a}{b}{\ln x} \\
  \end{array}
$$

\newpage

{\bf Juntando fórmulas estranhas}

$$\begin{array}{rcl}
  f(g(x)) &=& x \\
    g'(x) &=& \frac{1}{f'(g(x))} \\
  \sen(\arcsen x) &=& x \\
  \arcsen' x &=& \frac{1}{\cos(\arcsen x)} \\
             &=& \frac{1}{\sqrt{(\cos(\arcsen x))^2}} \\
             &=& \frac{1}{\sqrt{1 - (\sen(\arcsen x))^2}} \\
             &=& \frac{1}{\sqrt{1 - x^2}} \\
  \Intx{a}{b}{\arcsen' x} &=& \difx{a}{b}{\arcsen x} \\
  \Intx{a}{b}{\frac{1}{\sqrt{1-x^2}}} &=& \difx{a}{b}{\arcsen x} \\
  \end{array}
$$





\newpage

% «um-exemplo»  (to ".um-exemplo")
% (c2m212intsp 13 "um-exemplo")
% (c2m212intsa    "um-exemplo")

{\bf Um exemplo de mudança de variável}

\def\P  #1{\left(    #1 \right)}
\def\Pga#1{\left(\ga{#1}\right)}

\sa{Emv1}{[\text{EMV1}]}
\sa{Emv2}{[\text{EMV2}]}
\sa{Emv3}{[\text{EMV3}]}
\sa{Emv4}{[\text{EMV4}]}
\sa{Emv5}{[\text{EMV5}]}
\sa{EMV1}{
 \begin{array}{rcl}
       \D \Intx{  a }{  b }{f'(g(x))g'(x)}
   &=& \D \difx{  a }{  b }{f (g(x))     } \\
   &=& f(g(b))            - f (g(a))       \\[7.5pt]
   &=& \D \difu{g(a)}{g(b)}{f (u)}         \\[7.5pt]
   &=& \D \Intu{g(a)}{g(b)}{f'(u)}
 \end{array}}
\sa{EMV2}{
 \begin{array}{rcl}
       \D \Intx{  a }{  b }{f'(2x)·2}
   &=& \D \difx{  a }{  b }{f (2x)     } \\
   &=& f(2b)              - f (2a)       \\[7.5pt]
   &=& \D \difu{2a}{2b}{f (u)}         \\[7.5pt]
   &=& \D \Intu{2a}{2b}{f'(u)}
 \end{array}}
\sa{EMV3}{
 \begin{array}{rcl}
       \D \Intx{  a }{  b }{\sen(2x)·2}
   &=& \D \difx{  a }{  b }{(-\cos(2x))} \\
   &=& (-\cos(2b))         -(-\cos(2a))  \\[7.5pt]
   &=& \D \difu{2a}{2b}{(-\cos(u))}         \\[7.5pt]
   &=& \D \Intu{2a}{2b}{\sen(u)}
 \end{array}}
\sa{EMV4}{
 \begin{array}{rcl}
       \D \Intu{2a}{2b}{\sen(u)}
   &=& \D \Intx{ a}{ b}{\sen(2x)·2 \,}
 \end{array}}
\sa{EMV5}{
 \begin{array}{rcl}
       \D \Intu{a  }{b  }{\sen(u)}
   &=& \D \Intx{a/2}{b/2}{2\sen(2x)}
 \end{array}}



\msk

\scalebox{0.49}{\def\colwidth{9cm}\firstcol{

$\begin{array}{rcl}
  \ga{Emv1} &=& \Pga{EMV1} \\[55pt]
  \ga{Emv2} \;\;=\;\;
            \ga{Emv1} \bmat{g(x) := 2x \\ g'(x) := 2}
            &=& \Pga{EMV2} \\[55pt]
  \ga{Emv3} \;\;=\;\;
            \ga{Emv2} \bmat{f(x) := -\cos x \\ f'(x) := \sen x}
            &=& \Pga{EMV3} \\[55pt]
  \ga{Emv4} &=& \Pga{EMV4} \\[15pt]
  \ga{Emv5} &=& \Pga{EMV5} \\
  \end{array}
$

%}\anothercol{
}}


\newpage

{\bf Outro exemplo de mudança de variável}

\def\P  #1{\left(    #1 \right)}
\def\Pga#1{\left(\ga{#1}\right)}

\sa{Oemv3}{[\text{OEMV3}]}
\sa{Oemv4}{[\text{OEMV4}]}
\sa{Oemv5}{[\text{OEMV5}]}
\sa{OEMV3}{
 \begin{array}{rcl}
       \D \Intx{  a }{  b }{\tan(2x)·2}
   &=& \D \difx{  a }{  b }{(f(2x))} \\
   &=& (f(2b))         -(f(2a))  \\[7.5pt]
   &=& \D \difu{2a}{2b}{(f(u))}         \\[7.5pt]
   &=& \D \Intu{2a}{2b}{\tan(u)}
 \end{array}}
\sa{OEMV4}{
 \begin{array}{rcl}
       \D \Intu{2a}{2b}{\tan(u)}
   &=& \D \Intx{ a}{ b}{\tan(2x)·2 \,}
 \end{array}}
\sa{OEMV5}{
 \begin{array}{rcl}
       \D \Intu{a  }{b  }{\tan(u)}
   &=& \D \Intx{a/2}{b/2}{2\tan(2x)}
 \end{array}}

\msk

\scalebox{0.7}{\def\colwidth{9cm}\firstcol{

Aqui a gente não substitui a $f$, só a $f'$...

Digamos que $f(x) = \Intt{c}{x}{\tan t}$,

e portanto $f'(x) = \tan x$.

$\begin{array}{rcl}
  %\ga{Emv1} &=& \Pga{EMV1} \\[55pt]
  %\ga{Emv2} \;\;=\;\;
  %          \ga{Emv1} \bmat{g(x) := 2x \\ g'(x) := 2}
  %          &=& \Pga{EMV2} \\[55pt]
  \ga{Oemv3} \;\;=\;\;
            \ga{Emv2} \bmat{f'(x) := \tan x}
            &=& \Pga{OEMV3} \\[55pt]
  \ga{Oemv4} &=& \Pga{OEMV4} \\[15pt]
  \ga{Oemv5} &=& \Pga{OEMV5} \\
  \end{array}
$

%}\anothercol{
}}



% Nós vimos lá atrás que cada igualdade daqui
% era verdade (com as hipóteses certas)...




% (c2m211isp 13 "hipotese")
% (c2m211isa    "hipotese")


\newpage

% (c2m211isp 27 "exercicio-3")
% (c2m211isa    "exercicio-3")
% (c2m211isp 27 "exercicio-4")
% (c2m211isa    "exercicio-4")

{\scriptsize

% (find-books "__analysis/__analysis.el" "martins-martins")
% (find-martinscdipage (+ 10 109) "4.2       Integral")
% (find-martinscditext (+ 10 109) "4.2       Integral")
% (find-martinscdipage (+ 10 165) "6" "Metodos de Integracao")
% (find-martinscditext (+ 10 165) "6" "Metodos de Integracao")
% (find-martinscdipage (+ 10 165) "6.1       Metodo da Substituicao")
% (find-martinscditext (+ 10 165) "6.1       Metodo da Substituicao")
% http://angg.twu.net/2021.1-C2/martins_martins__sec_6.1.pdf
\url{http://angg.twu.net/2021.1-C2/martins_martins__sec_6.1.pdf}

% (find-books "__analysis/__analysis.el" "apex-calculus")
% (find-apexcalculuspage (+ 10 263) "6.1 Substitution")
% (find-apexcalculuspage (+ 10 280)     "Exercises 6.1")
% (find-twusfile "2021.1-C2/")
% http://angg.twu.net/2021.1-C2/APEX_Calculus_Version_4_BW_secs_6.1_6.2.pdf
\url{http://angg.twu.net/2021.1-C2/APEX_Calculus_Version_4_BW_secs_6.1_6.2.pdf}

% (find-books "__analysis/__analysis.el" "thomas")
% (find-thomas11-1page (+  61  368) "5.5 Indefinite integrals and the substituion rule")
% (find-thomas11-1page (+  61  369)     "Example 1")
% (find-thomas11-1page (+  61  370)     "Example 2")
% (find-thomas11-1page (+  61  371)     "Example 3")

}


\newpage

% «exemplo-contas»  (to ".exemplo-contas")
% (c2m211isp 6 "exemplo-contas")
% (c2m211isa   "exemplo-contas")
% (c2m202isp 9 "exemplo-gamb")
% (c2m202isa   "exemplo-gamb")

{\bf Um exemplo com contas}

Isto aqui é um exemplo de como contas com integração

por substituição costumam ser feitas na prática:
%
$$\scalebox{0.95}{$
  \begin{array}{l}
  \D \intx{2 \cos(3x+4)} \\[8pt]
  = \;\; \D \intu {2 (\cos u) · \frac13}
    \\[8pt]
  = \;\; \D \frac23 \intu{\cos u} \\[8pt]
  = \;\; \D \frac23 \sen u \\[8pt]
  = \;\; \D \frac23 \sen (3x+4) \\
  \end{array}
  $}
$$

É necessário indicar em algum lugar que a relação

entre a variável nova e a antiga é esta: $u=3x+4$.

\newpage

% «exemplo-contas-2»  (to ".exemplo-contas-2")
% (c2m211isp 7 "exemplo-contas-2")
% (c2m211isa   "exemplo-contas-2")

{\bf Outro exemplo com contas}
%
\def\S{\sen x}
\def\C{\cos x}
\def\und#1#2{\underbrace{#1}_{#2}}
%
$$\begin{array}[t]{l}
  \D \intx{(\S)^5 (\C)^3} \\
  \D = \;\; \intx{(\S)^5 (\C)^2 (\C)} \\
  \D = \;\; \intx{(\und{\S}{s})^5 \und{(\C)^2}{1-s^2} \und{(\C)}{\frac{ds}{dx}}} \\
  \D = \;\; \ints{s^5 (1-s^2)} \\
  \D = \;\; \ints{s^5 - s^7} \\
  \D = \;\; \frac{s^6}{6} - \frac{s^8}{8} \\
  \D = \;\; \frac{(\S)^6}{6} - \frac{(\S)^8}{8} \\
  \end{array}
  \qquad
  \begin{array}[t]{c}
  \\ \\
    \bmat{s = \sen x \\
          \frac{ds}{dx} = \cos x \\
          \sen x = s \\
          (\cos x)^2 = 1 - s^2 \\
          \cos x \, dx = ds
    }
  \end{array}
$$

\newpage

% «subst-int-def»  (to ".subst-int-def")
% (c2m211isp 8 "subst-int-def")
% (c2m211isa   "subst-int-def")

{\bf Substituição na integral definida}

Eu vou chamar a \ColorRed{demonstração} abaixo de \pfo{S2}.

Ela é uma série de três igualdades: o `$=$' de cima,

o `$=$' de baixo, e o `$=$' da esquerda (que é um `$\,\rotl{=}$').

Eu vou chamar o ``$F'(u)=f(u)$'' de a \ColorRed{hipótese} do \pfo{S2}.

Obs: nós \ColorRed{ainda} não acreditamos nessa demonstração...

vamos verificar as igualdades dela daqui a alguns slides.
%
% (c2m202isp 3 "def-S2-S2I")
% (c2m202isa   "def-S2-S2I")
%
$$\begin{array}{rcc}
 \pfo{S2} &=& \Stwo \\
 % \\
 % \pfo{S2I} &=& \StwoI \\
 \end{array}
$$


\newpage

% «so-alguns-simbolos»  (to ".so-alguns-simbolos")
% (c2m211isp 9 "so-alguns-simbolos")
% (c2m211isa   "so-alguns-simbolos")

Lembre que dá pra substituir só alguns símbolos...

Por exemplo:
%
\def\Stwotmp{
  \isubstboxT
    {\Difmx{a}{b}{F(2x)}}   {\Intx{a}{b}{f(2x)·2}}
    {\ph{mmm}}
    {\Difmu{2a}{2b}{F(u)}}  {\Intu{2a}{2b}{f(u)}}
    {Se $F'(u)=f(u)$ então:}
}
%
$$\scalebox{0.9}{$
  \begin{array}{c}
 \pfo{S2} \;\;=\;\; \Stwo \\
 [50pt]
 \pfo{S2}[g(x):=2x] \;\;=\;\; \Stwotmp \\
 \end{array}
  $}
$$

\newpage

% «hip-triv-true»  (to ".hip-triv-true")
% (c2m211isp 10 "hip-triv-true")
% (c2m211isa    "hip-triv-true")

Também podemos substituir o $f$ por $F'$...

E aí a hipótese passa a ser ``trivialmente verdadeira'':
%
\def\Stwotmp{
  \isubstboxT
    {\Difmx{a}{b}{F(g(x))}}   {\Intx{a}{b}{F'(g(x))g'(x)}}
    {\ph{mmm}}
    {\Difmu{g(a)}{g(b)}{F(u)}}  {\Intu{g(a)}{g(b)}{F'(u)}}
    {Se $F'(u)=F'(u)$ então:}
}
%
$$\scalebox{0.9}{$
  \begin{array}{c}
  \pfo{S2} \;\;=\;\; \Stwo \\
  [50pt]
  \pfo{S2}[f(u):=F'(u)] \;\;=\;\; \Stwotmp \\
  \end{array}
  $}
$$

\newpage

% «exercicio-1»  (to ".exercicio-1")
% (c2m211isp 11 "exercicio-1")
% (c2m211isa    "exercicio-1")

{\bf Exercício 1.}

Lembre que:
%
$$\pfo{TFC2}
  \;\;=\;\;
  \Ps{
       \D \Intx{a}{b}{\ddx F(x)} \;\;=\;\; \difx{a}{b}{F(x)}
     }
$$

\msk

Calcule os resultados destas expansões:

a) $\pfo{TFC2} \bmat{F(x):=F(g(x))}$

b) $\pfo{TFC2} \bmat{x:=u} \bmat{a:=g(a) \\ b:=g(b)}$

\bsk
\bsk

...e verifique que \ColorRed{se $f(u)=F'(u)$ então}:

c) o que você obteve no (a) prova o `$=$' de cima da \pfo{S2},

d) o que você obteve no (b) prova o `$=$' de baixo da \pfo{S2},


\newpage

% «esquerda»  (to ".esquerda")
% (c2m211isp 12 "esquerda")
% (c2m211isa    "esquerda")

O `$\,\rotl{=}$' à esquerda na \pfo{S2}

é bem fácil de verificar... ó:

$$\begin{array}{rcl}
  \difx{a}{b}{F(g(x))} &=& F(g(b)) - F(g(a)) \\
                       &=& \difu{g(a)}{g(b)}{F(u)}
  \end{array}
$$

\bsk
\bsk

Se você conseguiu fazer todos os itens

do exercício 1 e conseguiu entender isso aí

então \ColorRed{agora} você entende o $\pfo{S2}$ como uma

demonstração --- você entende todas as

igualdades dele.


\newpage

% «hipotese»  (to ".hipotese")
% (c2m211isp 13 "hipotese")
% (c2m211isa    "hipotese")

{\bf Pra que serve a hipótese do \pfo{S2}?}

Ela serve pra gente lidar com `$f$'s que a gente

não sabe integrar! Por exemplo:
%
\def\Stwotmp{
  \isubstboxT
    {\Difmx{a}{b}{F(g(x))}}   {\Intx{a}{b}{\tan(g(x))\tan'(x)}}
    {\ph{mmm}}
    {\Difmu{g(a)}{g(b)}{F(u)}}  {\Intu{g(a)}{g(b)}{\tan(u)}}
    {Se $F'(u)=F'(u)$ então:}
}
%
\def\Stwotmp{
  \isubstboxT
    {\Difmx{a}{b}{F(2x)}}   {\Intx{a}{b}{\tan(2x)·2}}
    {\ph{mmm}}
    {\Difmu{g(a)}{g(b)}{F(u)}}  {\Intu{2a}{2b}{\tan(u)}}
    {\ColorRed{Se $F'(u)=\tan u$ então:}}
}
%
$$\scalebox{0.90}{$
  \begin{array}{c}
  \pfo{S2} \;\;=\;\; \Stwo \\
  [50pt]
  % \pfo{S2}[f(x):=\tan x] \;\;=\;\; \Stwotmp \\
    \pfo{S2}\bmat{f(x):=\tan x \\ g(u):=2u} \;\;=\;\; \Stwotmp \\
  \end{array}
  $}
$$

\newpage

{\bf Uma versão do \pfo{S2} para integrais indefinidas}

Compare... e repare no ``\ColorRed{Obs: $u = g(x)$}''.
%
\def\StwoItmp{
  \isubstboxTT
    {F(g(x))}  {\intx{f(g(x))g'(x)}}
    {\ph{m}}
    {F(u)}     {\intu{f(u)}}
    {Se $F'(u)=f(u)$ então:}
    {\ColorRed{Obs: $u=g(x)$.}}
}
%
$$\scalebox{0.90}{$
  \begin{array}{c}
  \pfo{S2} \;\;=\;\; \Stwo \\
  [50pt]
  \pfo{S2I} \;\;=\;\; \StwoItmp \\
  \end{array}
  $}
$$

\newpage

{\bf Versões sem a parte da esquerda}

Compare:
%
$$\scalebox{0.90}{$
  \begin{array}{c}
  \pfo{S2} \;\;=\;\; \Stwo \\
  [50pt]
  \pfo{S3} \;\;=\;\; \Sthree \\
  \end{array}
  $}
$$

\newpage

{\bf Versões sem a parte da esquerda (2)}

...e compare:
%
$$\scalebox{0.90}{$
  \begin{array}{c}
  \pfo{S2I} \;\;=\;\; \StwoI \\
  [50pt]
  \pfo{S3I} \;\;=\;\; \SthreeI \\
  \end{array}
  $}
$$

\newpage

% «encontre-a-subst»  (to ".encontre-a-subst")
% (c2m211isp 17 "encontre-a-subst")
% (c2m211isa    "encontre-a-subst")

As pessoas costumam usar variações da $\pfo{S3I}$,

geralmente sem darem um nome pra função $g(u)$...


Lembre que em vários exercícios que nós já fizemos

ficava implícito que vocês tinham que descobrir qual

era a substituição certa... por exemplo:
%
$$\begin{array}{rcl}
  \difx{4}{5}{x^2} &=& \Rdq \\[5pt]
  \Ps{\difx{a}{b}{f(x)} = f(b)-f(a)} \bmat{f(x):=\Rdq \\ a:=\Rdq \\ b:=\Rdq} &=& \Rdq
  \\[20pt]
  \Ps{\difx{a}{b}{f(x)} = f(b)-f(a)} \bmat{f(x):=x^2  \\ a:=4    \\ b:=5} &=&
  \Ps{\difx{4}{5}{x^2} = 5^2-4^2} \\
  [20pt]
  \difx{4}{5}{x^2} &=& 5^2 - 4^2 \\
  \end{array}
$$


\newpage

% «exercicio-2»  (to ".exercicio-2")
% (c2m211isp 18 "exercicio-2")
% (c2m211isa    "exercicio-2")

{\bf Exercício 2.}

Nos livros e nas notas de aula que você vai encontrar por aí

o ``\ColorRed{Obs: $u = g(x)$}'' da nossa \pfo{S3I} quase sempre aparece escrito

de (ZILHÕES DE!!!) outros jeitos, então o melhor que a gente

pode fazer é tentar encontrar as substituições que transformam

a nossa \pfo{S3I} em algo ``mais ou menos equivalente'' às

igualdades complicadas que eu mostrei no vídeo e que eu disse

que a gente iria tentar decifrar...

\msk

Nos itens a e b deste exercício você vai tentar encontrar

as substituições --- que eu vou escrever como `$[\Rdq]$' --- que

transformam a $\pfo{S3I}$ em algo ``mais ou menos equivalente''

às igualdades da direita.

\newpage

% «exercicio-2-cont»  (to ".exercicio-2-cont")
% (c2m211isp 19 "exercicio-2-cont")
% (c2m211isa    "exercicio-2-cont")

{\bf Exercício 2 (cont.)}

Encontre as substituições `$[\Rdq]$'s que façam com que:

\bsk

a) $\SthreeI [\Rdq]$ vire algo como
   $\pmat{ \D \intx{2 \cos(3x+4)} \\
           \rotl{=} \\
           \D \intu {2 (\cos u) · \frac13} \\
         }$

\msk

b) $\pfo{S3I} \, [\Rdq]$ vire algo como
   $\pmat{ \D \intx{(\S)^5 (1 - \S^2) (\C)} \\
           \rotl{=} \\
           \D \ints{s^5 (1-s^2)} \\
         }$


\newpage

{\bf Gambiarras}

Em geral é mais prático a gente usar umas gambiarras

como ``$\frac{du}{dx}dx = du$'' ao invés do método ``mais honesto''

que a gente usou no exercício 2...

\msk

Às vezes essas gambiarras vão usar uma versão disfarçada

do teorema da derivada da função inversa: $\frac{du}{dx} = \frac{1}{\frac{dx}{du}}$,

e umas outras manipulações esquisitas de `$dx$'s e `$du$'s

que só aparecem explicadas direito nos capítulos sobre

``diferenciais'' dos livros de Cálculo.

\msk

Nós vamos começar usando elas como gambiarras mesmo,

e acho que nesse semestre não vai dar pra ver como

traduzir cada uma delas pra algo formal...

\newpage

% «gambiarras-2»  (to ".gambiarras-2")
% (c2m211isp 21 "gambiarras-2")
% (c2m211isa    "gambiarras-2")

{\bf Gambiarras (2)}

Quando a gente está começando e ainda não tem prática

este modo de por anotações embaixo de chaves ajuda muito:
%
$$%\begin{array}{c}
  \D \int  (\und{\S}{s})^5
           (1 - (\und{\S}{s})^2)
           \und{
           \und{(\C)}{\frac{ds}{dx}} \, dx
           }{ds}
            \\
  % \rotl{=} \\
  \;\; = \;\;
  \D \ints{s^5 (1-s^2)} \\
  % \end{array}
$$

Quando a gente já tem mais prática acaba sendo melhor

pôr todas as anotações dentro de caixinhas --- por exemplo:

$$\bmat{
  \sen x = s \\
  \frac{ds}{dx} = \frac{d}{dx} \sen x = \cos x \\
  \cos x \, dx = ds \\
  }
$$


\newpage

% «gambiarras-3»  (to ".gambiarras-3")
% (c2m211isp 22 "gambiarras-3")
% (c2m211isa    "gambiarras-3")

{\bf Gambiarras (3)}

Essas caixinhas, como
%
$$\bmat{
  \sen x = s \\
  \frac{ds}{dx} = \frac{d}{dx} \sen x = \cos x \\
  \cos x \, dx = ds \\
  }
$$

vão ser os únicos lugares em que nós vamos permitir

esses `$dx$'s e `$ds$' ``soltos'', que não estão nem em

derivadas e nem associados a um sinal `$∫$'...

\msk

E esses `$dx$'s e `$ds$' ``soltos'' só vão aparecer em linhas

que dizem como traduzir uma expressão que termina em `$dx$'

numa integral em $x$ pra uma expressão que termina em `$ds$'

numa integral na \ColorRed{variável} $s$.

\msk

Nós vamos \ColorRed{evitar} usar $s$ como uma \ColorRed{abreviação} para $\sen x$.


\newpage

{\bf Mais sobre as caixinhas de anotações}

Tudo numa caixinha de anotações é \ColorRed{consequência}

da primeira linha dela, que é a que define a variável

nova. Por exemplo, se definimos a variável nova como

$c=\cos x$ então $\frac{dc}{dx} = \frac{d}{dx} \cos x = - \sen x$, e podemos

reescrever isso na ``versão gambiarra'' como:

$dc = - \sen x \, dx$, \ColorRed{e também como} $\sen x \, dx = (-1) dc$.

\msk

A caixinha vai ser:
%
$$\bmat{c = \cos x \\
        \frac{dc}{dx} = \frac{d}{dx} \cos x = - \sen x \\
        dc = - \sen x \, dx \\
        \sen x \, dx = (-1) \, dc \\
       }
$$

\newpage

{\bf Mais sobre as caixinhas de anotações (2)}

\ColorRed{Muito importante:} cada linha das caixinhas

é uma série de igualdades --- por exemplo

$𝐬{expr}_1 = 𝐬{expr}_2 = 𝐬{expr}_3$ --- e cada uma dessas

expressões $𝐬{expr}_1, \ldots, 𝐬{expr}_n$ só pode mencionar

\ColorRed{ou} a variável antiga \ColorRed{ou} a variável nova...

\msk

Então:

\msk


\ColorRed{Bom:} $dc = - \sen x \, dx$

\ColorRed{Mau:} $\frac{1}{- \sen x} dc =  dx$

\ColorRed{Bom:} $\frac{dc}{dx} = \frac{d}{dx} \cos x$

\bsk

Truque: em $\frac{dc}{dx}$ o $c$ faz o papel de uma \ColorRed{abreviação}

para $\cos x$, não de uma variável.


\newpage

{\bf Mais sobre as caixinhas de anotações (3)}

Quando a gente faz algo como
%
$$\D \int  (\und{\S}{s})^5
           (1 - (\und{\S}{s})^2)
           \und{
           \und{(\C)}{\frac{ds}{dx}} \, dx
           }{ds}
            \\
  \;\; = \;\;
  \D \ints{s^5 (1-s^2)} \\
$$

Cada chave é como uma igualdade da caixa de anotações

``escrita na vertical''... por exemplo, ``$\und{\S}{s}$'' é $s = \sen x$.

\msk

As outras chaves correspondem a outras igualdades da

caixa de anotações --- \ColorRed{que têm que ser consequências

desse $s = \sen x$.}


\newpage

\vspace*{-0.5cm}

{\bf Mais sobre as caixinhas de anotações (3)}

Isto aqui está errado:
%
$$\D \int %(\und{\S}{s})^5
           (     \S    )^5
           (1 - (\und{\S}{s})^2)
           \und{
           \und{(\C)}{\frac{ds}{dx}} \, dx
           }{ds}
            \\
  \;\; = \;\;
  \D \ints{(\ColorRed{\S})^5 (1-s^2)} \\
$$

À esquerda do `$=$' a gente tem uma integral na qual

só aparece a ``variável antiga'', que é $x$, e à direita do `$=$'

a gente tem uma integral na qual aparecem tanto a variável

antiga, $x$, quanto a nova, que é $s$... \quad \frown

\msk

Lembre que tanto o truque das caixinhas quanto o truque das

chaves servem pra gente conseguir aplicar a $\pfo{S3I}$ de um jeito

mais fácil, e no $\pfo{S3I}$ uma integral usa só a variável antiga

e a outra usa só a nova.








\newpage

% «exercicio-3»  (to ".exercicio-3")
% (c2m211isp 27 "exercicio-3")
% (c2m211isa    "exercicio-3")

{\bf Exercício 3.}

Leia o início da seção 6.1 do APEX Calculus

e faça os exercíos 25 até 32 da página 280 dele. Link:

\ssk

{\scriptsize

% (find-books "__analysis/__analysis.el" "apex-calculus")
% (find-apexcalculuspage (+ 10 263) "6.1 Substitution")
% (find-apexcalculuspage (+ 10 280)     "Exercises 6.1")
% (find-twusfile "2021.1-C2/")
% http://angg.twu.net/2021.1-C2/APEX_Calculus_Version_4_BW_secs_6.1_6.2.pdf
\url{http://angg.twu.net/2021.1-C2/APEX_Calculus_Version_4_BW_secs_6.1_6.2.pdf}

}

\bsk
\bsk

% «exercicio-4»  (to ".exercicio-4")
% (c2m211isp 27 "exercicio-4")
% (c2m211isa    "exercicio-4")

{\bf Exercício 4.}

Leia o início da seção 6.1 do Martins/Martins

e refaça os exercícios resolvidos 1 a 6 dele

usando ou as nossas anotações sob chaves ou

as nossas anotações em caixinhas. Link:

\ssk

{\scriptsize

% (find-books "__analysis/__analysis.el" "martins-martins")
% (find-martinscdipage (+ 10 109) "4.2       Integral")
% (find-martinscditext (+ 10 109) "4.2       Integral")
% (find-martinscdipage (+ 10 165) "6" "Metodos de Integracao")
% (find-martinscditext (+ 10 165) "6" "Metodos de Integracao")
% (find-martinscdipage (+ 10 165) "6.1       Metodo da Substituicao")
% (find-martinscditext (+ 10 165) "6.1       Metodo da Substituicao")
% http://angg.twu.net/2021.1-C2/martins_martins__sec_6.1.pdf
\url{http://angg.twu.net/2021.1-C2/martins_martins__sec_6.1.pdf}

}


\newpage

% «exercicio-5»  (to ".exercicio-5")
% (c2m211isp 28 "exercicio-5")
% (c2m211isa    "exercicio-5")

{\bf Exercício 5.}

\msk

A questão 2 da P1 do semestre passado dizia que:
%
\begin{quote}
{\sl Toda integral que pode ser resolvida por uma sequência de
  mudanças de variável (ou: ``por uma sequência de integrações por
  substituição'') pode ser resolvida por uma mudança de variável só.}
\end{quote}

E ela pedia pra vocês verificarem isso num caso específico.

Tente fazer essa questão olhando poucas vezes pro gabarito dela.

Link:

\ssk

{\footnotesize

% (c2m202p1p 4)
%    http://angg.twu.net/LATEX/2020-2-C2-P1.pdf#page=4
\url{http://angg.twu.net/LATEX/2020-2-C2-P1.pdf#page=4}

}


% (c2m202p1p 4 "questao-2")
% (c2m202p1a   "questao-2")


\newpage


\sa{x}{xx}
\sa{u}{uu}
\sa{gx}{g(xx)}
\sa{nw}{F(g(x))}
\sa{ne}{f(g(x))g'(x)}
\sa{sw}{F(u)}
\sa{se}{f(u)}

\def\StwoIsetargs#1{\StwoIsetargsss#1}
\def\StwoIsetargsss#1#2#3#4#5#6#7{
  \sa{x}{#1} \sa{u}{#2} \sa{gx}{#3}
  \sa{nw}{#4} \sa{ne}{#5}
  \sa{sw}{#6} \sa{se}{#7}
  }

% (c2m202p1p 9 "gabarito-2")
% (c2m202p1a   "gabarito-2")

\StwoIsetargsss {xx} {uu} {gguu} {NW} {NE} {SW} {SE}
\StwoIsetargsss
    {v} {w} {\sqrt{v}}
    {F(\sqrt{v})} {\cos(2+\sqrt{v})·(2\sqrt{v})^{-1}}
    {F(w)}        {\cos(2+w)}

\def\StwoItmp{
  \isubstboxTT
    {\ga{nw}}  {\int \ga{ne} \, d\ga{x}}
    {\ph{m}}
    {\ga{sw}}  {\int \ga{se} \, d\ga{u}}
    {Se $F'(\ga{u})=\ga{se}$ então:}
    {Obs: $\ga{u}=\ga{gx}$.}
}
%
$$\scalebox{0.9}{$
  \begin{array}{c}
  \pfo{S2} \;\;=\;\; \Stwo \\
  [50pt]
  \pfo{S2}[f(u):=F'(u)] \;\;=\;\; \StwoItmp \\
  \end{array}
  $}
$$




%\printbibliography

\GenericWarning{Success:}{Success!!!}  % Used by `M-x cv'

\end{document}

%  ____  _             _         
% |  _ \(_)_   ___   _(_)_______ 
% | | | | \ \ / / | | | |_  / _ \
% | |_| | |\ V /| |_| | |/ /  __/
% |____// | \_/  \__,_|_/___\___|
%     |__/                       
%
% «djvuize»  (to ".djvuize")
% (find-LATEXgrep "grep --color -nH --null -e djvuize 2020-1*.tex")

 (eepitch-shell)
 (eepitch-kill)
 (eepitch-shell)
# (find-fline "~/2021.2-C2/")
# (find-fline "~/LATEX/2021-2-C2/")
# (find-fline "~/bin/djvuize")

cd /tmp/
for i in *.jpg; do echo f $(basename $i .jpg); done

f () { rm -v $1.pdf;  textcleaner -f 50 -o  5 $1.jpg $1.png; djvuize $1.pdf; xpdf $1.pdf }
f () { rm -v $1.pdf;  textcleaner -f 50 -o 10 $1.jpg $1.png; djvuize $1.pdf; xpdf $1.pdf }
f () { rm -v $1.pdf;  textcleaner -f 50 -o 20 $1.jpg $1.png; djvuize $1.pdf; xpdf $1.pdf }

f () { rm -fv $1.png $1.pdf; djvuize $1.pdf }
f () { rm -fv $1.png $1.pdf; djvuize WHITEBOARDOPTS="-m 1.0 -f 15" $1.pdf; xpdf $1.pdf }
f () { rm -fv $1.png $1.pdf; djvuize WHITEBOARDOPTS="-m 1.0 -f 30" $1.pdf; xpdf $1.pdf }
f () { rm -fv $1.png $1.pdf; djvuize WHITEBOARDOPTS="-m 1.0 -f 45" $1.pdf; xpdf $1.pdf }
f () { rm -fv $1.png $1.pdf; djvuize WHITEBOARDOPTS="-m 0.5" $1.pdf; xpdf $1.pdf }
f () { rm -fv $1.png $1.pdf; djvuize WHITEBOARDOPTS="-m 0.25" $1.pdf; xpdf $1.pdf }
f () { cp -fv $1.png $1.pdf       ~/2021.2-C2/
       cp -fv        $1.pdf ~/LATEX/2021-2-C2/
       cat <<%%%
% (find-latexscan-links "C2" "$1")
%%%
}

f 20201213_area_em_funcao_de_theta
f 20201213_area_em_funcao_de_x
f 20201213_area_fatias_pizza



%  __  __       _        
% |  \/  | __ _| | _____ 
% | |\/| |/ _` | |/ / _ \
% | |  | | (_| |   <  __/
% |_|  |_|\__,_|_|\_\___|
%                        
% <make>

 (eepitch-shell)
 (eepitch-kill)
 (eepitch-shell)
# (find-LATEXfile "2019planar-has-1.mk")
make -f 2019.mk STEM=2021-2-C2-int-subst veryclean
make -f 2019.mk STEM=2021-2-C2-int-subst pdf

% Local Variables:
% coding: utf-8-unix
% ee-tla: "c2ints"
% ee-tla: "c2m212ints"
% End:
