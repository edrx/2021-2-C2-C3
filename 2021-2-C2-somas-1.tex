% (find-LATEX "2021-2-C2-somas-1.tex")
% (defun c () (interactive) (find-LATEXsh "lualatex -record 2021-2-C2-somas-1.tex" :end))
% (defun C () (interactive) (find-LATEXsh "lualatex 2021-2-C2-somas-1.tex" "Success!!!"))
% (defun D () (interactive) (find-pdf-page      "~/LATEX/2021-2-C2-somas-1.pdf"))
% (defun d () (interactive) (find-pdftools-page "~/LATEX/2021-2-C2-somas-1.pdf"))
% (defun e () (interactive) (find-LATEX "2021-2-C2-somas-1.tex"))
% (defun o () (interactive) (find-LATEX "2021-1-C2-somas-1.tex"))
% (defun oo () (interactive) (find-LATEX "2020-2-C2-somas-1.tex"))
% (defun u () (interactive) (find-latex-upload-links "2021-2-C2-somas-1"))
% (defun v () (interactive) (find-2a '(e) '(d)))
% (defun d0 () (interactive) (find-ebuffer "2021-2-C2-somas-1.pdf"))
% (defun cv () (interactive) (C) (ee-kill-this-buffer) (v) (g))
%          (code-eec-LATEX "2021-2-C2-somas-1")
% (find-pdf-page   "~/LATEX/2021-2-C2-somas-1.pdf")
% (find-sh0 "cp -v  ~/LATEX/2021-2-C2-somas-1.pdf /tmp/")
% (find-sh0 "cp -v  ~/LATEX/2021-2-C2-somas-1.pdf /tmp/pen/")
%     (find-xournalpp "/tmp/2021-2-C2-somas-1.pdf")
%   file:///home/edrx/LATEX/2021-2-C2-somas-1.pdf
%               file:///tmp/2021-2-C2-somas-1.pdf
%           file:///tmp/pen/2021-2-C2-somas-1.pdf
% http://angg.twu.net/LATEX/2021-2-C2-somas-1.pdf
% (find-LATEX "2019.mk")
% (find-CN-aula-links "2021-2-C2-somas-1" "2" "c2m212somas1" "c2so")

% «.videos-antigos»		(to "videos-antigos")
%
% «.defs»			(to "defs")
% «.title»			(to "title")
% «.links-para-videos»		(to "links-para-videos")
%
% «.area-trapezio»		(to "area-trapezio")
% «.jeito-esperto»		(to "jeito-esperto")
% «.particoes»			(to "particoes")
% «.exercicio-4»		(to "exercicio-4")
% «.exercicio-5»		(to "exercicio-5")
%
% «.trapezios»			(to "trapezios")
% «.trap-e-rect»		(to "trap-e-rect")
% «.exercicio-9»		(to "exercicio-9")
% «.exercicio-11»		(to "exercicio-11")
% «.exercicio-12»		(to "exercicio-12")
% «.metodos-nomes»		(to "metodos-nomes")
%
% «.djvuize»			(to "djvuize")


% «videos-antigos»  (to ".videos-antigos")
% (c2m211somas1a "videos")
% (c2m202somas1a "video")
%
% Video (not yet):
% (find-ssr-links      "c2m212somas1" "2021-2-C2-somas-1" "{naoexiste}")
% (code-eevvideo       "c2m212somas1" "2021-2-C2-somas-1")
% (code-eevvideo-local "c2m212somas1" "2021-2-C2-somas-1")
% (find-c2m212somas1video "0:00")


\documentclass[oneside,12pt]{article}
\usepackage[colorlinks,citecolor=DarkRed,urlcolor=DarkRed]{hyperref} % (find-es "tex" "hyperref")
\usepackage{amsmath}
\usepackage{amsfonts}
\usepackage{amssymb}
\usepackage{pict2e}
\usepackage[x11names,svgnames]{xcolor} % (find-es "tex" "xcolor")
\usepackage{colorweb}                  % (find-es "tex" "colorweb")
%\usepackage{tikz}
%
% (find-dn6 "preamble6.lua" "preamble0")
%\usepackage{proof}   % For derivation trees ("%:" lines)
%\input diagxy        % For 2D diagrams ("%D" lines)
%\xyoption{curve}     % For the ".curve=" feature in 2D diagrams
%
\usepackage{edrx21}               % (find-LATEX "edrx21.sty")
\input edrxaccents.tex            % (find-LATEX "edrxaccents.tex")
\input edrx21chars.tex            % (find-LATEX "edrx21chars.tex")
\input edrxheadfoot.tex           % (find-LATEX "edrxheadfoot.tex")
\input edrxgac2.tex               % (find-LATEX "edrxgac2.tex")
%
%\usepackage[backend=biber,
%   style=alphabetic]{biblatex}            % (find-es "tex" "biber")
%\addbibresource{catsem-slides.bib}        % (find-LATEX "catsem-slides.bib")
%
% (find-es "tex" "geometry")
\usepackage[a6paper, landscape,
            top=1.5cm, bottom=.25cm, left=1cm, right=1cm, includefoot
           ]{geometry}
%
\begin{document}

\catcode`\^^J=10
\directlua{dofile "dednat6load.lua"}  % (find-LATEX "dednat6load.lua")

%L dofile "edrxtikz.lua"  -- (find-LATEX "edrxtikz.lua")
%L dofile "edrxpict.lua"  -- (find-LATEX "edrxpict.lua")
\pu

% «defs»  (to ".defs")
% (find-LATEX "edrx21defs.tex" "colors")
% (find-LATEX "edrx21.sty")

\def\drafturl{http://angg.twu.net/LATEX/2021-2-C2.pdf}
\def\drafturl{http://angg.twu.net/2021.2-C2.html}
\def\draftfooter{\tiny \href{\drafturl}{\jobname{}} \ColorBrown{\shorttoday{} \hours}}



%  _____ _ _   _                               
% |_   _(_) |_| | ___   _ __   __ _  __ _  ___ 
%   | | | | __| |/ _ \ | '_ \ / _` |/ _` |/ _ \
%   | | | | |_| |  __/ | |_) | (_| | (_| |  __/
%   |_| |_|\__|_|\___| | .__/ \__,_|\__, |\___|
%                      |_|          |___/      
%
% «title»  (to ".title")
% (c2m212somas1p 1 "title")
% (c2m212somas1a   "title")

\thispagestyle{empty}

\begin{center}

\vspace*{1.2cm}

{\bf \Large Cálculo 2 - 2021.2}

\bsk

Aula 4: integrais como somas de retângulos (1)

\bsk

Eduardo Ochs - RCN/PURO/UFF

\url{http://angg.twu.net/2021.2-C2.html}

\end{center}

\newpage

% «links-para-videos»  (to ".links-para-videos")

{\bf Links pra vídeos antigos}

\ssk

Vamos usar estes vídeos antigos aqui:

% (c2m202somas1a "video")
%
este de 2020.2,

\ssk

{\footnotesize

\url{http://angg.twu.net/eev-videos/2020.2-C2-somas-1.mp4}

\url{https://www.youtube.com/watch?v=bbZfQmtFCSw}

}

\msk

% (c2m211somas1a "videos")
%
e este de 2021.1, sobre o exercício 9:

\ssk

{\footnotesize

\url{http://angg.twu.net/eev-videos/2021-1-C2-somas-1.mp4}

\url{https://www.youtube.com/watch?v=Ht5iLKGlysM}

}

\msk

% (c2m211somas1da "video-1")

Este sobre não tou entendendo nada

também vai ser bem importante:

\ssk

{\footnotesize

\url{http://angg.twu.net/eev-videos/2021-1-C2-somas-1-dicas.mp4}

\url{https://www.youtube.com/watch?v=pCD1p9FZYdI}

}







\newpage

% «fig-hernandez-1»  (to ".fig-hernandez-1")
% (c2sop 4 "fig-hernandez-1")
% (c2soa   "fig-hernandez-1")

{\bf Algumas figuras}

Dê uma olhada nas notas de aula da Cristiane Hernández, linkadas

na página do curso... ela usa várias figuras como essa aqui:

% (find-c2crishernandezpage (+ 10  2) "")
% (find-latexgimp-links "2020-1-C2/area-hernandez-1")

\includegraphics[width=8cm]{2020-1-C2/area-hernandez-1.png}


\newpage

% «area-trapezio»  (to ".area-trapezio")
% «shoelace»  (to ".shoelace")
% (c2m212somas1p 5 "shoelace")
% (c2m212somas1a   "shoelace")

{\bf Algumas áreas fáceis de calcular}

Por enquanto a gente sabe calcular a área de algumas figuras simples:

retângulos, triângulos, trapézios, e figuras formadas por vários

retângulos, triângulos e trapézios.

% (find-es "mathologer" "shoelace")

\msk

Algumas pessoas viram no Ensino Médio um método de calcular

a área de qualquer polígono. Vamos rever isto agora.

Assista os primeiros 10 minutos deste vídeo do Mathologer:

% http://www.youtube.com/watch?v=0KjG8Pg6LGk
\url{http://www.youtube.com/watch?v=0KjG8Pg6LGk}

\msk

Algumas idéias dele vão ser muito importantes pra gente depois.

Por exemplo, que áreas podem ser calculadas de vários jeitos,

e que alguns pedaços devem ser contados ``negativamente''.

\msk

Na verdade nós vamos usar principalmente retângulos e

trapézios...


% Nós em geral vamos usar métodos mais fáceis do que o desse
% vídeo

\newpage

% «areas-de-trapezios»  (to ".areas-de-trapezios")
% (c2sop 5 "areas-de-trapezios")
% (c2soa   "areas-de-trapezios")

{\bf Áreas de trapézios}

O truque pra calcular a área de um trapézio é transformá-lo

num retângulo com a mesma área que ele por cortar-e-colar. Veja:

$
% (find-latexscan-links "C2" "20210205_trapezios_1")
% (find-xpdf-page "~/LATEX/2020-2-C2/20210205_trapezios_1.pdf")
\myvcenter{
\includegraphics[height=3.5cm]{2020-2-C2/20210205_trapezios_1.pdf}
}
%
\quad
% (find-latexscan-links "C2" "20210205_trapezios_3")
% (find-xpdf-page "~/LATEX/2020-2-C2/20210205_trapezios_3.pdf")
\myvcenter{
\includegraphics[height=4cm]{2020-2-C2/20210205_trapezios_3.pdf}
}
$






\newpage

% «exercicio-1»  (to ".exercicio-1")
% (c2m211somas1p 6 "exercicio-1")
% (c2m211somas1a   "exercicio-1")
% (c2m202somas1p 4 "exercicio-1")
% (c2m202somas1    "exercicio-1")

{\bf Nossa função preferida}

Seja $f(x) = 4 - (x-2)^2$.

Isto é uma parábola com a concavidade pra baixo.

Verifique que:

$f(0)=4-4=0$,

$f(1)=4-1=3$,

$f(2)=4-0=4$,

$f(3)=4-1=3$,

$f(4)=4-4=0$.

\msk

Além disso $f'(x) = -2(x-2)$, $f'(1)=2$, $f'(3)=-2$, e

a reta tangente à curva $y=f(x)$ em $x=1$ tem coef.\ angular 2, e

a reta tangente à curva $y=f(x)$ em $x=3$ tem coef.\ angular -2.

\msk

{\bf Exercício 1:} use estas informações para traçar o gráfico de $f(x)$

entre $x=0$ e $x=4$.


\newpage

% «jeito-esperto»  (to ".jeito-esperto")
% (c2m212somas1p 7 "jeito-esperto")
% (c2m212somas1a   "jeito-esperto")
% (c2m202somas1p 5 "jeito-esperto")
% (c2m202somas1    "jeito-esperto")
% (c2m202somas1a "video")
% (c2m202somas1a "video" "jeito esperto")


{\bf Dois jeitos de visualizar $(x,f(x))$}

Jeito burro:

Em $x=2.5$ temos

$f(2.5) = 4 - (2.5-2)^2 = 4 - 0.5^2 = 4-0.25 = 3.75$.

Encontre o ponto $y=3.75$ no eixo $y$.

Desenhe o ponto $(2.5,3.75)$.

\msk

Jeito esperto/rápido:

Encontre no eixo $x$ o ponto $x=2.5$.

Suba esse ponto pra curva $y=f(x)$ --

você encontrou o ponto $(2.5,f(2.5))$!

\bsk

O ``jeito esperto'' está explicado neste vídeo aqui:

\ssk

{\footnotesize

\url{http://www.youtube.com/watch?v=bbZfQmtFCSw\#t=4m00s}

}

\ssk

Ele vai ser \standout{MUITO} importante!!!!!!!!!


\newpage

% «exercicios-2-e-3»  (to ".exercicios-2-e-3")
% (c2sop 8 "exercicios-2-e-3")
% (c2soa   "exercicios-2-e-3")
% (c2m202somas1p 6 "exercicios-2-e-3")
% (c2m202somas1    "exercicios-2-e-3")

{\bf Mais exercícios}

{\bf Exercício 2.} Desenhe o gráfico da nossa função preferida

(obs: sempre no intervalo entre $x=0$ e $x=4$!)

e desenhe sobre ele o retângulo ``cuja área é $f(0.5)·(1.5-0.5)$''.

Truque: isto é altura $·$ base, e a base vai de $x=0.5$ a $x=1.5$.

\msk

{\bf Exercício 3.} Desenhe em outro gráfico

a nossa função preferida e sobre ela os retângulos da soma abaixo:

$f(0.5)·(1.5-0.5) + f(1.5)·(2-1.5) + f(2)·(3-2) + f(3.5)(3.5-3)$

\newpage

% «particoes»  (to ".particoes")
% (c2m212somas1p 9 "particoes")
% (c2m212somas1a   "particoes")

{\bf Partições}

\ColorRed{Informalmente} uma partição de um intervalo $[a,b]$ é um
modo de

decompor $[a,b]$ em intervalos menores consecutivos. Por exemplo,
%
$$[2,7] = [2,3.5]∪[3.5,4]∪[4,6]∪[6,7]$$

A definição ``certa'' é mais complicada... vamos vê-la daqui a pouco.

Caso geral:
%
$$[a,b] = [a_1,b_1]∪[a_2,b_2]∪\ldots∪[a_N,b_N],$$

onde:

$N$ é o número de intervalos,

$a=a_1$, $b=b_N$, (``extremidades'')

$a_i<b_i$ para todo $i$ em que isto faz sentido ($i=1,\ldots,N$)

$b_i=a_{i+1}$ para todo $i$ e.q.i.f.s.; neste caso, $i=1,\ldots,N-1$

\newpage

% «exercicio-4»  (to ".exercicio-4")
% (c2m212somas1p 10 "exercicio-4")
% (c2m212somas1a    "exercicio-4")
% (c2sop 10 "exercicio-4")
% (c2soa    "exercicio-4")
% (c2m202somas1p 8 "exercicio-4")
% (c2m202somas1    "exercicio-4")

{\bf Partições (2)}

Um jeito prático de definir uma partição é usando uma tabela.

Por exemplo, esta tabela
%
$$\begin{array}{ccc}
  i & a_i & b_i \\\hline
  1 & 2 & 3.5 \\
  2 & 3.5 & 4 \\
  3 & 4 & 6 \\
  4 & 6 & 7 \\
  \end{array}
$$

corresponde à partição de $[2,7]$ do slide anterior.

{\bf Exercício 4.} Converta esta ``partição''
%
$$[4,12] = [4,5]∪[5,6]∪[6,9]∪[9,10]∪[10,12]$$

numa tabela. Neste caso quem são $a$, $b$ e $N$?

\newpage

% «exercicio-5»  (to ".exercicio-5")
% (c2m212somas1p 11 "exercicio-5")
% (c2m212somas1a    "exercicio-5")
% (c2m202somas1p 9 "exercicio-5")
% (c2m202somas1    "exercicio-5")

{\bf Partições (3)}

A definição \ColorRed{certa} de partição é a seguinte.

Digamos que $P$ seja um subconjunto não-vazio e finito de $\R$,

e que o menor elemento de $P$ seja $a$ e o maior seja $b$.

Então $P$ é uma \ColorRed{partição} do intervalo $[a,b]$.

\msk

Exemplo: a partição $P=\{2,3.5,4,6,7\}$ corresponde a:
%
$$[2,7] = [2,3.5]∪[3.5,4]∪[4,6]∪[6,7]$$

Pra fazer a tradução ponha os elementos de $P$ em ordem e chame-os de
$b_0,\ldots,b_N$; defina cada $a_i$ como sendo $b_{i-1}$ -- por
exemplo, $a_1 = b_0$ -- e encontre $a$, $b$, e $N$.

\msk

{\bf Exercício 5.} Converta a partição $P=\{2.5,3,4,6,10\}$ para o
formato tabela e para o formato $[a,b] = [a_1,b_1]∪\ldots∪[a_N,b_N].$


% «ponto-decimal»  (to ".ponto-decimal")
% Ah, obs, repara que eu vou usar a convencao internacional e vou
% sempre escrever "1.5" ao inves de "1,5" - e recomendo que voces usem
% ela tambem pra gente poder usar a virgula pra outras coisas. Por
% exemplo, na pagina 9 temos P = {2, 3.5, 4, 6, 7}, e se a gente
% escrever "3,5" ao inves de "3.5" vamos ter que usar ";"s como
% separadores entres os numeros...

\newpage

% «partition-sum»  (to ".partition-sum")
% (c2m211somas1p 12 "partition-sum")
% (c2m211somas1a    "partition-sum")

{\bf Partições definem muitas coisas implicitamente}

Quando dizemos algo como ``Seja $P$ a partição $\{2.5,4,6\}$'' estamos
criando um contexto no qual há uma partição ``default'' definida... e
neste contexto vamos ter valores definidos para $N$, $a$, $b$, e para
cada $a_i$ e $b_i$. Por exemplo...

\msk

Seja $P$ a partição $\{2.5,4,6\}$. Então
%
$$\begin{array}{rcl}
  \sum_{i=1}^N f(b_i)·(b_i-a_i)
     &=& \sum_{i=1}^2 f(b_i)·(b_i-a_i) \\
     &=& f(b_1)·(b_1-a_1) + f(b_2)·(b_2-a_2) \\
     &=& f(4)·(4-2.5) + f(6)·(6-4) \\
  \end{array}
$$

\newpage

% «subst»  (to ".subst")
% (c2m202somas1p 11 "subst")
% (c2m202somas1     "subst")

Note que a expressão $\sum_{i=a}^b \mathsf{expr}$ quer dizer ``some
várias cópias da expressão $\mathsf{expr}$, a primeira com $i$
substituido por $a$, a segunda com $i$ substituido por $a+1$, etc etc,
até a cópia com $i$ substituido por $b$''...

Se você tiver dificuldade pra interpretar alguma expressão com
somatórios você pode calculá-la beeem passo a passo usando a operação
`$[:=]$' da aula passada. Por exemplo:
%
$$\scalebox{0.90}{$
  \begin{array}{rcl}
  \sum_{i=4}^7 f(b_i)·(b_i-a_i)
     &=& (f(b_i)·(b_i-a_i))[i:=4] \\
     &+& (f(b_i)·(b_i-a_i))[i:=5] \\
     &+& (f(b_i)·(b_i-a_i))[i:=6] \\
     &+& (f(b_i)·(b_i-a_i))[i:=7] \\[5pt]
     &=&  f(b_4)·(b_4-a_4) \\
     &+&  f(b_5)·(b_5-a_5) \\
     &+&  f(b_6)·(b_6-a_6) \\
     &+&  f(b_7)·(b_7-a_7) \\[5pt]
     &=& \ldots \\
  \end{array}
  $}
$$



\newpage

% «exercicio-6»  (to ".exercicio-6")
% (c2m202somas1p 12 "exercicio-6")
% (c2m202somas1     "exercicio-6")

{\bf Alguns exercícios de visualizar somas de retângulos...}

\ssk

{\bf Exercício 6.} Seja $f$ a nossa função preferida e seja $P$ a
partição $\{0.5,1,2,2.5\}$. Represente num gráfico só a curva $y=f(x)$
e os retângulos da soma $\sum_{i=1}^N f(b_i)·(b_i-a_i)$.

\msk

{\bf Exercício 7.} Seja $f$ a nossa função preferida e seja $P$ a
mesma partição que no exercício anterior. Represente num gráfico só --
separado do gráfico do exercício anterior!!! -- a curva $y=f(x)$ e os
retângulos da soma $\sum_{i=1}^N f(a_i)·(b_i-a_i)$.

\msk

{\bf Exercício 8.} Usando a mesma função $f$ e a mesma partição $P$
dos exercícios anteriores, represente num outro gráfico a curva
$y=f(x)$ e os retângulos da soma $\sum_{i=1}^N
f(\frac{a_i+b_i}{2})·(b_i-a_i)$. Repare que $\frac{a_i+b_i}{2}$ é o
ponto médio do intervalo $[a_i,b_i]$, e é fácil encontrar pontos
médios no olhômetro.

\newpage

% «exercicio-9»  (to ".exercicio-9")
% (c2m212somas1p 15 "exercicio-9")
% (c2m212somas1a    "exercicio-9")
% (c2m211somas1p 15 "exercicio-9")
% (c2m211somas1a    "exercicio-9")
% (c2m202somas1p 13 "exercicio-9")
% (c2m202somas1     "exercicio-9")

{\bf Agora comparando com a Wikipedia}

\msk

{\bf Exercício 9.} Dê uma olhada na página

\ssk

\url{https://pt.wikipedia.org/wiki/Soma_de_Riemann}

\ssk

da Wikipedia. Vamos tentar entender alguns pedaços dela.

Seja $P$ a ``partição do intervalo $[0,3]$ em 6 subintervalos
iguais''. Tem um ponto em que a página da Wikipedia diz: ``os pontos
da partição serão...'' -- entenda as definições dela, descubra quem é
$Δx$ neste caso, e escreva quais são os pontos desta partição na
linguagem da página da Wikipedia e na linguagem que eu usei nos slides.

Expanda a fórmula da página da Wikipedia para a ``soma média'' neste
caso. Expanda também a nossa fórmula $\sum_{i=1}^N
f(\frac{a_i+b_i}{2})·(b_i-a_i)$ e compare as duas expansões.

\msk

(Vamos ver o que são ``ínfimos'' e ``supremos'' na aula que vem)

\newpage

% «exercicio-9-dicas»  (to ".exercicio-9-dicas")
% (c2m211somas1p 16 "exercicio-9-dicas")
% (c2m211somas1a    "exercicio-9-dicas")

{\bf Dicas pro exercício 9}

Eu pus um vídeo com várias dicas pro exercício 9 aqui:

\ssk


\url{http://angg.twu.net/eev-videos/2021-1-C2-somas-1.mp4}

\url{https://www.youtube.com/watch?v=Ht5iLKGlysM}

\msk

Uma dica extra... no Ensino Médio às vezes convencem a gente de que
uma fração como $\frac64$ \ColorRed{\underline{\underline{tem}}} que
ser simplificada pra $\frac32$, mas se a gente tem que listar uma
sequência de números começando em 0 em que cada número novo é o
anterior mais $\frac14$ eu acho bem melhor escrever essa sequência
como $\frac04, \frac14, \frac24, \frac34, \frac44, \frac54, \frac64$
do que como $0, \frac14, \frac12, \frac34, 1, \frac54, \frac32$...

\newpage

{\bf Dicas pro exercício 9 (2)}

Além disso no exercício 9 você vai ter alguns somatórios de expressões
como $f(\frac{a_i+b_i}{2})·(b_i-a_i)$ em que todos os `$(b_i-a_i)$'s
dão o mesmo valor. Você {\sl pode} reescrever todos esses
`$(b_i-a_i)$'s como números, mas se você parar as suas expansões e
simplificações um passo antes e mantiver eles como $(0.5 - 0)$, $(1 -
0.5)$, etc, aí vai ser fácil interpretar cada
$f(\frac{a_i+b_i}{2})·(b_i-a_i)$ como um retângulo.


\bsk

Sobre o ``jeito esperto'', leia isto aqui:

\ssk

{\scriptsize

\url{http://angg.twu.net/2021.1-C2/2021-jun-18-pergunta_sobre_jeito_esperto.pdf}

}



\newpage

% «trapezios»  (to ".trapezios")
% (c2m212somas1p 18 "trapezios")
% (c2m212somas1a    "trapezios")
% (c2m211somas1p 18 "trapezios")
% (c2m211somas1a    "trapezios")

{\bf Trapézios}

Tem dois modos diferentes da gente

interpretar geometricamente $\frac{f(a)+f(b)}{2} (b-a)$:

\msk

1) como um retângulo de altura $\frac{f(a)+f(b)}{2}$, ou

2) como um trapézio com vértices
%
$$(a,0), (b,0), (b,f(b)), (a,f(a))$$.

{\bf Exercício 10.} Sejam $f$ a nossa função preferida e $P$ a
partição $\{0,1,2\}$. Desenhe num gráfico só a curva $y=f(x)$ e os
trapézios da soma:
%
$$\sum_{i=1}^N \frac{f(a_i)+f(b_i)}{2} (b_i-a_i)$$

(Veja as figuras da ``Regra Trapezoidal'' na página da Wikipedia)


\newpage

% «trap-e-rect»  (to ".trap-e-rect")
% (c2m212somas1p 19 "trap-e-rect")
% (c2m212somas1a    "trap-e-rect")

\unitlength=10pt
\def\mypicture#1#2#3#4{{
    \sa{x0}{#1}
    \sa{y0}{#2}
    \sa{x1}{#3}
    \sa{y1}{#4}
    \vcenter{\hbox{%
    \beginpicture(0,0)(6,4)
    \pictgrid%
    \ColorOrange{
      \polygon*(\ga{x0},0)(\ga{x0},\ga{y0})(\ga{x1},\ga{y1})(\ga{x1},0)
    }
      \polygon(\ga{x0},0)(\ga{x0},\ga{y0})(\ga{x1},\ga{y1})(\ga{x1},0)
    \pictaxes%
    \end{picture}%
  }}}}

...ou seja, estas duas igualdades estão certas:
%
$$\begin{array}{rcl}
  \D \left( \frac{2+3}{2} \right)(5-4)
     &=& \mypicture{4}{2}{5}{3} \\
     [20pt]
  \D \left( \frac{2+3}{2} \right)(5-4)
     &=& \mypicture{4}{2.5}{5}{2.5} \\
  \end{array}
$$

A primeira mostra a representação geométrica do
$\left( \frac{2+3}{2} \right)(5-4)$

como um trapézio, e a segunda mostra a representação

geométrica do $\left( \frac{2+3}{2} \right)(5-4)$ como um retângulo de

altura $\frac{2+3}{2} = 2.5$.

\newpage

% «exercicio-11»  (to ".exercicio-11")
% (c2m212somas1p 20 "exercicio-11")
% (c2m212somas1a    "exercicio-11")
% (c2m211prp 35 "retangulos-degenerados")
% (c2m211pra    "retangulos-degenerados")

{\bf Exercício 11.}

Para cada uma das expressões abaixo faça as duas

representações gráficas dela: a que interpreta a expressão

como uma soma de trapézios e que interpreta ela como uma

soma de retângulos.

\def\r#1{(\frac{#1}{2})}

\msk

a) $\r{2+1}(2-1) + \r{2+4}(5-2)$

\ssk

b) $\r{0+1}(1-0) + \r{1+2}(2-1) + \r{2+3}(3-2) + \r{3+5}(5-3)$

\ssk

c) $\r{1+1}(2-1) + \r{1-0}(3-2) + \r{0-1}(4-3) + \r{-1-1}(5-4)$

\ssk

d) $\r{1+1}(2-1) + \r{1-1}(4-2) + \r{-1-1}(5-4)$

\ssk

e) $\r{-1+3}(5-1)$

\ssk

f) $\r{1-2}(4-1)$


\newpage

% «exercicio-12»  (to ".exercicio-12")
% (c2m212somas1p 21 "exercicio-12")
% (c2m212somas1a    "exercicio-12")

{\bf Exercício 12.}

\msk

% (find-latexscan-links "C2" "20211111_somas_1_ex_12")
% (find-xpdf-page "~/LATEX/2021-2-C2/20211111_somas_1_ex_12.pdf")
\centerline{$\includegraphics[height=4cm]{2021-2-C2/20211111_somas_1_ex_12.pdf}$}

Faça duas cópias à mão do desenho acima -- o gráfico da função $f$

com indicações de quais são os pontos $x_0, \ldots, x_4$ do eixo $x$.

Você vai fazer cada item do exercício sobre uma das cópias.

As cópias não precisam ser muito precisas.


\newpage

{\bf Exercício 12 (cont.)}

\msk

\def\F#1{f(x_#1)}
\def\G#1#2{\r{f(x_#1)+f(x_#2)}}
\def\H#1#2{f(x_#2)-f(x_#1)}
\def\T#1#2{\G#1#2(\H#1#2)}

a) Represente graficamente sobre a primeira cópia:

\ssk

$  \F1(x_1-x_0)
 + \F2(x_2-x_1)
 + \F3(x_3-x_2)
 + \F4(x_4-x_3)
$

\bsk

b) Represente graficamente sobre a segunda cópia

a interpretação da expressão abaixo como trapézios

\ColorRed{e} a interpretação dela como retângulos:

\msk

$\begin{array}{c}\T01 \\ + \; \T12 \\ \; + \T23 \\ \; + \T34\end{array}$

\newpage

% «metodos-nomes»  (to ".metodos-nomes")
% (c2m212somas1p 23 "metodos-nomes")
% (c2m212somas1a    "metodos-nomes")
% (c2m212somas2p 27 "metodos-nomes")
% (c2m212somas2a    "metodos-nomes")

{\bf Métodos de integração: nomes}

\def\sumiN#1{\sum_{i=1}^N #1 (b_i-a_i)}
\def\mname#1{\text{[#1]}}
%
$$\scalebox{0.95}{$
  \begin{array}{ccl}
  \mname{L}    &=& \sumiN {f(a_i)}                    \\[2pt]
  \mname{R}    &=& \sumiN {f(b_i)}                    \\[2pt]
  \mname{Trap} &=& \sumiN {\frac{f(a_i) + f(b_i)}{2}} \\[2pt]
  \mname{M}    &=& \sumiN {f(\frac{a_i+b_i}{2})}      \\[2pt]
  \mname{min}  &=& \sumiN {\min(f(a_i), f(b_i))}      \\[2pt]
  \mname{max}  &=& \sumiN {\max(f(a_i), f(b_i))}      \\[2pt]
  \mname{inf}  &=& \sumiN {\inf(F([a_i,b_i]))}        \\[2pt]
  \mname{sup}  &=& \sumiN {\sup(F([a_i,b_i]))}        \\
  \end{array}
  $}
$$

Cada uma dessas fórmulas é um ``método de integração''.

Todos esses ``métodos'' aparecem na página da Wikipedia,

mas com outros nomes e usando partições em que todos os

intervalos têm o mesmo comprimento.



\newpage

{\bf Exercício 13.}

Seja $f$ a nossa função preferida, e

seja $P$ a partição $P=\{1,4\}$.

\msk

a) Represente num gráfico só $f$ e $\mname{L}$.

b) Represente num gráfico só $f$ e $\mname{R}$.

c) Represente num gráfico só $f$ e $\mname{Trap}$ (dos dois jeitos).

d) Represente num gráfico só $f$ e $\mname{M}$.

e) Represente num gráfico só $f$ e $\mname{min}$.

f) Represente num gráfico só $f$ e $\mname{max}$.

%\printbibliography

\GenericWarning{Success:}{Success!!!}  % Used by `M-x cv'

\end{document}

%  ____  _             _         
% |  _ \(_)_   ___   _(_)_______ 
% | | | | \ \ / / | | | |_  / _ \
% | |_| | |\ V /| |_| | |/ /  __/
% |____// | \_/  \__,_|_/___\___|
%     |__/                       
%
% «djvuize»  (to ".djvuize")
% (find-LATEXgrep "grep --color -nH --null -e djvuize 2020-1*.tex")

 (eepitch-shell)
 (eepitch-kill)
 (eepitch-shell)
# (find-fline "~/2021.2-C2/")
# (find-fline "~/LATEX/2021-2-C2/")
# (find-fline "~/bin/djvuize")

cd /tmp/
for i in *.jpg; do echo f $(basename $i .jpg); done

f () { rm -v $1.pdf;  textcleaner -f 50 -o  5 $1.jpg $1.png; djvuize $1.pdf; xpdf $1.pdf }
f () { rm -v $1.pdf;  textcleaner -f 50 -o 10 $1.jpg $1.png; djvuize $1.pdf; xpdf $1.pdf }
f () { rm -v $1.pdf;  textcleaner -f 50 -o 20 $1.jpg $1.png; djvuize $1.pdf; xpdf $1.pdf }

f () { rm -fv $1.png $1.pdf; djvuize $1.pdf }
f () { rm -fv $1.png $1.pdf; djvuize WHITEBOARDOPTS="-m 1.0 -f 15" $1.pdf; xpdf $1.pdf }
f () { rm -fv $1.png $1.pdf; djvuize WHITEBOARDOPTS="-m 1.0 -f 30" $1.pdf; xpdf $1.pdf }
f () { rm -fv $1.png $1.pdf; djvuize WHITEBOARDOPTS="-m 1.0 -f 45" $1.pdf; xpdf $1.pdf }
f () { rm -fv $1.png $1.pdf; djvuize WHITEBOARDOPTS="-m 1.0 -f 90" $1.pdf; xpdf $1.pdf }
f () { rm -fv $1.png $1.pdf; djvuize WHITEBOARDOPTS="-m 0.5" $1.pdf; xpdf $1.pdf }
f () { rm -fv $1.png $1.pdf; djvuize WHITEBOARDOPTS="-m 0.25" $1.pdf; xpdf $1.pdf }
f () { cp -fv $1.png $1.pdf       ~/2021.2-C2/
       cp -fv        $1.pdf ~/LATEX/2021-2-C2/
       cat <<%%%
% (find-latexscan-links "C2" "$1")
%%%
}

f 20211111_somas_1_ex_12



%  __  __       _        
% |  \/  | __ _| | _____ 
% | |\/| |/ _` | |/ / _ \
% | |  | | (_| |   <  __/
% |_|  |_|\__,_|_|\_\___|
%                        
% <make>

 (eepitch-shell)
 (eepitch-kill)
 (eepitch-shell)
# (find-LATEXfile "2019planar-has-1.mk")
make -f 2019.mk STEM=2021-2-C2-somas-1 veryclean
make -f 2019.mk STEM=2021-2-C2-somas-1 pdf

% Local Variables:
% coding: utf-8-unix
% ee-tla: "c2so"
% ee-tla: "c2m212somas1"
% End:
