% (find-LATEX "2021-2-C2-revisao-pra-P2.tex")
% (defun c () (interactive) (find-LATEXsh "lualatex -record 2021-2-C2-revisao-pra-P2.tex" :end))
% (defun C () (interactive) (find-LATEXsh "lualatex 2021-2-C2-revisao-pra-P2.tex" "Success!!!"))
% (defun D () (interactive) (find-pdf-page      "~/LATEX/2021-2-C2-revisao-pra-P2.pdf"))
% (defun d () (interactive) (find-pdftools-page "~/LATEX/2021-2-C2-revisao-pra-P2.pdf"))
% (defun e () (interactive) (find-LATEX "2021-2-C2-revisao-pra-P2.tex"))
% (defun o () (interactive) (find-LATEX "2021-2-C2-revisao-pra-P2.tex"))
% (defun u () (interactive) (find-latex-upload-links "2021-2-C2-revisao-pra-P2"))
% (defun v () (interactive) (find-2a '(e) '(d)))
% (defun d0 () (interactive) (find-ebuffer "2021-2-C2-revisao-pra-P2.pdf"))
% (defun cv () (interactive) (C) (ee-kill-this-buffer) (v) (g))
%          (code-eec-LATEX "2021-2-C2-revisao-pra-P2")
% (find-pdf-page   "~/LATEX/2021-2-C2-revisao-pra-P2.pdf")
% (find-sh0 "cp -v  ~/LATEX/2021-2-C2-revisao-pra-P2.pdf /tmp/")
% (find-sh0 "cp -v  ~/LATEX/2021-2-C2-revisao-pra-P2.pdf /tmp/pen/")
%     (find-xournalpp "/tmp/2021-2-C2-revisao-pra-P2.pdf")
%   file:///home/edrx/LATEX/2021-2-C2-revisao-pra-P2.pdf
%               file:///tmp/2021-2-C2-revisao-pra-P2.pdf
%           file:///tmp/pen/2021-2-C2-revisao-pra-P2.pdf
% http://angg.twu.net/LATEX/2021-2-C2-revisao-pra-P2.pdf
% (find-LATEX "2019.mk")
% (find-CN-aula-links "2021-2-C2-revisao-pra-P2" "2" "c2m212rp2" "c2rp2")

% «.defs»		(to "defs")
% «.title»		(to "title")
% «.exercicio-1»	(to "exercicio-1")
% «.exercicio-2»	(to "exercicio-2")
% «.exercicio-3»	(to "exercicio-3")
%
% «.djvuize»		(to "djvuize")



% <videos>
% Video (not yet):
% (find-ssr-links     "c2m212rp2" "2021-2-C2-revisao-pra-P2" "{naoexiste}")
% (code-eevvideo      "c2m212rp2" "2021-2-C2-revisao-pra-P2")
% (code-eevlinksvideo "c2m212rp2" "2021-2-C2-revisao-pra-P2")
% (find-c2m212rp2video "0:00")

\documentclass[oneside,12pt]{article}
\usepackage[colorlinks,citecolor=DarkRed,urlcolor=DarkRed]{hyperref} % (find-es "tex" "hyperref")
\usepackage{amsmath}
\usepackage{amsfonts}
\usepackage{amssymb}
\usepackage{pict2e}
\usepackage[x11names,svgnames]{xcolor} % (find-es "tex" "xcolor")
\usepackage{colorweb}                  % (find-es "tex" "colorweb")
%\usepackage{tikz}
%
% (find-dn6 "preamble6.lua" "preamble0")
%\usepackage{proof}   % For derivation trees ("%:" lines)
%\input diagxy        % For 2D diagrams ("%D" lines)
%\xyoption{curve}     % For the ".curve=" feature in 2D diagrams
%
\usepackage{edrx21}               % (find-LATEX "edrx21.sty")
\input edrxaccents.tex            % (find-LATEX "edrxaccents.tex")
\input edrx21chars.tex            % (find-LATEX "edrx21chars.tex")
\input edrxheadfoot.tex           % (find-LATEX "edrxheadfoot.tex")
\input edrxgac2.tex               % (find-LATEX "edrxgac2.tex")
%\usepackage{emaxima}              % (find-LATEX "emaxima.sty")
%
%\usepackage[backend=biber,
%   style=alphabetic]{biblatex}            % (find-es "tex" "biber")
%\addbibresource{catsem-slides.bib}        % (find-LATEX "catsem-slides.bib")
%
% (find-es "tex" "geometry")
\usepackage[a6paper, landscape,
            top=1.5cm, bottom=.25cm, left=1cm, right=1cm, includefoot
           ]{geometry}
%
\begin{document}

%\catcode`\^^J=10
%\directlua{dofile "dednat6load.lua"}  % (find-LATEX "dednat6load.lua")
%
% %L dofile "2021pict2e.lua"           -- (find-LATEX "2021pict2e.lua")
% %L Pict2e.__index.suffix = "%"
% \pu
% \def\pictgridstyle{\color{GrayPale}\linethickness{0.3pt}}
% \def\pictaxesstyle{\linethickness{0.5pt}}

% %L dofile "edrxtikz.lua"  -- (find-LATEX "edrxtikz.lua")
% %L dofile "edrxpict.lua"  -- (find-LATEX "edrxpict.lua")
% \pu

% «defs»  (to ".defs")
% (find-LATEX "edrx21defs.tex" "colors")
% (find-LATEX "edrx21.sty")

\def\u#1{\par{\footnotesize \url{#1}}}

\def\drafturl{http://angg.twu.net/LATEX/2021-2-C2.pdf}
\def\drafturl{http://angg.twu.net/2021.2-C2.html}
\def\draftfooter{\tiny \href{\drafturl}{\jobname{}} \ColorBrown{\shorttoday{} \hours}}



%  _____ _ _   _                               
% |_   _(_) |_| | ___   _ __   __ _  __ _  ___ 
%   | | | | __| |/ _ \ | '_ \ / _` |/ _` |/ _ \
%   | | | | |_| |  __/ | |_) | (_| | (_| |  __/
%   |_| |_|\__|_|\___| | .__/ \__,_|\__, |\___|
%                      |_|          |___/      
%
% «title»  (to ".title")
% (c2m212rp2p 1 "title")
% (c2m212rp2a   "title")

\thispagestyle{empty}

\begin{center}

\vspace*{1.2cm}

{\bf \Large Cálculo 2 - 2021.2}

\bsk

Aula 28: revisão pra P2

\bsk

Eduardo Ochs - RCN/PURO/UFF

\url{http://angg.twu.net/2021.2-C2.html}

\end{center}

\newpage

A P2 vai ter duas questões sobre calcular integrais do jeito

rápido, omitindo os limites de integração e usando mudanças

de variável. Na prova vocês {\bf VÃO TER QUE} calcular essas

integrais usando {\bf EXATAMENTE} a notação das caixinhas de

truques da MVG, como aqui:

\msk

{\footnotesize

% (c2m212mvp 4 "exemplo")
% (c2m212mva   "exemplo")
%    http://angg.twu.net/LATEX/2021-2-C2-mud-var-gamb.pdf#page=4
\url{http://angg.twu.net/LATEX/2021-2-C2-mud-var-gamb.pdf#page=4}

}

\msk

Eu não conheço nenhum livro que explique direito porque é

que a gente não pode misturar variáveis... um ou outro livro

menciona isso brevemente e supõe que o leitor vai entender

o porquê depois de estudar dezenas de horas, mas só.

\newpage

% «exercicio-1»  (to ".exercicio-1")
% (c2m212rp2p 3 "exercicio-1")
% (c2m212rp2a   "exercicio-1")

{\bf Exercício 1.}

Reescreva os exemplos das páginas 189 a 194 do livro do

Daniel Miranda na notação que você vai ter que usar na P2,

em que as caixinhas de truques aparecem explicitamente

à direita das contas. Link:

\msk

{\footnotesize

\url{http://hostel.ufabc.edu.br/~daniel.miranda/calculo/calculo.pdf}

}

% (find-books "__analysis/__analysis.el" "miranda")
% (find-dmirandacalcpage 189 "6.2 Integração por Substituição")


\bsk
\bsk
\bsk

Os próximos dois exercícios são sobre coisas que o livro

do Daniel Miranda explica bem brevemente na página 195

e bem detalhadamente na seção 8.3.

\newpage

% «exercicio-2»  (to ".exercicio-2")
% (c2m212rp2p 4 "exercicio-2")
% (c2m212rp2a   "exercicio-2")

{\bf Exercício 2.}

A caixinha de truques da MVG pra substituição $s = \sen θ$

é essa aqui:
%
$$\bmat{
    s = \sen θ \\
    \frac{ds}{dθ} = \frac{d}{dθ} \sen θ = \cos θ \\
    ds = \cos θ \, dθ \\
    (\cosθ)^2 + (\senθ)^2 = 1 \\
    (\cosθ)^2 = 1 - (\senθ)^2 \\
    (\cosθ)^2 = 1 - s^2 \\
  }
$$

\bsk

As últimas linhas dela são opcionais mas são úteis.

\msk

a) Use esta caixinha pra integrar $\intth {(\sen θ)^2(\cos θ)^2·\cos θ}$.

b) Confira a sua resposta derivando o seu resultado.



\newpage

% «exercicio-3»  (to ".exercicio-3")
% (c2m212rp2p 5 "exercicio-3")
% (c2m212rp2a   "exercicio-3")

{\bf Exercício 3.}

\ssk

a) Faça uma caixinha de truques da MVG

pra substituição $c = \cos θ$.

\msk

Dica: pode ser que nela apareçam uma coisas como

$(-1)dc$ ou $(-1)dθ$. Alguns livros escrevem isso como

$-dc$ ou $-dθ$, mas eu acho que as contas ficam mais

claras com `$(-1)$' ao invés de `$-$'.

\msk

b) Use essa caixinha pra integrar
%
$$\intth {(\cos θ)^2(\sen θ)^2·\sen θ}.$$

c) Confira a sua resposta derivando o seu resultado.


%\printbibliography

\GenericWarning{Success:}{Success!!!}  % Used by `M-x cv'

\end{document}

%  ____  _             _         
% |  _ \(_)_   ___   _(_)_______ 
% | | | | \ \ / / | | | |_  / _ \
% | |_| | |\ V /| |_| | |/ /  __/
% |____// | \_/  \__,_|_/___\___|
%     |__/                       
%
% «djvuize»  (to ".djvuize")
% (find-LATEXgrep "grep --color -nH --null -e djvuize 2020-1*.tex")

 (eepitch-shell)
 (eepitch-kill)
 (eepitch-shell)
# (find-fline "~/2021.2-C2/")
# (find-fline "~/LATEX/2021-2-C2/")
# (find-fline "~/bin/djvuize")

cd /tmp/
for i in *.jpg; do echo f $(basename $i .jpg); done

f () { rm -v $1.pdf;  textcleaner -f 50 -o  5 $1.jpg $1.png; djvuize $1.pdf; xpdf $1.pdf }
f () { rm -v $1.pdf;  textcleaner -f 50 -o 10 $1.jpg $1.png; djvuize $1.pdf; xpdf $1.pdf }
f () { rm -v $1.pdf;  textcleaner -f 50 -o 20 $1.jpg $1.png; djvuize $1.pdf; xpdf $1.pdf }

f () { rm -fv $1.png $1.pdf; djvuize $1.pdf }
f () { rm -fv $1.png $1.pdf; djvuize WHITEBOARDOPTS="-m 1.0 -f 15" $1.pdf; xpdf $1.pdf }
f () { rm -fv $1.png $1.pdf; djvuize WHITEBOARDOPTS="-m 1.0 -f 30" $1.pdf; xpdf $1.pdf }
f () { rm -fv $1.png $1.pdf; djvuize WHITEBOARDOPTS="-m 1.0 -f 45" $1.pdf; xpdf $1.pdf }
f () { rm -fv $1.png $1.pdf; djvuize WHITEBOARDOPTS="-m 0.5" $1.pdf; xpdf $1.pdf }
f () { rm -fv $1.png $1.pdf; djvuize WHITEBOARDOPTS="-m 0.25" $1.pdf; xpdf $1.pdf }
f () { cp -fv $1.png $1.pdf       ~/2021.2-C2/
       cp -fv        $1.pdf ~/LATEX/2021-2-C2/
       cat <<%%%
% (find-latexscan-links "C2" "$1")
%%%
}

f 20201213_area_em_funcao_de_theta
f 20201213_area_em_funcao_de_x
f 20201213_area_fatias_pizza



%  __  __       _        
% |  \/  | __ _| | _____ 
% | |\/| |/ _` | |/ / _ \
% | |  | | (_| |   <  __/
% |_|  |_|\__,_|_|\_\___|
%                        
% <make>

 (eepitch-shell)
 (eepitch-kill)
 (eepitch-shell)
# (find-LATEXfile "2019planar-has-1.mk")
make -f 2019.mk STEM=2021-2-C2-revisao-pra-P2 veryclean
make -f 2019.mk STEM=2021-2-C2-revisao-pra-P2 pdf

% Local Variables:
% coding: utf-8-unix
% ee-tla: "c2rp2"
% ee-tla: "c2m212rp2"
% End:
