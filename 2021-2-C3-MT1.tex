% (find-LATEX "2021-2-C3-MT1.tex")
% (defun c () (interactive) (find-LATEXsh "lualatex -record 2021-2-C3-MT1.tex" :end))
% (defun C () (interactive) (find-LATEXsh "lualatex 2021-2-C3-MT1.tex" "Success!!!"))
% (defun D () (interactive) (find-pdf-page      "~/LATEX/2021-2-C3-MT1.pdf"))
% (defun d () (interactive) (find-pdftools-page "~/LATEX/2021-2-C3-MT1.pdf"))
% (defun e () (interactive) (find-LATEX "2021-2-C3-MT1.tex"))
% (defun o () (interactive) (find-LATEX "2021-1-C3-MT1.tex"))
% (defun u () (interactive) (find-latex-upload-links "2021-2-C3-MT1"))
% (defun v () (interactive) (find-2a '(e) '(d)))
% (defun d0 () (interactive) (find-ebuffer "2021-2-C3-MT1.pdf"))
% (defun cv () (interactive) (C) (ee-kill-this-buffer) (v) (g))
%          (code-eec-LATEX "2021-2-C3-MT1")
% (find-pdf-page   "~/LATEX/2021-2-C3-MT1.pdf")
% (find-sh0 "cp -v  ~/LATEX/2021-2-C3-MT1.pdf /tmp/")
% (find-sh0 "cp -v  ~/LATEX/2021-2-C3-MT1.pdf /tmp/pen/")
%     (find-xournalpp "/tmp/2021-2-C3-MT1.pdf")
%   file:///home/edrx/LATEX/2021-2-C3-MT1.pdf
%               file:///tmp/2021-2-C3-MT1.pdf
%           file:///tmp/pen/2021-2-C3-MT1.pdf
% http://angg.twu.net/LATEX/2021-2-C3-MT1.pdf
% (find-LATEX "2019.mk")
% (find-CN-aula-links "2021-2-C3-MT1" "3" "c3m212mt1" "c3mt1")
%
% Video (not yet):
% (find-ssr-links      "c3m212mt1" "2021-2-C3-MT1")
% (code-eevvideo       "c3m212mt1" "2021-2-C3-MT1")
% (code-eevvideo-local "c3m212mt1" "2021-2-C3-MT1")
% (find-c3m212mt1video "0:00")

% «.defs»		(to "defs")
% «.title»		(to "title")
% «.dica»		(to "dica")
% «.copie-a-figura»	(to "copie-a-figura")
%
% «.aviso»		(to "aviso")
% «.feedback-icaro»	(to "feedback-icaro")
% «.djvuize»		(to "djvuize")

\documentclass[oneside,12pt]{article}
\usepackage[colorlinks,citecolor=DarkRed,urlcolor=DarkRed]{hyperref} % (find-es "tex" "hyperref")
\usepackage{amsmath}
\usepackage{amsfonts}
\usepackage{amssymb}
\usepackage{pict2e}
\usepackage[x11names,svgnames]{xcolor} % (find-es "tex" "xcolor")
\usepackage{colorweb}                  % (find-es "tex" "colorweb")
%\usepackage{tikz}
%
% (find-dn6 "preamble6.lua" "preamble0")
%\usepackage{proof}   % For derivation trees ("%:" lines)
%\input diagxy        % For 2D diagrams ("%D" lines)
%\xyoption{curve}     % For the ".curve=" feature in 2D diagrams
%
\usepackage{edrx21}               % (find-LATEX "edrx21.sty")
\input edrxaccents.tex            % (find-LATEX "edrxaccents.tex")
\input edrx21chars.tex            % (find-LATEX "edrx21chars.tex")
\input edrxheadfoot.tex           % (find-LATEX "edrxheadfoot.tex")
\input edrxgac2.tex               % (find-LATEX "edrxgac2.tex")
%
%\usepackage[backend=biber,
%   style=alphabetic]{biblatex}            % (find-es "tex" "biber")
%\addbibresource{catsem-slides.bib}        % (find-LATEX "catsem-slides.bib")
%
% (find-es "tex" "geometry")
\usepackage[a6paper, landscape,
            top=1.5cm, bottom=.25cm, left=1cm, right=1cm, includefoot
           ]{geometry}
%
\begin{document}

\catcode`\^^J=10
\directlua{dofile "dednat6load.lua"}  % (find-LATEX "dednat6load.lua")

%L dofile "edrxtikz.lua"  -- (find-LATEX "edrxtikz.lua")
%L dofile "edrxpict.lua"  -- (find-LATEX "edrxpict.lua")
\pu

% (find-LATEX "2021-1-C2-critical-points.lua" "Approxer-tests")
%L dofile     "2021-1-C2-critical-points.lua"
\pu

% «defs»  (to ".defs")
% (find-LATEX "edrx21defs.tex" "colors")
% (find-LATEX "edrx21.sty")



\def\drafturl{http://angg.twu.net/LATEX/2021-2-C3.pdf}
\def\drafturl{http://angg.twu.net/2021.2-C3.html}
\def\draftfooter{\tiny \href{\drafturl}{\jobname{}} \ColorBrown{\shorttoday{} \hours}}



%  _____ _ _   _                               
% |_   _(_) |_| | ___   _ __   __ _  __ _  ___ 
%   | | | | __| |/ _ \ | '_ \ / _` |/ _` |/ _ \
%   | | | | |_| |  __/ | |_) | (_| | (_| |  __/
%   |_| |_|\__|_|\___| | .__/ \__,_|\__, |\___|
%                      |_|          |___/      
%
% «title»  (to ".title")
% (c3m212mt1p 1 "title")
% (c3m212mt1a   "title")

\thispagestyle{empty}

\begin{center}

\vspace*{1.2cm}

{\bf \Large Cálculo 3 - 2021.2}

\bsk

Mini-teste 1

\bsk

Eduardo Ochs - RCN/PURO/UFF

\url{http://angg.twu.net/2021.2-C3.html}

\end{center}

\newpage

As regras vão ser as mesmas dos

mini-testes dos semestres anteriores:

\ssk

{\footnotesize

% (c2m202mt1p 2 "regras")
% (c2m202mt1a   "regras")
\url{http://angg.twu.net/LATEX/2020-2-C2-MT1.pdf#page=2}

}

(Leia com muita atenção!!!!!!!!!!!)

\bsk

As questões vão ser disponibilizadas às 20:30 da sexta

12/novembro/2021 e vocês vão ter até as 20:30 do sábado

13/novembro/2021 pra entregar as respostas.



\newpage

%  ____  _           
% |  _ \(_) ___ __ _ 
% | | | | |/ __/ _` |
% | |_| | | (_| (_| |
% |____/|_|\___\__,_|
%                    

% «dica»  (to ".dica")
% (c3m212mt1p 2 "dica")
% (c3m212mt1a   "dica")

Eu acabei de acrescentar no PDF das curvas de Bézier

as páginas 7, 8 e 9, que são o início do ``Exercício 3''...

Se já fez os exercícios 2, 3 e 4 do PDF de vetores

tangentes talvez você entenda na hora como usar essas

figuras das páginas 7, 8 e 9 pra resolver os exercícios

2, 3 e 4 quase sem contas... nós vamos ver isto em

detalhes na próxima aula.

\msk

Links:

{\footnotesize

% (c3m212bezierp 7 "exercicio-3")
% (c3m212beziera   "exercicio-3")
%    http://angg.twu.net/LATEX/2021-2-C3-bezier.pdf#page=7
\url{http://angg.twu.net/LATEX/2021-2-C3-bezier.pdf#page=7}

% (c3m212vtp 6 "exercicio-2")
% (c3m212vta   "exercicio-2")
%    http://angg.twu.net/LATEX/2021-2-C3-vetor-tangente.pdf#page=6
\url{http://angg.twu.net/LATEX/2021-2-C3-vetor-tangente.pdf#page=6}

}

\msk

O que você vai fazer no mini-teste é mais simples que isso.



\newpage

% «copie-a-figura»  (to ".copie-a-figura")
% (c3m212mt1p 4 "copie-a-figura")
% (c3m212mt1a    "copie-a-figura")

Copie a figura abaixo à mão para uma folha de papel.

Não tem problema se a sua cópia ficar meio torta.

A curva vermelha é o gráfico da função $x(t)$, e

a curva laranja é o gráfico da função $y(t)$.

O eixo horizontal é o eixo $t$.

\def\CA{\ColorRed}
\def\CB{\ColorOrange}
\def\CC{\ColorGreen}
\def\CD{\ColorViolet}


% (c2m211prp 5 "parabola-complicada")
% (c2m211pra   "parabola-complicada")
%L
%L sin, cos, pi = math.sin, math.cos,math.pi
%L PW = function (s) return
%L   Piecewisify.new(L("t -> "..s), seq(0, 6, 6/96)):pw(0, 6)
%L end
\pu

\def\trajcomponents#1{%
  \vcenter{\hbox{%
    \beginpicture(0,-2)(6,2)
    \pictgrid%
    #1%
    \pictaxes%
    \end{picture}%
  }}}

\unitlength=25pt

$$P(t) = (\CA{x(t)}, \CB{y(t)}) \\
  \quad
  \trajcomponents{%
    \CA{\expr{PW" cos((t/6)*4*pi) "}}%
    \CB{\expr{PW" sin((t/6)*6*pi) "}}%
  }
$$

\newpage

Para cada valor de $t$ você consegue usar o gráfico

da página anterior pra descobrir \ColorRed{aproximadamente}

os valores de $x(t)$ e $y(t)$ para aquele $t$.

\msk

Por exemplo, para $t=0.25$ o meu olhômetro me diz que

$P(0.25) ≈ (0.7, 0.8)$. Como a gente não tem as fórmulas

das funções $x(t)$ e $y(t)$ e a gente está tentando fazer tudo

no olho sem usar régua essa aproximação vai ter que ser

considerada boa o suficiente...

\msk

Marque no plano $\R^2$ os pontos $P(0), P(0.5), P(1.0), \ldots, P(6)$

e use esses 13 pontos pra tentar ``adivinhar'' a trajetória $P(t)$,

neste sentido aqui:

\ssk

{\footnotesize

% (c3m212introp 2)
%    http://angg.twu.net/LATEX/2021-2-C3-intro.pdf#page=2
\url{http://angg.twu.net/LATEX/2021-2-C3-intro.pdf#page=2}

}







\GenericWarning{Success:}{Success!!!}  % Used by `M-x cv'

\end{document}

% «aviso»  (to ".aviso")
% Aviso sobre o mini-teste no Telegram e no Classroom:

Gente, um aviso: o primeiro mini-teste vai ser NESTA SEXTA, naquele
esquema de que vocês vão ter 24 horas pra fazer e entregar... e ele
vai ser TODO baseado em exercícios que a gente vai fazer nas aulas de
quarta e sexta e que vão ser pra melhorar o olhômetro e a capacidade
de visualização de vocês.

Se você é uma pessoa que costuma resolver tudo por contas eu recomendo
MUITO que você participe MUITO das próximas duas aulas e faça todos os
exercícios dela em tempo real discutindo eles com todo mundo do grupo
do Telegram.

Obs: se você é uma pessoa que sabe muito sobre assuntos como os deste
vídeo aqui - http://www.youtube.com/watch?v=aVwxzDHniEw - você vai
achar o conteúdo das próximas duas aulas trivial e vai conseguir fazer
os desenhos do mini-teste em menos de 15 minutos.


% «feedback-icaro»  (to ".feedback-icaro")

Professor, enviei o arquivo mais cedo, com a trajetória, o desenho a
mão livre e achei interessante enviar uma outra folha para o senhor,
com uma análise que eu fiz dos pontos pra me auxiliar a identificar os
pontos de P no olhometrô

E em todas as folhas eu coloquei as coordenadas que eu ia achando, e
eu achei incrível essa atividade, pois o meu olhar foi ficando mais
agussado, na primeira folha, eu tinha um olhar do desenho em mão livre
e tive uns valores aproximados, que foram melhorando ao longo do
tempo.

Na segunda folha que eu fiz, eu destaquei só os pontos do gráfico,
colocando o valor da forma que o senhor fez no exercício, de 0.5 em
0.5. E aí eu usei de um artifício que me deu um estalo durante o
exercício. Eu lembrei do que o senhor disse no exemplo da última aula,
que a gente pode dividir um segmento de reta, e aí eu dividi a reta na
vertical em 1/2 é em 1/4 pra trazer mais precisão a análise e saber
com exatidão onde estaria os pontos de P

Depois disso eu peguei outra folha e fiz a trajetória com os valores
aproximados que encontrei e fui inserindo os pontos e no final dei a
trajetória dos mesmos

No fim eu peguei as três folhas e olhei tudo, vi a folha de rascunho,
que foi o desenho a mão livre, vi a folha com a marcação do ponto e
esse artifício que usei pra trazer precisão e olhei a última folha com
a marcação dos pontos e a trajetória. E vi que o meu olhar ficou muito
mais preciso e os pontos ficaram bem simétricos e trouxeram a figura
da forma que deveria ser. Estou mandando essa mensagem pra agradecer
pela aula e que ajudou muito pra aguçar o olhar e eu consegui
identificar os momentos pra usar as ferramentas corretas pra me
auxiliar no meu desempenho. Obrigado pelas dicas durante a aula
professor. Ajudou muito mesmo, achei que seria importante vir aqui
agradecer pelas orientações e queria contar como fiz tudo, pois fiquei
bem empolgado e não ia esperar até a próxima aula. Tenha um bom fim de
semana.


OOOOBAAAAAA!!!!!! Super obrigado pelo feedback!!!! 

Acho que você vai gostar muito do que a gente vai ver nas próximas
aulas, sobre visualizar derivadas e sobre como usar elas pra fazer
desenhos no olhômetro!

Há um tempo atrás teve a Semana de Monitoria e eu fiz parte da banca
que avaliava as apresentações dos monitores... e acabei ficando amigo
do monitor de GA, que está fazendo umas coisas com Geogebra que eu
achei muito impressionantes, e do monitor de Cálculo 1, que está
aprendendo a usar Manim e que mostrou



%  ____  _             _         
% |  _ \(_)_   ___   _(_)_______ 
% | | | | \ \ / / | | | |_  / _ \
% | |_| | |\ V /| |_| | |/ /  __/
% |____// | \_/  \__,_|_/___\___|
%     |__/                       
%
% «djvuize»  (to ".djvuize")
% (find-LATEXgrep "grep --color -nH --null -e djvuize 2020-1*.tex")

 (eepitch-shell)
 (eepitch-kill)
 (eepitch-shell)
# (find-fline "~/2021.2-C3/")
# (find-fline "~/LATEX/2021-2-C3/")
# (find-fline "~/bin/djvuize")

cd /tmp/
for i in *.jpg; do echo f $(basename $i .jpg); done

f () { rm -v $1.pdf;  textcleaner -f 50 -o  5 $1.jpg $1.png; djvuize $1.pdf; xpdf $1.pdf }
f () { rm -v $1.pdf;  textcleaner -f 50 -o 10 $1.jpg $1.png; djvuize $1.pdf; xpdf $1.pdf }
f () { rm -v $1.pdf;  textcleaner -f 50 -o 20 $1.jpg $1.png; djvuize $1.pdf; xpdf $1.pdf }

f () { rm -fv $1.png $1.pdf; djvuize $1.pdf }
f () { rm -fv $1.png $1.pdf; djvuize WHITEBOARDOPTS="-m 1.0 -f 15" $1.pdf; xpdf $1.pdf }
f () { rm -fv $1.png $1.pdf; djvuize WHITEBOARDOPTS="-m 1.0 -f 30" $1.pdf; xpdf $1.pdf }
f () { rm -fv $1.png $1.pdf; djvuize WHITEBOARDOPTS="-m 1.0 -f 45" $1.pdf; xpdf $1.pdf }
f () { rm -fv $1.png $1.pdf; djvuize WHITEBOARDOPTS="-m 0.5" $1.pdf; xpdf $1.pdf }
f () { rm -fv $1.png $1.pdf; djvuize WHITEBOARDOPTS="-m 0.25" $1.pdf; xpdf $1.pdf }
f () { cp -fv $1.png $1.pdf       ~/2021.2-C3/
       cp -fv        $1.pdf ~/LATEX/2021-2-C3/
       cat <<%%%
% (find-latexscan-links "C3" "$1")
%%%
}

f 20201213_area_em_funcao_de_theta
f 20201213_area_em_funcao_de_x
f 20201213_area_fatias_pizza



%  __  __       _        
% |  \/  | __ _| | _____ 
% | |\/| |/ _` | |/ / _ \
% | |  | | (_| |   <  __/
% |_|  |_|\__,_|_|\_\___|
%                        
% <make>

 (eepitch-shell)
 (eepitch-kill)
 (eepitch-shell)
# (find-LATEXfile "2019planar-has-1.mk")
make -f 2019.mk STEM=2021-2-C3-MT1 veryclean
make -f 2019.mk STEM=2021-2-C3-MT1 pdf

% Local Variables:
% coding: utf-8-unix
% ee-tla: "c3mt1"
% ee-tla: "c3m212mt1"
% End:
