% (find-LATEX "2021-2-C3-taylor.tex")
% (defun c () (interactive) (find-LATEXsh "lualatex -record 2021-2-C3-taylor.tex" :end))
% (defun C () (interactive) (find-LATEXsh "lualatex 2021-2-C3-taylor.tex" "Success!!!"))
% (defun D () (interactive) (find-pdf-page      "~/LATEX/2021-2-C3-taylor.pdf"))
% (defun d () (interactive) (find-pdftools-page "~/LATEX/2021-2-C3-taylor.pdf"))
% (defun e () (interactive) (find-LATEX "2021-2-C3-taylor.tex"))
% (defun o () (interactive) (find-LATEX "2021-1-C3-taylor.tex"))
% (defun u () (interactive) (find-latex-upload-links "2021-2-C3-taylor"))
% (defun v () (interactive) (find-2a '(e) '(d)))
% (defun d0 () (interactive) (find-ebuffer "2021-2-C3-taylor.pdf"))
% (defun cv () (interactive) (C) (ee-kill-this-buffer) (v) (g))
%          (code-eec-LATEX "2021-2-C3-taylor")
% (find-pdf-page   "~/LATEX/2021-2-C3-taylor.pdf")
% (find-sh0 "cp -v  ~/LATEX/2021-2-C3-taylor.pdf /tmp/")
% (find-sh0 "cp -v  ~/LATEX/2021-2-C3-taylor.pdf /tmp/pen/")
%     (find-xournalpp "/tmp/2021-2-C3-taylor.pdf")
%   file:///home/edrx/LATEX/2021-2-C3-taylor.pdf
%               file:///tmp/2021-2-C3-taylor.pdf
%           file:///tmp/pen/2021-2-C3-taylor.pdf
% http://angg.twu.net/LATEX/2021-2-C3-taylor.pdf
% (find-LATEX "2019.mk")
% (find-CN-aula-links "2021-2-C3-taylor" "3" "c3m212ta" "c3ta")

% «.video-1»		(to "video-1")
% «.video-2»		(to "video-2")
% «.defs»		(to "defs")
% «.title»		(to "title")
% «.ideia-basica-1»	(to "ideia-basica-1")
% «.ideia-basica-2»	(to "ideia-basica-2")
% «.exercicio-1»	(to "exercicio-1")
% «.exercicio-2»	(to "exercicio-2")
% «.derivs-e-derivs0»	(to "derivs-e-derivs0")
% «.exercicio-3»	(to "exercicio-3")
% «.approx»		(to "approx")
% «.exercicio-4»	(to "exercicio-4")
% «.exercicio-4-maxima»	(to "exercicio-4-maxima")
% «.exercicio-5»	(to "exercicio-5")
%
% «.djvuize»		(to "djvuize")



% «video-1»  (to ".video-1")
% (find-ssr-links     "c3m212ta" "2021-2-C3-taylor" "nug1S2GbY3U")
% (code-eevvideo      "c3m212ta" "2021-2-C3-taylor" "nug1S2GbY3U")
% (code-eevlinksvideo "c3m212ta" "2021-2-C3-taylor" "nug1S2GbY3U")
% (find-c3m212tavideo "0:00" "19/jan/2022")
% (find-c3m212tavideo "6:05" "Algumas definições")

% «video-2»  (to ".video-2")
% (find-ssr-links     "c3m212ta2" "2021-2-C3-taylor-2" "KjlfSQvsFYU")
% (code-eevvideo      "c3m212ta2" "2021-2-C3-taylor-2" "KjlfSQvsFYU")
% (code-eevlinksvideo "c3m212ta2" "2021-2-C3-taylor-2" "KjlfSQvsFYU")
% (find-c3m212ta2video "0:00")

\documentclass[oneside,12pt]{article}
\usepackage[colorlinks,citecolor=DarkRed,urlcolor=DarkRed]{hyperref} % (find-es "tex" "hyperref")
\usepackage{amsmath}
\usepackage{amsfonts}
\usepackage{amssymb}
\usepackage{pict2e}
\usepackage[x11names,svgnames]{xcolor} % (find-es "tex" "xcolor")
\usepackage{colorweb}                  % (find-es "tex" "colorweb")
%\usepackage{tikz}
%
% (find-dn6 "preamble6.lua" "preamble0")
%\usepackage{proof}   % For derivation trees ("%:" lines)
%\input diagxy        % For 2D diagrams ("%D" lines)
%\xyoption{curve}     % For the ".curve=" feature in 2D diagrams
%
\usepackage{edrx21}               % (find-LATEX "edrx21.sty")
\input edrxaccents.tex            % (find-LATEX "edrxaccents.tex")
\input edrx21chars.tex            % (find-LATEX "edrx21chars.tex")
\input edrxheadfoot.tex           % (find-LATEX "edrxheadfoot.tex")
\input edrxgac2.tex               % (find-LATEX "edrxgac2.tex")
\usepackage{emaxima}              % (find-LATEX "emaxima.sty")
%
%\usepackage[backend=biber,
%   style=alphabetic]{biblatex}            % (find-es "tex" "biber")
%\addbibresource{catsem-slides.bib}        % (find-LATEX "catsem-slides.bib")
%
% (find-es "tex" "geometry")
\usepackage[a6paper, landscape,
            top=1.5cm, bottom=.25cm, left=1cm, right=1cm, includefoot
           ]{geometry}
%
\begin{document}

%\catcode`\^^J=10
%\directlua{dofile "dednat6load.lua"}  % (find-LATEX "dednat6load.lua")
%
% %L dofile "2021pict2e.lua"           -- (find-LATEX "2021pict2e.lua")
% %L Pict2e.__index.suffix = "%"
% \pu
% \def\pictgridstyle{\color{GrayPale}\linethickness{0.3pt}}
% \def\pictaxesstyle{\linethickness{0.5pt}}

% %L dofile "edrxtikz.lua"  -- (find-LATEX "edrxtikz.lua")
% %L dofile "edrxpict.lua"  -- (find-LATEX "edrxpict.lua")
% \pu

% «defs»  (to ".defs")
% (find-LATEX "edrx21defs.tex" "colors")
% (find-LATEX "edrx21.sty")

\def\u#1{\par{\footnotesize \url{#1}}}

\def\drafturl{http://angg.twu.net/LATEX/2021-2-C3.pdf}
\def\drafturl{http://angg.twu.net/2021.2-C3.html}
\def\draftfooter{\tiny \href{\drafturl}{\jobname{}} \ColorBrown{\shorttoday{} \hours}}

\def\derivs{\mathsf{derivs}}
\def\frt#1{\frac{f^{(#1)}(0)}{#1!}}



%  _____ _ _   _                               
% |_   _(_) |_| | ___   _ __   __ _  __ _  ___ 
%   | | | | __| |/ _ \ | '_ \ / _` |/ _` |/ _ \
%   | | | | |_| |  __/ | |_) | (_| | (_| |  __/
%   |_| |_|\__|_|\___| | .__/ \__,_|\__, |\___|
%                      |_|          |___/      
%
% «title»  (to ".title")
% (c3m212tap 1 "title")
% (c3m212taa   "title")

\thispagestyle{empty}

\begin{center}

\vspace*{1.2cm}

{\bf \Large Cálculo 3 - 2021.2}

\bsk

Aula 24: Séries de Taylor

\bsk

Eduardo Ochs - RCN/PURO/UFF

\url{http://angg.twu.net/2021.2-C3.html}

\end{center}

\newpage

% «ideia-basica-1»  (to ".ideia-basica-1")
% (c3m212tap 2 "ideia-basica-1")
% (c3m212taa   "ideia-basica-1")
% (c3m211tap 2 "taylor-1")
% (c3m211taa   "taylor-1")

{\bf A idéia básica}


\scalebox{0.75}{\def\colwidth{12cm}\firstcol{

Digamos que $f(x)$ é um polinômio.

Digamos que o grau dele é 4, pra simplificar.

Digamos que $f(x) = a + bx + cx^2 + dx^3 + ex^4$.

Então:
%
\def\b#1{\hbox to 25pt{\hss$#1$\hss}}
%
$$\begin{array}{rcl}
      f(x) &=& \b{a} + \b{bx} + \b{cx^2} + \b{dx^3} + \b{ex^4} \\
     f'(x) &=& \b{b} + \b{2cx} + \b{3dx^2} + \b{4ex^3} \\
    f''(x) &=& \b{2c} + \b{6dx} + \b{12ex^2} \\
   f'''(x) &=& \b{6d} + \b{24ex} \\
  f''''(x) &=& \b{24e} \\
  \end{array}
  %
  \begin{array}{rcl}
      f(0) &=& a \\
     f'(0) &=& b \\
    f''(0) &=& 2c \\
   f'''(0) &=& 6d \\
  f''''(0) &=& 24e \\
  \end{array}
  %
  \begin{array}{rcl}
    \phantom{m}
    a &=&     f(0) \\
    b &=&    f'(0) \\
    c &=&   f''(0)/2 \\
    d &=&  f'''(0)/6 \\
    e &=& f''''(0)/24 \\
  \end{array}
  %
$$

E portanto:
%
$$f(x) \;\; = \;\;
    f(0) 
  + f'(0) x
  + \frac{f''(0)}{2} x^2
  + \frac{f'''(0)}{6} x^3
  + \frac{f''''(0)}{24} x^4
$$ 

%}\anothercol{
}}



\newpage

% «ideia-basica-2»  (to ".ideia-basica-2")
% (c3m212tap 3 "ideia-basica-2")
% (c3m212taa   "ideia-basica-2")

{\bf A idéia básica (2)}

\scalebox{0.72}{\def\colwidth{14cm}\firstcol{

Agora vamos tentar generalizar isso.

Digamos que $f(x)$ é um polinômio.

Digamos que o grau dele é $k$, e que \ColorRed{por enquanto} $k=4$.

Digamos que $f(x) = a_0 + a_1x + a_2x^2 + a_3x^3 + a_4x^4$.

A notação $f^{(k)}$, como o $(k)$ entre parênteses, quer dizer

``$f$ derivada $k$ vezes''. Por exemplo, $f^{(4)} = f''''$, e $f^{(0)} = f$.

Então:
%
\def\b#1{\hbox to 27.5pt{\hss$#1$\hss}}
%
$$\begin{array}{rcl}
  f^{(0)}(x) &=&   \b{a_0} +   \b{a_1x} +   \b{a_2x^2} +  \b{a_3x^3} + \b{a_4x^4} \\
  f^{(1)}(x) &=&   \b{a_1} +  \b{2a_2x} +  \b{3a_3x^2} + \b{4a_4x^3} \\
  f^{(2)}(x) &=&  \b{2a_2} +  \b{6a_3x} + \b{12a_4x^2} \\
  f^{(3)}(x) &=&  \b{6a_3} + \b{24a_4x} \\
  f^{(4)}(x) &=& \b{24a_4} \\
  \end{array}
  %
  \begin{array}{rcl}
    \phantom{ii}
  f^{(0)}(0) &=& 0!\,a_0 \\
  f^{(1)}(0) &=& 1!\,a_1 \\
  f^{(2)}(0) &=& 2!\,a_2 \\
  f^{(3)}(0) &=& 3!\,a_3 \\
  f^{(4)}(0) &=& 4!\,a_4 \\
  \end{array}
  %
  \begin{array}{rcl}
    \phantom{m}
  a_0 &=& f^{(0)}(0)/0! \\
  a_1 &=& f^{(1)}(0)/1! \\
  a_2 &=& f^{(2)}(0)/2! \\
  a_3 &=& f^{(3)}(0)/3! \\
  a_4 &=& f^{(4)}(0)/4! \\
  \end{array}
  %
$$

E portanto:
%
$$\def\frt#1{\frac{f^{(#1)}(0)}{#1!}}
  f(x) \;\; = \;\;
    \frt0 x^0
  + \frt1 x^1
  + \frt2 x^2
  + \frt3 x^3
  + \frt4 x^4
    \;\; = \;\;
  \D \sum_{k=0}^{4} \frt{k} x^k
$$ 

%}\anothercol{
}}



\newpage

% «exercicio-1»  (to ".exercicio-1")
% (c3tap 4 "exercicio-1")
% (c3taa   "exercicio-1")

{\bf Exercício 1.}

A fórmula do slide anterior também funciona

pra polinômios com grau menor que 4.

Verifique o que ela faz quando
%
$$f(x) = 42x^2 + 99x + 200.$$

Lembre que no ensino médio você era obrigado

a ``simplificar'' $4·5·6·999$ para 119880, mas

em Cálculo 2 você tem que encontrar jeitos

de escrever que sejam mais simples de ler e

de verificar... pra gente \ColorRed{em certos contextos}

$4·5·6·999$ é mais ``simples'' que 119880.




\newpage

% «exercicio-2»  (to ".exercicio-2")
% (c3tap 5 "exercicio-2")
% (c3taa   "exercicio-2")

{\bf Exercício 2.}

\scalebox{0.9}{\def\colwidth{12cm}\firstcol{

Tente aplicar a fórmula $(*)$ abaixo
%
$$f(x) \;\; = \;\;
  \D \sum_{k=0}^{4} \frt{k} x^k
  \qquad \qquad (*)
$$ 

a esta $f$ aqui: $f(x) = 200x^5$.

\msk

a) O que acontece?

\msk

b) Tente escrever em detalhes o que dá errado.

Você vai precisar de notação matemática \ColorRed{E} português.

Tente aprender as convenções que eu usei nos PDFs

e as convenções que os livros usam, e lembre que se

você começar escrevendo uma igualdade qualquer leitor

que não seja muito seu amigo vai interpretá-la

como uma \ColorRed{afirmação}.

%}\anothercol{
}}




\newpage

% «derivs-e-derivs0»  (to ".derivs-e-derivs0")
% (c3m212tap 5 "derivs-e-derivs0")
% (c3m212taa    "derivs-e-derivs0")
% (c3m211tap 2 "taylor-1")
% (c3m211taa   "taylor-1")

{\bf As operações $\derivs$ e $\derivs_0$}


\scalebox{0.75}{\def\colwidth{12cm}\firstcol{

Sejam $\derivs$ e $\derivs_0$ as seguintes operações --

que vão nos ajudar muito nas contas:
%
$$\begin{array}{rcl}
  \derivs(f)   &=& (f, f', f'', f''', \ldots) \\
  \derivs_0(f) &=& (f(0), f'(0), f''(0), f'''(0), \ldots) \\
  \end{array}
$$

Repare que $\derivs(f)$ retorna uma sequência infinita de \ColorRed{funções}

e $\derivs_0(f)$ retorna uma sequência infinita de \ColorRed{números}.

\msk

Um exemplo: se $f(x) = ax^2 + bx + c$, então:
%
$$\begin{array}{rclcrcl}
  f(x)   &=& ax^2 + bx + c,    && f(0)   &=& c, \\
  f'(x)  &=& 2ax + b,          && f'(0)  &=& b, \\
  f''(x) &=& 2a,               && f''(0) &=& 2a, \\
  f'''(x) &=& 0,               && f'''(0) &=& 0, \\
  \end{array}
$$

e:
%
$$\begin{array}{rcl}
  \derivs(f) &=& (ax^2 + bx + c, \; 2ax + b, \; 2a, \; 0, 0, 0, \ldots) \\
  \derivs_0(f) &=& (c, b, 2a, 0, 0, 0, \ldots) \\
  \end{array}
$$

}\anothercol{
}}

\newpage

% «exercicio-3»  (to ".exercicio-3")
% (c3m212tap 7 "exercicio-3")
% (c3m212taa   "exercicio-3")

{\bf Algumas definições}

\scalebox{0.8}{\def\colwidth{6.5cm}\firstcol{

Isto aqui
%
$$\D \sum_{k=0}^{n} \frt{k} x^k
$$ 

é a {\sl série de Taylor da função $f$

no ponto 0 truncada até grau $n$},

e isto aqui
%

$$\begin{array}{rr}
  & \D \lim_{n→∞} \sum_{k=0}^{n} \frt{k} x^k, \\[15pt]
  \text{ou:}
  & \D            \sum_{k=0}^{∞} \frt{k} x^k
  \end{array}
$$ 

é a {\sl série de Taylor da função $f$

no ponto 0}.

}\def\colwidth{10cm}%
 \anothercol%
{

{\bf Exercício 3.}

Seja $f(x) = \sen x$.

\msk

a) Calcule as 8 primeiras

componentes de $\derivs(f)$.

\msk

b) Calcule as 8 primeiras

componentes de $\derivs_0(f)$.

\msk

c) Calcule a série de Taylor

de $\sen x$ truncada até grau 7.

\msk

d) Seja $g(x)$ a série de Taylor

de $\sen x$ truncada até grau 7;

Calcule $g(0.1)$ \ColorRed{na mão} e

compare o seu resultado com

o resultado de calcular $\sen 0.1$

na calculadora ou no computador.


}}


\newpage

% «approx»  (to ".approx")
% (c3m212tap 8 "approx")
% (c3m212taa   "approx")

{\bf A notação com `$≈$'}

\scalebox{0.9}{\def\colwidth{12cm}\firstcol{

O sinal `$≈$' que dizer ``é aproximadamente igual a'',

mas ele não diz quão boa é a aproximação...

Estas duas afirmações são ambas verdadeiras:
%
$$\begin{array}{rcl}
  f(0.42) &≈& f(0) + f'(0)·0.42 + \frac{f''(0)}{2}(0.42)^2 \\[5pt]
  f(0.42) &≈& f(0) + f'(0)·0.42
              + \frac{f'' (0)}{2}(0.42)^2
              + \frac{f'''(0)}{6}(0.42)^3 \\
  \end{array}
$$

Até daria pra formalizar essa afirmação aqui

usando um limite,
%
$$f(x) \;\;=\;\;
  f(0) + f'(0)·x + \frac{f''(0)}{2}x^2
$$

Mas eu não sei como formalizar precisamente

a versão com $0.42$ no lugar do $x$... \quad $\frown$

%}\anothercol{
}}

\newpage

{\bf A tradução pra notação de físicos}

Temos:
%
$$\begin{array}{rcl}
  f(x) &≈& f(0) + f'(0)x + \frac{f''(0)}{2}x^2 \\
  \end{array}
$$

Acho que vocês devem conseguir acreditar nisso aqui...

(a gente pode checar os detalhes depois!)
%
$$\begin{array}{rcl}
  g(x_0 + Δx) &≈& g(x_0) + g'(x_0)Δx + \frac{g''(x_0)}{2}(Δx)^2 \\[2.5pt]
  h(x + Δx) &≈& h(x) + h'(x)Δx + \frac{h''(x)}{2}(Δx)^2 \\
  \end{array}
$$

E se $y=y(x)$ então:
%
$$\begin{array}{rcl}
  y(x+Δx) &≈& y + y_x Δx + \frac{y_{xx}}{2} (Δx)^2 \\[2.5pt]
  y(x+Δx) &≈& y + y_x Δx + \frac{y_{xx}}{2} (Δx)^2 + \frac{y_{xxx}}{6} (Δx)^3 \\
  \end{array}
$$

\newpage

{\bf As versões truncadas de $\derivs$, $\derivs_0$ e $\derivs_p$}


\scalebox{0.85}{\def\colwidth{13cm}\firstcol{

Vamos definir $\derivs^n$ e $\derivs_0^n$ como as versões ``truncadas até grau $n$''

de $\derivs$ e $\derivs_0$...

\msk


$\derivs^n(f)$ vai ser a lista com as primeiras $\ColorRed{n+1}$ entradas de $\derivs^n(f)$, e

$\derivs^n_0(f)$ vai ser a lista com as primeiras $\ColorRed{n+1}$ entradas de $\derivs_0^n(f)$.

\msk

Além disso $\derivs_p(f)$ vai ser a lista infinita $(f(p), f'(p), f''(p), \ldots)$, e

$\derivs_p^n(f)$ vai ser a lista com as primeiras $\ColorRed{n+1}$ entradas de $\derivs_p^n(f)$.

\msk

Exemplo:
%
$$\derivs_{42}^2(f) \;\;=\;\; (f(42), f'(42), f''(42)).
$$

Vamos nos referir a $\derivs_p^n(f)$ como ``as derivadas de $f$ até grau $n$ no

ponto $p$''. Repare que $f(42)$ é a ``derivada de $f$ de grau 0 no ponto 42'',

$f'(42)$ é a ``derivada de $f$ de grau 1 no ponto 42'', etc...

\msk

Antes o termo ``grau'' não servia pra falar de número de vezes que uma

função foi derivada, mas agora passou a servir. \quad $\smile$


%}\anothercol{
}}

\newpage

% «exercicio-4»  (to ".exercicio-4")
% (c3m212tap 11 "exercicio-4")
% (c3m212taa    "exercicio-4")

{\bf Exercício 4.}

Digamos que $x_0 = 10$, $f(x)=x^3$, $y_0=f(x_0)$, $g(y)=\sen y$.

\msk

a) Calcule $\derivs_{x_0}^1(f(x))$.

\ssk

b) Calcule $\derivs_{y_0}^1(g(y))$.

\ssk

c) Calcule $\derivs_{x_0}^1(g(f(x)))$.


\bsk

Seja $h(x) = g(f(x))$ --- ou seja, $h = g∘f$.

\msk

d) Calcule $\derivs_{x_0}^2(h(x))$.

\newpage

% «exercicio-4-maxima»  (to ".exercicio-4-maxima")

{\bf Exercício 4: gabarito em código}



% (setq eepitch-preprocess-regexp "^")
% (setq eepitch-preprocess-regexp "^%T ")
%
%T  (eepitch-maxima)
%T  (eepitch-kill)
%T  (eepitch-maxima)
%T load("/usr/share/emacs/site-lisp/maxima/emaxima.lisp")$
%T display2d:'emaxima$
%T f  : x^3;
%T g  : sin(y);
%T h  : subst([y=f], g);
%T diff(h, x);
%T [h, diff(h, x)];
%T x0 : 10;
%T y0 : subst([x=x0], f);
%T z0 : subst([x=x0], h);
%T subst([x=x0], [h, diff(h, x)]);

{\footnotesize

\begin{maximasession}
\maximaoutput*
\i3. f  : x^3; \\
\o3. x^3 \\
\i4. g  : sin(y); \\
\o4. \sin y \\
\i5. h  : subst([y=f], g); \\
\o5. \sin x^3 \\
\i6. diff(h, x); \\
\o6. 3\,x^2\,\cos x^3 \\
\i7. [h, diff(h, x)]; \\
\o7. \left[ \sin x^3 , 3\,x^2\,\cos x^3 \right]  \\
\i8. x0 : 10; \\
\o8. 10 \\
\i9. y0 : subst([x=x0], f); \\
\o9. 1000 \\
\i10. z0 : subst([x=x0], h); \\
\o10. \sin 1000 \\
\i11. subst([x=x0], [h, diff(h, x)]); \\
\o11. \left[ \sin 1000 , 300\,\cos 1000 \right]  \\
\end{maximasession}

Obs: aí não tem a resposta do item d...

}

\newpage


% «exercicio-5»  (to ".exercicio-5")
% (c3m212tap 15 "exercicio-5")
% (c3m212taa    "exercicio-5")

{\bf Exercício 5.}


Este exercício é uma versão mais geral do exercício 4.

Digamos que $f$ e $g$ são funções suaves de $\R$ em $\R$.

(Uma função é ``suave'' quando ela pode ser derivada

infinitas vezes. A função $|x|$ não é suave).

Digamos que $x_0∈\R$, $y_0=f(x_0)$, e $h=g∘f$.

\msk

a) Calcule $\derivs_{x_0}^2(h(x))$.

\msk

Repare que neste caso ``calcule'' quer dizer algo como

``expanda e simplifique  a expressão que você obtiver''...

Existem vários tipos de expansão e simplificação, e os

programas de computação simbólica dão um nome pra cada

tipo e permitem que você escolha quais vão ser aplicadas.



\newpage

{\bf Exercício 5 (cont.)}


\scalebox{0.9}{\def\colwidth{12cm}\firstcol{

Agora sejam $y=y(x)=f(x)$ e $z=z(y)=g(y)$.

\msk

b) Traduza o seu $\derivs_{x_0}^2(h(x))$ do item (a)

pra notação de físicos.

\msk

Dica (pequena): $\ddx g(f(x_0)) = z_y y_x$.


\bsk
\bsk

c) Calcule $\derivs_{x_0}^{\ColorRed{3}}(z)$ usando notação de físicos.

\msk

Nas próximas páginas eu pus um ``gabarito em código'' do

item (b). O modo mais fácil de usar a ``notação de físicos''

no Maxima é traduzir entre ela e a ``notação de matemáticos''

sempre que necessário. No item (c) as contas em ``notação de

matemáticos'' ficam gigantescas, mas se você conseguir fazer

elas todas em ``notação de físicos'' elas ficam pequenas.

%}\anothercol{
}}


\newpage

% (setq eepitch-preprocess-regexp "^")
% (setq eepitch-preprocess-regexp "^%T ")
%
%T  (eepitch-maxima)
%T  (eepitch-kill)
%T  (eepitch-maxima)
%T load("/usr/share/emacs/site-lisp/maxima/emaxima.lisp")$
%T display2d:'emaxima$
%T gradef(y  (x), y_x (x));
%T gradef(y_x(x), y_xx(x));
%T gradef(z  (y), z_y (y));
%T gradef(z_y(y), z_yy(y));
%T z     : z(y(x));
%T z__x  : diff(z,    x);
%T z__xx : diff(z__x, x);
%T gradefs;
%T ex : z__xx;
%T ex : subst([y   (x)=y],      ex);
%T ex : subst([y_x (x)=y_x],    ex);
%T ex : subst([y_xx(x)=y_xx],   ex);
%T ex : subst([z   (y)=z],      ex);
%T ex : subst([z_y (y)=z_y],    ex);
%T ex : subst([z_yy(y)=z_yy],   ex);
%T ex : expand(ex);

{\footnotesize

\begin{maximasession}
\maximaoutput*
\i3. gradef(y  (x), y_x (x)); \\
\o3. y\left(x\right) \\
\i4. gradef(y_x(x), y_xx(x)); \\
\o4. \mathrm{y\_x}\left(x\right) \\
\i5. gradef(z  (y), z_y (y)); \\
\o5. z\left(y\right) \\
\i6. gradef(z_y(y), z_yy(y)); \\
\o6. \mathrm{z\_y}\left(y\right) \\
\i7. z     : z(y(x)); \\
\o7. z\left(y\left(x\right)\right) \\
\i8. z__x  : diff(z,    x); \\
\o8. \mathrm{y\_x}\left(x\right)\,\mathrm{z\_y}\left(y\left(x\right)\right) \\
\i9. z__xx : diff(z__x, x); \\
\o9. \mathrm{y\_x}\left(x\right)^2\,\mathrm{z\_yy}\left(y\left(x\right)\right)+\mathrm{y\_xx}\left(x\right)\,\mathrm{z\_y}\left(y\left(x\right)\right) \\
\i10. gradefs; \\
\o10. \left[ y\left(x\right) , \mathrm{y\_x}\left(x\right) , z\left(y\right) , \mathrm{z\_y}\left(y\right) \right]  \\
\i11. ex : z__xx; \\
\o11. \mathrm{y\_x}\left(x\right)^2\,\mathrm{z\_yy}\left(y\left(x\right)\right)+\mathrm{y\_xx}\left(x\right)\,\mathrm{z\_y}\left(y\left(x\right)\right) \\
\i12. ex : subst([y   (x)=y],      ex); \\
\o12. \mathrm{y\_x}\left(x\right)^2\,\mathrm{z\_yy}\left(y\right)+\mathrm{y\_xx}\left(x\right)\,\mathrm{z\_y}\left(y\right) \\
\i13. ex : subst([y_x (x)=y_x],    ex); \\
\o13. \mathrm{z\_yy}\left(y\right)\,\mathrm{y\_x}^2+\mathrm{y\_xx}\left(x\right)\,\mathrm{z\_y}\left(y\right) \\
\i14. ex : subst([y_xx(x)=y_xx],   ex); \\
\o14. \mathrm{z\_y}\left(y\right)\,\mathrm{y\_xx}+\mathrm{z\_yy}\left(y\right)\,\mathrm{y\_x}^2 \\
\i15. ex : subst([z   (y)=z],      ex); \\
\o15. \mathrm{z\_y}\left(y\right)\,\mathrm{y\_xx}+\mathrm{z\_yy}\left(y\right)\,\mathrm{y\_x}^2 \\
\i16. ex : subst([z_y (y)=z_y],    ex); \\
\o16. \mathrm{y\_xx}\,\mathrm{z\_y}+\mathrm{z\_yy}\left(y\right)\,\mathrm{y\_x}^2 \\
\i17. ex : subst([z_yy(y)=z_yy],   ex); \\
\o17. \mathrm{y\_x}^2\,\mathrm{z\_yy}+\mathrm{y\_xx}\,\mathrm{z\_y} \\
\end{maximasession}

}


%\printbibliography

\GenericWarning{Success:}{Success!!!}  % Used by `M-x cv'

\end{document}

%  ____  _             _         
% |  _ \(_)_   ___   _(_)_______ 
% | | | | \ \ / / | | | |_  / _ \
% | |_| | |\ V /| |_| | |/ /  __/
% |____// | \_/  \__,_|_/___\___|
%     |__/                       
%
% «djvuize»  (to ".djvuize")
% (find-LATEXgrep "grep --color -nH --null -e djvuize 2020-1*.tex")

 (eepitch-shell)
 (eepitch-kill)
 (eepitch-shell)
# (find-fline "~/2021.2-C3/")
# (find-fline "~/LATEX/2021-2-C3/")
# (find-fline "~/bin/djvuize")

cd /tmp/
for i in *.jpg; do echo f $(basename $i .jpg); done

f () { rm -v $1.pdf;  textcleaner -f 50 -o  5 $1.jpg $1.png; djvuize $1.pdf; xpdf $1.pdf }
f () { rm -v $1.pdf;  textcleaner -f 50 -o 10 $1.jpg $1.png; djvuize $1.pdf; xpdf $1.pdf }
f () { rm -v $1.pdf;  textcleaner -f 50 -o 20 $1.jpg $1.png; djvuize $1.pdf; xpdf $1.pdf }

f () { rm -fv $1.png $1.pdf; djvuize $1.pdf }
f () { rm -fv $1.png $1.pdf; djvuize WHITEBOARDOPTS="-m 1.0 -f 15" $1.pdf; xpdf $1.pdf }
f () { rm -fv $1.png $1.pdf; djvuize WHITEBOARDOPTS="-m 1.0 -f 30" $1.pdf; xpdf $1.pdf }
f () { rm -fv $1.png $1.pdf; djvuize WHITEBOARDOPTS="-m 1.0 -f 45" $1.pdf; xpdf $1.pdf }
f () { rm -fv $1.png $1.pdf; djvuize WHITEBOARDOPTS="-m 0.5" $1.pdf; xpdf $1.pdf }
f () { rm -fv $1.png $1.pdf; djvuize WHITEBOARDOPTS="-m 0.25" $1.pdf; xpdf $1.pdf }
f () { cp -fv $1.png $1.pdf       ~/2021.2-C3/
       cp -fv        $1.pdf ~/LATEX/2021-2-C3/
       cat <<%%%
% (find-latexscan-links "C3" "$1")
%%%
}

f 20201213_area_em_funcao_de_theta
f 20201213_area_em_funcao_de_x
f 20201213_area_fatias_pizza



%  __  __       _        
% |  \/  | __ _| | _____ 
% | |\/| |/ _` | |/ / _ \
% | |  | | (_| |   <  __/
% |_|  |_|\__,_|_|\_\___|
%                        
% <make>

 (eepitch-shell)
 (eepitch-kill)
 (eepitch-shell)
# (find-LATEXfile "2019planar-has-1.mk")
make -f 2019.mk STEM=2021-2-C3-taylor veryclean
make -f 2019.mk STEM=2021-2-C3-taylor pdf

% Local Variables:
% coding: utf-8-unix
% ee-tla: "c3ta"
% ee-tla: "c3m212ta"
% End:
