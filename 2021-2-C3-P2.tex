% (find-LATEX "2021-2-C3-P2.tex")
% (defun c () (interactive) (find-LATEXsh "lualatex -record 2021-2-C3-P2.tex" :end))
% (defun C () (interactive) (find-LATEXsh "lualatex 2021-2-C3-P2.tex" "Success!!!"))
% (defun D () (interactive) (find-pdf-page      "~/LATEX/2021-2-C3-P2.pdf"))
% (defun d () (interactive) (find-pdftools-page "~/LATEX/2021-2-C3-P2.pdf"))
% (defun e () (interactive) (find-LATEX "2021-2-C3-P2.tex"))
% (defun o () (interactive) (find-LATEX "2021-2-C3-P2.tex"))
% (defun u () (interactive) (find-latex-upload-links "2021-2-C3-P2"))
% (defun v () (interactive) (find-2a '(e) '(d)))
% (defun d0 () (interactive) (find-ebuffer "2021-2-C3-P2.pdf"))
% (defun cv () (interactive) (C) (ee-kill-this-buffer) (v) (g))
%          (code-eec-LATEX "2021-2-C3-P2")
% (find-pdf-page   "~/LATEX/2021-2-C3-P2.pdf")
% (find-sh0 "cp -v  ~/LATEX/2021-2-C3-P2.pdf /tmp/")
% (find-sh0 "cp -v  ~/LATEX/2021-2-C3-P2.pdf /tmp/pen/")
%     (find-xournalpp "/tmp/2021-2-C3-P2.pdf")
%   file:///home/edrx/LATEX/2021-2-C3-P2.pdf
%               file:///tmp/2021-2-C3-P2.pdf
%           file:///tmp/pen/2021-2-C3-P2.pdf
% http://angg.twu.net/LATEX/2021-2-C3-P2.pdf
% (find-LATEX "2019.mk")
% (find-CN-aula-links "2021-2-C3-P2" "3" "c3m212p2" "c3p2")

% «.defs»		(to "defs")
% «.defs-T-and-B»	(to "defs-T-and-B")
% «.title»		(to "title")
% «.regras-e-dicas»	(to "regras-e-dicas")
% «.questao-2»		(to "questao-2")
% «.ideias»		(to "ideias")
%
% «.djvuize»		(to "djvuize")



% <videos>
% Video (not yet):
% (find-ssr-links     "c3m212p2" "2021-2-C3-P2")
% (code-eevvideo      "c3m212p2" "2021-2-C3-P2")
% (code-eevlinksvideo "c3m212p2" "2021-2-C3-P2")
% (find-c3m212p2video "0:00")

\documentclass[oneside,12pt]{article}
\usepackage[colorlinks,citecolor=DarkRed,urlcolor=DarkRed]{hyperref} % (find-es "tex" "hyperref")
\usepackage{amsmath}
\usepackage{amsfonts}
\usepackage{amssymb}
\usepackage{pict2e}
\usepackage[x11names,svgnames]{xcolor} % (find-es "tex" "xcolor")
\usepackage{colorweb}                  % (find-es "tex" "colorweb")
%\usepackage{tikz}
%
% (find-dn6 "preamble6.lua" "preamble0")
%\usepackage{proof}   % For derivation trees ("%:" lines)
%\input diagxy        % For 2D diagrams ("%D" lines)
%\xyoption{curve}     % For the ".curve=" feature in 2D diagrams
%
\usepackage{edrx21}               % (find-LATEX "edrx21.sty")
\input edrxaccents.tex            % (find-LATEX "edrxaccents.tex")
\input edrx21chars.tex            % (find-LATEX "edrx21chars.tex")
\input edrxheadfoot.tex           % (find-LATEX "edrxheadfoot.tex")
\input edrxgac2.tex               % (find-LATEX "edrxgac2.tex")
%\usepackage{emaxima}              % (find-LATEX "emaxima.sty")
%
%\usepackage[backend=biber,
%   style=alphabetic]{biblatex}            % (find-es "tex" "biber")
%\addbibresource{catsem-slides.bib}        % (find-LATEX "catsem-slides.bib")
%
% (find-es "tex" "geometry")
\usepackage[a6paper, landscape,
            top=1.5cm, bottom=.25cm, left=1cm, right=1cm, includefoot
           ]{geometry}
%
\begin{document}

\catcode`\^^J=10
\directlua{dofile "dednat6load.lua"}  % (find-LATEX "dednat6load.lua")
%
%L dofile "2021pict2e.lua"           -- (find-LATEX "2021pict2e.lua")
%L Pict2e.__index.suffix = "%"
\pu
\def\pictgridstyle{\color{GrayPale}\linethickness{0.3pt}}
\def\pictaxesstyle{\linethickness{0.5pt}}

% %L dofile "edrxtikz.lua"  -- (find-LATEX "edrxtikz.lua")
% %L dofile "edrxpict.lua"  -- (find-LATEX "edrxpict.lua")
% \pu

% «defs»  (to ".defs")
% (find-LATEX "edrx21defs.tex" "colors")
% (find-LATEX "edrx21.sty")

\def\u#1{\par{\footnotesize \url{#1}}}

\def\drafturl{http://angg.twu.net/LATEX/2021-2-C3.pdf}
\def\drafturl{http://angg.twu.net/2021.2-C3.html}
\def\draftfooter{\tiny \href{\drafturl}{\jobname{}} \ColorBrown{\shorttoday{} \hours}}

% «defs-T-and-B»  (to ".defs-T-and-B")
% (c3m202p1p 6 "questao-2")
% (c3m202p1a   "questao-2")
\long\def\ColorOrange#1{{\color{orange!90!black}#1}}
\def\T(Total: #1 pts){{\bf(Total: #1)}}
\def\T(Total: #1 pts){{\bf(Total: #1 pts)}}
\def\T(Total: #1 pts){\ColorRed{\bf(Total: #1 pts)}}
\def\B       (#1 pts){\ColorOrange{\bf(#1 pts)}}





%  _____ _ _   _                               
% |_   _(_) |_| | ___   _ __   __ _  __ _  ___ 
%   | | | | __| |/ _ \ | '_ \ / _` |/ _` |/ _ \
%   | | | | |_| |  __/ | |_) | (_| | (_| |  __/
%   |_| |_|\__|_|\___| | .__/ \__,_|\__, |\___|
%                      |_|          |___/      
%
% «title»  (to ".title")
% (c3m212p2p 1 "title")
% (c3m212p2a   "title")

\thispagestyle{empty}

\begin{center}

\vspace*{1.2cm}

{\bf \Large Cálculo 3 - 2021.2}

\bsk

Segunda prova (P2)

\bsk

Eduardo Ochs - RCN/PURO/UFF

\url{http://angg.twu.net/2021.2-C3.html}

\end{center}

\newpage

% «regras-e-dicas»  (to ".regras-e-dicas")
% (c3m212p2p 2 "regras-e-dicas")
% (c3m212p2a   "regras-e-dicas")

As regras e dicas são as mesmas dos mini-testes,

exceto que a prova será disponibilizada às 21:00

da quarta, 2/fev/2022, e você deverá entregá-la

até as 21:00 da quinta, 3/fev/2022.

\msk

Várias das questões desta prova são baseadas nos

exercícios sobre funções homogêneas que nós discutimos

na última aula antes da prova. Leia o log aqui:

\msk

{\footnotesize

%    http://angg.twu.net/tmp/C3-M1-RCN-PURO-2021.2-4.pdf#page=6
\url{http://angg.twu.net/tmp/C3-M1-RCN-PURO-2021.2-4.pdf\#page=6}

}

\msk

Obs: esse log vai ser deletado logo depois da prova.

\newpage

{\bf Questão 1.}

\T(Total: 2.5 pts)

Digamos que $(x_0,y_0)=(4,3)$ e que $f(x,y)$ é uma

função que obedece isto aqui:
%
$$f(x_0+kΔx, y_0+kΔy) \;\;=\;\; k^3 f(x_0+Δx, y_0+Δy).$$

Complete o diagrama de numerozinhos da

página seguinte. Os números dele indicam valores

que você sabe, como por exemplo $f(0,1)=3$, e

os pontos de interrogação dele indicam valores que

você tem que descobrir... por exemplo ``$g(0,2)=\ColorRed{?}$''.


\newpage

% (find-LATEX "edrx21.sty" "picture-cells")
\unitlength=8pt
\celllower=2.5pt
\def\cellfont{\scriptsize}
\def\cellfont{\footnotesize}

\unitlength=20pt

%L Pict2e.__index.axes_ = "\\linethickness{0.1pt}"
%L Pict2e.__index.axes_ = "\\color{gray}"
%L
%L Pict2e.new()
%L   :setbounds(v(0,0), v(6,5))
%L     :axesandticks()
%L     :run(function (p)
%L        local f = function(x, y, z)
%L            p:puttext(v(x+4,y+3), z or "\\ColorRed{?}")
%L          end
%L        f(1,1,2); f(2,2); f(0,0); f(-1,-1); f(-2,-2)
%L        f(1,0,5); f(2,0); f(-1,0); f(-2,0)
%L        f(-1,1); f(-2,2,8); f(1,-1); f(2,-2);
%L        f(-1,2,4); f(1,-2)
%L       end)
%L   :bepc()
%L   :def("Foo")
%L   :output()
\pu

$$\Foo
$$


\newpage

% «questao-2»  (to ".questao-2")
% (c3m212p2p 5 "questao-2")
% (c3m212p2a   "questao-2")

{\bf Questão 2.}

\T(Total: 7.5 pts)

\scalebox{0.8}{\def\colwidth{8cm}\firstcol{

Digamos que $f(x,y)$ é uma função

que obedece isto aqui:
%
$$\begin{array}{c}
  ∀x,y,k. \; f(kx,ky) = k^3 f(x,y), \\[2.5pt]
  ∀y. \; f(1,y) = (y+1)·(y-0.5),    \\[2.5pt]
  ∀y. \; f(0,y) = 0.                \\
  \end{array}
$$


a) \B(2.0 pts) Faça o diagrama de sinais

da função $f(x,y)$ na reta $x=1$. 

\ssk

b) \B(1.0 pts) Faça o diagrama de sinais

da função $f(x,y)$ na reta $x=2$. 

\ssk

c) \B(1.0 pts) Faça o diagrama de sinais

da função $f(x,y)$ na reta $x=-1$. 

\ssk

d) \B(3.5 pts) Faça o diagrama de sinais

da função $f(x,y)$ em todo o plano $\R^2$.

%}\anothercol{
}}




\newpage

% «ideias»  (to ".ideias")
% (c3m212p2p 2 "ideias")
% (c3m212p2a   "ideias")

% Função quadrática centrada num ponto estranho

% Função homogênea

%\printbibliography

\GenericWarning{Success:}{Success!!!}  % Used by `M-x cv'

\end{document}

%  ____  _             _         
% |  _ \(_)_   ___   _(_)_______ 
% | | | | \ \ / / | | | |_  / _ \
% | |_| | |\ V /| |_| | |/ /  __/
% |____// | \_/  \__,_|_/___\___|
%     |__/                       
%
% «djvuize»  (to ".djvuize")
% (find-LATEXgrep "grep --color -nH --null -e djvuize 2020-1*.tex")

 (eepitch-shell)
 (eepitch-kill)
 (eepitch-shell)
# (find-fline "~/2021.2-C3/")
# (find-fline "~/LATEX/2021-2-C3/")
# (find-fline "~/bin/djvuize")

cd /tmp/
for i in *.jpg; do echo f $(basename $i .jpg); done

f () { rm -v $1.pdf;  textcleaner -f 50 -o  5 $1.jpg $1.png; djvuize $1.pdf; xpdf $1.pdf }
f () { rm -v $1.pdf;  textcleaner -f 50 -o 10 $1.jpg $1.png; djvuize $1.pdf; xpdf $1.pdf }
f () { rm -v $1.pdf;  textcleaner -f 50 -o 20 $1.jpg $1.png; djvuize $1.pdf; xpdf $1.pdf }

f () { rm -fv $1.png $1.pdf; djvuize $1.pdf }
f () { rm -fv $1.png $1.pdf; djvuize WHITEBOARDOPTS="-m 1.0 -f 15" $1.pdf; xpdf $1.pdf }
f () { rm -fv $1.png $1.pdf; djvuize WHITEBOARDOPTS="-m 1.0 -f 30" $1.pdf; xpdf $1.pdf }
f () { rm -fv $1.png $1.pdf; djvuize WHITEBOARDOPTS="-m 1.0 -f 45" $1.pdf; xpdf $1.pdf }
f () { rm -fv $1.png $1.pdf; djvuize WHITEBOARDOPTS="-m 0.5" $1.pdf; xpdf $1.pdf }
f () { rm -fv $1.png $1.pdf; djvuize WHITEBOARDOPTS="-m 0.25" $1.pdf; xpdf $1.pdf }
f () { cp -fv $1.png $1.pdf       ~/2021.2-C3/
       cp -fv        $1.pdf ~/LATEX/2021-2-C3/
       cat <<%%%
% (find-latexscan-links "C3" "$1")
%%%
}

f 20201213_area_em_funcao_de_theta
f 20201213_area_em_funcao_de_x
f 20201213_area_fatias_pizza



%  __  __       _        
% |  \/  | __ _| | _____ 
% | |\/| |/ _` | |/ / _ \
% | |  | | (_| |   <  __/
% |_|  |_|\__,_|_|\_\___|
%                        
% <make>

 (eepitch-shell)
 (eepitch-kill)
 (eepitch-shell)
# (find-LATEXfile "2019planar-has-1.mk")
make -f 2019.mk STEM=2021-2-C3-P2 veryclean
make -f 2019.mk STEM=2021-2-C3-P2 pdf

% Local Variables:
% coding: utf-8-unix
% ee-tla: "c3p2"
% ee-tla: "c3m212p2"
% End:

