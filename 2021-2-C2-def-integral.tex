% (find-LATEX "2021-2-C2-def-integral.tex")
% (defun c () (interactive) (find-LATEXsh "lualatex -record 2021-2-C2-def-integral.tex" :end))
% (defun C () (interactive) (find-LATEXsh "lualatex 2021-2-C2-def-integral.tex" "Success!!!"))
% (defun D () (interactive) (find-pdf-page      "~/LATEX/2021-2-C2-def-integral.pdf"))
% (defun d () (interactive) (find-pdftools-page "~/LATEX/2021-2-C2-def-integral.pdf"))
% (defun e () (interactive) (find-LATEX "2021-2-C2-def-integral.tex"))
% (defun l () (interactive) (find-LATEX "2021-2-C2-def-integral.lua"))
% (defun o () (interactive) (find-LATEX "2021-1-C2-propriedades-da-integral.tex"))
% (defun p () (interactive) (find-LATEX "2021pict2e.lua"))
% (defun u () (interactive) (find-latex-upload-links "2021-2-C2-def-integral"))
% (defun v () (interactive) (find-2a '(e) '(d)))
% (defun d0 () (interactive) (find-ebuffer "2021-2-C2-def-integral.pdf"))
% (defun cv () (interactive) (C) (ee-kill-this-buffer) (v) (g))
%          (code-eec-LATEX "2021-2-C2-def-integral")
% (find-pdf-page   "~/LATEX/2021-2-C2-def-integral.pdf")
% (find-sh0 "cp -v  ~/LATEX/2021-2-C2-def-integral.pdf /tmp/")
% (find-sh0 "cp -v  ~/LATEX/2021-2-C2-def-integral.pdf /tmp/pen/")
%     (find-xournalpp "/tmp/2021-2-C2-def-integral.pdf")
%   file:///home/edrx/LATEX/2021-2-C2-def-integral.pdf
%               file:///tmp/2021-2-C2-def-integral.pdf
%           file:///tmp/pen/2021-2-C2-def-integral.pdf
% http://angg.twu.net/LATEX/2021-2-C2-def-integral.pdf
% (find-LATEX "2019.mk")
% (find-CN-aula-links "2021-2-C2-def-integral" "2" "c2m212di" "c2di")

% «.video-1»			(to "video-1")
% «.videos-antigos»		(to "videos-antigos")
% «.defs»			(to "defs")
% «.title»			(to "title")
% «.triangle-wave»		(to "triangle-wave")
% «.triangle-wave-more-defs»	(to "triangle-wave-more-defs")
% «.SixApproxsAndIntegral»	(to "SixApproxsAndIntegral")
% «.flutuando-4»		(to "flutuando-4")
% «.flutuando-8»		(to "flutuando-8")
% «.flutuando-16»		(to "flutuando-16")
% «.metodos-nomes»		(to "metodos-nomes")
% «.metodos-nomes-5»		(to "metodos-nomes-5")
% «.particoes-preferidas»	(to "particoes-preferidas")
% «.aproximacoes-por-cima»	(to "aproximacoes-por-cima")
% «.aproximacoes-por-baixo»	(to "aproximacoes-por-baixo")
% «.definicao-integral»		(to "definicao-integral")
% «.exercicio-2»		(to "exercicio-2")
%
% «.djvuize»		(to "djvuize")



% «video-1»  (to ".video-1")
% (find-ssr-links     "c2m212di" "2021-2-C2-def-integral" "GSxsbNGoUbo")
% (code-eevvideo      "c2m212di" "2021-2-C2-def-integral" "GSxsbNGoUbo")
% (code-eevlinksvideo "c2m212di" "2021-2-C2-def-integral" "GSxsbNGoUbo")
% (find-c2m212divideo "0:00")
% (find-c2m212divideo "4:58" "Aqui eu pus uma apresentação")
% (find-c2m212divideo "6:20" "integral de x=2 até x=10")

% (c2m212isp 19 "exercicio-10")
% (c2m212isa    "exercicio-10")


% «videos-antigos»  (to ".videos-antigos")
% (c2m211somas2a  "video-1")
% (c2m211somas2a  "video-2")

\documentclass[oneside,12pt]{article}
\usepackage[colorlinks,citecolor=DarkRed,urlcolor=DarkRed]{hyperref} % (find-es "tex" "hyperref")
\usepackage{amsmath}
\usepackage{amsfonts}
\usepackage{amssymb}
\usepackage{pict2e}
\usepackage[x11names,svgnames]{xcolor} % (find-es "tex" "xcolor")
\usepackage{colorweb}                  % (find-es "tex" "colorweb")
%\usepackage{tikz}
%
% (find-dn6 "preamble6.lua" "preamble0")
%\usepackage{proof}   % For derivation trees ("%:" lines)
%\input diagxy        % For 2D diagrams ("%D" lines)
%\xyoption{curve}     % For the ".curve=" feature in 2D diagrams
%
\usepackage{edrx21}               % (find-LATEX "edrx21.sty")
\input edrxaccents.tex            % (find-LATEX "edrxaccents.tex")
\input edrx21chars.tex            % (find-LATEX "edrx21chars.tex")
\input edrxheadfoot.tex           % (find-LATEX "edrxheadfoot.tex")
\input edrxgac2.tex               % (find-LATEX "edrxgac2.tex")
%
%\usepackage[backend=biber,
%   style=alphabetic]{biblatex}            % (find-es "tex" "biber")
%\addbibresource{catsem-slides.bib}        % (find-LATEX "catsem-slides.bib")
%
% (find-es "tex" "geometry")
\usepackage[a6paper, landscape,
            top=1.5cm, bottom=.25cm, left=1cm, right=1cm, includefoot
           ]{geometry}
%
\begin{document}

\catcode`\^^J=10
\directlua{dofile "dednat6load.lua"}  % (find-LATEX "dednat6load.lua")
\directlua{dofile "2021-2-C2-def-integral.lua"}
%L -- (find-LATEX "2021-2-C2-def-integral.lua")
%L Pict2e.__index.suffix = "%"
\pu
\def\pictgridstyle{\color{GrayPale}\linethickness{0.3pt}}
\def\pictaxesstyle{\linethickness{0.5pt}}
\def\pictcurvestyle{\linethickness{0.7pt}}
\def\pictcurvestyle{\linethickness{1.0pt}}

%L dofile "edrxtikz.lua"  -- (find-LATEX "edrxtikz.lua")
%L dofile "edrxpict.lua"  -- (find-LATEX "edrxpict.lua")
\pu


% «defs»  (to ".defs")
% (find-LATEX "edrx21defs.tex" "colors")
% (find-LATEX "edrx21.sty")

\def\u#1{\par{\footnotesize \url{#1}}}

\def\drafturl{http://angg.twu.net/LATEX/2021-2-C2.pdf}
\def\drafturl{http://angg.twu.net/2021.2-C2.html}
\def\draftfooter{\tiny \href{\drafturl}{\jobname{}} \ColorBrown{\shorttoday{} \hours}}

\def\Intover     #1#2{\overline {∫}_{#1}#2\,dx}
\def\Intunder    #1#2{\underline{∫}_{#1}#2\,dx}
\def\Intoverunder#1#2{\Intover{#1}{#2} - \Intunder{#1}{#2}}

\def\Intxover     #1#2#3{\overline {∫}_{x=#1}^{x=#2}#3\,dx}
\def\Intxunder    #1#2#3{\underline{∫}_{x=#1}^{x=#2}#3\,dx}

\def\Intoverunder   #1#2{\overline{\underline{∫}}_{#1}      #2\,dx}
\def\Intxoverunder#1#2#3{\overline{\underline{∫}}_{x=#1}^{x=#2} #3\,dx}

\def\sumiN#1{\sum_{i=1}^N #1 (b_i-a_i)}
\def\mname#1{\text{[#1]}}

\def\Io #1{\Intover      {[2,10]_{2^{#1}}} {f(x)}}
\def\Iu #1{\Intunder     {[2,10]_{2^{#1}}} {f(x)}}
\def\Iou#1{\Intoverunder {[2,10]_{2^{#1}}} {f(x)}}
%




%  _____ _ _   _                               
% |_   _(_) |_| | ___   _ __   __ _  __ _  ___ 
%   | | | | __| |/ _ \ | '_ \ / _` |/ _` |/ _ \
%   | | | | |_| |  __/ | |_) | (_| | (_| |  __/
%   |_| |_|\__|_|\___| | .__/ \__,_|\__, |\___|
%                      |_|          |___/      
%
% «title»  (to ".title")
% (c2m212dip 1 "title")
% (c2m212dia   "title")

\thispagestyle{empty}

\begin{center}

\vspace*{1.2cm}

{\bf \Large Cálculo 2 - 2021.2}

\bsk

Aula 19: a definição da integral

\bsk

Eduardo Ochs - RCN/PURO/UFF

\url{http://angg.twu.net/2021.2-C2.html}

\end{center}

\newpage

% Baseado em:
% (c2m211somas2p 37 "exercicio-16-defs")
% (c2m211somas2a    "exercicio-16-defs")
% (c2m211prp 2 "introducao")
% (c2m211pra   "introducao")

% (c2m211somas2p 25 "exercicio-10")
% (c2m211somas2a    "exercicio-10")



\unitlength=10pt


\newpage

{\bf Introdução}

A definição formal da integral é bem complicada.

\msk

A gente primeiro tem que definir aproximações por retângulos

por cima e por baixo usando sups e infs, depois a gente tem que

definir os limites dessas aproximações por cima e por baixo

do jeito certo, depois a gente tem que comparar esses limites...

\msk

Se o limite por cima e o limite por baixo dão o mesmo

resultado então a nossa função é integrável, e a integral

dela é o resultado desses limites --- mas existem funções

que não são integráveis.

\msk

A gente vai ter que definir um monte de abreviações

pras expressões matemáticas não ficarem grandes demais,

e a gente vai ter que aprender a interpretar graficamente

cada expressão... como nos próximos dois slides:

\newpage


% «triangle-wave»  (to ".triangle-wave")
% (c2m212dip 3 "triangle-wave")
% (c2m212dia   "triangle-wave")
%
%L f_triangle_wave = function (x)
%L     if x <= 3 then return x + 3 end
%L     if x <= 8 then return 9 - x end
%L     return x - 7
%L   end
%L 
%L pwitriw = Piecewisify.new(f_triangle_wave, 3, 8)
%L 
%L Pict2e.new()
%L   :setbounds(v(0,0), v(11,7))
%L     :grid()
%L     :add("#1")
%L     :axesandticks()
%L     :add("\\pictcurvestyle")
%L     :add(pwitriw:pw(0, 11))  -- f
%L   :bepc()
%L   :def("TriW#1")
%L   :output()
\pu

% Test:
%
% $\TriW{}$
% 
% $\celllower=2.5pt%
%  \def\cellfont{\scriptsize}%
%   \TriW{%
%     \put(3,6.5){\cell{(3,6)}}%
%     \put(8,0.5){\cell{(8,1)}}%
%  }
% $

% «triangle-wave-more-defs»  (to ".triangle-wave-more-defs")

% (c2m211somas2p 37 "exercicio-16-defs")
% (c2m211somas2a    "exercicio-16-defs")
% (c2m211figsa    "defs-int")

% (find-LATEX "2021-1-C2-critical-points.lua" "Approxer-tests")
%L dofile     "2021-1-C2-critical-points.lua"
%L appr = Approxer {
%L     f      = f_do_slide_8,
%L     allcps = {3,8},
%L     a      = 2,
%L     b      = 10,
%L     N      = 4,
%L     method = "supin",
%L     what   = "ac",
%L   }
\pu

\long\def\ColorUpperA#1{{\color{red!20!white}#1}}
\long\def\ColorUpperB#1{{\color{Gold1!20!white}#1}}
\long\def\ColorUpperC#1{{\color{Green1!20!white}#1}}
\long\def\ColorUpperD#1{{\color{Blue1!20!white}#1}}
\long\def\ColorLowerA#1{{\color{red!80!white}#1}}
\long\def\ColorLowerB#1{{\color{Gold1!80!white}#1}}
\long\def\ColorLowerC#1{{\color{Green1!80!white}#1}}
\long\def\ColorLowerD#1{{\color{Blue1!80!white}#1}}
\long\def\ColorRealInt#1{{\color{Purple0!90!white}#1}}

\def\UpperA{\ColorUpperA{\expr{appr:pict(2,  "supin", "c")}}}
\def\UpperB{\ColorUpperB{\expr{appr:pict(4,  "supin", "c")}}}
\def\UpperC{\ColorUpperC{\expr{appr:pict(8,  "supin", "c")}}}
\def\UpperD{\ColorUpperD{\expr{appr:pict(16, "supin", "c")}}}
\def\LowerD{\ColorLowerD{\expr{appr:pict(16, "infin", "c")}}}
\def\LowerC{\ColorLowerC{\expr{appr:pict(8,  "infin", "c")}}}
\def\LowerB{\ColorLowerB{\expr{appr:pict(4,  "infin", "c")}}}
\def\LowerA{\ColorLowerA{\expr{appr:pict(2,  "infin", "c")}}}

\def\fUpperA{\TriW{\UpperA}}
\def\fUpperB{\TriW{\UpperB}}
\def\fUpperC{\TriW{\UpperC}}
\def\fUpperD{\TriW{\UpperD}}
\def\fLowerA{\TriW{\LowerA}}
\def\fLowerB{\TriW{\LowerB}}
\def\fLowerC{\TriW{\LowerC}}
\def\fLowerD{\TriW{\LowerD}}
\def\fUpperLowerA{\TriW{\UpperA\LowerA}}
\def\fUpperLowerB{\TriW{\UpperB\LowerB}}
\def\fUpperLowerC{\TriW{\UpperC\LowerC}}
\def\fUpperLowerD{\TriW{\UpperD\LowerD}}
\def\fUpperLowerABCD{\TriW{%
    \UpperA%
    \UpperB%
    \UpperC%
    \UpperD%
    \LowerD%
    \LowerC%
    \LowerB%
    \LowerA%
  }}
\def\fRealInt{\TriW{\ColorRealInt{%
  \polygon*(2,0)(2,5)(3,6)(8,1)(10,3)(10,0)%
  }}}

\def\SixApproxsAndIntegral{
  \begin{array}{ccccccccc}
   \fUpperA &≥& \fUpperB &≥& \fUpperC &≥& \ldots \\
                                   &&&&&& \rotl{≤} \\
                                   &&&&&& \fRealInt \\
                                   &&&&&& \rotl{≤} \\
   \fLowerA &≤& \fLowerB &≤& \fLowerC &≤& \ldots \\
  \end{array}
  }
\def\SixApproxsAndIntegralB{{
  \def\fUpperA{\mname{sup}_{[2,10]_{2^1}}}
  \def\fUpperB{\mname{sup}_{[2,10]_{2^2}}}
  \def\fUpperC{\mname{sup}_{[2,10]_{2^3}}}
  \def\fLowerA{\mname{inf}_{[2,10]_{2^1}}}
  \def\fLowerB{\mname{inf}_{[2,10]_{2^2}}}
  \def\fLowerC{\mname{inf}_{[2,10]_{2^3}}}
  \def\fRealInt{\D\Intx{2}{10}{f(x)}}
  \SixApproxsAndIntegral
}}




\newpage

% «SixApproxsAndIntegral»  (to ".SixApproxsAndIntegral")
% (c2m212dip 4 "SixApproxsAndIntegral")
% (c2m212dia   "SixApproxsAndIntegral")
% (c2m211somas2p 38 "exercicio-16-fig2")
% (c2m211somas2a    "exercicio-16-fig2")

\unitlength=5pt

\vspace*{0.5cm}

$\scalebox{1.0}{$
  \SixApproxsAndIntegral
 $}
$

\newpage

$$\scalebox{0.7}{$
  \SixApproxsAndIntegral
 $}
$$

$$\scalebox{0.7}{$
     \SixApproxsAndIntegralB
 $}
$$



\newpage

Também podemos desenhar só a diferença entre

a aproximação por cima e a por baixo...

Aí o resultado vai ser formado por retângulos

``flutuando no ar''. Se $f(x)$ é esta função 

mais complicada aqui, então...

\unitlength=25pt

\def\Iou#1{\Intoverunder {[a,b]_{2^{#1}}} {f(x)}}
\def\Iou#1{\Intoverunder {[0,8]_{2^{#1}}} {f(x)}}

\def\FIG#1{%
  \ParCoWith{%
  \ColorOrange{%
  \expr{pwi:rects(Partition.new(0, 8):splitn(2^#1), "sup", "inf")}
  }}}
\def\FFIG#1{\Iou{#1} \;\; = \;\; \FIG{#1}}
\def\FFIGzero{f(x) \;\; = \;\; \ParCoWith{}}

$$\FFIGzero$$

\newpage

% «flutuando-4»  (to ".flutuando-4")
% (c2m212dip 6 "flutuando-4")
% (c2m212dia   "flutuando-4")

$$\FFIG2$$
\newpage

% «flutuando-8»  (to ".flutuando-8")
% (c2m212dip 7 "flutuando-8")
% (c2m212dia   "flutuando-8")

$$\FFIG3$$

\newpage

% «flutuando-16»  (to ".flutuando-16")
% (c2m212dip 8 "flutuando-16")
% (c2m212dia   "flutuando-16")

$$\FFIG4$$
\newpage

$$\FFIG5$$
\newpage

$$\FFIG6$$
\newpage

$$\FFIG7$$
\newpage




\newpage

% «metodos-nomes»  (to ".metodos-nomes")
% (c2m212dip 4 "metodos-nomes")
% (c2m212dia   "metodos-nomes")
% (c2m211somas2p 27 "metodos-nomes")
% (c2m211somas2a    "metodos-nomes")

{\bf Métodos de integração: nomes}

\def\sumiN#1{\sum_{i=1}^N #1 (b_i-a_i)}
\def\mname#1{\text{[#1]}}
%
$$\scalebox{0.95}{$
  \begin{array}{ccl}
  \mname{L}    &=& \sumiN {f(a_i)}                    \\[2pt]
  \mname{R}    &=& \sumiN {f(b_i)}                    \\[2pt]
  \mname{Trap} &=& \sumiN {\frac{f(a_i) + f(b_i)}{2}} \\[2pt]
  \mname{M}    &=& \sumiN {f(\frac{a_i+b_i}{2})}      \\[2pt]
  \mname{min}  &=& \sumiN {\min(f(a_i), f(b_i))}      \\[2pt]
  \mname{max}  &=& \sumiN {\max(f(a_i), f(b_i))}      \\[2pt]
  \mname{inf}  &=& \sumiN {\inf(F([a_i,b_i]))}        \\[2pt]
  \mname{sup}  &=& \sumiN {\sup(F([a_i,b_i]))}        \\
  \end{array}
  $}
$$

Cada uma dessas fórmulas é um ``método de integração''.

Todos esses ``métodos'' aparecem na página da Wikipedia,

mas com outros nomes e usando partições em que todos os

intervalos têm o mesmo comprimento.

\newpage

% «metodos-nomes-5»  (to ".metodos-nomes-5")
% (c2m211somas2p 27 "metodos-nomes-2")
% (c2m211somas2a    "metodos-nomes-2")

{\bf Métodos de integração: nomes (2)}

\ssk

Todas as fórmulas do slide anterior supõem que estamos

num contexto em que a partição $P$ está definida.

Se usamos elas com uma partição em subscrito,

como em $\mname{L}_{\{4,5,7\}}$, isso vai querer dizer

que a partição $P$ vai ser indicada no subscrito.

Por exemplo:
%
$$\scalebox{0.9}{$
  \begin{array}[t]{rcl}
  \mname{L}_{\{4,5,7\}} &=& \sumiN {f(a_i)} \\[5pt]
                        &=& f(a_1)(b_1-a_1) \\
                        &+& f(a_2)(b_2-a_2) \\[5pt]
                        &=& f(4)  (5-4) \\
                        &+& f(5)  (7-5,) \\
  \end{array}
  \quad
  \begin{array}[t]{rcl}
  \mname{L}_{\{6,7,8,9\}} &=& \sumiN {f(a_i)} \\[5pt]
                          &=& f(a_1)(b_1-a_1) \\
                          &+& f(a_2)(b_2-a_2) \\
                          &+& f(a_3)(b_3-a_2) \\[5pt]
                          &=& f(6)  (7-6) \\
                          &+& f(7)  (8-7) \\
                          &+& f(8)  (9-8). \\
  \end{array}
  $}
$$




\newpage

% «particoes-preferidas»  (to ".particoes-preferidas")
% (c2m212dip 7 "particoes-preferidas")
% (c2m212dia   "particoes-preferidas")
% (c2m211somas2p 20 "exercicio-13")
% (c2m211somas2a    "exercicio-13")

{\bf Nossas partições preferidas}

Agora eu vou definir uma notação pra partição que

divide um intervalo em $N$ subintervalos iguais:
%
\def\baN{\frac{b-a}{N}}
%
$$
  \textstyle
  [a,b]_N = \{a, \; a+\baN, \; a+2\baN, \; \ldots, \; b\}
$$

\msk

{\bf Exercício 1.}

Calcule:

a) $[4,6]_1$

b) $[4,6]_{2^3}$

\msk

Dicas: $2^3=8$, e releia isto aqui:

\ssk

{\footnotesize

% (c2m211somas1p 16 "exercicio-9-dicas")
% (c2m211somas1a    "exercicio-9-dicas")
% http://angg.twu.net/LATEX/2021-1-C2-somas-1.pdf#page=16
\url{http://angg.twu.net/LATEX/2021-1-C2-somas-1.pdf\#page=16}

}

\bsk


Obs: mais tarde no curso você vai (ter que!)

aprender a fazer as suas próprias definições...

\newpage

% «aproximacoes-por-cima»  (to ".aproximacoes-por-cima")
% (c2m212dip 15 "aproximacoes-por-cima")
% (c2m212dia    "aproximacoes-por-cima")

{\bf Aproximações por cima}

Mais duas definições:

A melhor aproximação por cima para a integral de $f$

na partição $P$ é:
%
$$\Intover{P}{f(x)} = \mname{sup}_P,$$

O limite das aproximações por cima pra integral de $f$

no intervalo $[a,b]$ é:
%
$$\Intxover{a}{b}{f(x)} = \lim_{k→∞} \mname{sup}_{[a,b]_{2^k}},$$

Esse limite também é chamado de a ``integral por cima de $f$

no intervalo $[a,b]$''.


\newpage

% «aproximacoes-por-baixo»  (to ".aproximacoes-por-baixo")
% (c2m212dip 16 "aproximacoes-por-baixo")
% (c2m212dia    "aproximacoes-por-baixo")

{\bf Aproximações por baixo}

Mais duas definições:

A melhor aproximação por baixo para a integral de $f$

na partição $P$ é:
%
$$\Intunder{P}{f(x)} = \mname{inf}_P,$$

O limite das aproximações por baixo pra integral de $f$

no intervalo $[a,b]$ é:
%
$$\Intxunder{a}{b}{f(x)} = \lim_{k→∞} \mname{inf}_{[a,b]_{2^k}},$$

Esse limite também é chamado de a ``integral por baixo de $f$

no intervalo $[a,b]$''.


\newpage

% «definicao-integral»  (to ".definicao-integral")
% (c2m212dip 17 "definicao-integral")
% (c2m212dia    "definicao-integral")
% (c2m211somas2p 34 "definicao-integral")
% (c2m211somas2a    "definicao-integral")

{\bf A definição de integral}

\ssk

\def\eqa{\overset{\ColorRed{\Downarrow}}{=}}

A nossa definição de $\Intx{a}{b}{f(x)}$ vai ser:
%
$$\Intx     {a}{b}{f(x)} \;\;=\;\;
  \Intxover {a}{b}{f(x)} \;\; \eqa \;\;
  \Intxunder{a}{b}{f(x)}
$$

se a igualdade marcada com `$\eqa$' for verdade.

\msk
\msk

Se a igualdade `$\eqa$' for falsa vamos dizer que:

``$f(x)$ não é integrável no intervalo $[a,b]$'',

``$\Intx{a}{b}{f(x)}$ não está definida'', ou

``$\Intx{a}{b}{f(x)}$ dá erro''.

\msk
\msk

(Compare com $\frac{42}{0}$, que também ``não está definido'', ou ``dá erro''...)

\newpage

% (c2m211somas2p 35 "intoverunder")
% (c2m211somas2a    "intoverunder")
% (c2m211somas2p 35 "exercicio-15")
% (c2m211somas2a    "exercicio-15")

{\bf Como esses limites funcionam?}

Em Cálculo 1 você viu que algumas funções não são deriváveis.

Agora nós vamos ver que algumas funções não são integráveis.

O melhor modo de visualizar isso é usando estas definições:
%
$$\begin{array}{rcl}
  \D \Intoverunder{P}{f(x)} &=&
  \D \Intover     {P}{f(x)} -
     \Intunder    {P}{f(x)}
  \\[15pt]
  \D \Intxoverunder{a}{b}{f(x)} &=&
  \D \Intxover     {a}{b}{f(x)} -
     \Intxunder    {a}{b}{f(x)}
  \end{array}
$$

\newpage

% «exercicio-2»  (to ".exercicio-2")
% (c2m212dip 19 "exercicio-2")
% (c2m212dia    "exercicio-2")

{\bf Exercício 2.}

Faça o exercício 1 do MT1 do semestre passado.

Ele tem gabarito, mas tente fazê-lo sem olhar o gabarito.

\msk

Link:


\ssk

{\footnotesize

% (c2m211mt1p 4 "questao-1")
% (c2m211mt1a   "questao-1")
%    http://angg.twu.net/LATEX/2021-1-C2-MT1.pdf#page=4
\url{http://angg.twu.net/LATEX/2021-1-C2-MT1.pdf#page=4}

}

\bsk

Dica: reveja o exercício 10 deste PDF:

% (c2m212isp 19 "exercicio-10")
% (c2m212isa    "exercicio-10")

\ssk

{\footnotesize

% (c2m212isp 19)
%    http://angg.twu.net/LATEX/2021-2-C2-infs-e-sups.pdf#page=19
\url{http://angg.twu.net/LATEX/2021-2-C2-infs-e-sups.pdf#page=19}

}


\newpage


(Tudo a partir daqui vai ser reescrito)

\newpage

{\bf Exercício 15.}

\def\iou#1{\Intoverunder{[2,10]_{2^#1}}{f(x)}}

a) Verifique que no exercício 14 você desenhou $\iou0$,

$\iou1$, $\iou2$, e $\iou3$.

\msk

b) Calcule a área dessas quatro diferenças. \ColorRed{Veja o vídeo!}




\newpage

% «exercicio-10»  (to ".exercicio-10")
% (c2m211somas2p 25 "exercicio-10")
% (c2m211somas2a    "exercicio-10")

{\bf Exercício 10.}

\ssk

% (find-LATEX "edrxpict.lua" "beginpicture")

Lembre que:

\bsk

$f(x)=
    \unitlength=10pt
    \celllower=2.5pt%
    \def\cellfont{\scriptsize}%
    %
    \TriW{%
      \put(3,6.5){\cell{(3,6)}}%
      \put(8,0.5){\cell{(8,1)}}%
    }
   $

\bsk

a) Calcule $\sup(F([2,4]))$.

b) Calcule $\inf(F([2,4]))$.

c) Calcule $\sup(F([4,7]))$.

d) Calcule $\inf(F([4,7]))$.

e) Calcule $\sup(F([7,9]))$.

f) Calcule $\inf(F([7,9]))$.


\newpage

% «exercicio-11»  (to ".exercicio-11")
% (c2m211somas2p 26 "exercicio-11")
% (c2m211somas2a    "exercicio-11")

{\bf Exercício 11.}

\ssk

% (find-LATEX "edrxpict.lua" "beginpicture")

Lembre que:

\bsk

$f(x)=
    \unitlength=8pt
    \celllower=2.5pt%
    \def\cellfont{\scriptsize}%
    %
    \TriW{%
    \put(3,6.5){\cell{(3,6)}}%
    \put(8,0.5){\cell{(8,1)}}%
    }
   $

\bsk

Digamos que $P=\{ 1, 2,4, 5,6, 7,9, 10 \}$.

Represente graficamente \ColorRed{num gráfico só}:

\msk

a) $\sum_{i=1}^{N} \sup(F([a_i,b_i])) (b_i - a_i)$,

b) a curva $y=f(x)$,

c) $\sum_{i=1}^{N} \inf(F([a_i,b_i])) (b_i - a_i)$.

\msk

e verifique que você obteve algo bem parecido

com a figura do slide 2.



\newpage

% «exercicio-12»  (to ".exercicio-12")
% (c2m211somas2p 29 "exercicio-12")
% (c2m211somas2a    "exercicio-12")

{\bf Exercício 12.}

\ssk

% (find-LATEX "edrxpict.lua" "beginpicture")

Lembre que:

\bsk

$f(x)=
    \unitlength=7pt
    \celllower=2.5pt%
    \def\cellfont{\scriptsize}%
    %
    \TriW{
    \put(3,6.5){\cell{(3,6)}}%
    \put(8,0.5){\cell{(8,1)}}%
    }
   $

\bsk

Em cada um dos itens abaixo represente graficamente

num gráfico só a curva $y=f(x)$ e os dois somatórios pedidos.

a) $\mname{sup}_{\{1,10\}}$, 
   $\mname{inf}_{\{1,10\}}$

\ssk

b) $\mname{sup}_{\{1,2,5,6,9,10\}}$, 
   $\mname{inf}_{\{1,2,5,6,9,10\}}$

\ssk

c) $\mname{sup}_{\{1,2,4,5,6,7,9,10\}}$, 
   $\mname{inf}_{\{1,2,4,5,6,7,9,10\}}$

\bsk

d) $\mname{max}_{\{1,10\}}$, 
   $\mname{min}_{\{1,10\}}$

\ssk

e) $\mname{max}_{\{1,2,5,6,9,10\}}$, 
   $\mname{min}_{\{1,2,5,6,9,10\}}$



\newpage

% «exercicio-14»  (to ".exercicio-14")
% (c2m211somas2p 30 "exercicio-14")
% (c2m211somas2a    "exercicio-14")

{\bf Exercício 14.}

Lembre que:

\bsk

$f(x)=
    \unitlength=10pt
    \celllower=2.5pt%
    \def\cellfont{\scriptsize}%
    %
    \TriW{
    \put(3,6.5){\cell{(3,6)}}%
    \put(8,0.5){\cell{(8,1)}}%
    }
   $

\bsk

Em cada um dos itens abaixo represente graficamente

num gráfico só a curva $y=f(x)$ e os dois somatórios pedidos.

a) $\mname{sup}_{[2,10]_{2^0}}$, 
   $\mname{inf}_{[2,10]_{2^0}}$

\ssk

b) $\mname{sup}_{[2,10]_{2^1}}$, 
   $\mname{inf}_{[2,10]_{2^1}}$

\ssk

c) $\mname{sup}_{[2,10]_{2^2}}$, 
   $\mname{inf}_{[2,10]_{2^2}}$

\ssk

d) $\mname{sup}_{[2,10]_{2^3}}$, 
   $\mname{inf}_{[2,10]_{2^3}}$


\newpage

% «exercicio-16»  (to ".exercicio-16")
% (c2m211somas2p 36 "exercicio-16")
% (c2m211somas2a    "exercicio-16")

{\bf Exercício 16.}

Identifique nas figuras dos próximos dois slides:
%
$$\scalebox{0.9}{$
  \begin{array}{cccc}
  \Io1,  & \Io2,  & \Io3,  & \Io4, \\[10pt]
  \Iu1,  & \Iu2,  & \Iu3,  & \Iu4, \\[10pt]
  \Iou1, & \Iou2, & \Iou3, & \Iou4, \\[10pt]
  \end{array}
  $}
$$

$$%\scalebox{0.9}{$
  \Intx{2}{10}{f(x)}.
  %$}
$$

\msk

Dica: os ``$\Intoverunder{P}{\ldots}$''s são feitos de ``retângulos flutuando no ar'',

não de retângulos cujas bases estão em $y=0$.


\newpage

% «exercicio-16-defs»  (to ".exercicio-16-defs")

% \newpage

% «exercicio-16-fig1»  (to ".exercicio-16-fig1")
% (c2m211somas2p 37 "exercicio-16-fig1")
% (c2m211somas2a    "exercicio-16-fig1")

\unitlength=6.5pt

$\fUpperLowerA
  \quad
  \fUpperLowerB
  \quad
  \fUpperLowerC
  \quad
  \fUpperLowerD
$

\bsk

\unitlength=20pt

$\fUpperLowerABCD
$

\newpage

% «exercicio-16-fig2»  (to ".exercicio-16-fig2")


\newpage



%\printbibliography

\GenericWarning{Success:}{Success!!!}  % Used by `M-x cv'

\end{document}

%  ____  _             _         
% |  _ \(_)_   ___   _(_)_______ 
% | | | | \ \ / / | | | |_  / _ \
% | |_| | |\ V /| |_| | |/ /  __/
% |____// | \_/  \__,_|_/___\___|
%     |__/                       
%
% «djvuize»  (to ".djvuize")
% (find-LATEXgrep "grep --color -nH --null -e djvuize 2020-1*.tex")

 (eepitch-shell)
 (eepitch-kill)
 (eepitch-shell)
# (find-fline "~/2021.2-C2/")
# (find-fline "~/LATEX/2021-2-C2/")
# (find-fline "~/bin/djvuize")

cd /tmp/
for i in *.jpg; do echo f $(basename $i .jpg); done

f () { rm -v $1.pdf;  textcleaner -f 50 -o  5 $1.jpg $1.png; djvuize $1.pdf; xpdf $1.pdf }
f () { rm -v $1.pdf;  textcleaner -f 50 -o 10 $1.jpg $1.png; djvuize $1.pdf; xpdf $1.pdf }
f () { rm -v $1.pdf;  textcleaner -f 50 -o 20 $1.jpg $1.png; djvuize $1.pdf; xpdf $1.pdf }

f () { rm -fv $1.png $1.pdf; djvuize $1.pdf }
f () { rm -fv $1.png $1.pdf; djvuize WHITEBOARDOPTS="-m 1.0 -f 15" $1.pdf; xpdf $1.pdf }
f () { rm -fv $1.png $1.pdf; djvuize WHITEBOARDOPTS="-m 1.0 -f 30" $1.pdf; xpdf $1.pdf }
f () { rm -fv $1.png $1.pdf; djvuize WHITEBOARDOPTS="-m 1.0 -f 45" $1.pdf; xpdf $1.pdf }
f () { rm -fv $1.png $1.pdf; djvuize WHITEBOARDOPTS="-m 0.5" $1.pdf; xpdf $1.pdf }
f () { rm -fv $1.png $1.pdf; djvuize WHITEBOARDOPTS="-m 0.25" $1.pdf; xpdf $1.pdf }
f () { cp -fv $1.png $1.pdf       ~/2021.2-C2/
       cp -fv        $1.pdf ~/LATEX/2021-2-C2/
       cat <<%%%
% (find-latexscan-links "C2" "$1")
%%%
}

f 20201213_area_em_funcao_de_theta
f 20201213_area_em_funcao_de_x
f 20201213_area_fatias_pizza



%  __  __       _        
% |  \/  | __ _| | _____ 
% | |\/| |/ _` | |/ / _ \
% | |  | | (_| |   <  __/
% |_|  |_|\__,_|_|\_\___|
%                        
% <make>

 (eepitch-shell)
 (eepitch-kill)
 (eepitch-shell)
# (find-LATEXfile "2019planar-has-1.mk")
make -f 2019.mk STEM=2021-2-C2-def-integral veryclean
make -f 2019.mk STEM=2021-2-C2-def-integral pdf

% Local Variables:
% coding: utf-8-unix
% ee-tla: "c2di"
% ee-tla: "c2m212di"
% End:
