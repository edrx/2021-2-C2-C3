% (find-LATEX "2021-2-C2-P1.tex")
% (defun c  () (interactive) (find-LATEXsh "lualatex -record 2021-2-C2-P1.tex" :end))
% (defun C  () (interactive) (find-LATEXsh "lualatex 2021-2-C2-P1.tex" "Success!!!"))
% (defun D  () (interactive) (find-pdf-page      "~/LATEX/2021-2-C2-P1.pdf"))
% (defun d  () (interactive) (find-pdftools-page "~/LATEX/2021-2-C2-P1.pdf"))
% (defun e  () (interactive) (find-LATEX "2021-2-C2-P1.tex"))
% (defun o  () (interactive) (find-LATEX "2021-2-C2-P1.tex"))
% (defun is () (interactive) (find-LATEX "2021-2-C2-int-subst.tex"))
% (defun u  () (interactive) (find-latex-upload-links "2021-2-C2-P1"))
% (defun v  () (interactive) (find-2a '(e) '(d)))
% (defun d0 () (interactive) (find-ebuffer "2021-2-C2-P1.pdf"))
% (defun cv () (interactive) (C) (ee-kill-this-buffer) (v) (g))
%          (code-eec-LATEX "2021-2-C2-P1")
% (find-pdf-page   "~/LATEX/2021-2-C2-P1.pdf")
% (find-sh0 "cp -v  ~/LATEX/2021-2-C2-P1.pdf /tmp/")
% (find-sh0 "cp -v  ~/LATEX/2021-2-C2-P1.pdf /tmp/pen/")
%     (find-xournalpp "/tmp/2021-2-C2-P1.pdf")
%   file:///home/edrx/LATEX/2021-2-C2-P1.pdf
%               file:///tmp/2021-2-C2-P1.pdf
%           file:///tmp/pen/2021-2-C2-P1.pdf
% http://angg.twu.net/LATEX/2021-2-C2-P1.pdf
% (find-LATEX "2019.mk")
% (find-CN-aula-links "2021-2-C2-P1" "2" "c2m212p1" "c2p1")

% «.defs»		(to "defs")
% «.defs-T-and-B»	(to "defs-T-and-B")
% «.title»		(to "title")
% «.questao-1»		(to "questao-1")
% «.questao-1b»		(to "questao-1b")
% «.questao-1c»		(to "questao-1c")
% «.questao-1d»		(to "questao-1d")
% «.questao-2»		(to "questao-2")
% «.questao-1-gab-ab»	(to "questao-1-gab-ab")
% «.questao-1-gab-c»	(to "questao-1-gab-c")
% «.questao-1-gab-d»	(to "questao-1-gab-d")
%
% «.djvuize»		(to "djvuize")



% <videos>
% Video (not yet):
% (find-ssr-links     "c2m212p1" "2021-2-C2-P1" "{naoexiste}")
% (code-eevvideo      "c2m212p1" "2021-2-C2-P1")
% (code-eevlinksvideo "c2m212p1" "2021-2-C2-P1")
% (find-c2m212p1video "0:00")

\documentclass[oneside,12pt]{article}
\usepackage[colorlinks,citecolor=DarkRed,urlcolor=DarkRed]{hyperref} % (find-es "tex" "hyperref")
\usepackage{amsmath}
\usepackage{amsfonts}
\usepackage{amssymb}
\usepackage{pict2e}
\usepackage[x11names,svgnames]{xcolor} % (find-es "tex" "xcolor")
\usepackage{colorweb}                  % (find-es "tex" "colorweb")
%\usepackage{tikz}
%
% (find-dn6 "preamble6.lua" "preamble0")
%\usepackage{proof}   % For derivation trees ("%:" lines)
%\input diagxy        % For 2D diagrams ("%D" lines)
%\xyoption{curve}     % For the ".curve=" feature in 2D diagrams
%
\usepackage{edrx21}               % (find-LATEX "edrx21.sty")
\input edrxaccents.tex            % (find-LATEX "edrxaccents.tex")
\input edrx21chars.tex            % (find-LATEX "edrx21chars.tex")
\input edrxheadfoot.tex           % (find-LATEX "edrxheadfoot.tex")
\input edrxgac2.tex               % (find-LATEX "edrxgac2.tex")
%\usepackage{emaxima}              % (find-LATEX "emaxima.sty")
%
%\usepackage[backend=biber,
%   style=alphabetic]{biblatex}            % (find-es "tex" "biber")
%\addbibresource{catsem-slides.bib}        % (find-LATEX "catsem-slides.bib")
%
% (find-es "tex" "geometry")
\usepackage[a6paper, landscape,
            top=1.5cm, bottom=.25cm, left=1cm, right=1cm, includefoot
           ]{geometry}
%
\begin{document}

%\catcode`\^^J=10
%\directlua{dofile "dednat6load.lua"}  % (find-LATEX "dednat6load.lua")
%
% %L dofile "2021pict2e.lua"           -- (find-LATEX "2021pict2e.lua")
% %L Pict2e.__index.suffix = "%"
% \pu
% \def\pictgridstyle{\color{GrayPale}\linethickness{0.3pt}}
% \def\pictaxesstyle{\linethickness{0.5pt}}

% %L dofile "edrxtikz.lua"  -- (find-LATEX "edrxtikz.lua")
% %L dofile "edrxpict.lua"  -- (find-LATEX "edrxpict.lua")
% \pu

% «defs»  (to ".defs")
% (find-LATEX "edrx21defs.tex" "colors")
% (find-LATEX "edrx21.sty")

\def\u#1{\par{\footnotesize \url{#1}}}

\def\drafturl{http://angg.twu.net/LATEX/2021-2-C2.pdf}
\def\drafturl{http://angg.twu.net/2021.2-C2.html}
\def\draftfooter{\tiny \href{\drafturl}{\jobname{}} \ColorBrown{\shorttoday{} \hours}}

% «defs-T-and-B»  (to ".defs-T-and-B")
% (c3m202p1p 6 "questao-2")
% (c3m202p1a   "questao-2")
\long\def\ColorOrange#1{{\color{orange!90!black}#1}}
\def\T(Total: #1 pts){{\bf(Total: #1)}}
\def\T(Total: #1 pts){{\bf(Total: #1 pts)}}
\def\T(Total: #1 pts){\ColorRed{\bf(Total: #1 pts)}}
\def\B       (#1 pts){\ColorOrange{\bf(#1 pts)}}



%  _____ _ _   _                               
% |_   _(_) |_| | ___   _ __   __ _  __ _  ___ 
%   | | | | __| |/ _ \ | '_ \ / _` |/ _` |/ _ \
%   | | | | |_| |  __/ | |_) | (_| | (_| |  __/
%   |_| |_|\__|_|\___| | .__/ \__,_|\__, |\___|
%                      |_|          |___/      
%
% «title»  (to ".title")
% (c2m212p1p 1 "title")
% (c2m212p1a   "title")

\thispagestyle{empty}

\begin{center}

\vspace*{1.2cm}

{\bf \Large Cálculo 2 - 2021.2}

\bsk

P1 (Primeira prova)

% \ColorRed{(Versão incompleta!)}

\bsk

Eduardo Ochs - RCN/PURO/UFF

\url{http://angg.twu.net/2021.2-C2.html}

\end{center}

\newpage


{\bf Regras e avisos}

\scalebox{0.95}{\def\colwidth{12cm}\firstcol{

As regras são as mesmas dos mini-testes e das

provas dos outros semestres -- veja por exemplo:

\msk

{\footnotesize

\url{http://angg.twu.net/LATEX/2020-2-C2-MT2.pdf}

}

\msk

Exceto que as questões serão disponibilizadas

às 0:40 da terça 25/jan/2022 e você vai ter até

as 10:00 da quinta 26/jan/2022 pra entregar as

respostas, e que eu vou responder perguntas tipo

``onde eu encontro mais informações sobre a questão

tal?'' se elas forem feitas no grupo da turma.

\msk

Quase todas as questões desta prova vão ser

pré-requisitos pra P2 -- a P2 vai supor que

você sabe ``encontrar a substituição certa''

muito bem e que você fez as questões desta

prova com muita atenção.

%}\anothercol{
}}

\newpage

% «questao-1»  (to ".questao-1")
% (c2m212p1p 3 "questao-1")
% (c2m212p1a   "questao-1")
% (c2m212intsp 7 "S2-proof-1")
% (c2m212intsa   "S2-proof-1")

\sa{Tfc2}   {[\text{TFC2}]}
\sa{DefDif} {[\text{DefDif}]}
\sa{Alface} {[\text{Alface}]}
\sa{Tomate} {[\text{Tomate}]}
\sa{Repolho}{[\text{Repolho}]}
\sa{DEFDIF} {\difx{a}{b}{F(x)}  \;\; = \;\; F(b) - F(a)       }
\sa{TFC2}   {\Intx{a}{b}{F'(x)} \;\; = \;\; \difx{a}{b}{F(x)} }
\def\TFCP    #1{ \D \left( #1 \right) }

\sa{TFC2-ap1}  { \TFCP{ \ga{TFC2-ap1-L}         \;\;=\;\; \ga{TFC2-ap1-R}   } }
\sa{TFC2-ap2}  { \TFCP{ \ga{TFC2-ap2-L}         \;\;=\;\; \ga{TFC2-ap2-R}   } }
\sa{DEFDIF-ap1}{ \TFCP{ \difx{a}{b}{f(g(x))}    \;\;=\;\; f(g(b)) - f(g(a)) } }
\sa{DEFDIF-ap2}{ \TFCP{ \difu{g(a)}{g(b)}{f(u)} \;\;=\;\; f(g(b)) - f(g(a)) } }

\sa{Emv1}{[\text{EMV1}]}
\sa{EMV1}{
 \begin{array}{rcl}
       \D \Intx{  a }{  b }{f'(g(x))g'(x)}
   &=& \D \difx{  a }{  b }{f (g(x))     } \\
   &=& f(g(b))            - f (g(a))       \\[7.5pt]
   &=& \D \difu{g(a)}{g(b)}{f (u)}         \\[7.5pt]
   &=& \D \Intu{g(a)}{g(b)}{f'(u)}
 \end{array}}
\sa{Alface} {[\text{Alface}]}
\sa{ALFACE}{
 \begin{array}{rcl}
       \D \Intx{  a }{  b }{f'(g(x))g'(x)}
   &=& \D \Intu{g(a)}{g(b)}{f'(u)}
 \end{array}}
\sa{Tomate} {[\text{Tomate}]}
\sa{TOMATE}{
 \begin{array}{rcl}
       \D \Intx{ a }{ b }{h(-x)·(-1)}
   &=& \D \Intu{\rq}{\rq}{h(u)}
 \end{array}}
\sa{Milho} {[\text{Milho}]}

\def\erro{\textsf{erro}}
\def\en#1{\overset{\scriptsize{\ColorRed{(#1)}}}{=}}

\sa{Trilho}{[\text{Trilho}]}
\sa{TRILHO}{
 \begin{array}{rcl}
          \Intx{ -3 }{ -2 }{\frac{1}{x}}
   &=& \D \difx{ -3 }{ -2 }{(\ln x))   } \\
   &=& \ln(-2) - \ln(-3)                 \\
   &=& \erro - \erro                     \\
   &=& \erro                             \\[7.5pt]
   %
          \Intx{ -3 }{ -2 }{\frac{1}{x}      }
   &\en1& \Intu{  3 }{  2 }{\frac{1}{-u}·(-1)} \\
   &=&    \Intu{  3 }{  2 }{\frac{1}{ u}     } \\
   &=&    \difu{  3 }{  2 }{(\ln u)          } \\
   &=& \ln(2) - \ln(3)                         \\
 \end{array}}

\sa{Trilho (1)}{[\text{Trilho (1)}]}
\sa{TRILHO (1)}{
              \Intx{ -3 }{ -2 }{\frac{1}{x}      }
   \; \en1 \; \Intu{  3 }{  2 }{\frac{1}{-u}·(-1)}
}

\def\rq{\ColorRed{?}}



{\bf Questão 1}

\T(Total: 9.5 pts)

\msk

Sejam:

$\scalebox{0.8}{$
  \begin{array}{rcl}
  \ga{DefDif} &=& \TFCP{\ga{DEFDIF}} \\
  \ga{Tfc2}   &=& \TFCP{\ga{TFC2}}   \\
  \ga{Emv1}   &=& \TFCP{\ga{EMV1}}   \\
  \ga{Alface} &=& \TFCP{\ga{ALFACE}}   \\
  \end{array}
 $}
$


\newpage

{\bf Questão 1 (cont.)}


\msk

a) \B(3.0 pts) Descubra qual é a substituição

\ssk

``da forma $\bsm{f(t) := f(t) \\
                 f'(t) := \rq \\
                 g(t) := \rq \\
                 g'(t) := \rq \\
                }$''
que faz com que isto seja verdade:

$$\scalebox{0.8}{$
  \ga{Alface}
  \bsm{f(t) := f(t) \\
                 f'(t) := \rq \\
                 g(t) := \rq \\
                 g'(t) := \rq \\
                }
  \;\; = \;\;
  \TFCP{\ga{TOMATE}}
  $}
$$

\ssk

Chame o resultado desta substituição de $\ga{Tomate}$

e ponha a sua resposta exatamente no mesmo formato

que as definições das fórmulas $[\text{EMV2}]$ e $[\text{EMV3}]$ daqui:

\msk

{\footnotesize

% (c2m212intsp 13 "um-exemplo")
% (c2m212intsa    "um-exemplo")
%    http://angg.twu.net/LATEX/2021-2-C2-int-subst.pdf#page=13
\url{http://angg.twu.net/LATEX/2021-2-C2-int-subst.pdf#page=13}

}

\msk

Ou seja, $\ga{Tomate} \; = \; \ga{Alface} [\rq] \;=\; (\rq)$.

\newpage

% «questao-1b»  (to ".questao-1b")
% (c2m212p1p 5 "questao-1b")
% (c2m212p1a   "questao-1b")

{\bf Questão 1 (cont.)}

\msk

b) \B(2.0 pts) Qual é o resultado de aplicar a substituição

que você obteve e usou no item (a) na ``fórmula'' $\ga{Emv1}$,

que na verdade é uma sequência de igualdades?

\msk

Chame a sua fórmula nova de $\ga{Repolho}$. A sua resposta

deve ser neste formato aqui:
%
$$\ga{Repolho} \;\;=\;\; \ga{Emv1} [\rq] \;\;=\;\; (\rq).$$

\newpage

% «questao-1c»  (to ".questao-1c")
% (c2m212p1p 6 "questao-1c")
% (c2m212p1a   "questao-1c")

{\bf Questão 1 (cont.)}

\msk

c) \B(1.0 pts) Seja
%
$$\ga{Milho} \;\;=\;\;
  \ga{Repolho} \bsm{
    b := 3 \\
    a := 2 \\
    f(t) := \ln t \\
    h(t) := \frac{1}{t} \\
  }.
$$

Escreva o resultado desta substituição explicitamente,

no formato:
%
$$\ga{Milho} \;\;=\;\; \ga{Repolho} [\rq] \;\;=\;\; (\rq),$$


\newpage

% «questao-1d»  (to ".questao-1d")
% (c2m212p1p 7 "questao-1d")
% (c2m212p1a   "questao-1d")

{\bf Questão 1 (cont.)}

\scalebox{0.75}{\def\colwidth{12cm}\firstcol{

d) \B(3.5 pts) Como a gente sabe muito pouco de números

complexos a gente considera que o domínio da função $\ln(x)$

é $(0,+∞)$, e que $\ln(x)$ não está definida, ou ``dá erro'', quando

$x∈(-∞,0]$. Alguns programas de computador vão dizer que

$\ln(-1) = \pi i$ --- mas eles estão usando uma outra definição do $\ln$.

\bsk

A demonstração $\ga{Trilho}$ da página seguinte mostra

dois modos diferentes de calcular uma certa integral ---

um modo dá erro, e o outro dá um valor fácil de calcular

(se você tiver uma calculadora que calcula log)...

\bsk

Os livros costumam fazer o passo `$\en1$' dela como

se ele fosse óbvio. Compare com:

\ssk

{\scriptsize

\url{http://hostel.ufabc.edu.br/~daniel.miranda/calculo/calculo.pdf\#page=189}

}

\bsk

Encontre uma substituição da forma $\ga{Tfc2}[\rq] = (\rq)$ que

justifique o passo `$\en1$' da $\ga{Trilho}$. Você não vai obter algo

exatamente igual à igualdade `$\en1$', só algo ``equivalente'' a ela.

%}\anothercol{
}}


\newpage

$$\ga{Trilho} \;\;=\;\; \TFCP{\ga{TRILHO}} $$

\newpage

% «questao-2»  (to ".questao-2")
% (c2m212p1p 9 "questao-2")
% (c2m212p1a   "questao-2")

{\bf Questão 2.}

\T(Total: 0.5 pts)

\msk

Nas próximas aulas nós vamos aprender os truques pra fazer

contas com integrais bem rápido --- como no livro do Daniel

Miranda; veja o link na questão (1d).

\msk

Na definição do `[:=]' que nós usamos até agora ele só

substituía variáveis e funções por ``expressões completas''...

por exemplo, ``$4+$'' e ``$)$'' \ColorRed{não são} expressões completas.

\msk

\def\lbparen{\mathopen{\vrule \lower.249em\vbox to1em{%
   \hrule height.2pt width.3em\vss \hrule height.2pt}%
   \kern-.32em(}}
\def\rbparen{\mathclose{)\kern-.32em\lower.249em\vbox to1em{%
   \hrule height.2pt width.3em\vss \hrule height.2pt}%
   \vrule}}
\def\banana#1{(\!| #1 |\!)}
\def\lens  #1{\lbparen #1 \rbparen}

\sa{PBL}{\biggl( \ga{BL} \biggr)}
\sa{BL}{\banana{f(x)} \;\;=\;\; \lens{g(y)}}
\sa{Bl}{[\text{BL}]}

\sa{PBL2}{\biggl( \ga{BL2} \biggr)}
\sa{BL2}{\banana{f(x)} \;\;=\;\; \lens{g(y)}}

Seja $\ga{Bl}$ a igualdade abaixo:
%
$$\ga{Bl} \;\;=\;\; \ga{PBL}$$

\newpage

{\bf Questão 2 (cont.)}

\msk

Eu sei que algumas pessoas de Linguagens Funcionais

usam as notações `$\banana{\ldots}$' e `$\lens{\ldots}$' como se fossem uns

tipos especiais de parênteses, e sei que a pronúncia

de `$\banana{f(x)}$' é ``$f(x)$ entre bananas'' e a

de `$\lens{g(y)}$' é ``$g(y)$ entre lentes'' -- mas

não sei o que eles significam.

\msk

Lá no início do curso a gente aprendeu a usar o `[:=]'

em expressões que a gente não entendia.

\newpage

{\bf Questão 2 (cont.)}

\msk

Em algumas gambiarras muito específicas a gente vai

autorizar o `[:=]' a substituir algumas expressões

incompletas (por outras expressões incompletas).

\msk

Digamos que \ColorRed{nesta questão} o `[:=]' está autorizado a

substituir o abre-banana, o fecha-banana, o abre-lente

e o fecha-lente por outras expressões incompletas.

\msk

\B(0.5 pts) Diga o resultado da substituição abaixo.
%
$$\ga{PBL} \bmat{
    \; |\!) \; := \; + 2 |\!) \;\; \\
    \rbparen \; := \; ·3 \rbparen \\
  } \;\;=\;\; \rq
$$



\newpage

% «questao-1-gab-ab»  (to ".questao-1-gab-ab")
% (c2m212p1p 12 "questao-1-gab-ab")
% (c2m212p1a    "questao-1-gab-ab")

{\bf Questão 1: gabarito}

\sa{TOMATE2}{
 \begin{array}{rcl}
       \D \Intx{ a }{ b }{h(-x)·(-1)}
   &=& \D \Intu{-a }{-b }{h(u)}
 \end{array}}

a)
%
$\scalebox{0.7}{$
 \begin{array}[t]{rcl}
  \ga{Alface} &=& \TFCP{\ga{ALFACE}} \\
  \ga{Tomate} \; = \; \ga{Alface}
                  \bsm{f(t) := f(t) \\
                       f'(t) := h(t) \\
                       g(t) := -t \\
                       g'(t) := -1 \\
                      }
              &=& \TFCP{\ga{TOMATE2}} \\
 \end{array}
 $}
$

\sa{REPOLHO}{
 \begin{array}{rcl}
       \D \Intx{  a }{  b }{h (-x)·(-1)}
   &=& \D \difx{  a }{  b }{f (-x)     } \\
   &=& f(-b)              - f (-a)       \\[7.5pt]
   &=& \D \difu{-a}{-b}{f (u)}         \\[7.5pt]
   &=& \D \Intu{-a}{-b}{h (u)}
 \end{array}}

b)
%
$\scalebox{0.7}{$
 \begin{array}[t]{rcl}
  \ga{Emv1} &=& \TFCP{\ga{EMV1}} \\
  \ga{Repolho} = \; \ga{Emv1}
                  \bsm{f(t) := f(t) \\
                       f'(t) := h(t) \\
                       g(t) := -t \\
                       g'(t) := -1 \\
                      }
              &=& \TFCP{\ga{REPOLHO}} \\
 \end{array}
 $}
$

\newpage

% «questao-1-gab-c»  (to ".questao-1-gab-c")
% (c2m212p1p 13 "questao-1-gab-c")
% (c2m212p1a    "questao-1-gab-c")

\sa{MILHO}{
 \begin{array}{rcl}
       \D \Intx{  2 }{  3 }{\frac{1}{-x}·(-1)}
   &=& \D \difx{  2 }{  3 }{(\ln -x)     } \\
   &=& (\ln -3)            - (\ln -2)       \\[7.5pt]
   &=& \D \difu{-2}{-3}{(\ln u)}         \\[7.5pt]
   &=& \D \Intu{-2}{-3}{\frac{1}{u}}
 \end{array}}

c)
%
$\scalebox{0.7}{$
 \begin{array}[t]{rcl}
  \ga{Repolho} &=& \TFCP{\ga{REPOLHO}} \\
  \ga{Milho} \; = \; \ga{Repolho}
                     \bsm{
                          b := 3 \\
                          a := 2 \\
                       f(t) := \ln t \\
                       h(t) := \frac{1}{t} \\
                          }
              &=& \TFCP{\ga{MILHO}} \\
 \end{array}
 $}
$



\newpage

% «questao-1-gab-d»  (to ".questao-1-gab-d")
% (c2m212p1p 14 "questao-1-gab-d")
% (c2m212p1a    "questao-1-gab-d")


\sa{ALFACE}{
 \begin{array}{rcl}
       \D \Intx{  a }{  b }{f'(g(x))g'(x)}
   &=& \D \Intu{g(a)}{g(b)}{f'(u)}
 \end{array}}
\sa{ALFACE2}{
 \begin{array}{rcl}
       \D \Intx{  a }{  b }{\frac{1}{-x}·(-1)}
   &=& \D \Intu{- a }{ -b }{\frac{1}{u}}
 \end{array}}
\sa{ALFACE3}{
 \begin{array}{rcl}
       \D \Intu{  a }{  b }{\frac{1}{-u}·(-1)}
   &=& \D \Intx{- a }{ -b }{\frac{1}{x}}
 \end{array}}
\sa{ALFACE4}{
 \begin{array}{rcl}
       \D \Intu{  3 }{  2 }{\frac{1}{-u}·(-1)}
   &=& \D \Intx{- 3 }{ -2 }{\frac{1}{x}}
 \end{array}}

d)
%
$\scalebox{0.7}{$
 \begin{array}[t]{rcl}
  \ga{Trilho (1)} &=& \TFCP{\ga{TRILHO (1)}} \\
  \ga{Alface}     &=& \TFCP{\ga{ALFACE}} \\
  \ga{Alface}
     \bsm{
        f(t) := f(t) \\
       f'(t) := \frac{1}{t} \\
        g(t) := -t \\
       g'(t) := -1 \\
          }
                  &=& \TFCP{\ga{ALFACE2}} \\
  \ga{Alface}
     \bsm{
        f(t) := f(t) \\
       f'(t) := \frac{1}{t} \\
        g(t) := -t \\
       g'(t) := -1 \\
          }
     \bsm{ x := u \\ u := x \\ }
                  &=& \TFCP{\ga{ALFACE3}} \\
  \ga{Alface}
     \bsm{
        f(t) := f(t) \\
       f'(t) := \frac{1}{t} \\
        g(t) := -t \\
       g'(t) := -1 \\
          }
     \bsm{ x := u \\ u := x \\ }
     \bsm{ a := 3 \\ b := 2 \\ }
                  &=& \TFCP{\ga{ALFACE4}} \\
  \ga{Alface}
     \bsm{
        f(t) := f(t) \\
       f'(t) := \frac{1}{t} \\
        g(t) := -t \\
       g'(t) := -1 \\
           x := u \\ u := x \\
           a := 3 \\ b := 2 \\ }
                  &=& \TFCP{\ga{ALFACE4}} \\
 \end{array}
 $}
$


\newpage



{\bf Questão 2: gabarito}

\sa{BL2}{\banana{f(x)+2} \;\;=\;\; \lens{g(y)·3}}

$$\scalebox{0.8}{$
  \ga{PBL} \bmat{
    \; |\!) \; := \; + 2 |\!) \;\; \\
    \rbparen \; := \; ·3 \rbparen \\
  } \;\;=\;\;
  \ga{PBL2}
  $}
$$


%
%\msk
%
%Algumas das subexpressões do $\ga{Milho}$ dão erro e
%
%outras não. Diga quais, e diga quais igualdades são 
%


% Gambiarra: substituir delimitadores

% (find-texbookpage)
% (find-texbooktext)

%$$\biggl\{\!\!\biggl|
%  \biggr\}
%$$


\newpage

%\printbibliography

\GenericWarning{Success:}{Success!!!}  % Used by `M-x cv'

\end{document}

%  ____  _             _         
% |  _ \(_)_   ___   _(_)_______ 
% | | | | \ \ / / | | | |_  / _ \
% | |_| | |\ V /| |_| | |/ /  __/
% |____// | \_/  \__,_|_/___\___|
%     |__/                       
%
% «djvuize»  (to ".djvuize")
% (find-LATEXgrep "grep --color -nH --null -e djvuize 2020-1*.tex")

 (eepitch-shell)
 (eepitch-kill)
 (eepitch-shell)
# (find-fline "~/2021.2-C2/")
# (find-fline "~/LATEX/2021-2-C2/")
# (find-fline "~/bin/djvuize")

cd /tmp/
for i in *.jpg; do echo f $(basename $i .jpg); done

f () { rm -v $1.pdf;  textcleaner -f 50 -o  5 $1.jpg $1.png; djvuize $1.pdf; xpdf $1.pdf }
f () { rm -v $1.pdf;  textcleaner -f 50 -o 10 $1.jpg $1.png; djvuize $1.pdf; xpdf $1.pdf }
f () { rm -v $1.pdf;  textcleaner -f 50 -o 20 $1.jpg $1.png; djvuize $1.pdf; xpdf $1.pdf }

f () { rm -fv $1.png $1.pdf; djvuize $1.pdf }
f () { rm -fv $1.png $1.pdf; djvuize WHITEBOARDOPTS="-m 1.0 -f 15" $1.pdf; xpdf $1.pdf }
f () { rm -fv $1.png $1.pdf; djvuize WHITEBOARDOPTS="-m 1.0 -f 30" $1.pdf; xpdf $1.pdf }
f () { rm -fv $1.png $1.pdf; djvuize WHITEBOARDOPTS="-m 1.0 -f 45" $1.pdf; xpdf $1.pdf }
f () { rm -fv $1.png $1.pdf; djvuize WHITEBOARDOPTS="-m 0.5" $1.pdf; xpdf $1.pdf }
f () { rm -fv $1.png $1.pdf; djvuize WHITEBOARDOPTS="-m 0.25" $1.pdf; xpdf $1.pdf }
f () { cp -fv $1.png $1.pdf       ~/2021.2-C2/
       cp -fv        $1.pdf ~/LATEX/2021-2-C2/
       cat <<%%%
% (find-latexscan-links "C2" "$1")
%%%
}

f 20201213_area_em_funcao_de_theta
f 20201213_area_em_funcao_de_x
f 20201213_area_fatias_pizza



%  __  __       _        
% |  \/  | __ _| | _____ 
% | |\/| |/ _` | |/ / _ \
% | |  | | (_| |   <  __/
% |_|  |_|\__,_|_|\_\___|
%                        
% <make>

 (eepitch-shell)
 (eepitch-kill)
 (eepitch-shell)
# (find-LATEXfile "2019planar-has-1.mk")
make -f 2019.mk STEM=2021-2-C2-P1 veryclean
make -f 2019.mk STEM=2021-2-C2-P1 pdf

% Local Variables:
% coding: utf-8-unix
% ee-tla: "c2p1"
% ee-tla: "c2m212p1"
% End:
