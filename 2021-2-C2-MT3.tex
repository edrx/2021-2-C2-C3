% (find-LATEX "2021-2-C2-MT3.tex")
% (defun c () (interactive) (find-LATEXsh "lualatex -record 2021-2-C2-MT3.tex" :end))
% (defun C () (interactive) (find-LATEXsh "lualatex 2021-2-C2-MT3.tex" "Success!!!"))
% (defun D () (interactive) (find-pdf-page      "~/LATEX/2021-2-C2-MT3.pdf"))
% (defun d () (interactive) (find-pdftools-page "~/LATEX/2021-2-C2-MT3.pdf"))
% (defun e () (interactive) (find-LATEX "2021-2-C2-MT3.tex"))
% (defun o () (interactive) (find-LATEX "2021-2-C2-MT3.tex"))
% (defun u () (interactive) (find-latex-upload-links "2021-2-C2-MT3"))
% (defun v () (interactive) (find-2a '(e) '(d)))
% (defun d0 () (interactive) (find-ebuffer "2021-2-C2-MT3.pdf"))
% (defun cv () (interactive) (C) (ee-kill-this-buffer) (v) (g))
%          (code-eec-LATEX "2021-2-C2-MT3")
% (find-pdf-page   "~/LATEX/2021-2-C2-MT3.pdf")
% (find-sh0 "cp -v  ~/LATEX/2021-2-C2-MT3.pdf /tmp/")
% (find-sh0 "cp -v  ~/LATEX/2021-2-C2-MT3.pdf /tmp/pen/")
%     (find-xournalpp "/tmp/2021-2-C2-MT3.pdf")
%   file:///home/edrx/LATEX/2021-2-C2-MT3.pdf
%               file:///tmp/2021-2-C2-MT3.pdf
%           file:///tmp/pen/2021-2-C2-MT3.pdf
% http://angg.twu.net/LATEX/2021-2-C2-MT3.pdf
% (find-LATEX "2019.mk")
% (find-CN-aula-links "2021-2-C2-MT3" "2" "c2m212mt3" "c2mt3")

% «.defs»		(to "defs")
% «.title»		(to "title")
% «.questao»		(to "questao")
% «.from-piecewise»	(to "from-piecewise")
% «.gabarito»		(to "gabarito")
%
% «.djvuize»		(to "djvuize")



% <videos>
% Video (not yet):
% (find-ssr-links     "c2m212mt3" "2021-2-C2-MT3" "{naoexiste}")
% (code-eevvideo      "c2m212mt3" "2021-2-C2-MT3")
% (code-eevlinksvideo "c2m212mt3" "2021-2-C2-MT3")
% (find-c2m212mt3video "0:00")

\documentclass[oneside,12pt]{article}
\usepackage[colorlinks,citecolor=DarkRed,urlcolor=DarkRed]{hyperref} % (find-es "tex" "hyperref")
\usepackage{amsmath}
\usepackage{amsfonts}
\usepackage{amssymb}
\usepackage{pict2e}
\usepackage[x11names,svgnames]{xcolor} % (find-es "tex" "xcolor")
\usepackage{colorweb}                  % (find-es "tex" "colorweb")
%\usepackage{tikz}
%
% (find-dn6 "preamble6.lua" "preamble0")
%\usepackage{proof}   % For derivation trees ("%:" lines)
%\input diagxy        % For 2D diagrams ("%D" lines)
%\xyoption{curve}     % For the ".curve=" feature in 2D diagrams
%
\usepackage{edrx21}               % (find-LATEX "edrx21.sty")
\input edrxaccents.tex            % (find-LATEX "edrxaccents.tex")
\input edrx21chars.tex            % (find-LATEX "edrx21chars.tex")
\input edrxheadfoot.tex           % (find-LATEX "edrxheadfoot.tex")
\input edrxgac2.tex               % (find-LATEX "edrxgac2.tex")
%
%\usepackage[backend=biber,
%   style=alphabetic]{biblatex}            % (find-es "tex" "biber")
%\addbibresource{catsem-slides.bib}        % (find-LATEX "catsem-slides.bib")
%
% (find-es "tex" "geometry")
\usepackage[a6paper, landscape,
            top=1.5cm, bottom=.25cm, left=1cm, right=1cm, includefoot
           ]{geometry}
%
\begin{document}

\catcode`\^^J=10
\directlua{dofile "dednat6load.lua"}  % (find-LATEX "dednat6load.lua")
%
%L dofile "2021pict2e.lua"           -- (find-LATEX "2021pict2e.lua")
%L Pict2e.__index.suffix = "%"
\pu
\def\pictgridstyle{\color{GrayPale}\linethickness{0.3pt}}
\def\pictaxesstyle{\linethickness{0.5pt}}

% %L dofile "edrxtikz.lua"  -- (find-LATEX "edrxtikz.lua")
% %L dofile "edrxpict.lua"  -- (find-LATEX "edrxpict.lua")
% \pu

% «defs»  (to ".defs")
% (find-LATEX "edrx21defs.tex" "colors")
% (find-LATEX "edrx21.sty")

\def\u#1{\par{\footnotesize \url{#1}}}

\def\drafturl{http://angg.twu.net/LATEX/2021-2-C2.pdf}
\def\drafturl{http://angg.twu.net/2021.2-C2.html}
\def\draftfooter{\tiny \href{\drafturl}{\jobname{}} \ColorBrown{\shorttoday{} \hours}}



%  _____ _ _   _                               
% |_   _(_) |_| | ___   _ __   __ _  __ _  ___ 
%   | | | | __| |/ _ \ | '_ \ / _` |/ _` |/ _ \
%   | | | | |_| |  __/ | |_) | (_| | (_| |  __/
%   |_| |_|\__|_|\___| | .__/ \__,_|\__, |\___|
%                      |_|          |___/      
%
% «title»  (to ".title")
% (c2m212mt3p 1 "title")
% (c2m212mt3a   "title")

\thispagestyle{empty}

\begin{center}

\vspace*{1.2cm}

{\bf \Large Cálculo 2 - 2021.2}

\bsk

Mini-teste 3

\bsk

Eduardo Ochs - RCN/PURO/UFF

\url{http://angg.twu.net/2021.2-C2.html}

\end{center}

\newpage

{\bf Avisos}

O mini-teste 3 vai acontecer no início de janeiro.

Ele vai ter problemas parecidos com estes aqui,

de integrar uma função escada no olhômetro:


\ssk

{\footnotesize

% (c2m202mt1p 4 "miniteste-funcao")
% (c2m202mt1a   "miniteste-funcao")
%    http://angg.twu.net/LATEX/2020-2-C2-MT1.pdf#page=4
\url{http://angg.twu.net/LATEX/2020-2-C2-MT1.pdf#page=4}

% (c2m211mt2p 4 "questoes-1-e-2")
% (c2m211mt2a   "questoes-1-e-2")
%    http://angg.twu.net/LATEX/2021-1-C2-MT2.pdf#page=4
\url{http://angg.twu.net/LATEX/2021-1-C2-MT2.pdf#page=4}

}



\bsk

As regras vão ser as mesmas dos outros mini-testes.

As questões do MT3 serão disponibilizadas às 20:00

da sexta 7/jan/2022 e você vai ter 24 horas pra

entregar as respostas.


\bsk

Lembre que tudo tem que ser feito à mão e em papel!

Desenhos feitos no computador não serão considerados.

\newpage

% «questao»  (to ".questao")
% (c2m212mt3p 3 "questao")
% (c2m212mt3a   "questao")

% (c2m212tfc1p 26 "exercicio-3")
% (c2m212tfc1a    "exercicio-3")
% (find-LATEX "2021pict2e.lua" "Piecewise-tests")
% (find-LATEX "2021pict2e.lua" "Piecewise-tests" ":fun0")

% «from-piecewise»  (to ".from-piecewise")
% (c2m212mt3p 4 "from-piecewise")
% (c2m212mt3a   "from-piecewise")
%
% (setq eepitch-preprocess-regexp "^")
% (setq eepitch-preprocess-regexp "^\\(% \\)?%L ?")
% %L  (eepitch-lua51)
% %L  (eepitch-kill)
% %L  (eepitch-lua51)
% %L dofile "2021pict2e.lua"           -- (find-LATEX "2021pict2e.lua")
%L
%L funcaof_piecewise_spec = [[
%L     (0,0)--(1,0)o
%L     (1,1)c--(2,1)o (2,-1)c--(3,-1)o
%L     (3,2)c--(4,2)o (4,-1)c--(5,-1)o
%L     (5,1)c--(6,1)o (6,0)c--(8,0)o
%L     (8,4)c--(9,4)o (9,-3)c--(10,-3)o
%L     (8,4)c--(9,4)o (9,-3)c--(10,-3)o
%L     (10,2)c--(11,2)o (11,-1)c--(12,-1)o
%L     (12,1)c--(13,1)o (13,-1)c--(15,-1)
%L   ]]
%L funcaof = Piecewise.new(funcaof_piecewise_spec):fun()
%L pwif    = Piecewisify.new(funcaof, seq(0, 12, 1))
%L pwifpol = function (a, b)
%L     return Pict2e.new():add(pwif:pol(a, b, "*")):color("orange")
%L   end
% %L = pwifpol(3.5, 4.5)
%L
%L Pict2e.new()
%L   :setbounds(v(0,-4), v(15,5))
%L     :grid()
%L     :add("#1")
%L     :axesandticks()
%L     :add(pictpiecewise(funcaof_piecewise_spec):as("\\linethickness{1pt}"))
%L   :add("#2")
%L   :bepc()
%L   :def("funcaof#1#2")
%L   :output()
\pu

\def\funcaofnumeros{%
    \def\cellfont{\scriptsize}%
    \put(5,-0.6){\cell{5}}%
    \put(10,-0.6){\cell{10}}%
  }

Sejam:
%
\vspace*{-0.25cm}
%
$$f(x) \;\;=\;\;
  \unitlength=12pt
  \funcaof{}{\funcaofnumeros}
  %\funcaof{\expr{pwifpol(3.5, 8.5)}}{\funcaofnumeros}
$$

e $G(x) = \Intt{3.5}{x}{f(t)}$.

\msk

a) Represente graficamente $G(8.5)$.

b) Dê o valor numérico de $G(8.5)$.

c) Desenhe o gráfico da função $G$. Dica: o domínio dela é $[0,15]$.

\newpage

% «gabarito»  (to ".gabarito")
% (c2m212mt3p 4 "gabarito")
% (c2m212mt3a   "gabarito")

{\bf Gabarito}

\unitlength=9pt

a) $G(8.5) \;\;=\;\;
    %\funcaof{}{\funcaofnumeros}
    \funcaof{\expr{pwifpol(3.5, 8.5)}}{\funcaofnumeros}
   $

b) $G(8.5) = 3$

%L funcaoG_piecewise_spec = 
%L   "(0,-1)--(1,-1)--(2,0)--(3,-1)--(3.5,0)--" ..
%L   "(4,1)--(5,0)--(6,1)--(8,1)--(9,5)--(10,2)--(11,4)--(12,3)--(13,4)--(15,2)"
%L funcaof = Piecewise.new(funcaof_piecewise_spec):fun()
%L pwif    = Piecewisify.new(funcaof, seq(0, 12, 1))
%L pwifpol = function (a, b)
%L     return Pict2e.new():add(pwif:pol(a, b, "*")):color("orange")
%L   end
% %L = pwifpol(3.5, 4.5)
%L
%L Pict2e.new()
%L   :setbounds(v(0,-4), v(15,5))
%L     :grid()
%L     :add("#1")
%L     :axesandticks()
%L     :add(pictpiecewise(funcaoG_piecewise_spec):as("\\linethickness{1pt}"))
%L   :add("#2")
%L   :bepc()
%L   :def("funcaoG#1#2")
%L   :output()
\pu

c) $G(x) \;\;\;=\;\;\;\, \funcaoG{}{\funcaofnumeros}$



\newpage

% Duvida da Alice:
%
% $G(0) \;\;=\;\;
%  %\funcaof{}{\funcaofnumeros}
%  \funcaof{\expr{pwifpol(0, 3.5)}}{\funcaofnumeros}
% $




%\printbibliography

\GenericWarning{Success:}{Success!!!}  % Used by `M-x cv'

\end{document}

%  ____  _             _         
% |  _ \(_)_   ___   _(_)_______ 
% | | | | \ \ / / | | | |_  / _ \
% | |_| | |\ V /| |_| | |/ /  __/
% |____// | \_/  \__,_|_/___\___|
%     |__/                       
%
% «djvuize»  (to ".djvuize")
% (find-LATEXgrep "grep --color -nH --null -e djvuize 2020-1*.tex")

 (eepitch-shell)
 (eepitch-kill)
 (eepitch-shell)
# (find-fline "~/2021.2-C2/")
# (find-fline "~/LATEX/2021-2-C2/")
# (find-fline "~/bin/djvuize")

cd /tmp/
for i in *.jpg; do echo f $(basename $i .jpg); done

f () { rm -v $1.pdf;  textcleaner -f 50 -o  5 $1.jpg $1.png; djvuize $1.pdf; xpdf $1.pdf }
f () { rm -v $1.pdf;  textcleaner -f 50 -o 10 $1.jpg $1.png; djvuize $1.pdf; xpdf $1.pdf }
f () { rm -v $1.pdf;  textcleaner -f 50 -o 20 $1.jpg $1.png; djvuize $1.pdf; xpdf $1.pdf }

f () { rm -fv $1.png $1.pdf; djvuize $1.pdf }
f () { rm -fv $1.png $1.pdf; djvuize WHITEBOARDOPTS="-m 1.0 -f 15" $1.pdf; xpdf $1.pdf }
f () { rm -fv $1.png $1.pdf; djvuize WHITEBOARDOPTS="-m 1.0 -f 30" $1.pdf; xpdf $1.pdf }
f () { rm -fv $1.png $1.pdf; djvuize WHITEBOARDOPTS="-m 1.0 -f 45" $1.pdf; xpdf $1.pdf }
f () { rm -fv $1.png $1.pdf; djvuize WHITEBOARDOPTS="-m 0.5" $1.pdf; xpdf $1.pdf }
f () { rm -fv $1.png $1.pdf; djvuize WHITEBOARDOPTS="-m 0.25" $1.pdf; xpdf $1.pdf }
f () { cp -fv $1.png $1.pdf       ~/2021.2-C2/
       cp -fv        $1.pdf ~/LATEX/2021-2-C2/
       cat <<%%%
% (find-latexscan-links "C2" "$1")
%%%
}

f 20201213_area_em_funcao_de_theta
f 20201213_area_em_funcao_de_x
f 20201213_area_fatias_pizza



%  __  __       _        
% |  \/  | __ _| | _____ 
% | |\/| |/ _` | |/ / _ \
% | |  | | (_| |   <  __/
% |_|  |_|\__,_|_|\_\___|
%                        
% <make>

 (eepitch-shell)
 (eepitch-kill)
 (eepitch-shell)
# (find-LATEXfile "2019planar-has-1.mk")
make -f 2019.mk STEM=2021-2-C2-MT3 veryclean
make -f 2019.mk STEM=2021-2-C2-MT3 pdf

% Local Variables:
% coding: utf-8-unix
% ee-tla: "c2mt3"
% ee-tla: "c2m212mt3"
% End:
