% (find-LATEX "2021-2-C3-funcoes-homogeneas.tex")
% (defun c () (interactive) (find-LATEXsh "lualatex -record 2021-2-C3-funcoes-homogeneas.tex" :end))
% (defun C () (interactive) (find-LATEXsh "lualatex 2021-2-C3-funcoes-homogeneas.tex" "Success!!!"))
% (defun D () (interactive) (find-pdf-page      "~/LATEX/2021-2-C3-funcoes-homogeneas.pdf"))
% (defun d () (interactive) (find-pdftools-page "~/LATEX/2021-2-C3-funcoes-homogeneas.pdf"))
% (defun e () (interactive) (find-LATEX "2021-2-C3-funcoes-homogeneas.tex"))
% (defun o () (interactive) (find-LATEX "2021-2-C3-funcoes-homogeneas.tex"))
% (defun u () (interactive) (find-latex-upload-links "2021-2-C3-funcoes-homogeneas"))
% (defun v () (interactive) (find-2a '(e) '(d)))
% (defun d0 () (interactive) (find-ebuffer "2021-2-C3-funcoes-homogeneas.pdf"))
% (defun cv () (interactive) (C) (ee-kill-this-buffer) (v) (g))
%          (code-eec-LATEX "2021-2-C3-funcoes-homogeneas")
% (find-pdf-page   "~/LATEX/2021-2-C3-funcoes-homogeneas.pdf")
% (find-sh0 "cp -v  ~/LATEX/2021-2-C3-funcoes-homogeneas.pdf /tmp/")
% (find-sh0 "cp -v  ~/LATEX/2021-2-C3-funcoes-homogeneas.pdf /tmp/pen/")
%     (find-xournalpp "/tmp/2021-2-C3-funcoes-homogeneas.pdf")
%   file:///home/edrx/LATEX/2021-2-C3-funcoes-homogeneas.pdf
%               file:///tmp/2021-2-C3-funcoes-homogeneas.pdf
%           file:///tmp/pen/2021-2-C3-funcoes-homogeneas.pdf
% http://angg.twu.net/LATEX/2021-2-C3-funcoes-homogeneas.pdf
% (find-LATEX "2019.mk")
% (find-CN-aula-links "2021-2-C3-funcoes-homogeneas" "3" "c3m212fh" "c3fh")

% «.defs»		(to "defs")
% «.title»		(to "title")
% «.exercicio-1»	(to "exercicio-1")
% «.exercicio-2»	(to "exercicio-2")
% «.exercicio-2-fig»	(to "exercicio-2-fig")
% «.exercicio-3»	(to "exercicio-3")
%
% «.djvuize»		(to "djvuize")



% <videos>
% Video (not yet):
% (find-ssr-links     "c3m212fh" "2021-2-C3-funcoes-homogeneas")
% (code-eevvideo      "c3m212fh" "2021-2-C3-funcoes-homogeneas")
% (code-eevlinksvideo "c3m212fh" "2021-2-C3-funcoes-homogeneas")
% (find-c3m212fhvideo "0:00")

\documentclass[oneside,12pt]{article}
\usepackage[colorlinks,citecolor=DarkRed,urlcolor=DarkRed]{hyperref} % (find-es "tex" "hyperref")
\usepackage{amsmath}
\usepackage{amsfonts}
\usepackage{amssymb}
\usepackage{pict2e}
\usepackage[x11names,svgnames]{xcolor} % (find-es "tex" "xcolor")
\usepackage{colorweb}                  % (find-es "tex" "colorweb")
%\usepackage{tikz}
%
% (find-dn6 "preamble6.lua" "preamble0")
%\usepackage{proof}   % For derivation trees ("%:" lines)
%\input diagxy        % For 2D diagrams ("%D" lines)
%\xyoption{curve}     % For the ".curve=" feature in 2D diagrams
%
\usepackage{edrx21}               % (find-LATEX "edrx21.sty")
\input edrxaccents.tex            % (find-LATEX "edrxaccents.tex")
\input edrx21chars.tex            % (find-LATEX "edrx21chars.tex")
\input edrxheadfoot.tex           % (find-LATEX "edrxheadfoot.tex")
\input edrxgac2.tex               % (find-LATEX "edrxgac2.tex")
%\usepackage{emaxima}              % (find-LATEX "emaxima.sty")
%
%\usepackage[backend=biber,
%   style=alphabetic]{biblatex}            % (find-es "tex" "biber")
%\addbibresource{catsem-slides.bib}        % (find-LATEX "catsem-slides.bib")
%
% (find-es "tex" "geometry")
\usepackage[a6paper, landscape,
            top=1.5cm, bottom=.25cm, left=1cm, right=1cm, includefoot
           ]{geometry}
%
\begin{document}

\catcode`\^^J=10
\directlua{dofile "dednat6load.lua"}  % (find-LATEX "dednat6load.lua")
%
%L dofile "2021pict2e.lua"           -- (find-LATEX "2021pict2e.lua")
%L Pict2e.__index.suffix = "%"
\pu
\def\pictgridstyle{\color{GrayPale}\linethickness{0.3pt}}
\def\pictaxesstyle{\linethickness{0.5pt}}


% %L dofile "edrxtikz.lua"  -- (find-LATEX "edrxtikz.lua")
% %L dofile "edrxpict.lua"  -- (find-LATEX "edrxpict.lua")
% \pu

% «defs»  (to ".defs")
% (find-LATEX "edrx21defs.tex" "colors")
% (find-LATEX "edrx21.sty")

\def\u#1{\par{\footnotesize \url{#1}}}

\def\drafturl{http://angg.twu.net/LATEX/2021-2-C3.pdf}
\def\drafturl{http://angg.twu.net/2021.2-C3.html}
\def\draftfooter{\tiny \href{\drafturl}{\jobname{}} \ColorBrown{\shorttoday{} \hours}}



%  _____ _ _   _                               
% |_   _(_) |_| | ___   _ __   __ _  __ _  ___ 
%   | | | | __| |/ _ \ | '_ \ / _` |/ _` |/ _ \
%   | | | | |_| |  __/ | |_) | (_| | (_| |  __/
%   |_| |_|\__|_|\___| | .__/ \__,_|\__, |\___|
%                      |_|          |___/      
%
% «title»  (to ".title")
% (c3m212fhp 1 "title")
% (c3m212fha   "title")

\thispagestyle{empty}

\begin{center}

\vspace*{1.2cm}

{\bf \Large Cálculo 3 - 2021.2}

\bsk

Aula 23: funções homogêneas

\bsk

Eduardo Ochs - RCN/PURO/UFF

\url{http://angg.twu.net/2021.2-C3.html}

\end{center}

\newpage

Assista os vídeos que eu gravei sobre

o mini-teste 2:

\msk

% (c3m212mt2a     "video-a")

{\footnotesize

\url{http://angg.twu.net/LATEX/2021-2-C3-MT2.pdf}

\msk

\url{http://angg.twu.net/eev-videos/2021-2-C3-MT2.mp4}

\url{https://www.youtube.com/watch?v=Rz01pLaL9Z0}

\ssk

\url{http://angg.twu.net/eev-videos/2021-2-C3-MT2-2.mp4}

\url{https://www.youtube.com/watch?v=lDvAMU5aNvc}

\ssk

\url{http://angg.twu.net/eev-videos/2021-2-C3-MT2-3.mp4}

\url{https://www.youtube.com/watch?v=cL3G-t3mILs}


}

\msk

O mais importante vai ser o vídeo 3.

\newpage

% «exercicio-1»  (to ".exercicio-1")
% (c3m212fhp 3 "exercicio-1")
% (c3m212fha   "exercicio-1")

{\bf Exercício 1.}

Digamos que $f(x,y)$ é uma função que obedece isso aqui:
%
$$∀x,y,k. \; f(kx,ky) = k^3 f(x,y).$$

Digamos que f(2,3) = 5.

\msk

Descubra os valores de:

\ssk

a) f(4,6)

b) f(20,30)

c) f(-2,-3)

d) f(0,0).


\bsk

{\bf Dica:} no vídeo eu usei $k^2$ em todo lugar.

Neste exercício eu tou usando $k^3$.



\newpage

% «exercicio-2»  (to ".exercicio-2")
% (c3m212fhp 4 "exercicio-2")
% (c3m212fha   "exercicio-2")

{\bf Exercício 2.}

Este exercício é uma versão mais visual do exercício 1.

Digamos que $g(x,y)$ obedece isto:
%
$$∀x,y,k. \; g(kx,ky) = k^3 g(x,y).$$

Complete o diagrama de numerozinhos da página seguinte.

Os números dele indicam valores que você sabe,

como por exemplo $g(0,1)=3$.


Os pontos de interrogação dele indicam valores que

você tem que descobrir... por exemplo ``$g(0,2)=\ColorRed{?}$''.

\newpage

% «exercicio-2-fig»  (to ".exercicio-2-fig")
% (c3m221fhp 5 "exercicio-2-fig")
% (c3m221fha   "exercicio-2-fig")
% (c3m212mt2p 7 "figura-homogeneas")
% (c3m212mt2a   "figura-homogeneas")

% (find-LATEX "edrx21.sty" "picture-cells")
\unitlength=8pt
\celllower=2.5pt
\def\cellfont{\footnotesize}
\def\cellfont{\scriptsize}

\unitlength=15pt

%L Pict2e.__index.axes_ = "\\linethickness{0.1pt}"
%L Pict2e.__index.axes_ = "\\color{gray}"
%L
%L Pict2e.new()
%L   :setbounds(v(-5,-5), v(5,5))
%L     :axesandticks()
%L     :run(function (p)
%L        local f = function(x, y, z)
%L            p:puttext(v(x,y), z or "\\ColorRed{?}")
%L          end
%L        f(1,1,2); f(2,2); f(3,3); f(0,0); f(-1,-1); f(-2,-2)
%L        f(2,1,5); f(4,2); f(-2,-1); f(-4,-2);
%L        f(0,1,3); f(0,2); f(0,3); f(0,-1); f(0,-2)
%L        f(1,-2,4); f(2,-4); f(-1,2); f(-2,4)
%L        f(1,0,8); f(2,0); f(3,0); f(-1,0); f(-2,0)
%L       end)
%L   :bepc()
%L   :def("Foo")
%L   :output()
\pu

$$\Foo
$$


\newpage

% «exercicio-3»  (to ".exercicio-3")
% (c3m212fhp 6 "exercicio-3")
% (c3m212fha   "exercicio-3")

{\bf Exercício 3.}


\scalebox{0.8}{\def\colwidth{12cm}\firstcol{

Digamos que $h(x,y)$ é uma função que obedece estas duas coisas:
%
$$∀x,y,k. \; h(kx,ky) = k^3 h(x,y),$$
$$∀x. \; h(x,1) = (x-2)(x+3).$$

a) Encontre dois pontos da reta $y=1$ nos quais vale $h(x,1)=0$.

b) Faça o diagrama de sinais da função $h(x,y)$ na reta $y=1$.

c) Encontre dois pontos da reta $y=2$ nos quais vale $h(x,2)=0$.

d) Faça o diagrama de sinais da função $h(x,y)$ na reta $y=2$.

e) Encontre dois pontos da reta $y=-1$ nos quais vale $h(x,-1)=0$.

f) Faça o diagrama de sinais da função $h(x,y)$ na reta $y=-1$.

g) Faça o diagrama de sinais da função $h(x,y)$ no conjunto
%
$$\setofxyst{y≠0}.$$

Note que não sabemos o comportamento da função $h(x,y)$

na reta $y=0$ --- exceto no ponto $(0,0)$.

%}\anothercol{
}}





\newpage

%\printbibliography

\GenericWarning{Success:}{Success!!!}  % Used by `M-x cv'

\end{document}

%  ____  _             _         
% |  _ \(_)_   ___   _(_)_______ 
% | | | | \ \ / / | | | |_  / _ \
% | |_| | |\ V /| |_| | |/ /  __/
% |____// | \_/  \__,_|_/___\___|
%     |__/                       
%
% «djvuize»  (to ".djvuize")
% (find-LATEXgrep "grep --color -nH --null -e djvuize 2020-1*.tex")

 (eepitch-shell)
 (eepitch-kill)
 (eepitch-shell)
# (find-fline "~/2021.2-C3/")
# (find-fline "~/LATEX/2021-2-C3/")
# (find-fline "~/bin/djvuize")

cd /tmp/
for i in *.jpg; do echo f $(basename $i .jpg); done

f () { rm -v $1.pdf;  textcleaner -f 50 -o  5 $1.jpg $1.png; djvuize $1.pdf; xpdf $1.pdf }
f () { rm -v $1.pdf;  textcleaner -f 50 -o 10 $1.jpg $1.png; djvuize $1.pdf; xpdf $1.pdf }
f () { rm -v $1.pdf;  textcleaner -f 50 -o 20 $1.jpg $1.png; djvuize $1.pdf; xpdf $1.pdf }

f () { rm -fv $1.png $1.pdf; djvuize $1.pdf }
f () { rm -fv $1.png $1.pdf; djvuize WHITEBOARDOPTS="-m 1.0 -f 15" $1.pdf; xpdf $1.pdf }
f () { rm -fv $1.png $1.pdf; djvuize WHITEBOARDOPTS="-m 1.0 -f 30" $1.pdf; xpdf $1.pdf }
f () { rm -fv $1.png $1.pdf; djvuize WHITEBOARDOPTS="-m 1.0 -f 45" $1.pdf; xpdf $1.pdf }
f () { rm -fv $1.png $1.pdf; djvuize WHITEBOARDOPTS="-m 0.5" $1.pdf; xpdf $1.pdf }
f () { rm -fv $1.png $1.pdf; djvuize WHITEBOARDOPTS="-m 0.25" $1.pdf; xpdf $1.pdf }
f () { cp -fv $1.png $1.pdf       ~/2021.2-C3/
       cp -fv        $1.pdf ~/LATEX/2021-2-C3/
       cat <<%%%
% (find-latexscan-links "C3" "$1")
%%%
}

f 20201213_area_em_funcao_de_theta
f 20201213_area_em_funcao_de_x
f 20201213_area_fatias_pizza



%  __  __       _        
% |  \/  | __ _| | _____ 
% | |\/| |/ _` | |/ / _ \
% | |  | | (_| |   <  __/
% |_|  |_|\__,_|_|\_\___|
%                        
% <make>

 (eepitch-shell)
 (eepitch-kill)
 (eepitch-shell)
# (find-LATEXfile "2019planar-has-1.mk")
make -f 2019.mk STEM=2021-2-C3-funcoes-homogeneas veryclean
make -f 2019.mk STEM=2021-2-C3-funcoes-homogeneas pdf

% Local Variables:
% coding: utf-8-unix
% ee-tla: "c3fh"
% ee-tla: "c3m212fh"
% End:
